\documentclass[12pt]{article}
\usepackage[hcentering,bindingoffset=20mm]{geometry}
\usepackage{placeins}
\usepackage[numbib]{tocbibind}
\usepackage{rotating}
\usepackage[square,sort,comma,numbers]{natbib}
\usepackage{graphicx}
\usepackage{tabularx}
\linespread{1.3}
%\fontsize{8cm}{1.3em}\selectfont
\usepackage{gensymb}
\usepackage{longtable}
\usepackage{lscape}
\usepackage{url}
\addtolength{\textwidth}{2cm}
\addtolength{\hoffset}{-1cm}


\addtolength{\textheight}{2cm}
\addtolength{\voffset}{-1cm}
\setlength{\parindent}{0pt}

\title{Characterisation of \emph{Gambierdiscus petramus} sp. nov. (Gonyaulacales): a new dinoflagellate from the Great Barrier Reef (Australia) that produces a ciguatoxin analogue.}
\author{Anna Liza Kretzschmar \thanks{School Plant Functional Ecology and Climate Change Cluster (C3), University of Technology Sydney, Ultimo, NSW, Australia}, Arjun Verma \footnotemark[1], Tim Harwood \thanks{Cawthron Institute, Nelson, New Zealand},\\
Mona Hoppenrath \thanks{Senckenberg Research Institute, German Centre for Marine Biodiversity Research, Wilhelmshaven, Germany}, Shauna Murray \footnotemark[1]}
\date{}

\begin{document}
\maketitle
\newpage
\section{Abstract}
\textit{Gambierdiscus} is a genus of benthic dinoflagellates which is found worldwide. Some species produce neurotoxins (maitotoxins and ciguatoxins) which  bioaccumulate and cause ciguatera fish poisoning. In this study we characterised five strains of \textit{Gambierdiscus} collected from Heron Island,  Australia, a region in which ciguatera is endemic. Clonal cultures were processed using (i) light microscopy; (ii) scanning electron microscopy; (iii) DNA sequencing based on the nuclear encoded ribosomal  SSU and D8-D10 LSU regions; (iv) toxicity via mouse bioassay (MBA) and; (v) toxin profile as determined by LC-MS. Correlation of morphological and phylogenetic data indicated that these strains represent a new species of Gambierdiscus, \emph{Gambierdiscus petramus} sp. nov.. Cultures produced maitotoxin 3 and a yet uncharacterised congener of ciaguatoxin. The investigation of toxigenic species of Gambierdiscus in CFP endemic regions continues the investigation for the causative agent of ciguatera in Australia.
\newpage

\section{Introduction}
%In general, your introduction comes across as fairly random rather than structured. 
%You need to make each paragraph about one key point that is essential background to this this particular piece of work.
%You can make up a title just for your own guidance, and then delete afterwards. 
%Remember, essential writing structure is that the first sentence of the paragraph  is a summary of the key point. Then the evidence is elaborated. Then the last sentence of the paragraph sums up the key message of the paragraph.
%Ie (in no particular order) 
%Paragraph 1. Prior knowledge of Gambierdiscus in Australia, especially the GBR.
%2. Cryptic species of Gambierdsicus and morphological variability within species .
%3. Toxicity in Gambierdiscus species.
%4. Background on CFP - symptoms, distribution of CFP in the GBR region.


\subsection{overview, history, diff sp}
\emph{Gambierdiscus} is a genus of benthic dinoflagellate, epiphytic on many substrates in shallow tropical and sub-tropical waters that can produce neurotoxins responsible for ciguatera fish poisoning (CFP) through bioaccumulation \cite{berdalet2012global}. 
Since the discovery of \emph{Gambierdiscus} in 1977 \cite{yasumoto1977finding}, extensive research has established that members of the genus are the primary sources of ciguatoxins (CTXs), the causative agent of CFP \cite{chinain1997intraspecific,holmes1998gambierdiscus}. Maitotoxins (MTXs) are also commonly produced, however whether they play a role in CFP is yet to be established \cite{kohli2014feeding}. 

For over 15 years \emph{Gambierdiscus} was considered to be a monotypic taxon, but since 1995 new species have been discovered with more extensive sampling in the Atlantic Ocean, Indian Ocean and Pacific \cite{faust1995observation,holmes1998gambierdiscus,litaker2009taxonomy,chinain1999morphology,fraga2011gambierdiscus,nishimura2014morphology}.

Currently 11 species, and six unnamed types of \emph{Gambierdiscus} have been described based on their distinct morphological and genetic characteristics \cite{adachi1979thecal,faust1995observation,chinain1999morphology,litaker2009taxonomy,nishimura2014morphology,fraga2011gambierdiscus}. Data of \emph{Gambierdiscus} spp. distribution and toxin production pre-dating the discovery of new species, sheds uncertainty of the reported species identity. Ascertaining the distribution of species and their toxin production is essential for assessing risk of CFP to local populations as well as the seafood industry.

\subsection{toxins and CFP}
CFP is the most common non bacterial illness associated with seafood consumption \cite{friedman2008ciguatera}, however the correct diagnosis of the illness is difficult due to extensive list of neurological, gastrointestinal and cardiovascular symptoms \cite{sims1987theoretical}. CFP is a highly morbid disease, predicted to cost \$2,421,080 in annual monetary loss in the USA alone \cite{minor2014per}.  UNESCO's international panel for harmful algal blooms has proposed a coordinated global ciguatera strategy to address the impact of the disease through \emph{Gambierdiscus} spp. detection and monitoring, improved toxin detection in seafood and epidemiological data collation \cite{globalcig}.
The suite of toxins produced by \emph{Gambierdiscus} varies between species. Bioassays are commonly used to assess toxicity, however it does not suffice to elucidate a detailed toxin profile - LC-MS or tandem LC-MS/MS analysis is required \cite{diogened2014chemistry}. Some species exclusively produce MTX (\emph{G. australes} \cite{rhodes2014production}) while others produce CTX (\emph{G. polynesiensis} \cite{rhodes2014production}) or neither (\emph{G. carpenteri} \cite{kohli2014high}) as confirmed by LC-MS. 

The genus \emph{Fukuyoa} came into effect in 2015 \cite{gomez2015fukuyoa}, but as \emph{F. rutzleri}, \emph{F. yasumotoi} and \emph{F. paulensis} produce MTX and hence may contribute to CFP \cite{kohli2014feeding}, they will be referred to alongside \emph{Gambierdiscus}.


\subsection{abundance and distribution}
Some \emph{Gambierdiscus} species have been classified as endemic to either the Pacific or Atlantic Oceans, while others have been found globally distributed \cite{berdalet2012global,litaker2010global}. %It has been suggested that with more extensive sampling, the distribution is likely to be global for all \emph{Gambierdiscus} species \cite{testerICHA}. 
However, the current understanding about \emph{Gambierdiscus} distribution and abundance is fragmentary due to the relatively recent progress regarding the phylogeny within the genus, the continued discovery of new species, difficulty with species identification with light microscopy \cite{berdalet2012global} and the comparatively lower sampling frequency in the Atlantic Ocean. 

Different species of \emph{Gambierdiscus} usually co-occur at sample sites \cite{litaker2010global}. As species can show morphological variability within species \cite{bravo2014cellular}, but can also be highly morphologically similar to one another \cite{kohli2014high}, molecular genetic techniques are necessary to complement morphology in elucidating species composition. Prior to Litaker et al's study in 2009, the taxonomy of the genus was poorly defined which makes species assignment prior to that study uncertain \cite{berdalet2012global}. %Morphology alongside genetic tools can identify species with a high certainty.

%Annual reported CFP cases are estimated up to 500,000 globally and are responsible for 80 to 96 \% of human poisoning from consuming seafood \cite{fleming1998seafood,grandjean2008centers}. ---> check ref
 As CFP is increasing in frequency in the Pacific \cite{skinner2011ciguatera} and in other regions, an understanding of the distribution and species specific toxicology of \emph{Gambierdiscus} spp. is essential \cite{globalcig}.
 



\subsection{australia}
\FloatBarrier
\begin{figure} 
\includegraphics[scale=.1]{oz-gamb-grey-map.png} 
\caption{Sites where Gambierdiscus isolates have been reported in Australia. Site 1 indicates Raine Island, QLD; site 2 indicates Townsville, QLD; Site 3 indicates Heron Island, QLD; Site 4 indicated Platypus Bay, QLD; site 5 indicates Wapengo Lagoon, NSW; site 6 indicates Merimbula, NSW; site 7 indicates Eden, NSW; and site 8 indicates Exmouth, WA.​} 
\label{fig:OzSites}
\end{figure}
\FloatBarrier 
%CFP is endemic to the Great Barrier Reef region os Australia, though the report rate is predicted to only constitute 20\% \cite{lewis2006ciguatera}.
Cases of CFP have been reported in Australia, mostly around the Great Barrier Reef, but a CTX producing species of \emph{Gambierdiscus} has not yet been recorded \cite{lewis2006ciguatera}. This is most likely due to Australia having been relatively under sampled (figure ~\ref{fig:OzSites}). Sites 1, 2, 3 and 4 of fig. ~\ref{fig:OzSites} encompass the length of the GBR.
%CFP is endemic to Queensland \cite{lewis2006ciguatera} and sampling sites 1, 2, 3 and 4 in figure ~\ref{fig:OzMap} reflects the breadth of the Great Barrier Reef. 

In 1994 Lewis et al reported three analogues of MTX, designated MTX-1, MTX-2 and MTX-3, isolated from a \emph{Gambierdiscus} sp. in Queensland, Australia \cite{holmes1994purification}. As this discovery pre-dates the understanding that \emph{Gambierdiscus} includes more species than \emph{G. toxicus}, it is unclear from which species these analogues were isolated.  \\


\emph{G. belizeanus} was found in temperate waters of Heron Island, Great Barrier Reef (site 3 in fig. ~\ref{fig:OzSites}) \cite{murray2014molecular}. \emph{G. belizeanus} has tested positive for CTX via RBA \cite{chinain2010growth} and MTX via HELA \cite{holland2013differences}. These results have not yet been verified by LC-MS.\\

\emph{F. yasumotoi} has been found in locations spanning a 1550km length of the Great Barrier Reef. Isolates were found at Raine Island (site 1 in fig. ~\ref{fig:OzSites}), Nelly Bay at Townsville (site 2 in fig. ~\ref{fig:OzSites}) and Heron Island (site 3 in fig. ~\ref{fig:OzSites}) \cite{murray2014molecular}. \emph{F. yasumotoi} was MTX positive for MBA \cite{holmes1998gambierdiscus}, however in a New Zealand strain no known CTX or MTX analogues were detected via LC-MS, though a variant of MTX may be present \cite{rhodes2014gambierdiscus}.

\emph{G. toxicus} has been found in both NSW and QLD. In NSW, the samples were found in Eden (site 7 in fig. ~\ref{fig:OzSites}) \cite{hallegraeff2010algae}, while in QLD it was found at Platypus Bay, Townsville (site 2 in fig. ~\ref{fig:OzSites}) \cite{hallegraeff2010algae}. \emph{G. toxicus} tested CTX positive via RBA \cite{chinain2010growth} and MTX positive via MBA \cite{chinain1999morphology} but these bioassay results have not yet been verified by LC-MS.\\

The species \emph{G. carpenteri} was found to be the only \emph{Gambierdiscus} species present in temperate New South Wales waters \cite{kohli2014high}. The species was detected at two sites at Merimbula (site 6 in fig. ~\ref{fig:OzSites}) and at Wapengo Lagoon (site 5 in fig. ~\ref{fig:OzSites}). No CTX or MTX was detected in these strains via LC-MS/MS, however the MTX-like fraction did display toxicity via MBA  \cite{kohli2014high}.\\ 

Kohli et al identified \emph{Gambierdiscus} spp. as part of benthic dinoflagellate communities via pyrosequencing in Western Australia at Town Beach, Ningaloo reef, Exmouth (site 8 in fig. ~\ref{fig:OzSites}).  \emph{Gambierdiscus} was identified at the genus level, so which species are fount in WA has not yet been determined \cite{kohli2014cob}. \\

It is highly likely that many more \emph{Gambierdiscus} species are present at the sites, as in each case, the studies represent only single or short term sampling events at one location within a site. Hence the toxicity of strains from Australia has not been examined, and their distribution along the Australian coast are largely not known. Consequently, the causative agent for CFP in Australia has not yet been determined. \\

\FloatBarrier 
The aim of the current study was to further the continued hunt for the causative agent of CFP in Australia by characterising a new species of \emph{Gambierdiscus} from a ciguateric region using molecular and morphologic methods, and investigate the toxicology of the isolates.


 \newpage
\section{Materials and methods}

\subsection{Specimen collection and culture conditions}
Specimens were collected from Heron Island (23.4420$^{\circ}$ S, 151.9140$^{\circ}$ E), as part of the Great Barrier Reef (GBR). 
The culture HG1 was isolated in October 2013 from \emph{Halimeda} sp. with an average sea surface temperature was 23.8 $^{\circ}$C. 
The cultures HG4, HG5, HG6, HG7 and HG 26 were isolated in July 2014 from \emph{Padina} sp. with an average sea surface temperature was 19.8 $^{\circ}$C.
Macroalgal samples were taken around the Heron Island research station (figure ~\ref{fig:HeronMap}). The samples were shaken in seawater to dislodge protists, the samples was then concentrated by filtration \cite{litaker2010global}. Single cells were isolated from the samples using a micropipette under an inverted light microscope, washed thrice in sterile seawater and transfered into F/10 medium \cite{holmes1991strain} to establish clonal cultures.
The cultures were maintained in F/10 medium at 26 $^{\circ}$C, 60mol/ m$^{2}$/ s light in 12hr:12hr light to dark cycles.

\FloatBarrier 
\begin{figure} 
%\includegraphics[scale=.4]{} 
\caption{Sample collection site at Heron Island - Arjun.} 
\label{fig:HeronMap}
\end{figure} 
\FloatBarrier 
%\newpage



\subsection{Morphological analyses}
The cultures HG4, HG6, HG7 and HG26 were fixed with 10 \% Lugol's iodide and examined with an inverted light microscope ( xxx) equipped with xx camera for cell size. Depth (D: measured along dorsal-ventral axis) and width (W: measured along the lateral axis) was measured for 30 to 36 cells per strain.
For HG4, HG5, HG6, HG7 and HG26 30ml of culture at stationary growth phase were centrifuged at 15,000 rcf and concentrated to 2ml, preserved with 10 \% Lugol's iodide, and analysed by scanning electron microscopy (SEM) at the Forschungsinstitut Senckenberg (Wilhelmshaven, Germany).  

\subsection{Genomic DNA extraction}
Genomic DNA was extracted using the CTAB method \cite{zhou1999analysis}. Purity and concentration of the extract was measured by Nanodrop, the integrity was visualised on 1\% agarose gel.

\subsection{PCR amplification and sequencing}
Extracted DNA was used as a template to amplify rDNA sequences in 25$\mu$l mixture in PCR tubes. The mixture contains 0.6 $\mu$M forward and reverse primer, 0.4 $\mu$M BSA, 2 - 20 ng DNA, 12.5 $\mu$l 2xEconoTaq (Lucigen) and 7.5 $\mu$l PCR grade water.\\
The PCR cycling comprised of an initial 10 min step at 94
The LSU D1-D3  and D8-D10 regions as well as the SSU region were amplified with the D1R-F-, FD8-RB and 18ScomF1-18ScomR1 primer sets respectively (table ~\ref{tbl:PrimerTable}).\\
Sanger sequencing was conducted by Macrogen (Korea).


\subsection{Sequence alignment and phylogenetic analysis}
Sequencing data was aligned with \emph{Gambierdiscus} spp. data from the GenBank reference database (accession numbers as part of ~\ref{fig:HGD8D10} and ~\ref{fig:HGSSU}). The allignment algorithm MUSCLE, with a maximum of 8 iterations \cite{edgar2004muscle}, was used through the Geneious software, version 8.1.7 \cite{kearse2012geneious}. Alignments were truncated to the same length (D10-D8 and SSU at 799bp and 1690bp respectively), and discrepancies removed.
Phylogenetic trees was calculated with Bayesian inference as well as maximum likelihood. Bayesian inference, via Mr. Bayes 3.2.2, was used to estimate the posterior probability distribution with Metropolis-Coupled Markov Chain Monte Carlo \cite{ronquist2003mrbayes}. A random staring tree with three heated and one cold chain with a temperature set at 0.2. Trees were sampled every 100th generation for the 2,000,000 generations generated.
PHYML was used for maximum lilkelihood (ML) analysis with 1000 bootstraps \cite{guindon2003simple}.
The general time reversal (GTR) model was used for both analyses.\\

\subsection{Toxin production}

\subsubsection{Toxicity via mouse bioassay}
The cultures HG4, HG6 and HG7 were freeze-dried after 36 days in culture and processed by Rex Munday at AgResearch, New Zealand. The weights of the samples were 57.0 mg, 155.0 mg and 23.0 mg respectively. The samples were extracted exhaustively with methanol. The extracts were evaporated to a small volume at 35$^{\circ}$C using a rotary evaporator, and the solution aliquotted into glass vials. The remaining methanol was removed under a stream of nitrogen, and the extracts freeze-dried overnight. The total weight of the extracts, which were all brownishgreen gums, were: HG4, 22.5 mg HG6, 42.9 mg HG7, 15.1 mg The median lethal doses of the test materials by intraperitoneal injection were determined according to the principles of OECD Guideline 425. Samples were dissolved in 1\% Tween 60 in normal saline immediately before dosing. For toxicity by intraperitoneal injection, aliquots of this solution, made up to a total volume of 1 ml with the same solvent, were injected into female Swiss mice, of initial body weight 18-22 g. The toxicity of the extract of HG6 by gavage was also investigated, when aliquots of the extract solution were made up to 200 µl with Tween-saline. The mice were monitored intensively during the day of dosing. Survivors were examined and weighed each day for the following 13 days, after which they were killed and necropsied. Tap water and food (Rat and Mouse Cubes, Speciality Feeds Ltd, Glen Forrest, Western Australia) were available at all times.

\subsection{Toxin profile via LC-MS}
After 36 days in culture 2 litres of HG4, HG6 and HG7 were centrifuged (  x g) and the resultant pellet freeze-dried. Pellets were extracted with methanol and screened for selected CTX and MTX analogues with quantitative LC-MS/MS methods developed by the Cawthron Institute \cite{kohli2014feeding}.
\textbf{Tim}, I've taken this part of the method from Lesley's 2014 paper on the Cook Island \emph{Gambierdiscus} spp. Please amend as appropriate. 

%For toxin profile elucidation, xxL of culture at the late exponential phase of growth were collected on 5 $\mu$m Millipore filters, washed off with sterile seawater and freeze dried for transportation to the Cawthron Institute (Nelson, New Zealand).
\newpage
\section{Results}

\subsection{\emph{Gambierdiscus petramus}}
\subsubsection{Morphology}
Cells of \emph{Gambierdiscus petramus} are photosynthetic, anterioposteriorly compressed and round to ellipsoid in apical view. \emph{Gambierdiscus petramus} cells are smaller than all other characterised \emph{Gambierdiscus} spp. (table ~\ref{tbl:GlobalSizeTable}), dorsoventral depth of 40.6 $\mu$m (range of 34.4 - 50.9 $\mu$m; standard deviation 3.3$\mu$m) and width of 39 $\mu$m (range of 32 - 46.9 $\mu$m; standard deviation 3.2$\mu$m)) with a depth to width ratio of 1.04 (standard deviation 0.06$\mu$m)  (n=128). The cell surface is highly areolated with pores. Flagella 2. The plate formula, via Kofoid tabulation, is Po, 4', 0a, 6'', 6c, 8s, 6''', 1p, 2''''. 
The apical pore complex (Po) is oval with a fish-hook shaped slit, located at the center of the epitheca (fig. ~\ref{fig:PetSEM}F). The largest apical plate is the hatchet-shaped 2' plate, followed by the pentagonal 3' plate and the hexagonal 1' plate is the smallest. The largest precingular plate is the 3'', followed by the 4'', 5'', 6'' then 2'' plates. The 1'' plate is the smallest (fig. ~\ref{fig:PetSEM}A, B and C) . The largest postcingular plate is the 4''' plate, followed by the 2''', 3''', 5''' then 1''' plates. The 1'''' antapical plate is smaller than the 2'''' plate, both are quadrangular (~\ref{fig:PetSEM}D). The 1p plate is narrow and pentagonal. The sulcus is deep excavated opening boardering on the cingulum (fig. ~\ref{fig:PetSEM}E).
The most distinguishing difference between \emph{G. petramus} and other \emph{Gambierdiscus} spp. is the species' size (table ~\ref{tbl:GlobalSizeTable} and fig. ~\ref{fig:SizeGraph}). The closest morphological relative to \emph{G. petramus} is \emph{G. belizeanus}. \emph{G. petramus} differs from \emph{G. belizeanus} in the 2' plate where the former's is hatchet shaped (fig. ~\ref{fig:PetSEM}A, B and C) and the latter's is rectangular.  

%-alt to petramus: cavus, citius, colo, florens

\FloatBarrier 
\begin{figure} 
%\includegraphics[scale=.02]{petramus-LM.png} 
\caption{Light micrographs of \emph{Gambierdiscus petramus}, strains HG4 (A); HG6 (B); HG7 (C); and HG26 (D). Scale bar equal 10 $\mu$m.​} 
\label{fig:PetLM}
\end{figure} 


\begin{figure} 
%\includegraphics[scale=.02]{petramus_SEM.png} 
\caption{SEM micrographs of \emph{Gambierdiscus petramus}, strains HG4 (D); HG6 (A and B); and HG7 (C, E and F). Scale bar equal 20 $\mu$m unless otherwise specified.} 
\label{fig:PetSEM}
\end{figure} 
\FloatBarrier 


\FloatBarrier 
\begin{figure} 
%\includegraphics[scale=.1]{size_chart-grey.png} 
\caption{Visual representation of the size difference between \emph{Gambierdiscus} species (table ~\ref{fig:SizeGraph}). Error bars denote standard deviation of meassurements for \emph{G. petramus}. Data taken from publications as follows: a) Chinain et al \cite{chinain1999morphology}; b) Fraga et al \cite{fraga2014genus}; c) Faust et al \cite{faust1995observation}; d) Gomez et al \cite{gomez2015fukuyoa}; e) Litaker et al \cite{litaker2009taxonomy}; f) Nishimura et al \cite{nishimura2014morphology}; and g) Xu et al \cite{xu2014distribution}} 
\label{fig:SizeGraph}
\end{figure} 
\FloatBarrier 
%\newpage

\subsubsection{Phylogenetics}

The D8D10 LSU (fig. ~\ref{fig:HGD8D10}) and SSU (fig. ~\ref{fig:HGSSU}) phylogenies of the Heron Island isolates HG1, HG4, HG6, HG7 and HG26 show that these strains fall into a new, well supported clade.\\
\begin{itemize}
\item Compare Bayesian inference and maximum likelihood topologies
\item Discuss discrepancies and link to previous papers eg Nishimuar '13
\item Comment on origin of isolates in trees, whether they form distinct clusters or if they are genetically indistinguishable
-discuss by clade
\end{itemize}
\FloatBarrier 

\begin{figure} 
%\includegraphics[scale=.1]{SSU_complex_geo_merge-f.png} 
\caption{Maximum likelihood phylogeny of \textit{Gambierdiscus} species/phylotypes of the SSU rDNA region. Nodal support are Bayesian posterior probability (pp) and bootstrap (bt) values obtained from Bayesian inference analysis and maximum likelihood analysis, respectively. Nodes with strong support (pp/bt = 1.00 / 100) are shown as thick lines.}
\label{fig:HGSSU} 
\end{figure} 
\FloatBarrier 

%\newpage
\begin{figure} 
%\includegraphics[scale=.1]{D8D10_complex_geo_merge-F.png} 
\caption{Maximum likelihood phylogeny of \textit{Gambierdiscus} species/phylotypes of the LSU D8-D10 rDNA region. Nodal support are Bayesian posterior probability (pp) and bootstrap (bt) values obtained from Bayesian inference analysis and maximum likelihood analysis, respectively. Nodes with strong support (pp/bt = 1.00 / 100) are shown as thick lines.} 
\label{fig:HGD8D10}
\end{figure} 
\FloatBarrier 
%\newpage


\subsubsection{Toxicology}

\paragraph{Toxicity via i.p. injection.}
The intraperitoneal LD50 of the extract of HG6 was 0.78 mg/kg, with 95\% confidence limits between 0.40 and 1.60 mg/kg. At high dose-levels, stretching movements were observed immediately after injection of the test material, most likely due to irritation. This was rapidly followed by cessation of movement and abdominal breathing. Respiration rates declined and death occurred within 1.5-4 hours after dosing. At necropsy, erythema was observed throughout the gastrointestinal tract of these animals At lower doses, death occurred at up to 24 hours after dosing. The stomachs of these animals were grossly distended with gas and with abnormally fluid contents. An abnormally large amount of material was present in the duodenum and upper jejunum, which was reddish-brown in colour and gelatinous in consistency. Caeca were also enlarged. Erythema in the glandular stomach was noted. No gross changes were observed in any other organ. 
The acute intraperitoneal toxicity of the extract of HG7 was lower than that of HD6, with an LD50 of 12.5 mg/kg, with 95\% confidence interval between 10.1 and 15.3 mg/kg. The symptoms of intoxication and the gross pathology in mice injected with HG7 were the same as those recorded with HG6. 
The acute intraperitoneal toxicity of the extract of HG4 was even lower. No deaths were recorded at 100 mg/kg. A dose of 150 mg/kg was lethal, however, with the same effects as seen with the other 2 extracts. Insufficient material was available to determine a precise LD50, though from the available data it may be concluded that it lies between 100 and 150 mg/kg.

\paragraph{Toxicity via gavage.}
 HG6 was much less toxic by oral administration than by intraperitoneal injection. No effects were observed at a dose of 300 mg/kg, which is 385 times the median lethal dose by intraperitoneal injection. Insufficient material was available for dosing at higher levels.​



\paragraph{Toxin profile.}
Toxin profile analysis for HG4, HG6 and HG7 was conducted by Tim Harwood at the Cawthron Institute, Nelson, New Zealand, via LC-MS. The protocol is set up to monitor for CTX-3B, CTX-3C, CTX-4A, CTX-4B, MTX-1 and MTX-3. Preliminary results show that solely MTX-3 was detected in all samples. Dr. Harwood reported unusual peaks in some samples in the CTX phase, which are subject to further investigation. \\

\newpage
\section{Discussion}

The main morphological characteristics for currently described \emph{Gambierdiscus} spp. are cell size and shape, and the shape of particular thecal plates ascertained from scanning electron microscopy (SEM).
\begin{itemize}
\item problems with morphology due to variable plates. sp nov closest relative morphologically is belizeanus, but somewhat removed in phylogenetic analysis. discuss and cite two of Shauna's papers, i.e. there only being one species of Fukuyoa and the carpentri/toxicus business. 

\item Discuss plasticity in 2' thecal plate, varying from hatchet shaped to rectangular and 5-sided as well as 6-sided.
These descriptions are meant to be species specific, however in a study conducted by Kohli et al they found that strains of \emph{G. carpenteri} (as confirmed by phylogenetics) displayed characteristics typical of \emph{G. toxicus}. The 2' plate had a variety of shapes ranging from hatchet shaped (as \emph{G. toxicus}) to rectangular (as \emph{G. carpenteri}). 

\item Some \emph{carpenteri} strains exhibited broad 1p plates with with a pointed dorsal end which is also characteristic of \emph{G. toxicus}. All the strains lack the thecal grove and  dorsal rostrum \cite{litaker2009taxonomy}. In short, some of the \emph{G. carpenteri} isolates exhibited the key characteristics of \emph{G. toxicus} identification \cite{kohli2014high}.
The variability in intra-species morphology makes this a support tool for classification which should be verified with genetic analysis.

\item  As species can show morphological variability within species \cite{bravo2014cellular}, but can also be highly morphologically similar to one another \cite{kohli2014high}, molecular genetic techniques are necessary to complement morphology in elucidating species composition.

\item change in cell shape and size from clonal strains \cite{bravo2014cellular} - statistically significant different cell size for HG6 compared to the other strains.


\item Comment of on difficulty comparing new species to type species description when limited or obscured pictures are supplied, example: belizeanus.

\item size distinctive, closest is F. rutzleri. Size used previously to distinguish species from each other
\end{itemize}

Variability in toxicity 
\begin{itemize}
\item seen in other dinoflagellates? How about between Alexandrium species? May be one of the other toxins  rather than CTX or MTX. Could be linked to variable copy number of toxin gene.
\item compare toxicity to Litaker '10 MBA table - why test toxicity of whole cell extract rather than separate fractions into MTX and CTX?
\item link to CFP in GBR
\end{itemize}

Mention gonyaulacales needing to be elucidated for tox stuff 
\begin{itemize}
\item \emph{Gambierdiscus} is a genus within the order \emph{Gonyaulacales}, whose evolutionary relationship has not yet been resolved \cite{gentekaki2014large}. To understand when and where the threat of CFP originates, it is essential to identify the species that produce the toxin at the base of the food chain. Currently \emph{Gambierdiscus} spp. have been implicated, but whether the production of CTX and MTX is restricted to a \emph{Gambierdiscus} clade or if it  persists in closely related genera is unknown.
\item link alexandrium saxitoxin distribution
\end{itemize}


 The toxicity of strains from Australia have not been examined, and their distribution along the Australian coast are largely not known. It is highly likely that many more \emph{Gambierdiscus} species are present at the sites, as in each case, the studies represent only single or short term sampling events at one location within a site
 \newline
 \paragraph{Description:} Cells photosynthetic, anterioposteriorly compressed and round to ellipsoid in apical view. Size of cells smaller than all other characterised \emph{Gambierdiscus} spp., dorsoventral depth 40.6 $\pm$3.3 $\mu$m, width 39 $\pm$3.2 $\mu$m and depth to width ratio 1.04 $\pm$0.06.The cell surface is highly areolated with pores. Flagella 2. The plate formula is Po, 4', 0a, 6'', 6c, 8s, 6''', 1p, 2''''. 
Apical pore complex Po is oval with a fish-hook shaped slit central of epitheca. Distinguishing feature to \emph{G. belizeanus}, closest morphological relative, is hatchet-shaped 2' plate. Largest apical plate is hatchet-shaped 2' plate, then pentagonal 3' plate and smallest hexagonal 1' plate. Largest precingular plate is 3'', then 4'', 5'', 6'', 2'' plates, 1'' plate is smallest. Largest postcingular plate is the 4''' plate, then 2''', 3''', 5''' and smallest 1''' plates. Antapical 1'''' plate is smaller than the 2'''' plate, both quadrangular. Pentagonal and narrow 1p plate. Deep and excavated sulcus bordering on cingulum.
 \paragraph{Etymology:} The epithet refers to disposition of one of the strains, HG6. The growth rate exceeded all other strains in culture at 27 $^{\circ}$C and subsequently terminated. The colloquialism that it lived fast and died young was applied, the strain was designated the rock and roll strain or \emph{G. petramus} (Latin petram = rock).
\paragraph{Holotype:} Fig. xx; SEM-stub (designation xxx) deposited at the Senckenberg Research Institute, German Centre for Marine Biodiversity Research, Wilhelmshaven, Germany.
\paragraph{Isotype:}
\paragraph{Type locality:} Heron Island (23.4420$^{\circ}$ S, 151.9140$^{\circ}$ E), Great Barrier Reef, Australia, South Pacific Ocean.
%\paragraph{Distribution:} Marine, associated as epiphyte to seaweeds. Observed at Heron Island, Australia, attached to  \emph{Halimeda} sp. and \emph{Padina} sp. with an average sea surface temperature was 23.8 $^{\circ}$C. October 2013 fromThe cultures HG4, HG5, HG6, HG7 and HG 26 were isolated in July 2014 from 
\paragraph{Remarks:} \emph{Gambierdiscus petramus} can be genetically identified rDNA sequences deposited in GenBank SSU and D8-D10 LSU.

\newpage
\section{Acknowledgements}

\FloatBarrier
\begin{table}
\caption{List of primers used for phylogenetic analysis of \emph{Gambierdiscus} strains, synthesised by Integrated DNA technologies.}
\label{tbl:PrimerTable}
\begin{tabular}{  | p{2cm} | p{7.5cm} | p{2.5cm} | p{2cm} | }
\hline
\textbf{Name} & \textbf{Sequence (5'-3')} & \textbf{Purpose} & \textbf{Reference} \\
\hline
    \multicolumn{4}{| c |}{\textbf{LSU D8-D10 region}}\\
    \hline
   FD8   & GGATTGGCTCTGAGGGTTGGG & Amplification \& sequencing & \cite{chinain1999morphology} \\
   \hline
 GLD8\_421F   & ACAGCCAAGGGAACGGGCTT & Sequencing & \cite{nishimura2013genetic} \\
 \hline
 GLD8\_677R   & TGTGCCGCCCCAGCCAAACT & Sequencing & \cite{nishimura2013genetic} \\
 \hline
   RB   & GATAGGAAGAGCCGACATCGA & Amplification \& sequencing &\cite{chinain1999morphology}  \\
    \hline
  \multicolumn{4}{| c |}{\textbf{SSU region}}\\
    \hline
 18ScomF1 & GCTTGTCTCAAAGATTAAGCCATGC & Amplification \& sequencing & \cite{zhang2005phylogeny} \\
 \hline
 18ScomR1  & CACCTACGGAAACCTTGTTACGAC & Amplification \& sequencing &  \cite{zhang2005phylogeny}  \\
 \hline
 Dino18SF2  & ATTAATAGGGATAGTTGGGGGC & Sequencing &  \cite{zhang2008mitochondrial}\\
 \hline
 Dino18SR1    & GAGCCAGATRCDCACCCA & Sequencing &  \cite{zhang2008mitochondrial}\\ 
 \hline
G10'F    & TGGAGGGCAAGTCTGGTG & Sequencing & \cite{nishimura2013genetic} \\
\hline
G18'R    & GCATCACAGACCTGTTATTG & Sequencing &  \cite{litaker2005reclassification} \\
 \hline
\end{tabular}
\end{table}
\FloatBarrier


\FloatBarrier
\begin{table}
\caption{Morphological meassurements of \emph{G. petramus }sp. nov. strains collected from Heron Island, Australia. Between 30 to 36 cells were counted per strain. Values in parentheses are $\pm$ standard deviation. \textbf{--- for supplementary data}}
\label{tbl:SizeTable}
\begin{tabular}{ | p{2cm} | p{2.5cm} | p{2.5cm} | p{2.5cm} | }
\hline
 \textbf{Strain} & \textbf{D ($\mu$m)} & \textbf{W ($\mu$m)}  & \textbf{D:W ratio}  \\
 \hline
 HG4  & 39 ($\pm$2.6) & 38.1 ($\pm$3.1) & 1.03 ($\pm$0.06) \\

 HG6  & 43.2 ($\pm$3.0) & 40.9 ($\pm$3.0) & 1.06 ($\pm$0.04)  \\

 HG7  & 39.2 ($\pm$2.8) & 38.4 ($\pm$2.8) & 1.02 ($\pm$0.07)  \\

 HG26  & 40.7 ($\pm$2.7) & 38.5 ($\pm$3.2) & 1.06 ($\pm$0.05) \\
  \hline
\textbf{sp. nov.}  & 40.6 ($\pm$3.3) & 39 ($\pm$3.2) & 1.04 ($\pm$0.06) \\
 \hline
\end{tabular}
\end{table}
\FloatBarrier


\FloatBarrier
\begin{table}
\caption{Morphological meassurements for all characterised \emph{Gambierdiscus} spp. and \emph{Fukoyoa} spp. Values in parentheses are $\pm$ standard deviation. NA stands for not available.}
\label{tbl:GlobalSizeTable}
\begin{tabular}{ | p{3.5cm} | p{2.5cm} | p{2.5cm} | p{2.5cm} | p{2.5cm} | p{1.8cm} | }
\hline
\textbf{Taxa} &  \textbf{D ($\mu$m)} & \textbf{W ($\mu$m)}  & \textbf{D:W ratio} & \textbf{Reference} \\
 \hline
\textit{G. australes}	& 86 ($\pm$5.1) & 77 ($\pm$3.7) & 1.12 (NA) & \cite{chinain1999morphology} \\
 \hline
 \textit{G. belizeanus}	& 63 ($\pm$2.2) & 58 ($\pm$2.5) & 1.07 ($\pm$0.08) & \cite{chinain1999morphology} \\
 \hline
 \textit{G. caribaeus}	& 82.2 ($\pm$7.6)	& 81.9 ($\pm$7.9)	& 1 (NA) & \cite{litaker2009taxonomy}\\
 \hline
 \textit{G. carolinianus} & 78.2 ($\pm$4.8) & 87.1 ($\pm$7.1) & 0.9 (NA) & \cite{litaker2009taxonomy} \\
 \hline
\textit{G. carpenteri} &	81.7 ($\pm$6.4) &	74.8 ($\pm$5.9) & 1.09 (NA) & \cite{litaker2009taxonomy} \\
 \hline
\textit{G. excentricus	}& 97.8 ($\pm$8.0) &	83.0 ($\pm$10.0) & 1.18 (NA) & \cite{litaker2009taxonomy} \\
 \hline
\textit{G. pacificus}	& 70 ($\pm$4.7) & 63 ($\pm$3.6) & 1.11 (NA) & \cite{chinain1999morphology}\\
 \hline
\textit{G. polynesiensis} & 69 ($\pm$4.5) & 69 ($\pm$3.6) & 1.1 (NA) &	\cite{chinain1999morphology} \\ 
 \hline
\textit{G. scabrosus}	& 63.2 ($\pm$5.7) & 58.2 ($\pm$5.7) & 1.09 ($\pm$0.07) & \cite{nishimura2014morphology}\\
 \hline
 \textit{G. silvae}	& 69 ($\pm$8.0) & 64 ($\pm$9.0) & 1.1 (NA) & \cite{fraga2014genus,litaker2010global}\\
 \hline
\textit{G. toxicus}	& 93 ($\pm$5.7) & 83 ($\pm$2.3) & 1.12 (NA) & \cite{litaker2009taxonomy}\\
 \hline
 \emph{Gambierdiscus} sp. type 4	& 67 ($\pm$5) & 68.8 ($\pm$5.6) & 0.97 (NA) & \cite{xu2014distribution} \\
 \hline
 \emph{Gambierdiscus} sp. type 5 & 54.8 ($\pm$4.6)	& 53.7 ($\pm$6.3)& 1.02 (NA) & \cite{xu2014distribution} \\
 \hline
 \textit{F. pauliensis} & 50($\pm$3) & 45 ($\pm$2) & 1.2 (NA) & \cite{gomez2015fukuyoa} \\
 \hline
\textit{F. rutzleri }& 43 ($\pm$5.1)	& 36 ($\pm$6.0) & 1.19 (NA) & \cite{litaker2009taxonomy}\\
 \hline
\textit{F. yasumotoi }& 56.8 ($\pm$5.6)	& 51.7 ($\pm$5.6) & 1.1 (NA) & \cite{litaker2009taxonomy} \\
 \hline
\textit{G. petramus}  & 40.6 ($\pm$3.3) & 39 ($\pm$3.2) & 1.04 ($\pm$0.06) & This study \\
   \hline
\end{tabular}
\end{table}
\FloatBarrier

\newpage
\bibliographystyle{acm}
\bibliography{references.bib}


\end{document}
