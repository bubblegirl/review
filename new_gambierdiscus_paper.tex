\documentclass[12pt]{article}
\usepackage[hcentering,bindingoffset=20mm]{geometry}
\usepackage{placeins}
%\usepackage[numbib]{tocbibind}
\usepackage{rotating}
\usepackage[square,sort,comma,numbers]{natbib}
\usepackage{graphicx}
\usepackage{tabularx}
\linespread{1.3}
%\fontsize{8cm}{1.3em}\selectfont
\usepackage{gensymb}
\usepackage{longtable}
\usepackage{lscape}
\usepackage{url}
\addtolength{\textwidth}{2cm}
\addtolength{\hoffset}{-1cm}


\addtolength{\textheight}{2cm}
\addtolength{\voffset}{-1cm}
\setlength{\parindent}{0pt}

\title{Characterisation of \emph{Gambierdiscus petramus} sp. nov. (Gonyaulacales): a new dinoflagellate from the Great Barrier Reef (Australia) that produces a ciguatoxin analogue. $^{1}$}
%\author{Anna Liza Kretzschmar \thanks{School Plant Functional Ecology and Climate Change Cluster (C3), University of Technology Sydney, Ultimo, NSW, Australia}, Arjun Verma \footnotemark[1], Tim Harwood \thanks{Cawthron Institute, Nelson, New Zealand},\\
%Mona Hoppenrath \thanks{Senckenberg Research Institute, German Centre for Marine Biodiversity Research, Wilhelmshaven, Germany}, Shauna Murray$^{2}$ \footnotemark[1]}
\date{}

\begin{document}
\maketitle
\paragraph{Anna Liza Kretzschmar} Plant Functional Ecology and Climate Change Cluster (C3), University of Technology Sydney, Ultimo, 2007 NSW, Australia
\paragraph{Arjun Verma} Plant Functional Ecology and Climate Change Cluster (C3), University of Technology Sydney, Ultimo, 2007 NSW, Australia
\paragraph{Tim Harwood} Cawthron Institute, The Wood, Nelson 7010, New Zealand
\paragraph{Mona Hoppenrath}Senckenberg Research Institute, German Centre for Marine Biodiversity Research, 26382 Wilhelmshaven, Germany
\paragraph{Shauna Murray$^{2}$} Plant Functional Ecology and Climate Change Cluster (C3), University of Technology Sydney, Ultimo, 2007 NSW, Australia, +61 2 9514 4006
\newpage
\section{Abstract}
\textit{Gambierdiscus} is a genus of benthic dinoflagellates which is found worldwide. Some species produce neurotoxins (maitotoxins and ciguatoxins) which  bioaccumulate and cause ciguatera fish poisoning, a potentially fatal food-borne illness that is common worldwide in tropical regions. In this study we characterised five strains of \textit{Gambierdiscus} collected from Heron Island,  Australia, a region in which ciguatera is endemic. Clonal cultures were processed using (i) light microscopy; (ii) scanning electron microscopy; (iii) DNA sequencing based on the nuclear encoded ribosomal  SSU and D8-D10 LSU regions; (iv) toxicity via mouse bioassay (MBA) and; (v) toxin profile as determined by LC-MS. Correlation of morphological and phylogenetic data indicated that these strains represent a new, toxic species of \emph{Gambierdiscus}, \emph{Gambierdiscus petramus} sp. nov. Cultures produced maitotoxin 3 and a yet uncharacterised congener of ciguatoxin. The investigation of toxigenic species of \textit{Gambierdiscus }in CFP endemic regions in Australia is necessary as a first step in order to determine which species of Gambierdiscus are related to CFP cases occurring in this region.

\paragraph{Keywords:} \emph{Gambierdiscus}, ciguatoxin, ciguatera
\paragraph{Abbreviations:} CFP, ciguatera fish poisoning; CTX, ciguatoxin; LC-MS, liquid chromatography - mass spectrometry; MBA, mouse bioassay; MTX, maitotoxin
\newpage

%-alt to petramus: cavus, florens, vegrandis
\section{Introduction}

%\subsection{overview, history, diff sp}
\emph{Gambierdiscus} is a genus of benthic dinoflagellate, epiphytic on many substrates in shallow tropical and sub-tropical waters that can produce neurotoxins responsible for ciguatera fish poisoning (CFP) through bioaccumulation \citep{berdalet2012global}. 
Since the discovery of \emph{Gambierdiscus} in 1977 \citep{yasumoto1977finding}, extensive research has established that members of the genus are the primary sources of ciguatoxins (CTXs), the causative agent of CFP \citep{chinain1997intraspecific,holmes1998gambierdiscus}. Maitotoxins (MTXs) are also commonly produced, however whether they play a role in CFP is yet to be established \citep{kohli2014feeding}. 

For over 15 years \emph{Gambierdiscus} was considered to be a monotypic taxon, but since 1995 new species have been discovered with more extensive sampling in the Atlantic Ocean, Indian Ocean and Pacific \citep{faust1995observation,holmes1998gambierdiscus,litaker2009taxonomy,chinain1999morphology,fraga2011gambierdiscus,nishimura2014morphology}.
Currently 11 species, and six unnamed clades of \emph{Gambierdiscus} have been described based on their distinct morphological and genetic characteristics \citep{adachi1979thecal,faust1995observation,chinain1999morphology,litaker2009taxonomy,nishimura2014morphology,fraga2011gambierdiscus}.  
The genus \emph{Fukuyoa} was described in 2015 \citep{gomez2015fukuyoa}. \emph{Fukuyoa} spp. produce MTX \citep{holmes1998gambierdiscus,holland2013differences} and hence may contribute to CFP \citep{kohli2014feeding}, they will be referred to alongside \emph{Gambierdiscus} where appropriate.
Ascertaining the distribution of species and their toxin production is essential for assessing risk of CFP to local populations as well as the seafood industry.

%\subsection{toxins and CFP}
CFP is the most common non bacterial illness associated with seafood consumption \citep{friedman2008ciguatera}, however the correct diagnosis of the illness is difficult due to extensive list of neurological, gastrointestinal and cardiovascular symptoms \citep{sims1987theoretical}. CFP is a highly morbid disease, predicted to cost \$2,421,080 in annual monetary loss in the USA alone \citep{minor2014per}.  UNESCO's international panel for harmful algal blooms has proposed a coordinated global ciguatera strategy to address the impact of the disease through \emph{Gambierdiscus} spp. detection and monitoring, improved toxin detection in seafood and epidemiological data collation \citep{globalcig}.
The suite of toxins produced by \emph{Gambierdiscus} varies between species. Bioassays are commonly used to assess toxicity, however LC-MS/MS analysis is required in order for a detailed and accurate toxin profile to be determined\citep{diogened2014chemistry}. Some species exclusively produce MTX (\emph{G. australes} \citep{rhodes2014production}) while others solely produce CTX (\emph{G. polynesiensis} \citep{rhodes2014production}) or neither (\emph{G. carpenteri} \citep{kohli2014high}), as confirmed by LC-MS. 

%\subsection{abundance and distribution}
Some \emph{Gambierdiscus} species have been classified as endemic to either the Pacific or Atlantic Oceans, while others have been found globally distributed \citep{berdalet2012global,litaker2010global}. %It has been suggested that with more extensive sampling, the distribution is likely to be global for all \emph{Gambierdiscus} species \citep{testerICHA}. 
However, the current understanding about \emph{Gambierdiscus} distribution and abundance is fragmentary due to the paucity of studies worldwide, the continued discovery of new species, difficulty with species identification with light microscopy and the comparatively lower sampling frequency in the Atlantic Ocean \citep{berdalet2012global,nishimura2014morphology}. 

Different species of \emph{Gambierdiscus} usually co-occur at sample sites \citep{litaker2010global}. As species can show morphological variability within species \citep{bravo2014cellular}, but can also be highly morphologically similar to one another \citep{kohli2014high}, molecular genetic techniques are necessary to complement the analysis of morphology to determine species. Taxonomy of the genus was poorly defined until relatively recently (ie \citep{litaker2009taxonomy,richlen2008phylogeography} which means that species identifications in most older studies are uncertain \citep{berdalet2012global}. %Morphology alongside genetic tools can identify species with a high certainty.

%Annual reported CFP cases are estimated up to 500,000 globally and are responsible for 80 to 96 \% of human poisoning from consuming seafood \citep{fleming1998seafood,grandjean2008centers}. ---> check ref
 %As CFP is increasing in frequency in the Pacific \citep{skinner2011ciguatera} and in other regions, an understanding of the distribution and species specific toxicology of \emph{Gambierdiscus} spp. is essential \citep{globalcig}.
 
%\subsection{australia}
Cases of CFP have been reported in Australia, mostly around the Great Barrier Reef, but a CTX producing species of \emph{Gambierdiscus} has not yet been identified \citep{lewis2006ciguatera}. This is most likely due the fact that Australia has a tropical and sub-tropical coastline of XX km, most of which has never been sampled for \textit{Gambierdiscus} species presence  (figure ~\ref{fig:OzSites}). Sites 1, 2, 3 and 4 of fig. ~\ref{fig:OzSites} encompass the length of the GBR and species isolated from this region are likely part of the ciguateric web. To date, four species of \emph{Gambierdiscus} have been identified in Australia (table ~\ref{tbl:OzTable}) in Queensland and New South Wales. \emph{Gambierdiscus} was also detected as part of benthic dinoflagellate communities via pyrosequencing in Western Australia, however could only be identified to genus level \citep{kohli2014cob}.
In 1994 Lewis et al reported three analogues of MTX, designated MTX-1, MTX-2 and MTX-3, isolated from a \emph{Gambierdiscus} sp. in Queensland, Australia \citep{holmes1994purification}. As this discovery pre-dates the understanding that \emph{Gambierdiscus} includes more species than \emph{G. toxicus}, it is unclear from which species these analogues were isolated.  
It is highly likely that many more \emph{Gambierdiscus} species are present at the sites, as in each case, the studies represent only single or short term sampling events at one location within a site. Hence the toxicity of strains from Australia is largely unknown, and their distribution along the Australian coast are relatively unexplored. %Consequently, the causative agent for CFP in Australia has not yet been determined. \\

The aim of the current study was to further the continued hunt for the causative agent of CFP in Australia by characterising a new species of \emph{Gambierdiscus} from a ciguateric region using molecular and morphologic methods, and investigate the toxicology of the isolates.


%\emph{G. belizeanus} was found in temperate waters of Heron Island, Great Barrier Reef (site 3 in fig. ~\ref{fig:OzSites}) \citep{murray2014molecular}. \emph{G. belizeanus} has tested positive for CTX via RBA \citep{chinain2010growth} and MTX via HELA \citep{holland2013differences}. These results have not yet been verified by LC-MS.\\

%\emph{F. yasumotoi} has been found in locations spanning a 1550km length of the Great Barrier Reef. Isolates were found at Raine Island (site 1 in fig. ~\ref{fig:OzSites}), Nelly Bay at Townsville (site 2 in fig. ~\ref{fig:OzSites}) and Heron Island (site 3 in fig. ~\ref{fig:OzSites}) \citep{murray2014molecular}. \emph{F. yasumotoi} was MTX positive for MBA \citep{holmes1998gambierdiscus}, however in a New Zealand strain no known CTX or MTX analogues were detected via LC-MS, though a variant of MTX may be present \citep{rhodes2014gambierdiscus}.

%\emph{G. toxicus} has been found in both NSW and QLD. In NSW, the samples were found in Eden (site 7 in fig. ~\ref{fig:OzSites}) \citep{hallegraeff2010algae}, while in QLD it was found at Platypus Bay, Townsville (site 2 in fig. ~\ref{fig:OzSites}) \citep{hallegraeff2010algae}. \emph{G. toxicus} tested CTX positive via RBA \citep{chinain2010growth} and MTX positive via MBA \citep{chinain1999morphology} but these bioassay results have not yet been verified by LC-MS.\\

%The species \emph{G. carpenteri} was found to be the only \emph{Gambierdiscus} species present in temperate New South Wales waters \citep{kohli2014high}. The species was detected at two sites at Merimbula (site 6 in fig. ~\ref{fig:OzSites}) and at Wapengo Lagoon (site 5 in fig. ~\ref{fig:OzSites}). No CTX or MTX was detected in these strains via LC-MS/MS, however the MTX-like fraction did display toxicity via MBA  \citep{kohli2014high}.\\ 

%Kohli et al identified \emph{Gambierdiscus} spp. as part of benthic dinoflagellate communities via pyrosequencing in Western Australia at Town Beach, Ningaloo reef, Exmouth (site 8 in fig. ~\ref{fig:OzSites}).  \emph{Gambierdiscus} was identified at the genus level, so which species are fount in WA has not yet been determined \citep{kohli2014cob}. \\




 \newpage
\section{Materials and methods}

\subsection{Specimen collection and culture conditions}
Specimens were collected from Heron Island (23.4420$^{\circ}$ S, 151.9140$^{\circ}$ E), as part of the Great Barrier Reef (GBR). 
The culture HG1 was isolated in October 2013 from \emph{Halimeda} sp. with an average sea surface temperature was 23.8 $^{\circ}$C. 
The cultures HG4, HG5, HG6, HG7 and HG 26 were isolated in July 2014 from \emph{Padina} sp. with an average sea surface temperature was 19.8 $^{\circ}$C.
Macroalgal samples were taken around the Heron Island research station (figure ~\ref{fig:HeronMap}). The samples were shaken in seawater to dislodge protists, the samples was then concentrated by filtration \citep{litaker2010global}. Single cells were isolated from the samples using a micropipette under an inverted light microscope, washed thrice in sterile seawater and transfered into F/10 medium \citep{holmes1991strain} to establish clonal cultures.
The cultures were maintained in F/10 medium at 27 $^{\circ}$C, 60mol/ m$^{2}$/ s light in 12hr:12hr light to dark cycles.


%\newpage



\subsection{Morphological analyses}
The cultures HG4, HG6, HG7 and HG26 were fixed with 10 \% Lugol's iodide and examined with an inverted light microscope ( xxx) equipped with xx camera for cell size. Depth (D: measured along dorsal-ventral axis) and width (W: measured along the lateral axis) was measured for 30 to 36 cells per strain.
For HG4, HG5, HG6, HG7 and HG26 30ml of culture at stationary growth phase were centrifuged at 15,000 rcf and concentrated to 2ml, preserved with 10 \% Lugol's iodide, and analysed by scanning electron microscopy (SEM) at the Forschungsinstitut Senckenberg (Wilhelmshaven, Germany). 
\textbf{Mona}, could you please supplement the appropriate details here. 

\subsection{Genomic DNA extraction}
Genomic DNA was extracted using the CTAB method \citep{zhou1999analysis}. Purity and concentration of the extract was measured by Nanodrop, the integrity was visualised on 1\% agarose gel.

\subsection{PCR amplification and sequencing}
Extracted DNA was used as a template to amplify rDNA sequences in 25$\mu$l mixture in PCR tubes. The mixture contains 0.6 $\mu$M forward and reverse primer, 0.4 $\mu$M BSA, 2 - 20 ng DNA, 12.5 $\mu$l 2xEconoTaq (Lucigen) and 7.5 $\mu$l PCR grade water.\\
The PCR cycling comprised of an initial 10 min step at 94 $^{\circ}$C, followed by 30 cycles of denaturing at 94 $^{\circ}$C for 30 sec, annealing at 55 $^{\circ}$C for 30 sec and extension at 72 $^{\circ}$C for 1 min, finalised with 10 minutes of extention at 72 $^{\circ}$C.
The LSU D1-D3  and D8-D10 regions as well as the SSU region were amplified with the D1R-F-, FD8-RB and 18ScomF1-18ScomR1 primer sets respectively (table ~\ref{tbl:PrimerTable}).\\
Sanger sequencing was conducted by Macrogen (Korea).


\subsection{Sequence alignment and phylogenetic analysis}
Sequencing data was aligned with \emph{Gambierdiscus} spp. data from the GenBank reference database (accession numbers as part of fig. ~\ref{fig:HGD8D10} and fig. ~\ref{fig:HGSSU}). The allignment algorithm MUSCLE, with a maximum of 8 iterations \citep{edgar2004muscle}, was used through the Geneious software, version 8.1.7 \citep{kearse2012geneious}. Alignments were truncated to the same length (D10-D8 and SSU at 799bp and 1690bp respectively), and discrepancies removed.
Phylogenetic trees were calculated with Bayesian inference as well as maximum likelihood. Bayesian inference, via Mr. Bayes 3.2.2, was used to estimate the posterior probability distribution with Metropolis-Coupled Markov Chain Monte Carlo \citep{ronquist2003mrbayes}. A random staring tree with three heated and one cold chain with a temperature set at 0.2. Trees were sampled every 100th generation for the 2,000,000 generations generated.
PHYML was used for maximum lilkelihood (ML) analysis with 1000 bootstraps \citep{guindon2003simple}.
The general time reversal (GTR) model was used for both analyses.\\

\subsection{Toxin production}

\subsubsection{Toxicity via mouse bioassay}
The cultures HG4, HG6 and HG7 were freeze-dried after 36 days in culture. The weights of the samples were 57.0 mg, 155.0 mg and 23.0 mg respectively. The samples were extracted exhaustively with methanol. The extracts were evaporated to a small volume at 35$^{\circ}$C using a rotary evaporator, and the solution aliquotted into glass vials. The remaining methanol was removed under a stream of nitrogen, and the extracts freeze-dried overnight. The total weight of the extracts, which were all brownishgreen gums, were: HG4, 22.5 mg HG6, 42.9 mg HG7, 15.1 mg The median lethal doses of the test materials by intraperitoneal injection were determined according to the principles of OECD Guideline 425. Samples were dissolved in 1\% Tween 60 in normal saline immediately before dosing. For toxicity by intraperitoneal injection, aliquots of this solution, made up to a total volume of 1 ml with the same solvent, were injected into female Swiss mice, of initial body weight 18-22 g. The toxicity of the extract of HG6 by gavage was also investigated, when aliquots of the extract solution were made up to 200 µl with Tween-saline. The mice were monitored intensively during the day of dosing. Survivors were examined and weighed each day for the following 13 days, after which they were killed and necropsied. Tap water and food (Rat and Mouse Cubes, Speciality Feeds Ltd, Glen Forrest, Western Australia) were available at all times.

\subsubsection{Toxin profile via LC-MS}
After 36 days in culture 2 litres of HG4, HG6 and HG7 were centrifuged (  x g) and the resultant pellet freeze-dried. Pellets were extracted with methanol and screened for CTX-3B, CTX-3C, CTX-4A, CTX-4B, MTX-1 and MTX-3 with quantitative LC-MS/MS methods developed by the Cawthron Institute \citep{kohli2014feeding}.
\textbf{Tim}, I've taken this part of the method from Lesley's 2014 paper on the Cook Island \emph{Gambierdiscus} spp. Please amend as appropriate. 

%For toxin profile elucidation, xxL of culture at the late exponential phase of growth were collected on 5 $\mu$m Millipore filters, washed off with sterile seawater and freeze dried for transportation to the Cawthron Institute (Nelson, New Zealand).
\newpage
\section{Results}

\subsection{\emph{Gambierdiscus petramus}}
\subsubsection{Morphology}
Cells of \emph{Gambierdiscus petramus} are photosynthetic, anterioposteriorly compressed and round to ellipsoid in apical view (fig. ~\ref{fig:PetLM}). \emph{Gambierdiscus petramus} cells are smaller than all other characterised \emph{Gambierdiscus} spp. (table ~\ref{tbl:GlobalSizeTable}, fig. ~\ref{fig:SizeGraph}), dorsoventral depth of 40.6 $\mu$m (range of 34.4 - 50.9 $\mu$m; standard deviation 3.3$\mu$m) and width of 39 $\mu$m (range of 32 - 46.9 $\mu$m; standard deviation 3.2$\mu$m) with a depth to width ratio of 1.04 (standard deviation 0.06$\mu$m)  (n=128). The cell surface is highly areolated with pores. Flagella 2. The plate formula, via Kofoid tabulation, is Po, 4', 0a, 6'', 6c, 8s, 6''', 1p, 2''''. 
The apical pore complex (Po) is oval with a fish-hook shaped slit, located at the center of the epitheca (fig. ~\ref{fig:PetSEM}F). The largest apical plate is the hatchet-shaped 2' plate, followed by the pentagonal 3' plate and the hexagonal 1' plate is the smallest. The largest precingular plate is the 3'', followed by the 4'', 5'', 6'' then 2'' plates. The 1'' plate is the smallest (fig. ~\ref{fig:PetSEM}A, B and C) . The largest postcingular plate is the 4''' plate, followed by the 2''', 3''', 5''' then 1''' plates. The 1'''' antapical plate is smaller than the 2'''' plate, both are quadrangular (~\ref{fig:PetSEM}D). The 1p plate is narrow and pentagonal. The sulcus is a deep excavated opening bordering on the cingulum (fig. ~\ref{fig:PetSEM}E).\\
A small subset of cells examined by SEM displayed variable 2' plate structure. The morphology of the 2' plate varied from hatchet shaped to rectangular and 5-sided as well as 6-sided. The shape of some of the cells in this subset were tear-drop shaped rather than ellipsoid. \\
\textbf{Mona}, please let me know if you agree with my assessment here. I think I got the plate formula right but validation would be appreciated.


%\newpage

\subsubsection{Phylogenetics}

Phylogenetic trees for two rDNA regions were calculated and compared Bayesian inference and maximum likelihood. Both D8D10 LSU (fig. ~\ref{fig:HGD8D10}) and SSU (fig. ~\ref{fig:HGSSU}) phylogenies reconstructed the same relationships of the Heron Island isolates HG1, HG4, HG6, HG7 and HG26 to the previously described \emph{Gambierdiscus} spp. as well as uncharacterised ribotypes and species types. The phylogenies showed that these strains are clearly separated from other \emph{Gambierdiscus} spp. including \emph{G. belizeanus} and \emph{G. scabrosus} and fall into a new, well supported clade.\\
%emph{G. silvae} was also isolated alongside HG4, HG6 , HG7 and HG26.

%\newpage

\subsubsection{Toxicology}

\paragraph{Toxicity via i.p. injection.}
The intraperitoneal LD50 of the extract of HG6 was 0.78 mg/kg, with 95\% confidence limits between 0.40 and 1.60 mg/kg. At high dose-levels, stretching movements were observed immediately after injection of the test material, most likely due to irritation. This was rapidly followed by cessation of movement and abdominal breathing. Respiration rates declined and death occurred within 1.5-4 hours after dosing. At necropsy, erythema was observed throughout the gastrointestinal tract of these animals At lower doses, death occurred at up to 24 hours after dosing. The stomachs of these animals were grossly distended with gas and with abnormally fluid contents. An abnormally large amount of material was present in the duodenum and upper jejunum, which was reddish-brown in colour and gelatinous in consistency. Caeca were also enlarged. Erythema in the glandular stomach was noted. No gross changes were observed in any other organ. 
The acute intraperitoneal toxicity of the extract of HG7 was lower than that of HD6, with an LD50 of 12.5 mg/kg, with 95\% confidence interval between 10.1 and 15.3 mg/kg. The symptoms of intoxication and the gross pathology in mice injected with HG7 were the same as those recorded with HG6. 
The acute intraperitoneal toxicity of the extract of HG4 was even lower. No deaths were recorded at 100 mg/kg. A dose of 150 mg/kg was lethal, however, with the same effects as seen with the other 2 extracts. Insufficient material was available to determine a precise LD50, though from the available data it may be concluded that it lies between 100 and 150 mg/kg.

\paragraph{Toxicity via gavage.}
 HG6 was much less toxic by oral administration than by intraperitoneal injection. No effects were observed at a dose of 300 mg/kg, which is 385 times the median lethal dose by intraperitoneal injection. Insufficient material was available for dosing at higher levels.​

\paragraph{Toxin profile.}
The toxin profile showed isolates of \emph{G. petramus} positive for MTX-3 only. An unusual peak in the CTX fraction was observed, indicative of an yet uncharacterised CTX analogue. This peak is subject to further investigation. \\ \textbf{Tim}, please include the relative area for the LC-MS peaks here.\\

\newpage
\section{Discussion}
According to the dichotomous key put forward by Litaker et al for morphologically  identifying \emph{Gambierdiscus} spp., \emph{G. petramus} proceeds to the same end point as \emph{G. belizeanus}. The cells are anterio-posteriorly compressed, poses a narrow 1p plate and heavily areolated cell surface \citep{litaker2009taxonomy}. \emph{G. scabrosus} also ends up in the \emph{G. belizeanus} classification, but can be distinguished by the asymmetry of the 4'' plate (designated 3'' by Nishimura et al) compared to the symmetry of the 4'' plate in \emph{G. belizeanus} \citep{nishimura2014morphology}. Hence the closest morphological relatives to \emph{G. petramus} are \emph{G. belizeanus} and \emph{G. scabrosus}. The most distinguishing difference between \emph{G. petramus} and all other \emph{Gambierdiscus} spp. is the species' diminutive size (table ~\ref{tbl:GlobalSizeTable} and fig. ~\ref{fig:SizeGraph}). Size is a feature that has been previously used to distinguish species of \emph{Gambierdiscus} from each other \citep{litaker2009taxonomy}. 
More specifically, \emph{G. petramus} differs to \emph{G. scabrosus} due to it's symmetric 4'' plate (fig. ~\ref{fig:PetSEM}), hence it is closer morphologically to \emph{G. belizeanus}. This is, however, not reflected in their evolutionary relationship which shows the two species several branches removed (fig. ~\ref{fig:HGD8D10} and fig. ~\ref{fig:HGSSU}). The morphological point of difference between the two species lies in the 2' plate. \emph{G. petramus}'s 2' plate is hatchet shaped (fig. ~\ref{fig:PetSEM}A, B and C) while the \emph{G. belizeanus}'s is rectangular \citep{faust1995observation}. \\
A small subset of isolates displayed different morphological features, in the 2' plate as well as cell shape. Bravo et al recorded a change in cell shape and size during the life cycle of a clonal strain of \emph{Gambierdiscus} sp., where a small proportion of the cells observed displayed this variability. They postulate that this was previously unrecorded due to their large number of observations compared to the rest of the literature \citep{bravo2014cellular}. We suggest that the variability we observed is similarly due to screening numerous cells of 4 strains of \emph{G. petramus} and that the variability in 2' plate is negligible. The hatchet shape of this plate should still be used as a diagnostic tool, though confirmation by cell size is essential.\\
Kohli et al observed similar morphological deviation from species description in  \emph{G. carpenteri} \citep{kohli2014high}. Strains of \emph{G. carpenteri}, as verified by sequencing, exhibited the key morphological characteristics of \emph{G. toxicus} (broad 1p plates pointed dorsal ends and lack of dorsal rostrum) \citep{kohli2014high,litaker2009taxonomy}.
The potential variability in intra-species morphology indicates that identification of \emph{Gambierdiscus} spp. needs to be verified with genetic analysis.\\
The closest relatives to \emph{G. petramus} genetically are \emph{G. pacificus} and \emph{G. toxicus} (fig. ~\ref{fig:HGD8D10} and fig. ~\ref{fig:HGSSU}). The confidence of \emph{G. petramus} constituting a new species distinct from its close relatives is high. The nodal support for both BI and ML for the SSU rDNA region is 1.00 and 100 respectively, for differentiating \emph{G. petramus} from \emph{G. pacificus}, \emph{G. toxicus} and \emph{G. scabrosus}. For the D8-D10 LSU region, the confidence of the relationship of \emph{G. petramus} with it's closest relatives is lower. The resolution of \emph{G. petramus} as a separate species is with high confidence, 1.00 and 100 for BI and ML respectively. The BI of the species' relationships was still high, whereas the ML confidence was reduced. %ref other papers.. xu for example
The discrepancy in confidence between SSU and D8-D10 LSU could be due to the length of the fragment, as the SSU fragment is almost twice the length as the LSU fragment. Alternatively it could be due to lower selection pressure, in effect higher mutation rate, in the D8-D10 region compared to the SSU region. Murray et al found that the D1-D6 LSU region in dinoflagellates had a substitution rate 4-8\% times that of the SSU region \citep{murray2005improving}. Whether this faster substitution rate applies to the D8-D10 region is unknown, however the possibility should be taken into consideration. \\
Morphologically \emph{G. petramus} is distinct from \emph{G. pacificus} due to it's smooth cell surface \citep{chinain1999morphology} while \emph{G. petramus} is discrete from \emph{G. toxicus} based on the latter's broad 1p plate \citep{litaker2009taxonomy}.\\

Berdalet et al postulated on the cosmopolitan distribution of \emph{Gambierdiscus} spp. but that the localised reports of some species are due to under sampling \citep{litaker2010global}. Also the idea that toxicity of isolates correlates to latitude has been put forward, with the observation that isolates in warmer areas display higher toxicity \citep{bomber1989epiphytism}. To test whether genetic divergence exists between strains from the same species from different localities, where possible a range of isolates from a variety of geographic locations were included in the phylogenetic study. The phylogenetic analysis of the D8-D10 LSU rDNA (fig. ~\ref{fig:HGD8D10})  \emph{G. belizeanus}, \emph{G. caribaeus}, \emph{G. carpenteri},  \emph{G. carolinianus} and \emph{G. toxicus} showed strains of one locality dispersed alongside strains from another geographic origin, rather than strains from different locations clustering together. Examining the strain distribution in light of their geographic origin for the SSU rDNA fragment (fig. ~\ref{fig:HGSSU}) shows a similar trend. Strains of \emph{G. carpenteri}, \emph{G. caribaeus} and \emph{G. toxicus} showed closer relationship to geographically removed strains than ones from their own location.
This indicates that there is no evolutionary pressure within the species based on their geographic separation, at least for the rDNA fragments.
As the genes involved in producing CTX and MTX are yet unknown, the rDNA markers used in this study serve as an indication of relatedness between the strains. The indication that geography does not facilitate divergence within \emph{Gambierdiscus} needs to be tested with concatenated multi gene phylogenetic analysis and ideally with genes involved in toxin production.\\
%genetic distsance calculation like in fraga 14?

CFP is endemic to the GBR in Australia, yet the causative agent has proven elusive. To our knowledge, this study constitutes the first report of a CTX producer in Australia. The LC-MS toxin profile of \emph{G. petramus} shows an undescribed congener of CTX, which indicates that this species is part of the ciguateric web in the GBR.
The toxicity of \emph{G. petramus} is equally unusual. Whole cell extract MBAs were conducted to query the toxicity as it would be encountered in an environmental setting. The pathology of the mice was unusual and not indicative of previously reported MTX or CTX intoxication (Pers. comm. Rex Munday). Whether the abnormal pathology is due to the unknown CTX congener or another toxin produced by this species is unknown. Further fractionation of whole cell extracts into aqueous methanol and dichloromethane partitions, which are expected to contain MTX from CTX respectively \citep{satake1993structure} could be conducted to elucidate which toxin is likely causing the pathology.
The toxicity of \emph{Gambierdiscus} spp. in culture have been observed to fluctuate wildly (pers. comm. Tim Harwood). This was observed in the highly variable toxicity of \emph{G. petramus} isolates in the MBA, where the toxicity of HG6 was 128- to 190-fold as high as HG4.  The recorded intra species toxicity variation of \emph{G. polynesiensis} via MBA was only 2-fold \citep{chinain2010growth}. Proposed reasons for changing toxin production are phosphorous limitation in the media, different stages of the growth phase and growth rate \citep{sperr1996variation,chinain2010growth}. However the HG4, HG6 and HG7 were grown under the same conditions and harvested simultaneously which indicates a intra strain variation in toxin production rather than external influence on toxin production regulation.\\
% whole cell extracts to mimc enviro - incl MTX because Kohli et al. Disprove  Litaker et al 2010


%Sampling around Australia has been limited (fig. ~\ref{fig:OzSites}) and constitute single points in time. As \emph{Gambierdiscus} spp. tend to co-habitate, it is likely that other \emph{Gambierdiscus} spp. are present at Heron Island which we were unable to establish in culture.

%\emph{Gambierdiscus} is a genus within the order \emph{Gonyaulacales}, whose evolutionary relationship has not yet been resolved \citep{gentekaki2014large}. To understand when and where the threat of CFP originates, it is essential to identify the species that produce the toxin at the base of the food chain. Currently \emph{Gambierdiscus} spp. have been implicated, whether the production of CTX and MTX is restricted to a \emph{Gambierdiscus} spp. or if it  persists in closely related genera is unknown. The identification of the genes involved in the toxin production is essential for monitoring CFP.

 \subsection{\emph{Gambierdiscus petramus} sp. nov. Kretzschmar et Murray}
 \paragraph{Description:} Cells photosynthetic, anterioposteriorly compressed and round to ellipsoid in apical view. Size of cells smaller than all other characterised \emph{Gambierdiscus} spp., dorsoventral depth 40.6 $\pm$3.3 $\mu$m, width 39 $\pm$3.2 $\mu$m and depth to width ratio 1.04 $\pm$0.06.The cell surface is highly areolated with pores. Flagella 2. The plate formula is Po, 4', 0a, 6'', 6c, 8s, 6''', 1p, 2''''. 
Apical pore complex Po is oval with a fish-hook shaped slit central of epitheca. Distinguishing feature to \emph{G. belizeanus}, closest morphological relative, is hatchet-shaped 2' plate. Largest apical plate is hatchet-shaped 2' plate, then pentagonal 3' plate and smallest hexagonal 1' plate. Largest precingular plate is 3'', then 4'', 5'', 6'', 2'' plates, 1'' plate is smallest. Largest postcingular plate is the 4''' plate, then 2''', 3''', 5''' and smallest 1''' plates. Antapical 1'''' plate is smaller than the 2'''' plate, both quadrangular. Pentagonal and narrow 1p plate. Deep and excavated sulcus bordering on cingulum.
 \paragraph{Etymology:} The epithet refers to disposition of one of the strains, HG6. The growth rate exceeded all other strains in culture at 27 $^{\circ}$C and subsequently terminated. The colloquialism that it lived fast and died young was applied, the strain was designated the rock and roll strain or \emph{G. petramus} (Latin petram = rock).
\paragraph{Holotype:} Fig. xx; SEM-stub (designation xxx) deposited at the Senckenberg Research Institute, German Centre for Marine Biodiversity Research, Wilhelmshaven, Germany. \\
\textbf{Mona} I assume these will be stored with you? Could you please fill in this section.
\paragraph{Isotype:} Fig. xx; SEM-stub (designation xxx) deposited at the Senckenberg Research Institute, German Centre for Marine Biodiversity Research, Wilhelmshaven, Germany. \\
\textbf{Mona} I assume these will be stored with you? Could you please fill in this section.
\paragraph{Type locality:} Heron Island (23.4420$^{\circ}$ S, 151.9140$^{\circ}$ E), Great Barrier Reef, Australia, South Pacific Ocean.
%\paragraph{Distribution:} Marine, associated as epiphyte to seaweeds. Observed at Heron Island, Australia, attached to  \emph{Halimeda} sp. and \emph{Padina} sp. with an average sea surface temperature was 23.8 $^{\circ}$C. October 2013 fromThe cultures HG4, HG5, HG6, HG7 and HG 26 were isolated in July 2014 from 
\paragraph{Remarks:} \emph{Gambierdiscus petramus} can be genetically identified rDNA sequences deposited in GenBank SSU and D8-D10 LSU. Awaiting accession numbers.

\newpage
\section{Acknowledgements}
Thank you to Dr Rex Munday of AgResearch for conducting the toxicology investigation.

\section{Conflict of interest disclosure}
The authors report no conflict of interest in conducting this study.
\newpage

\FloatBarrier
\begin{figure} 
%\includegraphics[scale=.1]{oz-gamb-grey-map.png} 
\caption{Sites where Gambierdiscus isolates have been reported in Australia. Site 1 indicates Raine Island, QLD; site 2 indicates Townsville, QLD; Site 3 indicates Heron Island, QLD; Site 4 indicated Platypus Bay, QLD; site 5 indicates Wapengo Lagoon, NSW; site 6 indicates Merimbula, NSW; site 7 indicates Eden, NSW; and site 8 indicates Exmouth, WA.​} 
\label{fig:OzSites}
\end{figure}
 \FloatBarrier
\begin{table}
\caption{\emph{Gambierdiscus} spp. identified around Australia, including toxicological data available. N/D denotes not detected. NA denotes not attempted.}
\label{tbl:OzTable}
\begin{tabular}{ | p{3.3cm} | p{4cm} | p{4.5cm} | p{2.3cm} | }
\hline
 \textbf{\emph{Gambierdiscus} sp.} & \textbf{Site identified from fig. ~\ref{fig:OzSites}} & \textbf{Toxicity}  & \textbf{LC-MS profile}  \\
 \hline
 \emph{G. belizeanus}  & Site 3: Heron Island, GBR \citep{murray2014molecular} & CTX +ve via RBA \citep{chinain2010growth}; MTX +ve via HELA \citep{holland2013differences} & NA  \\
 \hline
 \emph{G. carpenteri} &Site 5: Wapengo Lagoon, NSW \citep{kohli2014high}; site 6:  Merimbula, NSW  \citep{kohli2014high}& MTX +ve via MBA \citep{kohli2014high} & N/D \citep{kohli2014high}\\
 \hline
 \emph{G. toxicus} & Site 2: Townsville, GBR \citep{hallegraeff2010algae}; Site 7: Eden, NSW \citep{hallegraeff2010algae} & CTX +ve via RBA \citep{chinain2010growth}; MTX +ve via MBA \citep{chinain1999morphology} & NA \\
  \hline
  \emph{Gambierdiscus} sp. &Site 8: Exmouth, WA \citep{kohli2014cob}& NA & NA \\
  \hline
 \emph{F. yasumotoi}  & Site 1: Raine Island, GBR \citep{murray2014molecular}; Site 2: Townsville, GBR \citep{murray2014molecular}; site 3: Heron Island, GBR \citep{murray2014molecular}& MTX +ve via MBA \citep{holmes1998gambierdiscus} & N/D \citep{rhodes2014gambierdiscus}\\
  \hline
\end{tabular}
\end{table}

\FloatBarrier 
\begin{figure} 
%\includegraphics[scale=.4]{} 
\caption{Sample collection site at Heron Island - Arjun.} 
\label{fig:HeronMap}
\end{figure} 

\FloatBarrier
\begin{table}
\caption{List of primers used for phylogenetic analysis of \emph{Gambierdiscus} strains, synthesised by Integrated DNA technologies.}
\label{tbl:PrimerTable}
\begin{tabular}{  | p{2cm} | p{7.5cm} | p{2.5cm} | p{2cm} | }
\hline
\textbf{Name} & \textbf{Sequence (5'-3')} & \textbf{Purpose} & \textbf{Reference} \\
\hline
    \multicolumn{4}{| c |}{\textbf{LSU D8-D10 region}}\\
    \hline
   FD8   & GGATTGGCTCTGAGGGTTGGG & Amplification \& sequencing & \citep{chinain1999morphology} \\
   \hline
 GLD8\_421F   & ACAGCCAAGGGAACGGGCTT & Sequencing & \citep{nishimura2013genetic} \\
 \hline
 GLD8\_677R   & TGTGCCGCCCCAGCCAAACT & Sequencing & \citep{nishimura2013genetic} \\
 \hline
   RB   & GATAGGAAGAGCCGACATCGA & Amplification \& sequencing &\citep{chinain1999morphology}  \\
    \hline
  \multicolumn{4}{| c |}{\textbf{SSU region}}\\
    \hline
 18ScomF1 & GCTTGTCTCAAAGATTAAGCCATGC & Amplification \& sequencing & \citep{zhang2005phylogeny} \\
 \hline
 18ScomR1  & CACCTACGGAAACCTTGTTACGAC & Amplification \& sequencing &  \citep{zhang2005phylogeny}  \\
 \hline
 Dino18SF2  & ATTAATAGGGATAGTTGGGGGC & Sequencing &  \citep{zhang2008mitochondrial}\\
 \hline
 Dino18SR1    & GAGCCAGATRCDCACCCA & Sequencing &  \citep{zhang2008mitochondrial}\\ 
 \hline
G10'F    & TGGAGGGCAAGTCTGGTG & Sequencing & \citep{nishimura2013genetic} \\
\hline
G18'R    & GCATCACAGACCTGTTATTG & Sequencing &  \citep{litaker2005reclassification} \\
 \hline
\end{tabular}
\end{table}

\FloatBarrier
\begin{table}
\caption{Morphological meassurements for all characterised \emph{Gambierdiscus} spp. and \emph{Fukoyoa} spp. Values in parentheses are $\pm$ standard deviation. NA stands for not available.}
\label{tbl:GlobalSizeTable}
\begin{tabular}{ | p{3.5cm} | p{2.5cm} | p{2.5cm} | p{2.5cm} | p{2.5cm} | p{1.8cm} | }
\hline
\textbf{Taxa} &  \textbf{D ($\mu$m)} & \textbf{W ($\mu$m)}  & \textbf{D:W ratio} & \textbf{Reference} \\
 \hline
\textit{G. australes}	& 86 ($\pm$5.1) & 77 ($\pm$3.7) & 1.12 (NA) & \citep{chinain1999morphology} \\
 \hline
 \textit{G. belizeanus}	& 63 ($\pm$2.2) & 58 ($\pm$2.5) & 1.07 ($\pm$0.08) & \citep{chinain1999morphology} \\
 \hline
 \textit{G. caribaeus}	& 82.2 ($\pm$7.6)	& 81.9 ($\pm$7.9)	& 1 (NA) & \citep{litaker2009taxonomy}\\
 \hline
 \textit{G. carolinianus} & 78.2 ($\pm$4.8) & 87.1 ($\pm$7.1) & 0.9 (NA) & \citep{litaker2009taxonomy} \\
 \hline
\textit{G. carpenteri} &	81.7 ($\pm$6.4) &	74.8 ($\pm$5.9) & 1.09 (NA) & \citep{litaker2009taxonomy} \\
 \hline
\textit{G. excentricus	}& 97.8 ($\pm$8.0) &	83.0 ($\pm$10.0) & 1.18 (NA) & \citep{litaker2009taxonomy} \\
 \hline
\textit{G. pacificus}	& 70 ($\pm$4.7) & 63 ($\pm$3.6) & 1.11 (NA) & \citep{chinain1999morphology}\\
 \hline
\textit{G. polynesiensis} & 69 ($\pm$4.5) & 69 ($\pm$3.6) & 1.1 (NA) &	\citep{chinain1999morphology} \\ 
 \hline
\textit{G. scabrosus}	& 63.2 ($\pm$5.7) & 58.2 ($\pm$5.7) & 1.09 ($\pm$0.07) & \citep{nishimura2014morphology}\\
 \hline
 \textit{G. silvae}	& 69 ($\pm$8.0) & 64 ($\pm$9.0) & 1.1 (NA) & \citep{fraga2014genus,litaker2010global}\\
 \hline
\textit{G. toxicus}	& 93 ($\pm$5.7) & 83 ($\pm$2.3) & 1.12 (NA) & \citep{litaker2009taxonomy}\\
 \hline
 \emph{Gambierdiscus} sp. type 4	& 67 ($\pm$5) & 68.8 ($\pm$5.6) & 0.97 (NA) & \citep{xu2014distribution} \\
 \hline
 \emph{Gambierdiscus} sp. type 5 & 54.8 ($\pm$4.6)	& 53.7 ($\pm$6.3)& 1.02 (NA) & \citep{xu2014distribution} \\
 \hline
 \textit{F. pauliensis} & 50($\pm$3) & 45 ($\pm$2) & 1.2 (NA) & \citep{gomez2015fukuyoa} \\
 \hline
\textit{F. rutzleri }& 43 ($\pm$5.1)	& 36 ($\pm$6.0) & 1.19 (NA) & \citep{litaker2009taxonomy}\\
 \hline
\textit{F. yasumotoi }& 56.8 ($\pm$5.6)	& 51.7 ($\pm$5.6) & 1.1 (NA) & \citep{litaker2009taxonomy} \\
 \hline
\textit{G. petramus}  & 40.6 ($\pm$3.3) & 39 ($\pm$3.2) & 1.04 ($\pm$0.06) & This study \\
   \hline
\end{tabular}
\end{table}

\FloatBarrier 
\begin{figure} 
%\includegraphics[scale=.02]{petramus-LM.png} 
\caption{Light micrographs of \emph{Gambierdiscus petramus}, strains HG4 (A); HG6 (B); HG7 (C); and HG26 (D). Scale bar equal 10 $\mu$m.​} 
\label{fig:PetLM}
\end{figure} 

\FloatBarrier 
\begin{figure} 
%\includegraphics[scale=.02]{petramus_SEM.png} 
\caption{SEM micrographs of \emph{Gambierdiscus petramus}, strains HG4 (D); HG6 (A and B); and HG7 (C, E and F). Scale bar equal 20 $\mu$m unless otherwise specified.} 
\label{fig:PetSEM}
\end{figure} 

\FloatBarrier 
\begin{figure} 
%\includegraphics[scale=.1]{size_chart-grey.png} 
\caption{Visual representation of the size difference between \emph{Gambierdiscus} species (table ~\ref{fig:SizeGraph}). Error bars denote standard deviation of meassurements for \emph{G. petramus}. Data taken from publications as follows: a) Chinain et al \citep{chinain1999morphology}; b) Fraga et al \citep{fraga2014genus}; c) Faust et al \citep{faust1995observation}; d) Gomez et al \citep{gomez2015fukuyoa}; e) Litaker et al \citep{litaker2009taxonomy}; f) Nishimura et al \citep{nishimura2014morphology}; and g) Xu et al \citep{xu2014distribution}} 
\label{fig:SizeGraph}
\end{figure} 


\FloatBarrier 
\begin{figure} 
%\includegraphics[scale=.1]{SSU_complex_geo_merge-f.png} 
\caption{Maximum likelihood phylogeny of \textit{Gambierdiscus} species/phylotypes of the SSU rDNA region. Nodal support are Bayesian posterior probability (pp) and bootstrap (bt) values obtained from Bayesian inference analysis and maximum likelihood analysis, respectively. Nodes with strong support (pp/bt = 1.00 / 100) are shown as thick lines.}
\label{fig:HGSSU} 
\end{figure} 
\FloatBarrier 

%\newpage
\begin{figure} 
%\includegraphics[scale=.1]{D8D10_complex_geo_merge-F.png} 
\caption{Maximum likelihood phylogeny of \textit{Gambierdiscus} species/phylotypes of the LSU D8-D10 rDNA region. Nodal support are Bayesian posterior probability (pp) and bootstrap (bt) values obtained from Bayesian inference analysis and maximum likelihood analysis, respectively. Nodes with strong support (pp/bt = 1.00 / 100) are shown as thick lines.} 
\label{fig:HGD8D10}
\end{figure} 


\FloatBarrier
\begin{table}
\caption{Morphological meassurements of \emph{G. petramus }sp. nov. strains collected from Heron Island, Australia. Between 30 to 36 cells were counted per strain. Values in parentheses are $\pm$ standard deviation. \textbf{--- for supplementary data}}
\label{tbl:SizeTable}
\begin{tabular}{ | p{2cm} | p{2.5cm} | p{2.5cm} | p{2.5cm} | }
\hline
 \textbf{Strain} & \textbf{D ($\mu$m)} & \textbf{W ($\mu$m)}  & \textbf{D:W ratio}  \\
 \hline
 HG4  & 39 ($\pm$2.6) & 38.1 ($\pm$3.1) & 1.03 ($\pm$0.06) \\

 HG6  & 43.2 ($\pm$3.0) & 40.9 ($\pm$3.0) & 1.06 ($\pm$0.04)  \\

 HG7  & 39.2 ($\pm$2.8) & 38.4 ($\pm$2.8) & 1.02 ($\pm$0.07)  \\

 HG26  & 40.7 ($\pm$2.7) & 38.5 ($\pm$3.2) & 1.06 ($\pm$0.05) \\
  \hline
\textbf{sp. nov.}  & 40.6 ($\pm$3.3) & 39 ($\pm$3.2) & 1.04 ($\pm$0.06) \\
 \hline
\end{tabular}
\end{table}
\FloatBarrier

\newpage
\bibliographystyle{acm}
\bibliography{references.bib}


\end{document}
