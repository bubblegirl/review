\documentclass[12pt]{article}
\usepackage[hcentering,bindingoffset=20mm]{geometry}
\usepackage{placeins}
\usepackage[numbib]{tocbibind}
\usepackage{rotating}
\usepackage[square,sort,comma,numbers]{natbib}
\usepackage{graphicx}
\usepackage{tabularx}
\linespread{1.3}
%\fontsize{8cm}{1.3em}\selectfont
\usepackage{gensymb}
\usepackage{longtable}
\usepackage{lscape}

\addtolength{\textwidth}{2cm}
\addtolength{\hoffset}{-1cm}


\addtolength{\textheight}{2cm}
\addtolength{\voffset}{-1cm}
\setlength{\parindent}{0pt}

\title{\textbf{ Morphology and phylogenetics of\emph{Gambierdiscus xx} sp. nov. (Gonyaulacales): a new dinoflagellate associated with toxins from the Great Barrier Reef (Australia)}}
\author{Anna Liza Kretzschmar, Arjun Verma, Tim Harwood,\\
 Rex Munday, Mona Hoppenrath, Shauna Murray}
\date{}

\begin{document}
\maketitle

\section{Abstract}
\textit{Gambierdiscus} is a genus of benthic dinoflagellates which is found worldwide and forms harmful algal blooms (HABs). Some species produce neurotoxins (maitotoxins (MTX) and ciguatoxins (CTX)) which  bioaccumulate and cause ciguatera fish poisoning (CFP). In this study we characterised 5 strains of \textit{Gambierdiscus} collected from Heron Island,  Australia, a region in which CFP is endemic. Clonal cultures were processed using (i) light microscopy; (ii) scanning electron microscopy; (iii) DNA sequencing based on the nuclear encoded ribosomal  SSU and D8-D10 LSU regions; (iv) toxicity via mouse bioassay (MBA) and; (v) toxin profile as determined by LC-MS. Correlation of morphological and phylogenetic data has ascertained that these strains represent a new species of Gambierdiscus, \emph{Gambierdiscus sp. nov.}. Cultures produced MTX but no known congener of CTX, however the CTX toxin profile requires further investigation. The LD50 as meassured by MBA ranged between 0.78 mg/kg to 150 mg/kg between strains. The investigation of toxigenic species of Gambierdiscus in CFP endemic regions continues the investigation for the causative agent of CFP in Australia.

\section{Introduction}

\emph{Gambierdiscus} is a benthic dinoflagellate genus, epiphytic on many substrates in shallow tropical and sub-tropical waters. The first evidence that \emph{Gambierdiscus} may produce toxins causing ciguatera fish poisoning (CFP), a severe seafood-borne illness, was found in 1977 \cite{yasumoto1977finding}.  
Since then, extensive research has established that species of the genus are the primary sources of the toxins ciguatoxin (CTXs) and maitotoxin (MTXs), involved in CFP \cite{chinain1997intraspecific,holmes1998gambierdiscus}.  
For over 15 years \emph{Gambierdiscus} was considered to be a monotypic taxon, but since 1995 new species are being discovered with the aid of genetic analysis tools \cite{faust1995observation,holmes1998gambierdiscus,litaker2009taxonomy,chinain1999morphology,fraga2011gambierdiscus,nishimura2014morphology}.
Currently 11 species, and 6 unnamed types of \emph{Gambierdiscus} have been described based on their distinct morphological and genetic characteristics \cite{faust1995observation,chinain1999morphology,litaker2009taxonomy,nishimura2014morphology,fraga2014genus,adachi1979thecal,litaker2010global,nishimura2013genetic,kuno2010genetic,xu2014distribution}.
The genus was split earlier on this year resulting in the genus \emph{Fukuyoa}, members of which also produce MTX and hence may contribute to CFP \cite{kohli2014feeding}.
%need more on toxin composition
The current understanding about the global \emph{Gambierdiscus} distribution and abundance is fragmentary due to the relatively recent progress regarding the phylogeny within the genus, the continued discovery of new species, difficulty with species identification and the comparatively lower sampling frequency in the Atlantic Ocean. \\
Species of \emph{Gambierdiscus} have been reported from many countries, but due to the aforementioned reasons surrounding the unresolved taxonomy, the identities of species are often incorrect or uncertain. This is usually compounded by lack of culture or type material available \cite{marine2014}.
Annual reported CFP cases are estimated up to 500,000 globally and are responsible for 80 to 96 \% of human poisoning from consuming seafood \cite{fleming1998seafood,grandjean2008centers}. It is the most common non bacterial illness associated with seafood consumption \cite{friedman2008ciguatera}, however the correct diagnosis of the illness is difficult. As CFP is increasing in frequency in the Pacific \cite{skinner2011ciguatera} and in other regions, an understanding of the phylogeny and species specific toxicology of \emph{Gambierdiscus} spp. is essential.

We currently lack information on which species of \emph{Gambierdiscus} are present in Australia, though CFP is endemic to Queensland \cite{lewis2006ciguatera}. The report rate of CFP is predicted to only constitute 20\%, the majority of cases concentrated in the Great Barrier Reef (GBR) region \cite{lewis2006ciguatera}.
In 1994 Lewis et al reported three analogues of MTX, designated MTX-1, MTX-2 and MTX-3, isolated from a \emph{Gambierdiscus} sp. in Queensland, Australia \cite{holmes1994purification}. As this discovery pre-dates the understanding that \emph{Gambierdiscus} includes more species than \emph{G. toxicus}, it is unclear from which species these analogues were isolated.  
\emph{G. belizeanus} was found in temperate waters of Heron Island, Great Barrier Reef  \cite{murray2014molecular}. \emph{G. belizeanus} has tested positive for CTX via RBA \cite{chinain2010growth} and MTX via HELA \cite{holland2013differences}. These results have not yet been verified by LC-MS.
\emph{F. yasumotoi} has been found in locations spanning a 1550km length of the Great Barrier Reef. Isolates were found at Raine Island, Nelly Bay at Townsville and Heron Island \cite{murray2014molecular}. \emph{F. yasumotoi} was MTX positive for MBA \cite{holmes1998gambierdiscus}, however in a New Zealand strain no known CTX or MTX analogues were detected via LC-MS, though a variant of MTX may be present \cite{rhodes2014gambierdiscus}.
\emph{G. toxicus} has been found in both NSW and QLD. In NSW, the samples were found in Eden \cite{hallegraeff2010algae}, while in QLD it was found at Platypus Bay, Townsville  \cite{hallegraeff2010algae}. \emph{G. toxicus} tested CTX positive via RBA \cite{chinain2010growth} and MTX positive via MBA \cite{chinain1999morphology} but these bioassay results have not yet been verified by LC-MS.
The species \emph{G. carpenteri} was found to be the only \emph{Gambierdiscus} species present in temperate New South Wales waters. The species was detected at two sites at Merimbula and at Wapengo Lagoon. No CTX or MTX was detected in these strains via LC-MS/MS, however the MTX-like fraction did display toxicity via MBA  \cite{kohli2014high}. 
Pyrosequencing in Western Australia at Town Beach, Ningaloo reef, Exmouth found \emph{Gambierdiscus} spp. as part of benthic dinoflagellate communities.  \emph{Gambierdiscus} was identified at the genus level, so which species are found in WA is not yet determined \cite{kohli2014cob}. 
Hence the toxicity of strains from Australia has not been examined, and their distribution along the Australian coast are largely not known. It is highly likely that many more \emph{Gambierdiscus} species are present at the sites, as in each case, the studies represent only single or short term sampling events at one location within a site. Consequently, the causative agent for CFP in Australia has not yet been determined. \\

The aim of the current study was to further the continued hunt for the causative agent of CFP in Australia by characterising a new species of \emph{Gambierdiscus} from a ciguateric region using molecular and morphologic methods, and investigate the toxigenic nature of the isolates.
  
\section{Materials and methods}

\subsection{Study area}
The specimens were collected from Heron Island (23.4420$^{\circ}$ S, 151.9140$^{\circ}$ E), as part of the Great Barrier Reef (GBR). The majority of Australian CFP have been reported from the Great Barrier Reef area, but the causative agent has not yet been identified \cite{lewis2006ciguatera}. Heron Island is located in the Great Barrier Reef and was chosen as a sample site to search for toxic \emph{Gambierdiscus}.\\

\subsection{Specimen collection}
 
Macroalgal samples were taken around the Heron Island research station. The samples were shaken in seawater to dislodge protists, the samples was then concentrated by filtration. Single cells were isolated from concentrated samples.\\
HG1 was isolated in October 2013 from \emph{Halimedia} sp. The average sea surface temperature was 23.8 $^{\circ}$C. \\
HG4, HG5, HG6, HG7 and HG 26 were isolated in July 2014 from \emph{Padina} sp. The average sea surface temperature was 19.8 $^{\circ}$C.


\subsection{Culture conditions}
Isolates were separated by single cell isolation and grown as clonal cultures in F/10 medium. The cultures were maintained in F/10 medium at 26 $^{\circ}$C, 60mol/ m$^{2}$/ s light in 12hr:12hr light to dark cycles.

\subsection{Morphological examination}
HG4, HG6, HG7 and HG26 were fixed with 10 \% Lugol's iodide and examined with an inverted light microscope ( xxx) equipped with xx camera for cell size. Depth (D: measured along dorsal-ventral axis) and width (W: measured along the lateral axis) was meassured for 30 to 36 cells per strain.
For HG4, HG5, HG6, HG7 and HG26 30ml of culture at stationary growth phase were centrifuged at 15,000 rcf and concentrated to 2ml, preserved with 10 \% Lugol's iodide, and analysed by scanning electron microscopy (SEM) at the Forschungsinstitut Senckenberg (Wilhelmshaven, Germany).  

\subsection{Genomic DNA extraction}
Genomic DNA was extracted using the CTAB method \cite{zhou1999analysis}. Purity and concentration of the extract was measured by Nanodrop, the integrity was visualised on 1\% agarose gel.

\subsection{PCR amplification and sequencing}
Extracted DNA was used as a template to amplify rDNA sequences in 25$\mu$l mixture in PCR tubes. The mixture contains 0.6 $\mu$M forward and reverse primer, 0.4 $\mu$M BSA, 2 - 20 ng DNA, 12.5 $\mu$l 2xEconoTaq (Lucigen) and 7.5 $\mu$l PCR grade water.\\
The PCR cycling comprised of an initial 10 min step at 94
The LSU D1-D3  and D8-D10 regions as well as the SSU region were amplified with the D1R-F-, FD8-RB and 18ScomF1-18ScomR1 primer sets respectively (table ~\ref{tbl:PrimerTable}).\\
Sanger sequencing was conducted by Macrogen (Korea).
\FloatBarrier
\begin{table}
\caption{List of primers used for phylogenetic analysis of \emph{Gambierdiscus} strains.}
\label{tbl:PrimerTable}
\begin{tabular}{  | p{2cm} | p{7.5cm} | p{2.5cm} | p{2cm} | }
\hline
\textbf{Name} & \textbf{Sequence (5'-3')} & \textbf{Purpose} & \textbf{Reference} \\
\hline
    \multicolumn{4}{| c |}{\textbf{LSU D8-D10 region}}\\
    \hline
   FD8   & GGATTGGCTCTGAGGGTTGGG & Amplification \& sequencing & \cite{chinain1999morphology} \\
   \hline
 GLD8\_421F   & ACAGCCAAGGGAACGGGCTT & Sequencing & \cite{nishimura2013genetic} \\
 \hline
 GLD8\_677R   & TGTGCCGCCCCAGCCAAACT & Sequencing & \cite{nishimura2013genetic} \\
 \hline
   RB   & GATAGGAAGAGCCGACATCGA & Amplification \& sequencing &\cite{chinain1999morphology}  \\
    \hline
  \multicolumn{4}{| c |}{\textbf{SSU region}}\\
    \hline
 18ScomF1 & GCTTGTCTCAAAGATTAAGCCATGC & Amplification \& sequencing & \cite{zhang2005phylogeny} \\
 \hline
 18ScomR1  & CACCTACGGAAACCTTGTTACGAC & Amplification \& sequencing &  \cite{zhang2005phylogeny}  \\
 \hline
 Dino18SF2  & ATTAATAGGGATAGTTGGGGGC & Sequencing &  \cite{zhang2008mitochondrial}\\
 \hline
 Dino18SR1    & GAGCCAGATRCDCACCCA & Sequencing &  \cite{zhang2008mitochondrial}\\ 
 \hline
G10'F    & TGGAGGGCAAGTCTGGTG & Sequencing & \cite{nishimura2013genetic} \\
\hline
G18'R    & GCATCACAGACCTGTTATTG & Sequencing &  \cite{litaker2005reclassification} \\
 \hline
\end{tabular}
\end{table}
\FloatBarrier

\subsection{Sequence alignment and phylogenetic analysis}
The rRNA sequences were aligned using MUSCLE \cite{edgar2004muscle} against publicly available sequences from the NCBI database with a maximum of 8 iterations. The alignment was manually screened and adjusted. The 5' and 3' ends were truncated to render all sequences of the same length leaving D1-D3, D10-D8 and SSU. Outliers for LSU and SSU alignments were \emph{Alexandrium tamarense} AY831407.1 or \emph{Prorocentrum lima} AB189780.1.
PHYML was used for Maximum Likelihood (ML) tree generation with 1000 bootstraps in the general-time-reversal model. \\

\subsection{Toxin determination}

\subsubsection{Toxicity via mouse bioassay}
HG4, HG6 and HG7 were freeze-dried after 36 days in culture and processed by AgResearch. The weights of the samples were 57.0 mg, 155.0 mg and 23.0 mg respectively. The samples were extracted exhaustively with methanol. The extracts were evaporated to a small volume at 35$^{\circ}$C using a rotary evaporator, and the solution aliquotted into glass vials. The remaining methanol was removed under a stream of nitrogen, and the extracts freeze-dried overnight. The total weight of the extracts, which were all brownishgreen gums, were: HG4, 22.5 mg HG6, 42.9 mg HG7, 15.1 mg The median lethal doses of the test materials by intraperitoneal injection were determined according to the principles of OECD Guideline 425. Samples were dissolved in 1\% Tween 60 in normal saline immediately before dosing. For toxicity by intraperitoneal injection, aliquots of this solution, made up to a total volume of 1 ml with the same solvent, were injected into female Swiss mice, of initial body weight 18-22 g. The toxicity of the extract of HG6 by gavage was also investigated, when aliquots of the extract solution were made up to 200 µl with Tween-saline. The mice were monitored intensively during the day of dosing. Survivors were examined and weighed each day for the following 13 days, after which they were killed and necropsied. Tap water and food (Rat and Mouse Cubes, Speciality Feeds Ltd, Glen Forrest, Western Australia) were available at all times.
\subsection{Toxin profile via LC-MS}
Insert Tim methods
%For toxin profile elucidation, xxL of culture at the late exponential phase of growth were collected on 5 $\mu$m Millipore filters, washed off with sterile seawater and freeze dried for transportation to the Cawthron Institute (Nelson, New Zealand).

\section{Results}

\subsection{\emph{Gambierdiscus xx}}

\subsubsection{Morphology}
Light and SEM, which nomenclature used?


\FloatBarrier
\begin{table}
\caption{Morphological meassurements for details for \emph{Gambierdiscus sp. nov.} strains collected from Heron Island, Australia. Values in parentheses are $\pm$ standard deviation.}
\label{tbl:SizeTable}
\begin{tabular}{ | p{2cm} | p{2.5cm} | p{2.5cm} | p{2.5cm} | }
\hline
 \textbf{Strain} & \textbf{D ($\mu$m)} & \textbf{W ($\mu$m)}  & \textbf{D:W ratio}  \\
 \hline
 HG4  & 39 ($\pm$2.6) & 38.1 ($\pm$3.1) & 1.03 ($\pm$0.06) \\

 HG6  & 43.2 ($\pm$3.0) & 40.9 ($\pm$3.0) & 1.06 ($\pm$0.04)  \\

 HG7  & 39.2 ($\pm$2.8) & 38.4 ($\pm$2.8) & 1.02 ($\pm$0.07)  \\

 HG26  & 40.7 ($\pm$2.7) & 38.5 ($\pm$3.2) & 1.06 ($\pm$0.05) \\
  \hline
\textbf{sp. nov.}  & 40.6 ($\pm$3.3) & 39 ($\pm$3.2) & 1.04 ($\pm$0.06) \\
 \hline
\end{tabular}
\end{table}
\FloatBarrier


\FloatBarrier
\begin{table}
\caption{Morphological meassurements for all characterised \emph{Gambierdiscus} spp. and \emph{Fukoyoa} spp. Values in parentheses are $\pm$ standard deviation. NA stands for not available.}
\label{tbl:GlobalSizeTable}
\begin{tabular}{ | p{3.5cm} | p{2.5cm} | p{2.5cm} | p{2.5cm} | p{2.5cm} | p{1.8cm} | }
\hline
\textbf{Taxa} &  \textbf{D ($\mu$m)} & \textbf{W ($\mu$m)}  & \textbf{D:W ratio} & \textbf{Reference} \\
 \hline
 G. australes	& 72.5 (3.8) &	63.4 (5.0)	& 1.14 (NA) & \cite{chinain1999morphology,litaker2009taxonomy} \\
 \hline
 G. belizeanus	& 61 ($\pm$6) & 57.2 ($\pm$5.2) & 1.07 ($\pm$0.08) & \cite{nishimura2014morphology} \\
 \hline
 G. caribaeus	& 82.2 ($\pm$7.6)	& 81.9 ($\pm$7.9)	& 1 (NA) & \cite{litaker2009taxonomy}\\
 \hline
 G. carolinianus & 78.2 ($\pm$4.8) & 87.1 ($\pm$7.1) & 0.9 (NA) & \cite{litaker2009taxonomy} \\
 \hline
 G. carpenteri &	81.7 ($\pm$6.4) &	74.8 ($\pm$5.9) & 1.09 (NA) & \cite{litaker2009taxonomy} \\
 \hline
 G. excentricus	& 97.8 ($\pm$8.0) &	83.0 ($\pm$10.0) & 1.18 (NA) & \cite{litaker2009taxonomy} \\
 \hline
 G. pacificus	& 70 ($\pm$4.7) & 63 ($\pm$3.6) & 1.11 (NA) & \cite{litaker2009taxonomy}\\
 \hline
 G. polynesiensis & 66.3 ($\pm$3.0) & 60.5 ($\pm$5.9) & 1.1 (NA) &	\cite{litaker2009taxonomy,chinain1999morphology} \\ 
 \hline
 G. scabrosus	& 63.2 ($\pm$5.7) & 58.2 ($\pm$5.7) & 1.09 ($\pm$0.07) & \cite{nishimura2014morphology}\\
 \hline
 G. silvae	& 69 ($\pm$8.0) & 64 ($\pm$9.0) & 1.1 (NA) & \cite{fraga2014genus,litaker2010global}\\
 \hline
 G. toxicus	& 93 ($\pm$5.7) & 83 ($\pm$2.3) & 1.12 (NA) & \cite{litaker2009taxonomy}\\
 \hline
 %Gambierdiscus sp. type 4	& 65.9-72.4 ($\pm$4.1) & 64.5-68.9 ($\pm$5.0) & & \cite{xu2014distribution} \\
% \hline
 %Gambierdiscus sp. type 5 & 54.8 ($\pm$4.6)	& 53.7 ($\pm$6.3)& & \cite{xu2014distribution} \\
 %\hline
 F. pauliensis & 50($\pm$3) & 45 ($\pm$2) & 1.2 (NA) & \cite{gomez2015fukuyoa} \\
 \hline
 F. rutzleri & 43 ($\pm$5.1)	& 36 ($\pm$6.0) & 1.19 (NA) & \cite{litaker2009taxonomy}\\
 \hline
 F. yasumotoi & 56.8 ($\pm$5.6)	& 51.7 ($\pm$5.6) & 1.1 (NA) & \cite{litaker2009taxonomy} \\
 \hline
 \textbf{sp. nov.}  & 40.6 ($\pm$3.3) & 39 ($\pm$3.2) & 1.04 ($\pm$0.06) & This study \\
   \hline
\end{tabular}
\end{table}
\FloatBarrier





	
	


\subsubsection{Phylogenetics}
\FloatBarrier 
\begin{figure} 
\includegraphics[scale=.4]{HG_D8D10_phylo.jpg} 
\caption{Maximum Likelihood phylogeny of Heron Island \textit{Gambierdiscus} isolates using 791nt long sequences amplified from the D8D10 LSU region.} 
\label{fig:HGD8D10}
\end{figure} 
\FloatBarrier 
\newpage

\begin{figure} 
\includegraphics[scale=.4]{HG_SSU_phylo.jpg} 
\caption{Maximum Likelihood phylogeny of Heron Island \textit{Gambierdiscus} isolates using 1733nt long sequences amplified from the SSU region.}
\label{fig:HGSSU} 
\end{figure} 
\FloatBarrier 
\newpage

The D8D10 LSU (fig. ~\ref{fig:HGD8D10}) and SSU (fig. ~\ref{fig:HGSSU}) phylogenies of the Heron Island isolates HG1, HG4, HG6, HG7 and HG26 show that these strains fall into a new, well supported clade.\\
In the D8D10 LSU phylogeny (fig. ~\ref{fig:HGD8D10}), HG5 comes out between \emph{G. silvae} and the not yet characterised \emph{Gambierdiscus} sp. 4. whereas in the SSU phylogeny (fig. ~\ref{fig:HGSSU}) HG5 comes out next to \emph{G. polynesiensis}. SSU sequences for \emph{G. silvae} and \emph{Gambierdiscus} sp. 4 are not available from NCBI, hence it is not possible to determine whether HG5 is actually as closely related to \emph{G. polynesiensis} as this phylogeny suggests.\\

\subsubsection{Toxicology}

\subsubsection{Toxicity via i.p. injection}
The intraperitoneal LD50 of the extract of HG6 was 0.78 mg/kg, with 95\% confidence limits between 0.40 and 1.60 mg/kg. At high dose-levels, stretching movements were observed immediately after injection of the test material, most likely due to irritation. This was rapidly followed by cessation of movement and abdominal breathing. Respiration rates declined and death occurred within 1.5-4 hours after dosing. At necropsy, erythema was observed throughout the gastrointestinal tract of these animals At lower doses, death occurred at up to 24 hours after dosing. The stomachs of these animals were grossly distended with gas and with abnormally fluid contents. An abnormally large amount of material was present in the duodenum and upper jejunum, which was reddish-brown in colour and gelatinous in consistency. Caeca were also enlarged. Erythema in the glandular stomach was noted. No gross changes were observed in any other organ. 
The acute intraperitoneal toxicity of the extract of HG7 was lower than that of HD6, with an LD50 of 12.5 mg/kg, with 95\% confidence interval between 10.1 and 15.3 mg/kg. The symptoms of intoxication and the gross pathology in mice injected with HG7 were the same as those recorded with HG6. 
The acute intraperitoneal toxicity of the extract of HG4 was even lower. No deaths were recorded at 100 mg/kg. A dose of 150 mg/kg was lethal, however, with the same effects as seen with the other 2 extracts. Insufficient material was available to determine a precise LD50, though from the available data it may be concluded that it lies between 100 and 150 mg/kg.

\subsubsection{Toxicity via gavage}
 HG6 was much less toxic by oral administration than by intraperitoneal injection. No effects were observed at a dose of 300 mg/kg, which is 385 times the median lethal dose by intraperitoneal injection. Insufficient material was available for dosing at higher levels.​



\subsubsection{Toxin profile}
Toxin profile analysis for HG4, HG5, HG6 and HG7 was conducted by Tim Harwood at the Cawthron Institute, Nelson, New Zealand, via LC-MS. The protocol is set up to monitor for CTX-3B, CTX-3C, CTX-4A, CTX-4B, MTX-1 and MTX-3. Preliminary results show that solely MTX-3 was detected in all samples. Dr. Harwood reported unusual peaks in some samples in the CTX phase, which are subject to further investigation. \\


\section{Discussion}
The main morphological characteristics for currently described \emph{Gambierdiscus} spp. are cell size and shape, and the shape of particular thecal plates ascertained from scanning electron microscopy (SEM).
-size distinctive, closest is F. rutzleri
-problems with morphology due to variable plates. sp nov closest relative morphologically is belizeanus, but somewhat removed in phylogenetic analysis. discuss and cite two of Shauna's papers, ie there only being one species of Fukuyoa and the carpentri/toxicus business.

Variability in toxicity - seen in other dinos? How about between Alexandrium species? May be one of the other toxins  rather than CTX or MTX.

Different species of \emph{Gambierdiscus} usually co-occur at sample sites \cite{litaker2010global}. As species can show morphological variability within species (\cite{bravo2014cellular}, but can also be highly morphologically similar to one another \cite{kohli2014high}, molecular genetic techniques are necessary to complement morphology in elucidating species composition.
These descriptions are meant to be species specific, however in a study conducted by Kohli et al they found that strains of \emph{G. carpenteri} (as confirmed by phylogenetics) displayed characteristics typical of \emph{G. toxicus}. The 2' plate had a variety of shapes ranging from hatchet shaped (as \emph{G. toxicus}) to rectangular (as \emph{G. carpenteri}). Some strains exhibited broad 1p plates with with a pointed dorsal end which is also characteristic of \emph{G. toxicus}. All the strains lack the thecal grove and  dorsal rostrum \cite{litaker2009taxonomy}. In short, some of the \emph{G. carpenteri} isolates exhibited the key characteristics of \emph{G. toxicus} identification \cite{kohli2014high}.
The variability in intra-species morphology makes this a support tool for classification which should be verified with genetic analysis. \\

mention gonyaulacales needing to be elucidated for tox stuff - link alexandrium:
\emph{Gambierdiscus} is a genus within the order \emph{Gonyaulacales}, whose evolutionary relationship has not yet been resolved \cite{gentekaki2014large}. To understand when and where the threat of CFP originates, it is essential to identify the species that produce the toxin at the base of the food chain. Currently \emph{Gambierdiscus} spp. have been implicated, but whether the production of CTX and MTX is restricted to a \emph{Gambierdiscus} clade or if it  persists in closely related genera is unknown.

 The toxicity of strains from Australia have not been examined, and their distribution along the Australian coast are largely not known. It is highly likely that many more \emph{Gambierdiscus} species are present at the sites, as in each case, the studies represent only single or short term sampling events at one location within a site

\section{Acknowledgements}

\section{References}

\newpage
\bibliographystyle{acm}
\bibliography{references.bib}


\end{document}