\documentclass[12pt]{article}
\usepackage[hcentering,bindingoffset=20mm]{geometry}
\usepackage{placeins}
\usepackage[numbib]{tocbibind}
\usepackage{rotating}
\usepackage[square,sort,comma,numbers]{natbib}
\usepackage{graphicx}
\usepackage{tabularx}
\linespread{1.3}
\usepackage{gensymb}
\usepackage{longtable}
\usepackage{lscape}
\usepackage{url}
\addtolength{\textwidth}{2cm}
\addtolength{\hoffset}{-1cm}
\addtolength{\textheight}{2cm}
\addtolength{\voffset}{-1cm}
\setlength{\parindent}{0pt}
\title{Development of semi-quantitative PCR assays for the detection and enumeration of ciguatoxin producing \emph{Gambierdiscus lapillus} and \emph{Gambierdiscus polynesiensis} (Gonyaulacales, Dinophyceaes).}
\author{}
\date{}
\begin{document}
\maketitle

\begin{figure} 
%\includegraphics[scale=1.5]{Heron_sample-map2.png} 
\caption{(A) Map of Australia, with the position of Heron Island (grey circle); (B) Heron Island including surrounding reefs; (C) Sampling sites around Heron Island.} 
\label{fig:samplesites}
\end{figure} 
\FloatBarrier

\begin{figure}
%\includegraphics[scale=.22]{MH_variability.jpg}
\caption{qPCR cell based standard curves of (A) \emph{G. lapillus} strains HG4 (circle) and HG7 (triangle); and (B) \emph{G. polynesiensis} strains CG14 and CG15.} %Error bars represent the deviation of technical replicates during reactions.}
\label{fig:stdCurve}
\end{figure}
\FloatBarrier

\begin{figure}
%\includegraphics[scale=.22]{MH_variability.jpg}
\caption{qPCR gene based standard curves of (A) \emph{G. lapillus} and (B) \emph{G. polynesiensis}.} %Error bars represent the deviation of technical replicates during reactions.}
\label{fig:lapigblocks}
\end{figure}
\FloatBarrier

\FloatBarrier 
\begin{figure} 
%\includegraphics[scale=1.5]{Heron-positive-negative-samplingsites.png} 
\caption{\emph{G. lapillus} presence at the sampling sites around Heron Island. The spatial replicates for each site are set up as shown in (A); the sites in (B) linked to numbering in Fig. ~\ref{fig:samplesites} where positive (green) and negative (red) as per Supplementary Table 1.} 
\label{fig:envposneg}
\end{figure} 
\FloatBarrier

\FloatBarrier 
\begin{figure} 
%\includegraphics[scale=.3]{HG7-env.png} 
\caption{Detection of \emph{G. lapillus} per spatial replicate at each sampling site, cell numbers as normalised to HG7 standard curve (Fig. ~\ref{fig:stdCurve}A). Spatial replicates coloured as per macroalgal substrate,\ where \emph{Chnoospora} sp. are green, \emph{Sargassum} sp. are blue, \emph{Padina} sp. are red and mixed macroalgal substrates are yellow (see Supplementary Table 1.).} 
\label{fig:envHG7}
\end{figure} 
\FloatBarrier



\newpage
\bibliographystyle{acm}
\bibliography{references.bib}
\end{document}