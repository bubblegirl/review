\documentclass[12pt]{article}
\usepackage[hcentering,bindingoffset=20mm]{geometry}
\usepackage{placeins}
\usepackage[numbib]{tocbibind}
\usepackage{rotating}
\usepackage[square,sort,comma,numbers]{natbib}
\usepackage{graphicx}
\usepackage{tabularx}
\linespread{1.3}
\usepackage{gensymb}
\usepackage{longtable}
\usepackage{lscape}
\usepackage{url}
\addtolength{\textwidth}{2cm}
\addtolength{\hoffset}{-1cm}
\addtolength{\textheight}{2cm}
\addtolength{\voffset}{-1cm}
\setlength{\parindent}{0pt}
\title{The current state of knowledge of \emph{Gambierdiscus} and ciguatera fish poisoning in Australia.}
\author{Key words: Australia, Gambierdiscus, ciguatera fish poisoning, ciguatoxins, maitotoxins}
\date{}
\begin{document}
\maketitle
\paragraph{}Anna Liza Kretzschmar\\
Climate Change Cluster (C3), University of Technology Sydney, Ultimo, 2007 NSW, Australia, anna.kretzschmar@uts.edu.au
\paragraph{}Shauna A. Murray\\
Climate Change Cluster (C3), University of Technology Sydney, Ultimo, 2007 NSW, Australia
\newpage
\section*{Abstract}


\section*{Introduction}
\textit{Gambierdiscus} Adachi et Fukuyo is a genus of benthic microalgae, which colonize a variety of substrates in shallow tropical and sub-tropical waters, primarily macroalgae and dead corrals (summarized in \cite{hoppenrath2014marine}).
The type species \emph{G. toxicus} Adachi \& Fukuyo was discovered in 1977 and the genus was considered monophyletic till the discovery of \emph{G. belizeanus} Faust 1995. The current species concept encompasses 11 species and 6 unnamed clades, as based on both distinct morphological and genetic data.
Some species of \emph{Gambierdiscus} produce ciguatoxins (CTX), the causative agent of ciguatera fish poisoning (CFP) \cite{berdalet2012global}. CTX is introduced into the food chain when consumed by a vector species, usually because the toxic \emph{Gambierdiscus} is epiphytic to the macroalgal food source (Fig. ~\ref{fig:bioacc}). The CTXs are then passed up the food chain through bioaccumulation, a process which causes increased potency of the CTX congener through oxidation.
The populations of \emph{Gambierdiscus} vary between oceans, with some 

\FloatBarrier
\begin{figure} 
\includegraphics[scale=0.6]{CFP_diagram.png} 
\caption{The mechanism of bioaccumulation of ciguatoxins, with \emph{G. polynesiensis} at the base of the food web inhabiting macroalgae, e.g. a \emph{Padina} spp. \cite{padina}. A herbivore, e.g. white trevally (\emph{Pseudocaranx dentex}) \cite{trevally} consumes CTX from \emph{G. polynesiensis} along with the macroalgae, which is then either passes directly to humans through consumption, or through an intermediary piscivorous vector such as Australian spotted mackerel (\emph{Scomberomorus munroi}) \cite{mackerel}. Image of \emph{G. polynesiensis} (strain CG15) taken by A. L. Kretzschmar, 2016, Nikon Eclipse TS100 equipped with an Infinite Luminera 1 camera.} 
\label{fig:bioacc}
\end{figure} 
\FloatBarrier

Link CFP and Gambierdiscus presence, usual stuff about genus discovery, species states and global cig

labs working on CFP

\section*{History of ciguatera in Australia}
Lewis 06 paper \cite{lewis2006ciguatera}, qPCR mansucript data with QLD cig data, Captain Cook and all that. Type of fish? Eel and are they migratory?
Hamilton 10, Glazou 94 \cite{glaziou1994epidemiology}, Lucas 97 \cite{lucas1997pacific} Steward 97

\subsection*{Symptoms of ciguatera}
Lewis 06 \cite{lewis2006ciguatera}, Bagnis '87 \cite{bagnis1987use}

\section*{CTXs in AU}
Lapillus paper with potential CTX based on LC-MS/MS \cite{kretzschmar2017characterization}, Larsson paper on toxicity work \cite{larsson2018toxicology}, Lewis paper on toxin isolation from barracuda predating species division in 2009 \cite{lewis1984ciguatoxin}


\section*{Gambierdiscus in AU}
Link through bioacc. explanation 

yasumotoi Murray2014, Sparrow 17, lapillus \cite{kretzschmar2017characterization}, Holmes 91 \cite{holmes1991strain}, carpenteri \cite{kohli2014high} some unnamed shit from pyroseq \cite{kohli2014cob}

\section*{Range expansion of ciguatera in AU}
Farrell paper of NSW outbreaks \cite{farrellclinical} and spanish mackerel work \cite{kohli2017qualitative}

incl map of CFP range expansion into NSW


\section*{Future for AU}


\newpage
\bibliographystyle{acm}
\bibliography{references.bib}
\end{document}