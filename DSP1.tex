\documentclass[12pt]{article}
\usepackage[hcentering,bindingoffset=20mm]{geometry}
\usepackage{placeins}
\usepackage[numbib]{tocbibind}
\usepackage{rotating}
\usepackage[square,sort,comma,numbers]{natbib}
\usepackage{graphicx}
\usepackage{tabularx}
\linespread{1.3}
%\fontsize{8cm}{1.3em}\selectfont
\usepackage{gensymb}
\usepackage{longtable}
\usepackage{lscape}

\addtolength{\textwidth}{2cm}
\addtolength{\hoffset}{-1cm}


\addtolength{\textheight}{2cm}
\addtolength{\voffset}{-1cm}
\setlength{\parindent}{0pt}

\title{\textbf{Molecular genetics of \emph{Gambierdiscus}, producer of ciguatera fish poisoning}}
\author{\textbf{Candidate:} Anna Liza Kretzschmar\\
 \textbf{Chief Supervisor:} Shauna Murray\\
 {\small University of Technology Sydney, Plant Functional Biology and Climate Change Cluster}}
\date{}

\begin{document}
\maketitle
% need to replace all cite with citep

%need to intro gonyaulacales
\section{General Introduction}
\emph{Gambierdiscus toxicus} is a dinoflagellate, a type of marine microalgae, and is epiphytic on many substrates in shallow tropical and sub-tropical waters. It was first suggested to produce the causative agent of ciguatera fish poisoning (CFP), a severe seafood-borne illness, in 1977 \cite{yasumoto1977finding}.  
Since then, extensive research has established that species of the genus are the primary sources of the toxins ciguatoxin (CTXs) and maitotoxin (MTXs), which are known to be involved in CFP \cite{chinain1997intraspecific,holmes1998gambierdiscus}. CTXs bioaccumulate through the food web until seafood reaches the orally accumulative toxicity level which cause CFP \cite{bagnis1979clinical,gillespie1987possible,sims1987theoretical}. %Whether MTXs play a role in CFP needs to be further investigated \cite{kohli2014feeding}. 
For over 15 years \emph{Gambierdiscus} was considered to be a monotypic taxon, but since 1995 new species are being discovered with the aid of genetic analysis tools \cite{faust1995observation,holmes1998gambierdiscus,litaker2009taxonomy,chinain1999morphology,fraga2011gambierdiscus,nishimura2014morphology}. \emph{Gambierdiscus} is a genus within the order \emph{Gonyaulacales}, whose evolutionary relationship has eluded elucidation (cite Genetaki '14). Evidence suggests that the production of CTXs and MTXs varies between species \cite{chinain2010growth,holland2013differences}, similar to the situation in other dinoflagellates, in which toxin production is highly variable between species, and generally stable within species.
As CFP is increasing in frequency in the Pacific \cite{skinner2011ciguatera} and in other regions, an understanding of the genetics and toxicology of these organisms is becoming increasingly important. 

\section{Ciguatera fish poisoning}
CFP is the most common non bacterial illness associated with seafood consumption, posing a global problem \cite{friedman2008ciguatera}. %CFP problem ref! + seafood trade
Annual reported CFP cases are estimated up to 500,000 a year globally and are responsible for 80 to 96 \% of human poisoning from consuming seafood \cite{fleming1998seafood,grandjean2008centers}, however the correct diagnosis of the illness is problematic. 

Seafood contaminated with CTXs is characteristically impossible to discern during preparation as it does not exhibit any differences to non toxic seafood and the toxins are heat resistant \cite{withers1982ciguatera}. Toxin presence is usually only diagnosed retrospectively from CFP symptoms in conjunction with recent seafood consumption \cite{sims1987theoretical}. CFP is predicted to be widely under reported with as little of 20 \% of cases being reported  in Queensland, Australia \cite{lewis2006ciguatera}. The disease can manifest as over 180 symptoms, with varying levels of severity, and differ even between individuals at the same time and source of exposure \cite{sims1987theoretical}. The effects of CFP appear to be cumulative, as severity increases with previous exposure \cite{emerson1983preliminary}. Commonly several hours after consumption, the initial symptoms are gastrointestinal, which can be followed by cardiovascular and neurological manifestations \cite{sims1987theoretical}. They can persist between weeks to years. Recurrence is a common phenomenon induced by the factors: alcohol, fish or meat consumption \cite{lewis2006ciguatera} as well as sexual activity \cite{lange1992travel}. \\

Some CTXs undergo biotransformation during uptake of vector species  (find ref). %different types between different trophic levels
Some modifications to the toxin structure in the food chain can increase the potency of the toxins up to 10-fold, compared to the toxin produced by the \emph{Gambierdiscus} sp. under observation \cite{lewis2006ciguatera}.
In 2005 Hung et al established that the CFP of three family members was caused by CTX present in baracuda fish eggs \cite{hung2005persistent}. This indicates that bioaccumulated CTXs can be passed on to offspring which start their life cycle with CTX concentrations above the toxicity threshold to elicit CFP.  \\ 
There is no reliable bioassay for seafood samples due to CTXs structural variety and uncertainty of quantification \cite{dickey2010ciguatera}. The most commonly used bioassay is by mouse intraperitoneal injection but this is not indicative for orally active toxins \cite{botana2014seafood}.\\
A study correlating CFP related calls from the US National Poison Centre Data with storms and sea surface temperature showed that sea surface temperature was a relevant factor for CFP incidence and extrapolated that climate change related sea temperature rise could escalate CFP by 200 - 400 \% in the US \cite{garces2012habitat}. %incl the monetary figure on CFP
A study of preferential substrate colonisation showed that \emph{Gambierdiscus} spp. favoured dead corals, which will increase in availability with coral bleaching \cite{grzebyk1994ecology}. Over the last decade, CFP incidence has increased by 60\% in Pacific Islands \cite{skinner2011ciguatera}.
CFP has been reported in the Caribbean, Pacific and Indian oceans. In the Mediterranean Sea \cite{lejeusne2010climate}, \emph{Gambierdiscus} species have been detected only since the early 2000s \cite{aligizaki2008morphological}.
Understanding the phylogeny and species specific toxicology of \emph{Gambierdiscus} spp. is essential as part as global HAB species monitoring for early warning system for CFP outbreaks \cite{berdalet2012global}. Development and publishing of species specific qPCR primer sets is XXX of the global ciguatera strategy (cite). PCR is a technique that is commonly used by researchers and easily replicated. Hence qPCR primers could represent a quick and easy screening tool for which \emph{Gambierdiscus} spp. are present which can then be matched to toxicology, to infer imminent ciguatera risk.

%The under reporting and diagnostic issues mentioned, compounded by the absence of an adequate commercial test kit \cite{wong2005study}, the exact figure for number of global CFP cases cannot be determined. Carnivorous fish are the main cause of CFP, however herbivorous fish (eg. surgeonfish, parrotfish) pose important intermediate vector in the food chain \cite{cruz2006macroalgal,randall1958review,mak2013pacific}.

\section{The genus \emph{Gambierdiscus}}
We currently have little information of which species of \emph{Gambierdiscus} are present in Australia and surrounding countries. Species of \emph{Gambierdiscus} have been reported from many countries, but the unresolved taxonomy of the group has resulted in many uncertain identifications \cite{marine2014}.

Some \emph{Gambierdiscus} species have been classified as endemic to either the Pacific or Atlantic Oceans, while others have been found globally distributed \cite{berdalet2012global,litaker2010global}. It has been suggested that with more extensive sampling, the distribution is likely to be global for all \emph{Gambierdiscus} species \cite{testerICHA}. However, the current understanding about \emph{Gambierdiscus} distribution and abundance is fragmentary due to the relatively recent progress regarding the phylogeny within the genus, the continued discovery of new species, difficulty with species identification and the comparatively lower sampling frequency in the Atlantic Ocean. \\
%^ shithouse paragraph
Localised benthic HABs (BHABs) of \emph{Gambierdiscus} species have been noted in the literature in both the Pacific and Atlantic regions \cite{nakajima1981toxicity,withers1984ciguatera,chinain1999seasonal,darius2007ciguatera}.
%In general, when the density of a species of \emph{Gambierdiscus} 10,000 cell g$^{-1}$ wet weight algae, this has been referred to as a 'bloom' \cite{chinain1999seasonal}.%However, accurately estimating the abundance of benthic dinoflagellates including \emph{Gambierdiscus} is very difficult, due to the fact that cells can be patchily distributed in both space and time \cite{lobel1988assessment,ballantine1988population,litaker2010global}.

%This is due to two main factors: the fact that many regions have not yet been sampled for \emph{Gambierdiscus}, and the difficulty in identifying species. %For example,  In the hundreds of samples from Atlantic (Caribbean/Gulf of Mexico/West Indies/Southeast US coast Florida to North Carolina), no Pacific specific \emph{Gambierdiscus} sp. have been detected \cite{berdalet2012global,litaker2010global}. However,  most of the Pacific Ocean has not been sampled for \emph{Gambierdiscus} spp.,  so a similar conclusion about Atlantic species in the Pacific Ocean cannot be made and assertions about endemism or restricted distribution cannot be made conclusively.

%In terms of understanding difficulties in species identification, one example is \emph{G. yasumotoi}. \emph{G. yasumotoi} has been primarily reported in the Pacific, except for one report in the Mexican-Caribbean \cite{hernandez2004species}. However this singular record precedes the discovery of the other globular species \emph{G. ruetzleri}, which is considered to be endemic to the Atlantic region \cite{litaker2009taxonomy}. However a recent report of this species at the southern Kuwait coast and the Gulf of Aqaba in Jordan \cite{saburova2013new}, could indicate that this species is globally distributed also \cite{xu2014distribution}.\\ Shauna reckons o leave this out because the species are probably the same. so either i didnt write this very well to showcase how flawed morphology as an identification tool is, or its irrelevant.


In the absence of an estimate for cell density of \emph{Gambierdiscus} which leads to CFP, it has been proposed  that high cell concentrations of \emph{Gambierdiscus} species are likely to lead to increased CFP risk \cite{litaker2010global}. 
However, this hypothesis has not been consistently reported in the literature. Sampling within the Republic of Kiribati showed comparatively low colonisation of \emph{Gambierdiscus} species on the macro algae \emph{Halimeda} with 0 - 174 cells g$^{-1}$ wet weight algae. Yet 91 \% of fish sampled in the area were considered ciguateric as they exceeded the quarantine threshold of 0.01$\mu$g/kg P-CTX-1 equivalent \cite{xu2014distribution,chan2011spatial}. The authors postulate that \emph{Gambierdiscus} spp. could preferentially colonise another substrate. Xu et al suggested that an unidentified \emph{Gambierdiscus} sp. could be present alongside the 6 low to medium toxic strains reported \cite{xu2014distribution,bomber1988r}. Another option could be that low level, consistent exposure of CFP causes the concentration of CTXs to increase in fish over long periods of time.\\

Different species of \emph{Gambierdiscus} usually co-occur at sample sites \cite{litaker2010global}. As species can show morphological variability within species (ref gamb lifecycle paper), but can also be highly morphologically similar to one another (ref), molecular genetic techniques are necessary to complement morphology in elucidating species composition.

\subsection{\emph{Gambierdiscus} spp. in Australia}
To date, four species of \emph{Gambierdiscus} have been reported from 6 sites in Australia, the only sites yet examined \cite{kohli2014cob,murray2014molecular}. The toxicity of strains from Australia have not been examined, and their distribution along the Australian coast are largely not known. It is highly likely that many more \emph{Gambierdiscus} species are present at the sites, as in each case, the studies represent only single or short term sampling events at one location within a site.


In 1994 Lewis et al reported three analogues of MTX, designated MTX-1, MTX-2 and MTX-3, isolated from a \emph{Gambierdiscus} sp. in Queensland, Australia \cite{holmes1994purification}. As this discovery predates the understanding that \emph{Gambierdiscus} includes more species than \emph{G. toxicus}, it is unclear from which species these analogues were isolated.  \\

The species \emph{G. carpenteri} was found to be the only \emph{Gambierdiscus} species present at three separate sampling sites in temperate Australian waters. Recorded cell densities were 8256 g $^{-1}$ wet weight algae \cite{kohli2014high}. No CTX or MTX was detected in this strain via LC-MS/MS, however the MTX-like fraction did display toxicity via MBA. Whether this strain could have contributed to the CFP incidences in the region on the basis of MTX production is yet to be established  \cite{kohli2014high}. 
%the fuck? not even MTX-3?

\subsection{Morphology}

Currently 13 species, and 6 unnamed ribotypes of \emph{Gambierdiscus} have been described (Table ~\ref{tbl:MorphTable}) based on their distinct morphological and genetic characteristics. The new discovery of species with increased global sampling is likely. \\

The main morphological characteristics for currently described \emph{Gambierdiscus} spp. are cell size and shape, and the shape of particular thecal plates (Table ~\ref{tbl:MorphTable}). The terminology used has been conformed to the proposed terms from Hoppenrath et al \cite{hoppenrath2013taxonomy}. The initial species description detailed \emph{Gambierdiscus} to be large (60-100 $\mu$m), armoured with a distinct plate pattern and fishhook shaped accessory pore. Species fall into two categories - antero-posteriorly compressed, also described as lenticular, or slightly laterally compressed, or globular. Further classification is based on characteristics such as plate patterns and accessory pore shape. %ref

These features can be readily identified using scanning electron microscopy. However the strain specific variability of morphology within species, such as in size and shape of individual plates, makes this a support tool for classification which should be verified with genetic analysis. \\

\begin{longtable}{ |  p{1.6cm} | p{3.2cm} | p{6.7cm} |  p{2.5cm} | }
\caption{Presently characterised \emph{Gambierdiscus} spp. and their taxonomic identifications; table adapted from Kholi \cite{kohli2013Gambierdiscus}}\\
\hline
\label{tbl:MorphTable}
\textbf{Species} & \textbf{Cell size ($\mu$m) (dept h-width-length)} & \textbf{Morphological characteristics} & \textbf{References} \\
\hline
 \emph{G. australes} & (72.5$\pm$3.8)-(63.4$\pm$5.0)-(38.7$\pm$3.8) & Antero-posteriously compressed species; narrow 1p plate; smooth cell surface; 2' rectangular shaped; smaller than \emph{G. excentricus} &  \cite{chinain1999morphology,litaker2009taxonomy} \\
\hline
 \emph{G. belizeanus} & (61.7$\pm$3.1)-(60.0$\pm$4.5)-(48.1$\pm$4.2) & Antero-posteriously compressed species; narrow 1p plate; different 2' plate symmetry and size; heavily reticulate-foveate cell surface & \cite{litaker2009taxonomy,faust1995observation} \\
\hline
 \emph{G. caribaeus} & (82.2$\pm$7.6)-(81.9$\pm$7.9)-(60.0$\pm$6.2) & Antero-posteriously compressed species; broad 1p plate; 2’ rectangular shaped; symmetric 4’’  & \cite{litaker2009taxonomy} \\
\hline
 \emph{G. carolinianus} & (78.2$\pm$4.8)-(87.1$\pm$7.1)-(51.4$\pm$5.2) & Antero-posteriously compressed species; broad 1p plate; 2’ hatchet shaped; dorsal end 1p oblique; larger cell size than \emph{G. polynesiensis}  & \cite{litaker2009taxonomy} \\
\hline
 \emph{G. carpenteri} & (81.7$\pm$6.4)-(74.8$\pm$5.9)-(50.2$\pm$6.1) & Antero-posteriously compressed species; broad 1p plate; 2’ rectangular shaped; asymmetric 4''   & \cite{litaker2009taxonomy} \\
\hline
  \emph{G. excentricus} & (97.8$\pm$8.0)-(83.0$\pm$10.0)-(37.0$\pm$3.0) & Antero-posteriously compressed species; narrow 1p plate; smooth cell surface; 2’ rectangular shaped; cell size bigger than \emph{G. australes} (1.5 times wider and deeper)   & \cite{litaker2009taxonomy} \\
\hline
  \emph{G. pacificus} & (58.5$\pm$3.9)-(53.6$\pm$4.1)-(40.4$\pm$3.6) & Antero-posteriously compressed species; narrow 1p plate; smooth cell surface; 2' hatchet shaped   & \cite{litaker2009taxonomy,chinain1999morphology} \\
\hline
 \emph{G. polynesiensis} & (66.3$\pm$3.0)-(60.5$\pm$5.9)-(44.3$\pm$5.1) & Antero-posteriously compressed species; broad 1p plate; 2’ hatchet shaped; dorsal end 1p oblique; smaller cell size than \emph{G. carolinianus}   & \cite{litaker2009taxonomy,chinain1999morphology} \\
\hline 
 \emph{G. ruetzleri} & (45.5$\pm$3.3)-(37.5$\pm$3.0)-(51.6$\pm$4.9) & Globular species smaller than \emph{G. yasumotoi}; cell size less than 42$\mu$m  & \cite{litaker2009taxonomy} \\
 \hline
 \emph{G. scabrosus} & (63.2$\pm$5.7)-(58.2$\pm$5.7)-(37.3$\pm$3.5) & Antero-posteriously compressed species; narrow 1p plate; reticulate-foveate; asymmetric 4'' plate  & \cite{nishimura2013genetic,nishimura2014morphology,kuno2010genetic} \\ %is 3'' plate right?
\hline
\emph{G. silvae} & (69$\pm$8)-(64$\pm$9)-(46$\pm$5)  & Antero-posteriorly  compressed; narrow 1p plate; heavily reticulate-foveate;  & \cite{litaker2010global,fraga2014genus} \\
\hline
 \emph{G. toxicus} & (93.0$\pm$5.5)-(83.0$\pm$2.3)-(54.0$\pm$1.5) & Antero-posteriously compressed species; broad 1p plate; 2’ hatchet shaped; dorsal end 1p pointed  & \cite{litaker2009taxonomy,adachi1979thecal,chinain1997intraspecific,richlen2008phylogeography} \\
 \hline
  \emph{G. yasumotoi} & (56.8$\pm$5.6)-(51.7$\pm$5.6)-(62.4$\pm$4.3) & Globular species larger than \emph{G. ruetzleri}; cell size exceeds 42 $\mu$m  & \cite{holmes1998gambierdiscus,litaker2009taxonomy} \\
  \hline
  \multicolumn{4}{| c |}{\textbf{Genetically described phylotypes}}\\
    \hline
\emph{Gambier- discus} ribotype 2 & N/A & N/A & \cite{litaker2010global} \\
\hline
\emph{Gambier- discus} sp. type 2 & N/A & N/A  & \cite{kuno2010genetic,nishimura2013genetic} \\
\hline
\emph{Gambier- discus} sp. type 3 & N/A & N/A  & \cite{nishimura2013genetic} \\
\hline
\emph{Gambier- discus} sp. type 4  & (65.9-72.4$\pm$4.1-4.2)-(64.5-68.9$\pm$5.0)-(N/A) & Antero-posteriorly compressed species;evenly distributed pores; larger than \emph{Gambierdiscus} sp. type 5; 2' hatchet shaped; broad 1p plate  & \cite{xu2014distribution} \\
\hline
\emph{Gambier- discus} sp. type 5  & (54.8$\pm$4.6)-(53.7$\pm$6.3)- (N/A) & Antero-posteriorly compressed species;evenly distributed pores; smaller than \emph{Gambierdiscus} sp. type 4; 2' rectangular shaped; narrow 1p plate & \cite{xu2014distribution} \\
\hline
 \emph{Gambier- discus} sp. type 6 & N/A & N/A & \cite{xu2014distribution} \\
 \hline
\end{longtable}
\FloatBarrier

\subsection{Phylogenetics of \emph{Gambierdiscus}}
%Nishimura 14 - light micro unreliable due to subtle differences
%insert more spin, especially what Shauna said 

Phylogenetic analysis is a powerful tool that will be essential in elucidating the evolutionary relationships of species and genera. Molecular genetic tools for discriminating species of \emph{Gambierdiscus} from one another have been frequently applied, particularly in the last 10 years. These have consisted of ‘barcoding’ regions of genes such as large subunit (LSU) rRNA, small subunit (SSU) rRNA and internal transcribed spacer (ITS) rRNA. From such analysis, the current consensus is that \emph{Gambierdiscus} is monophyletic, with the lenticular and globular species forming two distinct clades \cite{chinain1999morphology,litaker2009taxonomy,fraga2011gambierdiscus,richlen2008phylogeography,kuno2010genetic,litaker2010global,nishimura2013genetic}. \\

The globular clade containing \emph{G. ruetzleri} and \emph{G. yasumotoi} appears to have diverged early in the evolution of the genus \cite{litaker2009taxonomy,nishimura2013genetic}. The two species are also the most closely related within the genus, and cannot be clearly distinguished from one another based on either molecular sequences of strains from different localities, or based on morphological features. \\

Based on D8-D10 LSU rRNA sequences, \emph{Gambierdiscus} ribotype 2 was designated a putative new phylotype by Litaker et al \cite{litaker2010global} as well as \emph{Gambierdiscus} species type 4, 5 and 6 by Xu et al \cite{xu2014distribution}. Based on SSU rRNA sequences \emph{Gambierdiscus} species types 2 and 3 were also found to be genetically distinct \cite{nishimura2013genetic,kuno2010genetic}. The species\emph{G. scabrosus}, previously \emph{Gambierdiscus} species type 1 \cite{nishimura2013genetic,nishimura2014morphology},  and \emph{G. silvae}, previously Gambierdiscus ribotype 1 \cite{fraga2014genus}, have been identified based initially only on their divergent genetic sequences. Following this, morphological features that confirmed their identities were also described. \\

An emerging issue in the identification of \emph{Gambierdiscus} spp. is that morphology is not as highly conserved within the species as previously assumed. For example, within \emph{G. yasumotoi} and \emph{G. carpenteri}, morphological features that were considered to be consistent within the species and hence used as an identifier of the species were found to vary considerably \cite{murray2014molecular,kohli2014high}. Therefore, while morphology can be an indicator, genetic analysis is essential for species identification.

%ok, so... p-values from Gurjeet's stuff?
% NZ strain of G yas Rhodes_14 shows super similarity between G yas and G rutz but was decided on G yas on SEM data --- insert what was requested during discussion - 
%Shauna, picture from your book will go in here and a paragraph on why morphology isn't enough for species identification, referring to both your and Gurjeet's work on \emph{G carpenteri} and \emph{G. yasumotoi} still to come.
\FloatBarrier
\begin{longtable}{ |  p{2.2cm} | p{2.8cm} | p{2.8cm} | p{2.8cm} | p{2.6cm} | }
\caption{Presently characterised \emph{Gambierdiscus} spp. and their genetic identifications}\\
\hline
\label{tbl:MorphTable}
\textbf{Species} & \textbf{LSU D1-D3 region (Genbank \#)} & \textbf{LSU D8-D10 region (Genbank \#)} & \textbf{SSU region (Genbank \#)} & \textbf{References} \\
\hline
 \emph{G. australes} & EF202969-72 & EF498072-74  & EF202891-96 & \cite{chinain1999morphology,litaker2009taxonomy} \\
\hline
 \emph{G. belizeanus} & EF202940-43 &  EF498028-34 & EF202876-77  & \cite{litaker2009taxonomy,faust1995observation} \\
\hline
 \emph{G. caribaeus} & EF202929-37, EF202983, EF202985 &  EF498045-71  & EF202914-28 & \cite{litaker2009taxonomy} \\
\hline
 \emph{G. carolinianus} &  EF202973-75 & EF498035-37 & EF202897-EF202901  & \cite{litaker2009taxonomy} \\
\hline
 \emph{G. carpenteri} &  EF202938-39, EF202984 & EF498038-44  & EF202908-13   & \cite{litaker2009taxonomy} \\
\hline
  \emph{G. excentricus} &  HQ877874, JF303063, JF303065-71 & JF303073-76 & N/A & \cite{litaker2009taxonomy} \\
\hline
  \emph{G. pacificus} &  EF202944-47 &  EF498012-13, EF498015-16 & SSU: EF202861-65  & \cite{litaker2009taxonomy,chinain1999morphology} \\
\hline
 \emph{G. polynesiensis} &  EF202976-82 & EF498076-80 & EF202902-07& \cite{litaker2009taxonomy,chinain1999morphology} \\
\hline 
 \emph{G. ruetzleri} &  EF202962-64 & EF498081-85 & EF202853-60 & \cite{litaker2009taxonomy} \\
 \hline
 \emph{G. scabrosus} &  AB605004.1 & AB765908-12  & AB764229-76 & \cite{nishimura2013genetic,nishimura2014morphology,kuno2010genetic} \\ 
\hline
\emph{G. silvae} &  KJ620013.1 & GU968512-20, GU968523 & N/A & \cite{litaker2010global,fraga2014genus} \\
\hline
 \emph{G. toxicus} &EF202951-61 & EF498017-27 &  EF202878-90 & \cite{litaker2009taxonomy,adachi1979thecal,chinain1997intraspecific,richlen2008phylogeography} \\
 \hline
  \emph{G. yasumotoi} & EF202965-68 & EF498087-89  &EF202846-52 & \cite{holmes1998gambierdiscus,litaker2009taxonomy} \\
  \hline
  \multicolumn{5}{| c |}{\textbf{Genetically described phylotypes}}\\
    \hline
\emph{Gambier- discus} ribotype 2 & N/A &  GU968499-500, GU968503, GU968505, GU968507-11 & N/A  & \cite{litaker2010global} \\
\hline
\emph{Gambier- discus} sp. type 2 & N/A & AB765913-18 & AB764277-96  & \cite{kuno2010genetic,nishimura2013genetic} \\
\hline
\emph{Gambier- discus} sp. type 3 & N/A & AB765923-24& AB764296-300  & \cite{nishimura2013genetic} \\
\hline
\emph{Gambier- discus} sp. type 4  & N/A &   KJ125080, KJ125114-15,KJ125119-20 & N/A  & \cite{xu2014distribution} \\
\hline
\emph{Gambier- discus} sp. type 5  &  N/A &  KJ125132-35 & N/A & \cite{xu2014distribution} \\
\hline
 \emph{Gambier- discus} sp. type 6 & N/A & KJ125108-09, KJ125111-13 & N/A  & \cite{xu2014distribution} \\
 \hline
\end{longtable}
\FloatBarrier


\section{Toxins produced by \emph{Gambierdiscus}}

Symptoms of CFP can vary between geographical locations \cite{molgo2000ciguatera,dickey2010ciguatera}, which likely contributes to the inefficiency of diagnoses. This could be due to the structural difference between CTX analogues ///REF//. Hence the accurate determination of CTX congeners is vital to understanding toxicology of CFP.\\

We do not yet have clear information available on the toxin profiles of each species of Gambierdiscus, particularly comparative assays of multiple species which have applied a consistent methodology. Bioassays provide an excellent indicator of a species' toxicity, however it does not suffice to elucidate a detailed toxin profile - liquid chromatography-mass spectrometry (LC-MS) or tandem LC-MS/MS analysis is required \cite{diogened2014chemistry}. \emph{Gambierdiscus} spp. produce more than just CTXs - notably MTXs are also commonly registered \cite{holmes1994purification,murata1993structure}. It has, until recently, been assumed in the literature that CTXs are the sole toxins causing CFP, CTXs have been isolated from the Pacific, Caribbean and the Indian Oceans. The  structure varies between the Pacific and the Caribbean congeners, with the structures of some of the Indian Ocean analogues still to be determined. \\

\subsection{CTX}
CTX and its analogues are lipid soluble polyether ladder compounds which act as sodium channel activators \cite{dechraoui1999ciguatoxins}. They are orally effective in humans at the picomolar (pM) and millimolar (mM) concentration range \cite{molgo2000ciguatera}, causing an influx of Na$^{+}$ ions and hence spontaneous action potentials in the cell, especially active on voltage sensitive channels along the nodes of Ranvier \cite{sims1987theoretical,mattei1999neurotoxins,lewis1992action,molgo2000ciguatera}. \\
Currently CTXs are prefixed by their origin, where members from the Pacific, Caribbean and Indian Oceans are designated P-CTX, C-CTX and I-CTX respectively. % don't know C-PTX subdivision ref... for P-CTX it's \cite{} I-CYX is\cite}

The congeners are further subdivided on a structural basis into type I and type II within P-CTXs, followed by all C-CTXs as type III - the molecular structure of I-CTXs have not been elucidated to such an extent as to assign a type \cite{legrand1997two,hamilton2002multiple,hamilton2002isolation} (Table ~\ref{tbl:CTXTable}).  \\

Type I P-CTXs consist of 13 rings with 60 C atoms \cite{murata1990structures,lewis1991purification,lewis1993origin}. The first completely defined CTX was isolated from moray eels, and was designated P-CTX-1B \cite{murata1990structures} or also described as CTX-1 \cite{lewis1991purification}. P-CTX-2 and P-CTX-3 were isolated from the same extracts but exhibited an alternate structure and toxicity in mice \cite{lewis1991purification}. It has been suggested by two research groups that P-CTX-1, P-CTX-2 and P-CTX-3 may be derivatives from the dinoflagellate precursors CTX-4A and CTX-4B (also GTX-4B by Murata et al \cite{murata1990structures}) \cite{lewis1993origin,yasumoto2000structural}. So far, P-CTX-4A and P-CTX-4B have been isolated from \emph{G. polynesiensis} culture extracts \cite{chinain2010growth}, while P-CTX-1, P-CTX-2 and P-CTX-3 have not. \\

The structure for type II P-CTX-3C consists of 13 ring and 57 C atoms and has been isolated first from \emph{G. toxicus} in 1993 \cite{satake1993structure}, when \emph{Gambierdiscus} was still designated as monophyletic, then in 2010 from \emph{G. polynesiensis} \cite{chinain2010growth}. There are two known congeners of P-CTX-3C which have been isolated - 49-epi-CTX-3C (or CTX-3B by Chinain et al \cite{chinain2010growth}) and M-seco-CTX-3C from \emph{Gambierdiscus} \cite{satake1993structure} and \emph{G. polynesiensis} \cite{chinain2010growth}. Two type II CTXs have been isolated from Moray eel, 2,3-dihydroxy-CTX-3C (or CTX-2A1) and 51-hydroxy-CTX-3C \cite{satake1998isolation}, and are suspected to constitute oxygenated metabolites of P-CTX-3. \\ %\cite{yasumoto2000structural} check ref as well as if P-CTX3 or P-CTX-3C 

Structurally C-CTXs were resolved to consist of 14 rings and 62 C atoms, with multiple congeners isolated from carnivorous fish - C-CTX-1, C-CTX-2, C-CTX-1127, C-CTX-1141, C-CTX-1143, C-CTX-1157, C-CTX-1159 \cite{vernoux1997isolation,lewis1998structure,pottier2003identification,pottier2002characterisation}. As yet, no C-CTXs have been isolated from any of the endemic \emph{Gambierdiscus} spp., Fraga et al have shown that \emph{G. excentricus} from the Canary Islands produce CTX and MTX like compounds as detected by N2A. The characterisation of these toxins using LC-MS is in process \cite{fraga2011gambierdiscus}. \\

The most recently discovered group of CTXs is from the Indian Ocean, where I-CTX-1, I-CTX-2, I-CTX-3 and I-CTX-4 were isolated from carnivorous fish. Their structure has not yet been established, preventing classification to a sub type \cite{hamilton2002multiple,hamilton2002isolation}. It has been elucidated that they have a higher molecular weight in comparison to P-CTXs and C-CTXs \cite{caillaud2010update,hamilton2002multiple,hamilton2002isolation}. \\ %I-CTX-1 toxic to mice via intraperitoneal injection \cite{hamilton2002isolation}. 

It has been suggested that the evident variety of CTX congeners, found even in the same ecosystem, is due to modification to the toxin structure as it passes up the food chain. This process can increase potency up to 10-fold \cite{hokama1996human,lewis2006ciguatera}. This theory is supported by a study conducted by Mak et al, examining the CTX composition of different members of a ciguateric food web \cite{mak2013pacific}.

%Indigenous population's understanding of ciguateric fish may help prevent CFP. 
%Experiemntation of P-CTX-1 injeted into rats intravenously showed rapid distribution to tissue and high bioavailibility of toxin via Neuro2a assay 
%cite{ledreux2013bioavailability}.

\begin{table}
\caption{Different known congeners of CTXs and their toxicity.}
\label{tbl:CTXTable}
\begin{tabular}{  | p{2cm} | p{1.5cm} | p{2.5cm} | p{4cm} | p{4cm} |}
\hline
\textbf{Toxin} & \textbf{Type} & \textbf{Molecular Ion [M +H]$^{+}$} & \textbf{Source} & \textbf{Toxicity (LD50, i.p. mice)} \\
\hline
P-CTX-1, P-CTX-1B & I & 1111.6 & Moray eel (\emph{Gymnothorax javanicus}) \cite{murata1990structures,lewis1991purification} & P-CTX-1 0.35 $\mu$g/kg \cite{murata1990structures}; P-CTX-1B 0.25$\mu$g/kg \cite{lewis1991purification} \\
\hline
P-CTX-2 & I & 1095.5 \cite{lewis1991purification} & Moray eel (\emph{Gymnothorax javanicus}) \cite{lewis1991purification} & 2.3 $\mu$g/kg \cite{lewis1991purification} \\
\hline
 P-CTX-3 & I & 1095.5 \cite{lewis1991purification} & Moray eel (\emph{Gymnothorax javanicus}) \cite{lewis1991purification} & 0.9 $\mu$g/kg \cite{lewis1991purification} \\
 \hline
 P-CTX-4A & I & 1061.6 \cite{yasumoto2000structural} & \emph{Gambierdiscus} sp. \cite{yasumoto2000structural}; \emph{G. polynesiensis} \cite{chinain2010growth} & 12$\mu$g/kg \cite{chinain2010growth} \\
 \hline
 P-CTX-4B & I & 1061.6 \cite{yasumoto2000structural} & \emph{Gambierdiscus} sp. \cite{yasumoto2000structural}; \emph{G. polynesiensis} \cite{chinain2010growth} & 20$\mu$g/kg \cite{chinain2010growth}\\
 \hline
 P-CTX-3C & II & 1023.6 \cite{satake1993structure} &  \emph{Gambierdiscus} sp. \cite{satake1993structure}; \emph{G. polynesiensis} \cite{chinain2010growth} & 2.5$\mu$g/kg \cite{chinain2010growth}\\
 \hline
 P-49-epi-CTX-3C & II & 1023.6 \cite{chinain2010growth} & \emph{Gambierdiscus} sp. \cite{satake1993structure}; \emph{G. polynesiensis} \cite{chinain2010growth} & 8$\mu$g/kg\cite{chinain2010growth}\\
 \hline
 P-M-seco-CTX-3C & II & 1041.6 \cite{chinain2010growth} &\emph{Gambierdiscus} sp. \cite{satake1993structure}; \emph{G. polynesiensis} \cite{chinain2010growth} & 10$\mu$g/kg \cite{chinain2010growth}\\
 \hline
 C-CTX-1 & III & 1141.6 \cite{vernoux1997isolation,pottier2002characterisation} & Horse-eye jack (\emph{Caranx latus}) \cite{vernoux1997isolation,pottier2002characterisation} & 3.6$\mu$g/kg \cite{vernoux1997isolation}\\
 \hline
 C-CTX-2 & III & 1141.6 \cite{vernoux1997isolation,pottier2002characterisation}& Horse-eye jack (\emph{Caranx latus}) \cite{vernoux1997isolation,pottier2002characterisation} & Toxic \cite{vernoux1997isolation}\\
 \hline
 I-CTX-1 & N/A & 1141.6 \cite{hamilton2002isolation}& Red bass (\emph{Lutjanus bohar}); Red emperor (\emph{Lutjanus sebae}) \cite{hamilton2002isolation} & Toxic \cite{hamilton2002isolation} \\
 \hline
\end{tabular}
\end{table}
\FloatBarrier

\subsection{MTX}

MTXs are the largest known proteinacious natural product \cite{yokoyama1988some,murata1993structure} and is a highly toxic Ca$^{2+}$ channel activator, with a LD$_{50}$ 0.05 $\mu$g/kg. There are only a couple of known bacterial, proteinacious toxins with higher potency \cite{yokoyama1988some,murata1993structure}. However the Ca$^{2+}$ activation is a secondary process of MTX activity, the primary mode of action in mammalian cells  is still unclear \cite{van2000diversity}. MTX is a water soluble, polyether ladder compound. It was first isolated in 1976 from the gut of a form herbivorous fish species with the common name 'Maito' \cite{yasumoto1976toxicity}. Their complex structure and stereochemistry was elucidated in the 1990s using stereoscopic studies and partial synthesis \cite{murata1993structure,murata1994structure,satake1995structural,nonomura1996complete,zheng1996complete}. \\

There are three known analogues of MTX (table ~\ref{tbl:MTXTable}) - designated MTX-1, MTX-2 and MTX-3, isolated from a \emph{Gambierdiscus} sp. in Queensland, Australia \cite{holmes1994purification}. MTX-1 and MTX-2 are structurally larger than the third. \\ %who isolated it first? yokoyama? 

%Biophysical characteristics of pacific MTXs different to caribbean MTX \cite{lu2013caribbean}. Carribean MTX not been charactrised, so whether the difference in action is structural is speculative. %check lit

%MTX easier to detect and quantify than CTX due to sulphate esters using liquid chromatography-electronspray ionisation-mass spectrometry (T. Harwood pers. com.). Solvolysis (desulphonation) reduces toxicity 100-fold or more \cite{murata1991effect}. 

\begin{table}
\caption{Different known congeners of MTXs and their toxicity.}
\label{tbl:MTXTable}
\begin{tabular}{ |  p{2cm} | p{2cm} | p{3cm} | p{3cm} | p{4cm} | }
\hline
\textbf{Origin} & \textbf{Toxin} & \textbf{Molecular Ion [M +H]$^{+}$} & \textbf{Source} & \textbf{Toxicity (LD50, i.p. mice)} \\
\hline
 Pacific & MTX-1 & 3422 \cite{holmes1994purification,murata1993structure} & \emph{Gambierdiscus} sp. \cite{holmes1994purification} & 0.05$\mu$g/kg \cite{murata1993structure}\\
 \hline
 Pacific & MTX-2 & 3298 \cite{holmes1994purification} & \emph{Gambierdiscus} sp. \cite{holmes1994purification} & 0.08$\mu$g/kg \cite{holmes1994purification}\\
 \hline
 Pacific & MTX-3 & 1060   \cite{holmes1994purification} & \emph{Gambierdiscus} sp. \cite{holmes1994purification}; \emph{G. australes}, \emph{G. pacificus} \cite{rhodes2014production} &  0.08$\mu$g/kg for \emph{G. australes}  \cite{rhodes2014production} \\
 \hline
\end{tabular}
\end{table}
\FloatBarrier
%can i use and cite gurjeets chemdraws?
\subsection{Other}
Further toxins reportedly synthesised by \emph{Gambierdiscus} spp. are are gambieric acid, gambierol and gambieroxide \cite{watanabe2013gambieroxide,satake1993gambierol,nagai1992gambieric}.
Gambierol is less toxic than CTX with 80 $\mu$g/kg i.p. and 150 $\mu$g/kg orally in MBA, but fast acting with lethality setting in within 3 hrs\cite{ito2003pathological}. Gambierol could play a potential role in CFP, which to date has not been investigated. Though less toxic than CTX, it is still a highly toxic substance which could bioaccumulate in the food chain \cite{rhodes2014production}.\\
Gambieric acids A, B, C and D showed potent antifungal properties but no toxicity via MBA at 1000 $\mu$g/kg i.p. and hence are not of concern as a health hazard, including CFP \cite{rhodes2014production,nagai1992gambieric}.
Recently gambieroxide was isolated from \emph{G. toxicus} by Watanabe et al \cite{watanabe2013gambieroxide}. Gambieroxide's stereostructure has been elucidated by NMR and LC-MS/MS, concluding that it is structurally similar to yessotoxin (YTX) which is a lipophillic toxin found in filter feeding shellfish  \cite{tubaro2010yessotoxins}. Known producers of YTX are \emph{Protoceratium reticulatum}, \emph{Lingulodinium polyedrum} and \emph{Gonyaulax spinifera} \cite{tubaro2010yessotoxins} which are not  phylogenetically closely related to \emph{Gambierdiscus}.  Gambieroxide needs to be further investigated for structure, toxicity and bioaccumulative potential for assessment as a health threat, including CFP. \\

\section{Toxicity of different \emph{Gambierdiscus} spp.}
\emph{Gambierdiscus} spp. produce CTXs and MTXs \cite{murata1990structures,holmes1991strain,satake1993structure,holmes1994purification,satake1996isolation}, however, some wild and culturable strains have been recorded to not produce either or both toxins in measurable quantities \cite{gillespie1985significance,holmes1990toxicity}. Hence toxin production varies between species. Most studies were conducted in the 16 year time period pre-dating the discovery of \emph{G. belizeanus} in 1995 \cite{faust1995observation}, and therefore the strain used was refered to as \emph{Gambierdiscus toxicus}, although it may have been any of the now known species. \\

The Caribbean species \emph{G. excentricus} was tested with a neuroblastoma cytotoxicity assay (N2A) which indicated both MTX and CTX toxicity \cite{fraga2011gambierdiscus}. However the toxin profile needs to be extrapolated with LC-MS.
Chinain et al identified the toxins produced by two strains of \emph{G. polynesiensis} with LC-MS. The major toxin produced was P-CTX-3C, other type II (M-seco-CTX-3C, 49-epi-CTX-3C) and type I (CTX-4A, CTX-4B) toxins were detected. The different strains produced the same toxins, but at different proportions \cite{chinain2010growth}. In an earlier study, Chinain et al detected toxicity of the water soluble fraction of \emph{G. polynesiensis} in a mouse bioassay (MBA), implicating MTX activity \cite{chinain1999morphology}. Rhodes et al produced evidence to the contrary, as they found no LC-MS measurable MTX in a strain of \emph{G. polynesiensis} from the Cook Islands \cite{rhodes2014production}.
\emph{G. australes} extracts isolated by Rhodes et al indicated CTX and MTX presence by displaying toxic activity in both lipid and water soluble fractions when tested with MBA. However, LC-MS/MS analysis did not detect any CTXs. A follow up study on the same strain with LC-MS also found no CTXs, but potent MTX-3 like compounds were detected \cite{rhodes2014production,rhodes2010toxic}. In contrast, three different \emph{G. australes} strains isolated from French Polynesia, tested with a receptor binding assay (RBA), produced P-CTX-3C like compounds \cite{chinain2010growth}.\\

Holland et al conducted a large scale two year toxicity screening of 56 strains from 6 species of \emph{G. belizeanus, G. caribaeus, G. carolinianus, G. carpenteri, G. rutzleri} and  \emph{Gambierdiscus} ribotype 2, using human erythrocyte lysis assay (HELA) \cite{holland2013differences}. Haemolytic activity is a well established property of MTX \cite{igarashi1999mechanisms}, which is the implication from Holland's study. It showed that the haemolytic activity varied marginally between strains but stayed consistent for each strain over the two year sampling period, indicating that MTX production, and congener suite, varies between the species but remains consistent over time within a strain \cite{holland2013differences}.
In contrast, a recent study of CTX-like toxicity using mouse neuroblastoma assay (MNA) of the new \emph{Gambierdiscus} ribotypes 4, 5 and 6 found that the isolates from a ciguateric site were up to 4\- fold more toxic compared to isolates from a geographically closely related non-ciguateric reference site \cite{xu2014distribution}. This observed difference in MTX and CTX production staying constant and varying respectively, could be due to CTX production increasing in a ciguateric setting, or something other than temperature variation changes MTX production. The findings of both of these studies need to be verified with LC-MS. \\

These case studies show that while bioassays are important to indicate toxicity, LC-MS studies of every species need to be conducted to define their toxin profiles.

\section{Dinoflagellate genetics and transcriptomics}
Dinoflaggelates have some unique genetic features in both structure and regulation. Their genetic material extensively features repetitive genomic elements, high gene copy number and a large genome size ranging betweeb 1.5 and 245 Gb \cite{lin2011genomic}. These features have confounded current sequencing technologies as assembly becomes unfeasable. DNA resides in a crystalised, permanently condensed state throughout the cell cycle except during DNA replication with loops of active regions extending for access by polymerases. Histones play a minor role in DNA packaging for dinoflaggelates.

While such large genomes cannot readily be sequenced, transcriptomics can be used for gene discovery and for examining genetic regulation \cite{murray2012transcriptomics}. This method involves the extraction of mRNA from an organism, which due to its instability, is then reverse transcribed into complimentary DNA (cDNA) which can then be explored. This allows a glimpse into the actively used genes of the organism and traverses the truncated or inactive copies of that gene.

Dinoflagellates are generally not axenically culturable. Hence if RNA is extracted from a dinoflagellate culture, it is likely to contain the transcripts of bacterial organisms also. This posed a major obstacle for assigning transcripts to an organism with any certainty. Recently dinoflagellate specific spliced leader (dinoSL) sequence was discovered \cite{zhang2007spliced}. This process whereby, during the maturation of mRNA, these dinoSL are \emph{trans}-spliced to the 5' end of transcripts along with the polyadenylated 3' tail characteristic of eukaryotes. The presence of this dinoSL allows certain designation of a transcript to a dinoflagellate  with a high degree of certainty as demonstrated in case studies in a grazer-prey setting \cite{lin2010dinoflagellate}, a natural aquatic ecosystem \cite{lin2010spliced} and in a mixed system with \emph{Symbiodinium kawagutii} \cite{zhang2013proof}. In the latter case study, no cDNAs from non-dinoflaggelate algae present at extraction were detected \cite{zhang2013proof}.

\section{Polyketide synthase genes in Dinoflagellates}
Dinoflagellates commonly produce of marine biotoxins, such as CTXs and MTXs. While the toxin structure and hence mode of action varies between species, a common feature is the basic polyketide structure. They are secondary metabolites which have undergone several steps of transformation from a linear polyketide precursor by catalytic domains of polyketide synthases (PKSs), like in fatty acid synthesis (FAS) \cite{rein1999polyketides}. %Pawlowiez paper
The minimum of catalytic domains involved in chain elongation are beta-ketosynthase (KS), acyltransferase (AT) and the acyl carrier protein (ACP); for chain modification ketoreductase (KR), dehydrase (DR) and enoylreductase (ER) are required; and finally the polyketide is released from the PKS complex by thioesterase (TE). There are three major types of PKSs. Type I, which contains all the catalytic domains required, or Type II, which comprise of mono-functional units with each catalytic domain on a separate peptide which assemble into complexes and Type III which is akin to type II but smaller (eg. \cite{eichholz2012putative})  . There are two functional groups within type I PKS - in modular Type I PKSs the catalytic domains are organised into modules according to each step, and that module is only used once during polyketide synthesis. In iterative type I PKSs the same set of catalytic domains are located on  one protein and used cyclically for chain elongation.\\ 
Dinoflagellate specific PKS encoding transcripts to date sport a singular catalytic domain for approximately 50 - 100 kDa proteins, reminiscent of the multi protein complex formation of type II PKS but sequence similarity is akin to type I PKS \cite{monroe2008toxic,eichholz2012putative}. Due to this unique feature, they are termed type I-like PKS \cite{monroe2010characterization}.
As PKS are active in many essential pathways, such as fatty acid synthesis for the cell wall, elucidating which PKS are essential for cell function and which are specific to toxin production is of special interest. Studies focusing on PKS have been conducted for \emph{Alexandrium ostenfeldii} \cite{eichholz2012putative}, \emph{Amphidinium} \cite{murray2012genetic}, \emph{Azadinium spinosum} \cite{meyer20152transcriptomic}, \emph{Gambierdiscus} spp. \cite{pawlowiez2014transcriptome,kohli2015polyketide}, \emph{Hererocapsa} spp. \cite{eichholz2012putative,salcedo2012dozens}, \emph{Karenia brevis} \cite{monroe2008toxic,ryan2014novo}.  \\ %ref kohli 15, google trawl dino pks
Comparative transcriptomics between toxic and non-toxic strains of dinoflagellates can help differentiate between PKS essential for cell functionality and PKS active in toxin production \cite{kohli2015polyketide}Stucken et al). The continued exploration and publication of toxic dinoflagellate transcriptomes is building the framework for toxin related PKS detection. The application for monitoring blooms could revolutionise the detection of ciguatoxic systems and warn local populations as well as the seafood industry.


\subsection{PKS genes in \emph{Gambierdiscus} spp.}
The transcriptome of the CTX producer \emph{G. polynesiensis} was sequenced with a focus on PKS production \cite{pawlowiez2014transcriptome}. The study uncovered 22 transcripts with sequence similarity to type I-like PKS, however no FAS conserved domains were found. The sequence depth and coverage may have been too shallow to detect these sequences \cite{kohli2015polyketide}.
In a recent study he MTX-1 producer \emph{G. australes} was compared to \emph{G. belizeanus}, whose toxin profile did not show any MTX-1 \cite{kohli2015polyketide}. Both strains produced MTX-3. This study was conducted with a focus on PKS detection and combined their transcriptomic data with the dinoflagellate transcriptomes available on Marine Microbial Eukaryote Transcriptome Sequencing Project (MMETSP). Based on the KS domain of PKS genes, they found 5 distinct clades. PKS of clades C, D, and E, containing 34, 20 and 74 sequences respectively, were found in all dinoflagellate taxa analysed in this study which indicates that PKS from these clades correspond to common cellular pathways such as FAS. PKS from clade A, containing 55 sequences, correlate with dinoflagellates that produce polyether ladder compounds while PKS from Clade B contained two sequences which were specific to \emph{G. australes} and \emph{G. belizeanus}. The authors postulate that clade A may contain PKS related to toxin production while clade B may related to MTX production \cite{kohli2015polyketide}. Further in depth transcriptomic coverage of toxin producing dinoflagellates for comparison, and in particular \emph{Gambierdiscus} spp., is required to add support to this theory.

\section{Phylogeny of \emph{Gonyaulacales}}
In order to investigate the evolution of toxin biosynthesis in \textit{Gambierdiscus}, it is necessary to understand the phylogeny of the genus, and its close relatives. The closest relatives of the genus \textit{Gambierdiscus} are not yet known.  As species of the order appear to have relatively fast evolutionary rates and/or constitute ancient lineages, determining their relationships to one another has been difficult. The majority of analyses of dinoflagelalte phylogeny have been conducted using rDNA. Murray et al investigated the evolutionary rate of rDNA using GenBank sequences for LSU and SSU, 42 and 56 sequences respectively. They found that evolution was non-linear. Their findings support that mutational sites were linked through co-evolution, in theory to keep the integrity of the stem loop, but also that different sites evolve at different rates as well as multiple substitutions can occur at the same site. Most phylogenetic programs assume compositional stationarity, which does not hold true for genes with these properties \cite{murray2005improving}. A potential solution to the problem of differential gene evolution between taxa is to concatenate a number of genes, taking into account the different parameter settings \cite{gontcharov2004combined,pupko2002combining}.\\

A phylum which poses similar problems of phylogenetic elucidation were the ciliophora, as the overwhelming majority of data available was from the rDNA region. Ciliophora are contained in the superphylum alveolata alongside the dinoflagellates. Genetaki et al assembled the AA sequences of 158 genes, protein coding as well as rDNA, from transcriptomic data. The breadth of ciliate candidates represented and number of concatenated genes used to analyse the phylogenomic structure of this phylum answered a number of previously unresolved questions such as confirming monophyly of the phylum and resolving the position of the position of a problematic taxon \cite{gentekaki2014large}. \\
Conducting a study on a scale that covers a wide range of genes and species inevitably accrues representatives for which not all data points are available (eg. \cite{gentekaki2014large,bachvaroff2014dinoflagellate}). Gontcharov et al addressed this problem by comparing the phylogenetic resolution of an algal data set comprising of 43 representatives with SSU rDNA and plastid encoded \textit{rbc}L, when taking the sequences in isolation vs. concatenation. They found that the concatenated model, even with gaps for missing genes for some representatives, gave a better resolution, statistical support for node allocation and a higher confidence level for the resulting phylogenetic tree \cite{gontcharov2004combined}. A further argument for inclusion of incomplete taxa for concatenated phylogenies comes from Orr et al, in a study where they scrutinised the evolutionary relationship of dinoflagellates with a focus on thecate evolution \cite{orr2012naked}. The distinction of armouring in dinoflagellates evolving from athecate, or unarmoured, lineages as a single event (monophyly) only became aparent when using the 8 gene dataset inclusive of gaps for some representatives, in effect with the broadest available dataset. The internal braching of the gonyaulacales was not resolved, though certainty increased with number of concatenated genes \cite{orr2012naked}. \\
The phylogenetic relationship of the alveolata was analysed by Bachvaroff et al, including ciliates and dinoflagellates.  The study concatenated 74 ribosomal, protein coding AA sequences and significantly increased support for the placement of problematic basal dinoflagellate taxa \textit{Perkinsus marinus } and \textit{Oxyrrhis marinara}. The core dinoflagellate taxa were not well supported, in all four permutations of the dataset used in the study. In particular the authors note that the gonyaulacoids, represented by \textit{Alexandrium} spp. and \textit{Lingulodinium} sp., were very sensitive to Gblocks trimming which was one of the factors that confounded solid phylogenetic resolution \cite{bachvaroff2014dinoflagellate}.\\

In 2014, MMETSP was published for open access. One of the aims of the project was to establish a reliable, curated base reference database for marine microbial eukaryotes, such as exists for bacteria. Approximately a quarter of sequenced dinoflagellates were \emph{Gonyaulacales}, which represents xxx genera \cite{keeling2014marine}. A limiting factor for elucidating the evolutionary relationship of this order has been limited data points \cite{bachvaroff2014dinoflagellate,gentekaki2014large,orr2012naked}. Access to this database, supplemented with published transcriptomes, can potentially overcome that obstacle. \\ 
%discuss studies that used transcriptomic sequencing to generate large scale concatenated alignments ---  check if shauna gave refs

\section{Aims of the study}
The central aims of this study are:
\begin{enumerate}
\item Elucidate the diversity of \emph{Gambierdiscus} at selected sites
\item Determine the evolution and phylogenetics of the \emph{Gonyaulacales} with a focus on \emph{Gambierdiscus}
\item Investigate the genetics of toxin production in species of \emph{Gambierdiscus}
\item Determine the ecological factors impacting differential expression of toxin related genes in \emph{Gambierdiscus}
\end{enumerate}

\section{Results to date}

\subsection{Characterisation of \emph{Gambierdiscus} from Heron Island and Cook Islands}

\subsubsection{Sample collection and culturing}
Environmental samples were collected from Heron Island and from the Cook Islands.
The isolates were separated by single cell isolation and grown as clonal cultures in F/10 medium. The cultures were maintained in F/10 medium at 26 degrees celcius, xx light in 12hr:12hr light to dark cycles.

\subsubsection{Genomic DNA extraction}
Genomic DNA was extracted using the CTAB method. Purity and concentration of the extract was measured by Nanodrop, the integrity was visualised on 1\% agarose gel.

\subsubsection{PCR amplification and sequencing}
Extracted DNA was used as a template to amplify rRDNA sequences in 25$\mu$l mixture in PCR tubes. The mixture contains 0.6 $\mu$M forward and reverse primer, 0.4 $\mu$M BSA, 2 - 20 ng DNA, 12.5 $\mu$l 2xEconoTaq (Lucigen) and 7.5 $\mu$l PCR grade water.\\
The PCR cycling comprised of an initial 10 min step at 94
The LSU D1-D3  and D8-D10 regions as well as the SSU region were amplified with the D1R-F-, FD8-RB and 18ScomF1-18ScomR1 primer sets respectively (table ~\ref{tbl:PrimerTable}).\\
Sequencing was conducted by Macrogen with Sanger sequencing.
\FloatBarrier
\begin{table}
\caption{List of primers used for phylogenetic elucidation for Heron Island and Cook Island \emph{Gambierdiscus}.}
\label{tbl:PrimerTable}
\begin{tabular}{  | p{2cm} | p{7.5cm} | p{2.5cm} | p{2cm} | }
\hline
\textbf{Name} & \textbf{Sequence} & \textbf{Purpose} & \textbf{Reference} \\
\hline
  \multicolumn{4}{| c |}{\textbf{LSU D1-D3 region}}\\
    \hline
  D1R-F   & ACCCGCTGAATTTAAGCATA & Amplification \& sequencing &  \\
  D3B & TCGGAGGGAACCAGCTACTA & Amplification \& sequencing &  \\
\hline
  \multicolumn{4}{| c |}{\textbf{LSU D8-D10 region}}\\
    \hline
   FD8   & GGATTGGCTCTGAGGGTTGGG & Amplification \& sequencing & \cite{chinain1999morphology} \\
   \hline
 GLD8\_421F   & ACAGCCAAGGGAACGGGCTT & Sequencing & \cite{nishimura2013genetic} \\
 \hline
 GLD8\_677R   & TGTGCCGCCCCAGCCAAACT & Sequencing & \cite{nishimura2013genetic} \\
 \hline
   RB   & GATAGGAAGAGCCGACATCGA & Amplification \& sequencing &\cite{chinain1999morphology}  \\
    \hline
  \multicolumn{4}{| c |}{\textbf{SSU region}}\\
    \hline
 18ScomF1 & GCTTGTCTCAAAGATTAAGCCATGC & Amplification \& sequencing & \cite{zhang2005phylogeny} \\
 \hline
 18ScomR1  & CACCTACGGAAACCTTGTTACGAC & Amplification \& sequencing &  \cite{zhang2005phylogeny}  \\
 \hline
 Dino18SF2  & ATTAATAGGGATAGTTGGGGGC & Amplification \& sequencing &  \cite{zhang2008mitochondrial}\\
 \hline
 Dino18SR1    & GAGCCAGATRCDCACCCA & Amplification \& sequencing &  \cite{zhang2008mitochondrial}\\ 
 \hline
G10'F    & TGGAGGGCAAGTCTGGTG & Sequencing & \cite{nishimura2013genetic} \\
\hline
G18'R    & GCATCACAGACCTGTTATTG & Sequencing &  \cite{litaker2005reclassification} \\
 \hline
\end{tabular}
\end{table}
\FloatBarrier

\subsubsection{Sequence alignment and phylogenetic analysis}
Phylogenetic analysis and sequence checking and manipulation was carried out using Geneious 8.2.1.
The rRNA sequences were aligned using MUSCLE against publicly available sequences from the NCBI database with a maximum of 8 iterations. The alignment was manually screened and adjusted. The 5' and 3' ends were truncated to render all sequences of the same length leaving D1-D3, D10-D8 and SSU at xx bp. Outliers for LSU and SSU alignments were \emph{Alexandrium tamarense} AY831407.1 or \emph{Prorocentrum lima} AB189780.1.
PHYML was used for Maximum Likelyhood (ML) tree generation with 1000 bootstraps in the general-time-reversal model. \\



\begin{figure} 
\includegraphics[scale=.7]{HG_D8D10_phylo.jpg} 
\caption{Maximum Likelyhood phylogeny of Heron Island \textit{Gambierdiscus} isolates using sequences amplified from the D8D10 LSU region.} 
\end{figure} 
\FloatBarrier 
\newpage

\begin{figure} 
\includegraphics[scale=.7]{HG_SSU_phylo.jpg} 
\caption{Maximum Likelyhood phylogeny of Heron Island \textit{Gambierdiscus} isolates using sequences amplified from the SSU region.} 
\end{figure} 
\FloatBarrier 
\newpage

The D8D10 LSU (fig x) and SSU (fig. y) phylogenies of the Heron Island isolates HG1, HG4, HG6, HG7 and HG26 fall into their own well supported clade. These isolates appear to be members of a new species , of the proposed name \emph{Gambierdiscus ludi} sp. nov.\\
In the D8D10 LSU phylogeny (fig. xx), HG5 comes out between \emph{G. silvae} and the not yet characterised \emph{Gambierdiscus} sp. 4. whereas in the SSU phylogeny (fig. xx) HG5 comes out amidst \emph{G. polynesiensis}. This indicates a problem with the \emph{G. polynesiensis} sequences obtained from NCBI. SSU sequences for \emph{G. silvae} and \emph{Gambierdiscus} sp. 4 are not available from NCBI, hence it is not possible to determine whether HG5 is actually as closely related to \emph{G. polynesiensis} as this phylogeny suggests.\\

\begin{figure} 
\includegraphics[scale=.85]{CG_D1D3_phylo.jpg} 
\caption{Maximum Likelyhood phylogeny of Cook Island \textit{Gambierdiscus} isolates using sequences amplified from the D1D3 LSU region.} 
\end{figure} 
\FloatBarrier 
\newpage

\begin{figure} 
\includegraphics[scale=.75]{CG_D8D10_phylo.jpg} 
\caption{Maximum Likelyhood phylogeny of Cook Island \textit{Gambierdiscus} isolates using sequences amplified from the D8D10 LSU region.} 
\end{figure} 
\FloatBarrier 

CG14 and CG15 are consistently located amidst \emph{G. polynesiensis} reference sequences with high support in both D1D3 and D8D10 LSU alignments (fig. xx and fig. yy). Alas it is likely that these isolates are members of that species.\\
Likewise, CG61 is consistently placed with \emph{G. australes} with high support in both analyses (fig. xx and fig. yy).



\subsubsection{Morphology}
Scanning electron microscopy (SEM) was conducted by Mona Hoppenrath at the xxx institute, Germany. The plate structure of isolates HG5 and HG7 was elucidated from the images as per xx.\\
--- include light microscopy?

\subsubsection{Toxin profile analysis}
Toxin profile analysis for HG4, HG5, HG6 and HG7 was conducted by Tim Harwood at the Cawthorne Institute, Nelson, New Zealand, via LC-MS. The protocol is set up to monitor for CTX-3B, CTX-3C, CTX-4A, CTX-4B, MTX-1 and MTX-3. Solely MTX-3 was detected in all samples. Dr. Harwood reported unusual peaks in some samples which are subject to further investigation. \\
Toxicity of HG4, HG6 and HG7 is currently under investigation via MBA by i. p. injection by Rex Munday at the Cawthorne Institute, Nelson, New Zealand.
 

\subsection{Phylogeny of \emph{Gonyaulacales}}

\subsubsection{RNA extraction and analysis}
Cells from HG4, HG5, HG6 and HG7 were harvested from 1.2L cultures on day 13 during the late exponential phase, using 5 $\mu$m Durapore filters. Cells were washed off filters with F/10 and pelleted by centrifugation at xx for 10 minutes. The cell pellet was washed thrice with PBS and cells broken down with Zirkonium beads by bead beating (setting 6, 30s, name of machine). The lysate wss extracted with Tri-Reagent (Life Technologies) as per manufacturer's protocol. The RNA was purified with the RNeasy Plant mini kit (Qiagen, Limberg, Netherlands) as per the manufacturer's protocol. The eluent was treated with the TURBO DNA-free Kit (Life Technologies) to remove residual DNA. RNA quantity and quality was measured by Nanodrop (name) and the integrity by Agilent 2100 Bioanalyser  at the Ramaciotti centre.

%\subsubsection{Transcriptomic databases used}
%Microbial marine eukaryote transcriptome project (MMETSP)


%\subsubsection{Concatenated phylogeny}

%\subsection{Search for PKS genes in \emph{Gambierdiscus} transcriptomes}

%\subsection{PKS expression in \emph{Gambierdiscus} strains}


%\section{Results}

%\subsection{Diversity of \emph{Gambierdiscus} at Heron Island and Cook Islands}

%\subsubsection{Phylogenetics}

%\subsubsection{Morphological analysis}

%\subsubsection{Toxin analysis}

\section{Future work and potential obstacles}

\subsection{qPCR primer development}
An easily accessible solution for sample screening for and enumeration of \emph{Gambierdiscus} spp. is  semi-quantitative PCR. A complete set of published qPCR primers for \emph{Gambierdiscus} spp. would be an adequate screening tool and could serve as an early detection system for ciguatera outbreaks when coupled with toxicological data. Development of such a set of qPCR primers is part of Element 1 of the global ciguatera strategy plan \cite{globalcig}. \\
Species specific PCR primers need to be developed. The design requires three steps - initially identification of species specific variable sites though sequence allignment and comparison to all other known \emph{Gambierdiscus} spp. in SSU or LSU region. Secondly the primer set is tested against the target species. If amplification is successful, the primer set is tested against at least the closest relatives of the target species as negative controls to rule out cross reactivity. \\
A roadblock in this part of the project will be acquiring DNA samples to test qPCR primers on as a negative control as well as getting permission from the suppliers of the target DNA extracts to develop qPCR primers.

Vandersea et al have published qPCR primers for \emph{G. belizeanus}, \emph{G. caribaeus}, \emph{G. carpenteri}, \emph{G. carolinianus}, \emph{F. rutzleri} and \emph{Gambierdiscus} sp. ribotype 2 \cite{vandersea2012development}. Litaker and Chinain are in the process of publishing qPCR primers for \emph{G. australes}, \emph{G. pacificus}, \emph{G. polynesiensis} and \emph{G. toxicus} (pers. comm. with W. Litaker).

\subsection{Phylogeny of \emph{Gonyaulacales}}
Transcriptomics is the solution for elucidating the phylogeny of \emph{Gonyaulaceles} as it focuses on the actively transcribed gene copies (cite). Ideally, every genus should be represented. However the limiting factor is access to transcriptomes. The main source for transcriptomes will be from the MMETSP, available online at xxxx. Transcriptomes retrieved by this database will be supplemented by transcriptomes available from published studies. Nucleotide sequences will be converted to AA sequences which masks silent mutation that have no impact on gene function (cite Bachvaroff '14). The methodology for sequence preparation and concatenation will be adapted from large scale, transcriptome based phylogenetc studies conducted by Bachvaroff et al and Genetaki et al (cite Bachvaroff and Genetaki).\\
It is necessary to account for differential evolutionary rates between genes with imposing paramteres for analysis, without imposing so many assumptions as to come to the wrong conclusion (cite Pupko 'o2). Hence, selecting the right models to analyse multiple sequences is essential. Pupko et al explore the impact of different branch length models (concatenated, proportional and separate) with different among site variation models (Homogenous, 1-GAM and N-GAM) on two different AA datasets and a mitochondrial DNA dataset, using maximum likelyhood (ML) pylogentic analysis. Invariably, the assumption that the mutation sites have separate rates of evolution (N-GAM) was the best model. The most appropriate branch length parameters was either the proportional or separate model, depending on the dataset (cite Pupko '02). Based on these findings, ML with the N-GAM parameter will be adopted for the generation of phylogenetic trees. Whether the separate or proportional model will be more adequate is yet to be determined.

%\FloatBarrier
%\begin{table}
%\caption{Potential candidates witha available transcriptomes for phylogenetic elucidation of %\emph{Gonyaulacales}.}
%\label{tbl:GonyaTransTable}
%\begin{tabular}{  | p{5cm} | p{4cm} | p{5cm} |}
%\hline
%\emph{Organism} & \emph{} & \emph{Source} \\
% \hline
%\end{tabular}
%\end{table}
%\FloatBarrier
Genera of Gonyaulacales which have accessible transcriptomes are Azadinium \cite{murray2012genetic}, Ceratium, Crypthecodinium, Gambierdiscus, Pyrodinium, Alexandrium, Gonyaulax, Protoceratium and Lingulodinium \cite{kohli2015polyketide}.
This leaves 12 genera without currently accessible representatives for the study.

\subsubsection{Selection of gene candidates for concatenated phylogeny}
Genes selected should be universally represented in the organisms, known as unigenes. A study by xx et al showed that an increased number of AA sites increases phylogenetic resolution. They also found that inclusion of representatives with incomplete concatenated sequences is superior to less nodes on the tree (cite Orr plus Wiens '06 within). While the emphasis will be on achieving maximum coverage of selected unigenes for each organism, incomplete conactenations will be included.
A study conducted by xx explores the phylogenetic relationship of cilliates using 158 unigenes (cite). The methodology for this study will be adapted for this projec

%\FloatBarrier
%\begin{table}
%\caption{Potential gene candidates for phylogenetic elucidation of \emph{Gonyaulacales}.}
%\label{tbl:GeneCandTable}
%\begin{tabular}{  | p{5cm} | p{4cm} | p{5cm} |}
%\hline
%\emph{Gene name} & \emph{rate of gene evolution} & \emph{Precedence for concatenation in %literature} \\
 %\hline
%\end{tabular}
%\end{table}
%\FloatBarrier

\subsection{Search for PKS genes in \emph{Gambierdiscus} transcriptomes}
Exploring the differential transcriptome of toxic an non-toxic isolates of \emph{Gambierdiscus} is designed to explore which PKS are involved in toxin production. By characterising the PKS found in the non-toxic strains, for FAS or other cellular processes, they can be eliminated when searching for PKS in the toxic strain. 

\subsection{PKS expression in \emph{Gambierdiscus} strains}
To address whether PKS expression is affected by environmental factors, RNA is extracted from cultures subjected to different conditions, such as: 
\begin{enumerate}
\item culturing in presence of copeopods - elicit defence mechanism in form of toxins?
\item culture with different media - some media said to be better for toxin production..?
\end{enumerate}
How the differential expression will be quantified depends on the outcome of the previous section. Microarray is an option, as is qPCR. These options will be explored closer to execution of this experiment.
\subsection{Proposed project timeline for the next year} 

\FloatBarrier
\begin{figure} 
\includegraphics[scale=.9]{Timeline_DSP1_smaller.png} 
\caption{.} 
\end{figure} 
\FloatBarrier 
c
\newpage

\bibliographystyle{plain}
\bibliography{review_ref.bib}


\end{document}
