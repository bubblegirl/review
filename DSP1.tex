\documentclass[12pt]{article}
\usepackage[hcentering,bindingoffset=20mm]{geometry}
\usepackage{placeins}
\usepackage[numbib]{tocbibind}
\usepackage{rotating}
\usepackage[square,sort,comma,numbers]{natbib}
\usepackage{graphicx}
\usepackage{tabularx}
\linespread{1.3}
%\fontsize{8cm}{1.3em}\selectfont
\usepackage{gensymb}
\usepackage{longtable}
\usepackage{lscape}

\addtolength{\textwidth}{2cm}
\addtolength{\hoffset}{-1cm}


\addtolength{\textheight}{2cm}
\addtolength{\voffset}{-1cm}
\setlength{\parindent}{0pt}

\title{\textbf{Questing for the genes involved in ciguatoxin production}}
\author{\textbf{Candidate:} Anna Liza Kretzschmar\\
 \textbf{Chief Supervisor:} Shauna Murray\\
 \textbf{Co-Supervisor:} \\
 {\small University of Technology Sydney, Plant Functional Biology and Climate Change Cluster}}
\date{}

\begin{document}
\maketitle
% need to replace all cite with citep
\section{General Introduction}
\emph{Gambierdiscus toxicus} is a dinoflagellate, a type of marine microalgae, and is epiphytic on many substrates in shallow tropical and sub-tropical waters. It was first suggested to produce the causative agent of ciguatera fish poisoning (CFP), a severe seafood-borne illness, in 1977 \cite{yasumoto1977finding}.  
Since then, extensive research has established that species of the genus are the primary sources of the toxins ciguatoxin (CTXs) and maitotoxin (MTXs), which are known to be involved in CFP \cite{chinain1997intraspecific,holmes1998gambierdiscus}. CTXs bioaccumulate through the food web until seafood reaches the orally accumulative toxicity level which cause CFP \cite{bagnis1979clinical,gillespie1987possible,sims1987theoretical}. %Whether MTXs play a role in CFP needs to be further investigated \cite{kohli2014feeding}. 
For over 15 years \emph{Gambierdiscus} was considered to be a monotypic taxon, but since 1995 new species are being discovered with the aid of genetic analysis tools \cite{faust1995observation,holmes1998gambierdiscus,litaker2009taxonomy,chinain1999morphology,fraga2011gambierdiscus,nishimura2014morphology}. Evidence suggests that the production of CTXs and MTXs varies between species \cite{chinain2010growth,holland2013differences}, similar to the situation in other dinoflagellates, in which toxin production is highly variable between species, and generally stable within species.
As CFP is increasing in frequency in the Pacific \cite{skinner2011ciguatera} and in other regions, an understanding of the genetics and toxicology of these organisms is becoming increasingly important. 

\section{Ciguatera}
CFP is the most common non bacterial illness associated with seafood consumption, posing a global problem \cite{friedman2008ciguatera}. %CFP problem ref! + seafood trade
Annual reported CFP cases are estimated up to 500,000 a year globally and are responsible for 80 to 96 \% of human poisoning from consuming seafood \cite{fleming1998seafood,grandjean2008centers}, however correct diagnosis is problematic. 

Seafood contaminated with CTXs is characteristically impossible to discern during preparation as it does not exhibit any differences to non toxic seafood and the toxins are heat resistant \cite{withers1982ciguatera}. Toxin presence is usually only diagnosed retrospectively from CFP symptoms in conjunction with recent seafood consumption \cite{sims1987theoretical}. CFP is predicted to be widely under reported with as little of 20 \% of cases being reported  in Queensland, Australia \cite{lewis2006ciguatera}. The disease can manifest as over 180 symptoms, with varying levels of severity, and differ even between individuals at the same time and source of exposure \cite{sims1987theoretical}. The effects of CFP appear to be cumulative, as severity increases with previous exposure \cite{emerson1983preliminary}. Commonly several hours after consumption, the initial symptoms are gastrointestinal, which can be followed by cardiovascular and neurological manifestations \cite{sims1987theoretical}. They can persist between weeks to years. Recurrence is a common phenomenon induced by the factors: alcohol, fish or meat consumption \cite{lewis2006ciguatera} as well as sexual activity \cite{lange1992travel}. \\

CTXs undergo biotransformation during uptake of vector species  (find ref). %different types between different trophic levels
Some modifications to the toxin structure in the food chain can increase the potency of the toxins up to 10-fold, compared to the toxin produced by the \emph{Gambierdiscus} sp. under observation \cite{lewis2006ciguatera}.
In 2005 Hung et al established that the CFP of three family members was caused by CTX present in Baracuda fish eggs \cite{hung2005persistent}. This indicates that bioaccumulated CTXs can be passed on to offspring which start their life cycle with CTX concentrations above the toxicity threshold to elicit CFP.  \\ 
There is no reliable bioassay for seafood samples due to CTXs structural variety and uncertainty of quantification \cite{dickey2010ciguatera}. The most commonly used bioassay is by mouse intraperitoneal injection but this is not indicative for orally active toxins \cite{botana2014seafood}.\\
A study correlating CFP related calls from the US National Poison Centre Data with storms and sea surface temperature showed that sea surface temperature was a relevant factor for CFP incidence and extrapolated that climate change related sea temperature rise could escalate CFP by 200 - 400 \% in the US \cite{garces2012habitat}. %incl the monetary figure on CFP
A study of preferential substrate colonisation showed that \emph{Gambierdiscus} spp. favoured dead corals, which will increase in availability with coral bleaching \cite{grzebyk1994ecology}. Over the last decade, CFP incidence has increased by 60\% in Pacific Islands \cite{skinner2011ciguatera}.
CFP has been reported in the Caribbean, Pacific and Indian oceans. In the Mediterranean Sea \cite{lejeusne2010climate}, \emph{Gambierdiscus} species have been detected only since the early 2000s \cite{aligizaki2008morphological}.
Understanding the phylogeny and species specific toxicology of \emph{Gambierdiscus} spp. is essential as part as global HAB species monitoring for early warning system for CFP outbreaks \cite{berdalet2012global}.

%The under reporting and diagnostic issues mentioned, compounded by the absence of an adequate commercial test kit \cite{wong2005study}, the exact figure for number of global CFP cases cannot be determined. Carnivorous fish are the main cause of CFP, however herbivorous fish (eg. surgeonfish, parrotfish) pose important intermediate vector in the food chain \cite{cruz2006macroalgal,randall1958review,mak2013pacific}.

\section{The genus \emph{Gambierdiscus}}
We currently have little information of which species of \emph{Gambierdiscus} are present in Australia and surrounding countries. Species of \emph{Gambierdiscus} have been reported from many countries, but the unresolved taxonomy of the group has resulted in many uncertain identifications \cite{marine2014}.

Some \emph{Gambierdiscus} species have been classified as endemic to either the Pacific or Atlantic Oceans, while others have been found globally distributed \cite{berdalet2012global,litaker2010global}. It has been suggested that with more extensive sampling, the distribution is likely to be global for all \emph{Gambierdiscus} species \cite{testerICHA}. However current understanding about \emph{Gambierdiscus} distribution and abundance is fragmentary due to the relatively recent progress regarding the phylogeny within the genus, the continued discovery of new species, difficulty with species identification and the comparatively lower sampling frequency in the Atlantic. \\
%^ shithouse paragraph
Localised benthic HABs (BHABs) of \emph{Gambierdiscus} species have been noted in literature in both the Pacific and Atlantic regions \cite{nakajima1981toxicity,withers1984ciguatera,chinain1999seasonal,darius2007ciguatera}.
%In general, when the density of a species of \emph{Gambierdiscus} 10,000 cell g$^{-1}$ wet weight algae, this has been referred to as a 'bloom' \cite{chinain1999seasonal}.%However, accurately estimating the abundance of benthic dinoflagellates including \emph{Gambierdiscus} is very difficult, due to the fact that cells can be patchily distributed in both space and time \cite{lobel1988assessment,ballantine1988population,litaker2010global}.

%This is due to two main factors: the fact that many regions have not yet been sampled for \emph{Gambierdiscus}, and the difficulty in identifying species. %For example,  In the hundreds of samples from Atlantic (Caribbean/Gulf of Mexico/West Indies/Southeast US coast Florida to North Carolina), no Pacific specific \emph{Gambierdiscus} sp. have been detected \cite{berdalet2012global,litaker2010global}. However,  most of the Pacific Ocean has not been sampled for \emph{Gambierdiscus} spp.,  so a similar conclusion about Atlantic species in the Pacific Ocean cannot be made and assertions about endemism or restricted distribution cannot be made conclusively.

%In terms of understanding difficulties in species identification, one example is \emph{G. yasumotoi}. \emph{G. yasumotoi} has been primarily reported in the Pacific, except for one report in the Mexican-Caribbean \cite{hernandez2004species}. However this singular record precedes the discovery of the other globular species \emph{G. ruetzleri}, which is considered to be endemic to the Atlantic region \cite{litaker2009taxonomy}. However a recent report of this species at the southern Kuwait coast and the Gulf of Aqaba in Jordan \cite{saburova2013new}, could indicate that this species is globally distributed also \cite{xu2014distribution}.\\ Shauna reckons o leave this out because the species are probably the same. so either i didnt write this very well to showcase how flawed morphology as an identification tool is, or its irrelevant.


In the absence of an estimate for cell density of \emph{Gambierdiscus} which leads to CFP, it has been proposed  that high cell concentrations of \emph{Gambierdiscus} species are likely to lead to increased CFP risk \cite{litaker2010global}. 
However, this hypothesis has not been consistently reported in the literature. Sampling within the Republic of Kiribati showed comparatively low colonisation of \emph{Gambierdiscus} species on the macro algae \emph{Halimeda} with 0 - 174 cells g$^{-1}$ wet weight algae. Yet 91 \% of fish sampled in the area were considered ciguateric as they exceeded the quarantine threshold of 0.01$\mu$g/kg P-CTX-1 equivalent \cite{xu2014distribution,chan2011spatial}. The authors postulate that \emph{Gambierdiscus} spp. could preferentially colonise another substrate. Xu et al suggested that an unidentified \emph{Gambierdiscus} sp. could be present alongside the 6 low to medium toxic strains reported \cite{xu2014distribution,bomber1988r}. Another option could be that low level, consistent exposure of CFP causes the concentration of CTXs to increase in fish over long periods of time.\\

Different species of \emph{Gambierdiscus} usually co-occur at sample sites \cite{litaker2010global}. As species can show morphological variability within species (ref gamb lifecycle paper), but can also be highly morphologically similar to one another (ref), molecular genetic techniques are necessary to complement morphology in elucidating species composition.

\subsection{In Australia}
To date, four species of \emph{Gambierdiscus} have been reported from only 6 sites in Australia, the only sites yet examined \cite{kohli2014cob,murray2014molecular}. The toxicity of strains from Australia have never before been examined, and their distribution along the Australian coast are largely not known. It is highly likely that many more \emph{Gambierdiscus} species are present, in each case, the studies represent only single or short term sampling events at one location within a site.


In 1994 Lewis et al reported three analogues of MTX, designated MTX-1, MTX-2 and MTX-3, isolated from a \emph{Gambierdiscus} sp. in Queensland, Australia \cite{holmes1994purification}. As this discovery predates the understanding that \emph{Gambierdiscus} includes more species than \emph{G. toxicus}, it is unclear from which species these analogues were isolated.  \\

The species \emph{G. carpenteri} was found to be the only \emph{Gambierdiscus} species present at three separate sampling sites in temperate Australian waters. Recorded cell densities were 8256 g $^{-1}$ wet weight algae \cite{kohli2014high}. No CTX or MTX was detected in this strain via LC-MS/MS, however the MTX-like fraction did display toxicity via MBA. Whether this strain could have contributed to the CFP incidences in the region on the basis of MTX production is yet to be established  \cite{kohli2014high}. 
%the fuck? not even MTX-3?

\subsection{Morphology}

Currently 13 species, and 6 unnamed ribotypes of \emph{Gambierdiscus} have been described (Table ~\ref{tbl:MorphTable}) based on their distinct morphological and genetic characteristics. New discovery of species with increased global sampling is likely. \\

The descriptions in Table ~\ref{tbl:MorphTable} contain an overview of the main morphological characteristics for currently described \emph{Gambierdiscus} sp. The terminology used has been conformed to the proposed terms from Hoppenrath et al \cite{hoppenrath2013taxonomy}. The initial species description detailed \emph{Gambierdiscus} to be large (60-100 $\mu$m), armoured with a distinct plate pattern and fishhook shaped accessory pore. Species fall into two categories - antero-posteriorly compressed, also described as lenticular, or slightly laterally compressed, or globular. Further classification is based on characteristics such as plate patterns and accessory pore shape. %ref

These features can be readily identified using scanning electron microscopy. However the strain specific variability of morphology within species, such as in size and shape of individual plates, makes this a support tool for classification which should be verified with genetic analysis. \\

\begin{longtable}{ |  p{1.6cm} | p{2.5cm} | p{6.7cm} |  p{3.2cm} | }
\caption{Presently characterised \emph{Gambierdiscus} spp. and their taxonomic identifications; table adapted from Kholi \cite{kohli2013Gambierdiscus}}\\
\hline
\label{tbl:MorphTable}
\textbf{Species} & \textbf{Cell size ($\mu$m) (depth-width-length)} & \textbf{Morphological characteristics} & \textbf{References} \\
\hline
 \emph{G. australes} & (72.5$\pm$3.8)-(63.4$\pm$5.0)-(38.7$\pm$3.8) & Antero-posteriously compressed species; narrow 1p plate; smooth cell surface; 2' rectangular shaped; smaller than \emph{G. excentricus} &  \cite{chinain1999morphology,litaker2009taxonomy} \\
\hline
 \emph{G. belizeanus} & (61.7$\pm$3.1)-(60.0$\pm$4.5)-(48.1$\pm$4.2) & Antero-posteriously compressed species; narrow 1p plate; different 2' plate symmetry and size; heavily reticulate-foveate cell surface & \cite{litaker2009taxonomy,faust1995observation} \\
\hline
 \emph{G. caribaeus} & (82.2$\pm$7.6)-(81.9$\pm$7.9)-(60.0$\pm$6.2) & Antero-posteriously compressed species; broad 1p plate; 2’ rectangular shaped; symmetric 4’’  & \cite{litaker2009taxonomy} \\
\hline
 \emph{G. carolinianus} & (78.2$\pm$4.8)-(87.1$\pm$7.1)-(51.4$\pm$5.2) & Antero-posteriously compressed species; broad 1p plate; 2’ hatchet shaped; dorsal end 1p oblique; larger cell size than \emph{G. polynesiensis}  & \cite{litaker2009taxonomy} \\
\hline
 \emph{G. carpenteri} & (81.7$\pm$6.4)-(74.8$\pm$5.9)-(50.2$\pm$6.1) & Antero-posteriously compressed species; broad 1p plate; 2’ rectangular shaped; asymmetric 4''   & \cite{litaker2009taxonomy} \\
\hline
  \emph{G. excentricus} & (97.8$\pm$8.0)-(83.0$\pm$10.0)-(37.0$\pm$3.0) & Antero-posteriously compressed species; narrow 1p plate; smooth cell surface; 2’ rectangular shaped; cell size bigger than \emph{G. australes} (1.5 times wider and deeper)   & \cite{litaker2009taxonomy} \\
\hline
  \emph{G. pacificus} & (58.5$\pm$3.9)-(53.6$\pm$4.1)-(40.4$\pm$3.6) & Antero-posteriously compressed species; narrow 1p plate; smooth cell surface; 2' hatchet shaped   & \cite{litaker2009taxonomy,chinain1999morphology} \\
\hline
 \emph{G. polynesiensis} & (66.3$\pm$3.0)-(60.5$\pm$5.9)-(44.3$\pm$5.1) & Antero-posteriously compressed species; broad 1p plate; 2’ hatchet shaped; dorsal end 1p oblique; smaller cell size than \emph{G. carolinianus}   & \cite{litaker2009taxonomy,chinain1999morphology} \\
\hline 
 \emph{G. ruetzleri} & (45.5$\pm$3.3)-(37.5$\pm$3.0)-(51.6$\pm$4.9) & Globular species smaller than \emph{G. yasumotoi}; cell size less than 42$\mu$m  & \cite{litaker2009taxonomy} \\
 \hline
 \emph{G. scabrosus} & (63.2$\pm$5.7)-(58.2$\pm$5.7)-(37.3$\pm$3.5) & Antero-posteriously compressed species; narrow 1p plate; reticulate-foveate; asymmetric 4'' plate  & \cite{nishimura2013genetic,nishimura2014morphology,kuno2010genetic} \\ %is 3'' plate right?
\hline
\emph{G. silvae} & (69$\pm$8)-(64$\pm$9)-(46$\pm$5)  & Antero-posteriorly  compressed; narrow 1p plate; heavily reticulate-foveate;  & \cite{litaker2010global,fraga2014genus} \\
\hline
 \emph{G. toxicus} & (93.0$\pm$5.5)-(83.0$\pm$2.3)-(54.0$\pm$1.5) & Antero-posteriously compressed species; broad 1p plate; 2’ hatchet shaped; dorsal end 1p pointed  & \cite{litaker2009taxonomy,adachi1979thecal,chinain1997intraspecific,richlen2008phylogeography} \\
 \hline
  \emph{G. yasumotoi} & (56.8$\pm$5.6)-(51.7$\pm$5.6)-(62.4$\pm$4.3) & Globular species larger than \emph{G. ruetzleri}; cell size exceeds 42 $\mu$m  & \cite{holmes1998gambierdiscus,litaker2009taxonomy} \\
  \hline
  \multicolumn{4}{| c |}{\textbf{Genetically described phylotypes}}\\
    \hline
\emph{Gambier- discus} ribotype 2 & N/A & N/A & \cite{litaker2010global} \\
\hline
\emph{Gambier- discus} sp. type 2 & N/A & N/A  & \cite{kuno2010genetic,nishimura2013genetic} \\
\hline
\emph{Gambier- discus} sp. type 3 & N/A & N/A  & \cite{nishimura2013genetic} \\
\hline
\emph{Gambier- discus} sp. type 4  & (65.9-72.4$\pm$4.1-4.2)-(64.5-68.9$\pm$5.0)-(N/A) & Antero-posteriorly compressed species;evenly distributed pores; larger than \emph{Gambierdiscus} sp. type 5; 2' hatchet shaped; broad 1p plate  & \cite{xu2014distribution} \\
\hline
\emph{Gambier- discus} sp. type 5  & (54.8$\pm$4.6)-(53.7$\pm$6.3)- (N/A) & Antero-posteriorly compressed species;evenly distributed pores; smaller than \emph{Gambierdiscus} sp. type 4; 2' rectangular shaped; narrow 1p plate & \cite{xu2014distribution} \\
\hline
 \emph{Gambier- discus} sp. type 6 & N/A & N/A & \cite{xu2014distribution} \\
 \hline
\end{longtable}
\FloatBarrier

\subsection{Phylogenetics}
%Nishimura 14 - light micro unreliable due to subtle differences
%insert more spin, especially what Shauna said 

Phylogenetic analysis is a powerful tool that will be essential in elucidating the evolutionary relationships of species and genera. Molecular genetic tools for discriminating species of \emph{Gambierdiscus} from one another have been frequently applied, particularly in the last 10 years. These have consisted of ‘barcoding’ regions of genes such as large subunit (LSU) rRNA, small subunit (SSU) rRNA and internal transcribed spacer (ITS) rRNA. From such analysis, the current consensus is that \emph{Gambierdiscus} is monophyletic, with the lenticular and globular species forming two distinct clades \cite{chinain1999morphology,litaker2009taxonomy,fraga2011gambierdiscus,richlen2008phylogeography,kuno2010genetic,litaker2010global,nishimura2013genetic}. \\

Phylogenetic analysis concluded that the globular clade containing \emph{G. ruetzleri} and \emph{G. yasumotoi} diverged early in the evolution of the genus \cite{litaker2009taxonomy,nishimura2013genetic}. The two species are also the most closely related within the genus, and cannot be clearly distinguished from one another based on either molecular sequences of strains from different localities, or based on morphological features. \\

Based on D8-D10 LSU rRNA sequences, \emph{Gambierdiscus} ribotype 2 was designated a putative new phylotype by Litaker et al \cite{litaker2010global} as well as \emph{Gambierdiscus} species type 4, 5 and 6 by Xu et al \cite{xu2014distribution}. Based on SSU rRNA sequences \emph{Gambierdiscus} species types 2 and 3 were also found to be genetically distinct \cite{nishimura2013genetic,kuno2010genetic}. The species\emph{G. scabrosus}, previously \emph{Gambierdiscus} species type 1 \cite{nishimura2013genetic,nishimura2014morphology},  and \emph{G. silvae}, previously Gambierdiscus ribotype 1 \cite{fraga2014genus}, have been identified based initially only on their divergent genetic sequences. Following this, morphological features that confirmed their identities were also described. \\

An emerging issue in the identification of \emph{Gambierdiscus} spp. is that morphology is not as highly conserved within the species as previously assumed. For example, within \emph{G. yasumotoi} and \emph{G. carpenteri}, morphological features that were considered to be consistent within the species and hence used as an identifier of the species were found to vary considerably \cite{murray2014molecular,kohli2014high}. Therefore, while morphology can be an indicator, genetic analysis is essential for species identification.

%ok, so... p-values from Gurjeet's stuff?
% NZ strain of G yas Rhodes_14 shows super similarity between G yas and G rutz but was decided on G yas on SEM data --- insert what was requested during discussion - 
%Shauna, picture from your book will go in here and a paragraph on why morphology isn't enough for species identification, referring to both your and Gurjeet's work on \emph{G carpenteri} and \emph{G. yasumotoi} still to come.
\FloatBarrier
\begin{longtable}{ |  p{2.2cm} | p{2.8cm} | p{2.8cm} | p{2.8cm} | p{2.6cm} | }
\caption{Presently characterised \emph{Gambierdiscus} spp. and their genetic identifications}\\
\hline
\label{tbl:MorphTable}
\textbf{Species} & \textbf{LSU D1-D3 region (Genbank \#)} & \textbf{LSU D8-D10 region (Genbank \#)} & \textbf{SSU region (Genbank \#)} & \textbf{References} \\
\hline
 \emph{G. australes} & EF202969-72 & EF498072-74  & EF202891-96 & \cite{chinain1999morphology,litaker2009taxonomy} \\
\hline
 \emph{G. belizeanus} & EF202940-43 &  EF498028-34 & EF202876-77  & \cite{litaker2009taxonomy,faust1995observation} \\
\hline
 \emph{G. caribaeus} & EF202929-37, EF202983, EF202985 &  EF498045-71  & EF202914-28 & \cite{litaker2009taxonomy} \\
\hline
 \emph{G. carolinianus} &  EF202973-75 & EF498035-37 & EF202897-EF202901  & \cite{litaker2009taxonomy} \\
\hline
 \emph{G. carpenteri} &  EF202938-39, EF202984 & EF498038-44  & EF202908-13   & \cite{litaker2009taxonomy} \\
\hline
  \emph{G. excentricus} &  HQ877874, JF303063, JF303065-71 & JF303073-76 & N/A & \cite{litaker2009taxonomy} \\
\hline
  \emph{G. pacificus} &  EF202944-47 &  EF498012-13, EF498015-16 & SSU: EF202861-65  & \cite{litaker2009taxonomy,chinain1999morphology} \\
\hline
 \emph{G. polynesiensis} &  EF202976-82 & EF498076-80 & EF202902-07& \cite{litaker2009taxonomy,chinain1999morphology} \\
\hline 
 \emph{G. ruetzleri} &  EF202962-64 & EF498081-85 & EF202853-60 & \cite{litaker2009taxonomy} \\
 \hline
 \emph{G. scabrosus} &  AB605004.1 & AB765908-12  & AB764229-76 & \cite{nishimura2013genetic,nishimura2014morphology,kuno2010genetic} \\ 
\hline
\emph{G. silvae} &  KJ620013.1 & GU968512-20, GU968523 & N/A & \cite{litaker2010global,fraga2014genus} \\
\hline
 \emph{G. toxicus} &EF202951-61 & EF498017-27 &  EF202878-90 & \cite{litaker2009taxonomy,adachi1979thecal,chinain1997intraspecific,richlen2008phylogeography} \\
 \hline
  \emph{G. yasumotoi} & EF202965-68 & EF498087-89  &EF202846-52 & \cite{holmes1998gambierdiscus,litaker2009taxonomy} \\
  \hline
  \multicolumn{5}{| c |}{\textbf{Genetically described phylotypes}}\\
    \hline
\emph{Gambier- discus} ribotype 2 & N/A &  GU968499-500, GU968503, GU968505, GU968507-11 & N/A  & \cite{litaker2010global} \\
\hline
\emph{Gambier- discus} sp. type 2 & N/A & AB765913-18 & AB764277-96  & \cite{kuno2010genetic,nishimura2013genetic} \\
\hline
\emph{Gambier- discus} sp. type 3 & N/A & AB765923-24& AB764296-300  & \cite{nishimura2013genetic} \\
\hline
\emph{Gambier- discus} sp. type 4  & N/A &   KJ125080, KJ125114-15,KJ125119-20 & N/A  & \cite{xu2014distribution} \\
\hline
\emph{Gambier- discus} sp. type 5  &  N/A &  KJ125132-35 & N/A & \cite{xu2014distribution} \\
\hline
 \emph{Gambier- discus} sp. type 6 & N/A & KJ125108-09, KJ125111-13 & N/A  & \cite{xu2014distribution} \\
 \hline
\end{longtable}
\FloatBarrier


\section{Toxins produced by \emph{Gambierdiscus}}

Symptoms of CFP can vary between geographical locations \cite{molgo2000ciguatera,dickey2010ciguatera}, which likely contributes to the inefficiency of diagnoses. This could be due to the structural difference between CTX analogues ///REF//. Hence the accurate determination of CTX congeners is vital to understanding toxicology of CFP.\\

We do not yet have clear information available on the toxin profiles of each species of Gambierdiscus, particularly comparative assays of multiple species which have applied a consistent methodology. Bioassays provide an excellent indicator of a species' toxicity, however it does not suffice to elucidate a detailed toxin profile - liquid chromatography-mass spectrometry (LC-MS) or tandem LC-MS/MS analysis is required \cite{diogened2014chemistry}. \emph{Gambierdiscus} spp. produce more than just CTXs - notably MTXs are also commonly registered \cite{holmes1994purification,murata1993structure}. It has, until recently, been assumed in the literature that CTXs are the sole toxins causing CFP, CTXs have been isolated from the Pacific, Caribbean and the Indian Oceans. The  structure varies between the Pacific and the Caribbean congeners, with the structures of some of the Indian Ocean analogues still to be determined. \\

\subsection{CTX}
CTX and its analogues are lipid soluble polyether ladder compounds which act as sodium channel activators \cite{dechraoui1999ciguatoxins}. They are orally effective in humans at the picomolar (pM) and millimolar (mM) concentration range \cite{molgo2000ciguatera}, causing an influx of Na$^{+}$ ions and hence spontaneous action potentials in the cell, especially active on voltage sensitive channels along the nodes of Ranvier \cite{sims1987theoretical,mattei1999neurotoxins,lewis1992action,molgo2000ciguatera}. \\
Currently CTXs are prefixed by their origin, where members from the Pacific, Caribbean and Indian Oceans are designated P-CTX, C-CTX and I-CTX respectively. % don't know C-PTX subdivision ref... for P-CTX it's \cite{} I-CYX is\cite}

The congeners are further subdivided on a structural basis into type I and type II within P-CTXs, followed by all C-CTXs as type III - the molecular structure of I-CTXs have not been elucidated to such an extent as to assign a type \cite{legrand1997two,hamilton2002multiple,hamilton2002isolation} (Table ~\ref{tbl:CTXTable}).  \\

Type I P-CTXs consist of 13 rings with 60 C atoms \cite{murata1990structures,lewis1991purification,lewis1993origin}. The first completely defined CTX was isolated from moray eels, and was designated P-CTX-1B \cite{murata1990structures} or also described as CTX-1 \cite{lewis1991purification}. P-CTX-2 and P-CTX-3 were isolated from the same extracts but exhibited an alternate structure and toxicity in mice \cite{lewis1991purification}. It has been suggested by two research groups that P-CTX-1, P-CTX-2 and P-CTX-3 may be derivatives from the dinoflagellate precursors CTX-4A and CTX-4B (also GTX-4B by Murata et al \cite{murata1990structures}) \cite{lewis1993origin,yasumoto2000structural}. So far, P-CTX-4A and P-CTX-4B have been isolated from \emph{G. polynesiensis} culture extracts \cite{chinain2010growth}, while P-CTX-1, P-CTX-2 and P-CTX-3 have not. \\

The structure for type II P-CTX-3C consists of 13 ring and 57 C atoms and has been isolated first from \emph{G. toxicus} in 1993 \cite{satake1993structure}, when \emph{Gambierdiscus} was still designated as monophyletic, then in 2010 from \emph{G. polynesiensis} \cite{chinain2010growth}. There are two known congeners of P-CTX-3C which have been isolated - 49-epi-CTX-3C (or CTX-3B by Chinain et al \cite{chinain2010growth}) and M-seco-CTX-3C from \emph{Gambierdiscus} \cite{satake1993structure} and \emph{G. polynesiensis} \cite{chinain2010growth}. Two type II CTXs have been isolated from Moray eel, 2,3-dihydroxy-CTX-3C (or CTX-2A1) and 51-hydroxy-CTX-3C \cite{satake1998isolation}, and are suspected to constitute oxygenated metabolites of P-CTX-3. \\ %\cite{yasumoto2000structural} check ref as well as if P-CTX3 or P-CTX-3C 

Structurally C-CTXs were resolved to consist of 14 rings and 62 C atoms, with multiple congeners isolated from carnivorous fish - C-CTX-1, C-CTX-2, C-CTX-1127, C-CTX-1141, C-CTX-1143, C-CTX-1157, C-CTX-1159 \cite{vernoux1997isolation,lewis1998structure,pottier2003identification,pottier2002characterisation}. As yet, no C-CTXs have been isolated from any of the endemic \emph{Gambierdiscus} spp., Fraga et al have shown that \emph{G. excentricus} from the Canary Islands produce CTX and MTX like compounds as detected by N2A. The characterisation of these toxins using LC-MS is in process \cite{fraga2011gambierdiscus}. \\

The most recently discovered group of CTXs is from the Indian Ocean, where I-CTX-1, I-CTX-2, I-CTX-3 and I-CTX-4 were isolated from carnivorous fish. Their structure has not yet been established, preventing classification to a sub type \cite{hamilton2002multiple,hamilton2002isolation}. It has been elucidated that they have a higher molecular weight in comparison to P-CTXs and C-CTXs \cite{caillaud2010update,hamilton2002multiple,hamilton2002isolation}. \\ %I-CTX-1 toxic to mice via intraperitoneal injection \cite{hamilton2002isolation}. 

It has been suggested that the evident variety of CTX congeners, found even in the same ecosystem, is due to modification to the toxin structure as it passes up the food chain. This process can increase potency up to 10-fold \cite{hokama1996human,lewis2006ciguatera}. This theory is supported by a study conducted by Mak et al, examining the CTX composition of different members of a ciguateric food web \cite{mak2013pacific}.

%Indigenous population's understanding of ciguateric fish may help prevent CFP. 
%Experiemntation of P-CTX-1 injeted into rats intravenously showed rapid distribution to tissue and high bioavailibility of toxin via Neuro2a assay 
%cite{ledreux2013bioavailability}.

\begin{table}
\caption{Different known congeners of CTXs and their toxicity.}
\label{tbl:CTXTable}
\begin{tabular}{  | p{2cm} | p{1.5cm} | p{2.5cm} | p{4cm} | p{4cm} |}
\hline
\textbf{Toxin} & \textbf{Type} & \textbf{Molecular Ion [M +H]$^{+}$} & \textbf{Source} & \textbf{Toxicity (LD50, i.p. mice)} \\
\hline
P-CTX-1, P-CTX-1B & I & 1111.6 & Moray eel (\emph{Gymnothorax javanicus}) \cite{murata1990structures,lewis1991purification} & P-CTX-1 0.35 $\mu$g/kg \cite{murata1990structures}; P-CTX-1B 0.25$\mu$g/kg \cite{lewis1991purification} \\
\hline
P-CTX-2 & I & 1095.5 \cite{lewis1991purification} & Moray eel (\emph{Gymnothorax javanicus}) \cite{lewis1991purification} & 2.3 $\mu$g/kg \cite{lewis1991purification} \\
\hline
 P-CTX-3 & I & 1095.5 \cite{lewis1991purification} & Moray eel (\emph{Gymnothorax javanicus}) \cite{lewis1991purification} & 0.9 $\mu$g/kg \cite{lewis1991purification} \\
 \hline
 P-CTX-4A & I & 1061.6 \cite{yasumoto2000structural} & \emph{Gambierdiscus} sp. \cite{yasumoto2000structural}; \emph{G. polynesiensis} \cite{chinain2010growth} & 12$\mu$g/kg \cite{chinain2010growth} \\
 \hline
 P-CTX-4B & I & 1061.6 \cite{yasumoto2000structural} & \emph{Gambierdiscus} sp. \cite{yasumoto2000structural}; \emph{G. polynesiensis} \cite{chinain2010growth} & 20$\mu$g/kg \cite{chinain2010growth}\\
 \hline
 P-CTX-3C & II & 1023.6 \cite{satake1993structure} &  \emph{Gambierdiscus} sp. \cite{satake1993structure}; \emph{G. polynesiensis} \cite{chinain2010growth} & 2.5$\mu$g/kg \cite{chinain2010growth}\\
 \hline
 P-49-epi-CTX-3C & II & 1023.6 \cite{chinain2010growth} & \emph{Gambierdiscus} sp. \cite{satake1993structure}; \emph{G. polynesiensis} \cite{chinain2010growth} & 8$\mu$g/kg\cite{chinain2010growth}\\
 \hline
 P-M-seco-CTX-3C & II & 1041.6 \cite{chinain2010growth} &\emph{Gambierdiscus} sp. \cite{satake1993structure}; \emph{G. polynesiensis} \cite{chinain2010growth} & 10$\mu$g/kg \cite{chinain2010growth}\\
 \hline
 C-CTX-1 & III & 1141.6 \cite{vernoux1997isolation,pottier2002characterisation} & Horse-eye jack (\emph{Caranx latus}) \cite{vernoux1997isolation,pottier2002characterisation} & 3.6$\mu$g/kg \cite{vernoux1997isolation}\\
 \hline
 C-CTX-2 & III & 1141.6 \cite{vernoux1997isolation,pottier2002characterisation}& Horse-eye jack (\emph{Caranx latus}) \cite{vernoux1997isolation,pottier2002characterisation} & Toxic \cite{vernoux1997isolation}\\
 \hline
 I-CTX-1 & N/A & 1141.6 \cite{hamilton2002isolation}& Red bass (\emph{Lutjanus bohar}); Red emperor (\emph{Lutjanus sebae}) \cite{hamilton2002isolation} & Toxic \cite{hamilton2002isolation} \\
 \hline
\end{tabular}
\end{table}
\FloatBarrier

\subsection{MTX}

MTXs are the largest known proteinacious natural product \cite{yokoyama1988some,murata1993structure} and is a highly toxic Ca$^{2+}$ channel activator, with a LD$_{50}$ 0.05 $\mu$g/kg. There are only a couple of known bacterial, proteinacious toxins with higher potency \cite{yokoyama1988some,murata1993structure}. However the Ca$^{2+}$ activation is a secondary process of MTX activity, the primary mode of action in mammalian cells  is still unclear \cite{van2000diversity}. MTX is a water soluble, polyether ladder compound. It was first isolated in 1976 from the gut of a form herbivorous fish species with the common name 'Maito' \cite{yasumoto1976toxicity}. Their complex structure and stereochemistry was elucidated in the 1990s using stereoscopic studies and partial synthesis \cite{murata1993structure,murata1994structure,satake1995structural,nonomura1996complete,zheng1996complete}. \\

There are three known analogues of MTX (table ~\ref{tbl:MTXTable}) - designated MTX-1, MTX-2 and MTX-3, isolated from a \emph{Gambierdiscus} sp. in Queensland, Australia \cite{holmes1994purification}. MTX-1 and MTX-2 are structurally larger than the third. \\ %who isolated it first? yokoyama? 

%Biophysical characteristics of pacific MTXs different to caribbean MTX \cite{lu2013caribbean}. Carribean MTX not been charactrised, so whether the difference in action is structural is speculative. %check lit

%MTX easier to detect and quantify than CTX due to sulphate esters using liquid chromatography-electronspray ionisation-mass spectrometry (T. Harwood pers. com.). Solvolysis (desulphonation) reduces toxicity 100-fold or more \cite{murata1991effect}. 

\begin{table}
\caption{Different known congeners of MTXs and their toxicity.}
\label{tbl:MTXTable}
\begin{tabular}{ |  p{2cm} | p{2cm} | p{3cm} | p{3cm} | p{4cm} | }
\hline
\textbf{Origin} & \textbf{Toxin} & \textbf{Molecular Ion [M +H]$^{+}$} & \textbf{Source} & \textbf{Toxicity (LD50, i.p. mice)} \\
\hline
 Pacific & MTX-1 & 3422 \cite{holmes1994purification,murata1993structure} & \emph{Gambierdiscus} sp. \cite{holmes1994purification} & 0.05$\mu$g/kg \cite{murata1993structure}\\
 \hline
 Pacific & MTX-2 & 3298 \cite{holmes1994purification} & \emph{Gambierdiscus} sp. \cite{holmes1994purification} & 0.08$\mu$g/kg \cite{holmes1994purification}\\
 \hline
 Pacific & MTX-3 & 1060   \cite{holmes1994purification} & \emph{Gambierdiscus} sp. \cite{holmes1994purification}; \emph{G. australes}, \emph{G. pacificus} \cite{rhodes2014production} &  0.08$\mu$g/kg for \emph{G. australes}  \cite{rhodes2014production} \\
 \hline
\end{tabular}
\end{table}
\FloatBarrier
%can i use and cite gurjeets chemdraws?
\subsection{Other}
Further toxins reportedly synthesised by \emph{Gambierdiscus} spp. are are gambieric acid, gambierol and gambieroxide \cite{watanabe2013gambieroxide,satake1993gambierol,nagai1992gambieric}.
Gambierol is less toxic than CTX with 80 $\mu$g/kg i.p. and 150 $\mu$g/kg orally in MBA, but fast acting with lethality setting in within 3 hrs\cite{ito2003pathological}. Gambierol could play a potential role in CFP, which to date has not been investigated. Though less toxic than CTX, it is still a highly toxic substance which could bioaccumulate in the food chain \cite{rhodes2014production}.\\
Gambieric acids A, B, C and D showed potent antifungal properties but no toxicity via MBA at 1000 $\mu$g/kg i.p. and hence are not of concern as a health hazard, including CFP \cite{rhodes2014production,nagai1992gambieric}.
Recently gambieroxide was isolated from \emph{G. toxicus} by Watanabe et al \cite{watanabe2013gambieroxide}. Gambieroxide's stereostructure has been elucidated by NMR and LC-MS/MS, concluding that it is structurally similar to yessotoxin (YTX) which is a lipophillic toxin found in filter feeding shellfish  \cite{tubaro2010yessotoxins}. Known producers of YTX are \emph{Protoceratium reticulatum}, \emph{Lingulodinium polyedrum} and \emph{Gonyaulax spinifera} \cite{tubaro2010yessotoxins} which are not  phylogenetically closely related to \emph{Gambierdiscus}.  Gambieroxide needs to be further investigated for structure, toxicity and bioaccumulative potential for assessment as a health threat, including CFP. \\

\section{Toxicity of different \emph{Gambierdiscus} spp.}
\emph{Gambierdiscus} spp. produce CTXs and MTXs \cite{murata1990structures,holmes1991strain,satake1993structure,holmes1994purification,satake1996isolation}, however, some wild and culturable strains have been recorded to not produce either or both toxins in measurable quantities \cite{gillespie1985significance,holmes1990toxicity}. Hence toxin production varies between species. Most studies were conducted in the 16 year time period pre-dating the discovery of \emph{G. belizeanus} in 1995 \cite{faust1995observation}, and therefore the strain used was refered to as \emph{Gambierdiscus toxicus}, although it may have been any of the now known species. Table ~\ref{tbl:GeoTable} shows known toxicity of each \emph{Gambierdiscus} species as detected via various essays and whether they have been characterised by LC-MS. \\

The Caribbean species \emph{G. excentricus} was tested with a neuroblastoma cytotoxicity assay (N2A) which indicated both MTX and CTX toxicity \cite{fraga2011gambierdiscus}. However the toxin profile needs to be extrapolated with LC-MS.
Chinain et al identified the toxins produced by two strains of \emph{G. polynesiensis} with LC-MS. The major toxin produced was P-CTX-3C, other type II (M-seco-CTX-3C, 49-epi-CTX-3C) and type I (CTX-4A, CTX-4B) toxins were detected. The different strains produced the same toxins, but at different proportions \cite{chinain2010growth}. In an earlier study, Chinain et al detected toxicity of the water soluble fraction of \emph{G. polynesiensis} in a mouse bioassay (MBA), implicating MTX activity \cite{chinain1999morphology}. Rhodes et al produced evidence to the contrary, as they found no LC-MS measurable MTX in a strain of \emph{G. polynesiensis} from the Cook Islands \cite{rhodes2014production}.
\emph{G. australes} extracts isolated by Rhodes et al indicated CTX and MTX presence by displaying toxic activity in both lipid and water soluble fractions when tested with MBA. However, LC-MS/MS analysis did not detect any CTXs. A follow up study on the same strain with LC-MS also found no CTXs, but potent MTX-3 like compounds were detected \cite{rhodes2014production,rhodes2010toxic}. In contrast, three different \emph{G. australes} strains isolated from French Polynesia, tested with a receptor binding assay (RBA), produced P-CTX-3C like compounds \cite{chinain2010growth}.\\

Holland et al conducted a large scale two year toxicity screening of 56 strains from 6 species of \emph{G. belizeanus, G. caribaeus, G. carolinianus, G. carpenteri, G. rutzleri} and  \emph{Gambierdiscus} ribotype 2, using human erythrocyte lysis assay (HELA) \cite{holland2013differences}. Haemolytic activity is a well established property of MTX \cite{igarashi1999mechanisms}, which is the implication from Holland's study. It showed that the haemolytic activity varied marginally between strains but stayed consistent for each strain over the two year sampling period, indicating that MTX production, and congener suite, varies between the species but remains consistent over time within a strain \cite{holland2013differences}.
In contrast, a recent study of CTX-like toxicity using mouse neuroblastoma assay (MNA) of the new \emph{Gambierdiscus} ribotypes 4, 5 and 6 found that the isolates from a ciguateric site were up to 4\- fold more toxic compared to isolates from a geographically closely related non-ciguateric reference site \cite{xu2014distribution}. This observed difference in MTX and CTX production staying constant and varying respectively, could be due to CTX production increasing in a ciguateric setting, or something other than temperature variation changes MTX production. The findings of both of these studies need to be verified with LC-MS. \\

These case studies show that while bioassays are important to indicate toxicity, LC-MS studies of every species need to be conducted to define their toxin profiles.

\section{Dinoflagellate genetics and transcriptomics}
EST library?

\section{Polyketide synthase genes in Dinoflagellates}

\section{Phylogeny of \emph{Gonyaulacales}}

\section{Aims of the study}
The central aims of this study are:
\begin{enumerate}
\item Elucidate the diversity of \emph{Gambierdiscus} in Heron Island as well as Cook Islands on a genetic basis
\item Shed light on the evolutionary relationship of \emph{Gonyaulacales} with a focus on \emph{Gambierdiscus}
\item Investigate and PKS genes potential involvement in toxin production in \emph{Gambierdiscus}
\item Understand the impact of ecological factors on differential expression of toxin related genes in \emph{Gambierdiscus}
\item Develop a diagnostic tool for assessing potential toxicity of \emph{Gambierdiscus} blooms
\end{enumerate}

\section{Results to date}

\subsection{Characterisation of \emph{Gambierdiscus} from Heron Island and Cook Islands}

\subsubsection{Sample collection and culturing}
Environmental samples were collected from Heron Island on the xx and from the Cook Islands on the xxx.
The isolates were separated by single cell isolation and grown as clonal cultures in F/10 medium. The cultures were maintained in F/10 medium at 26 degrees celcius and xxx light.

\subsubsection{Genomic DNA extraction}
Genomic DNA was extracted using the CTAB method.

\subsubsection{PCR amplification and sequencing}
Extracted DNA was used as a template to amplify rRDNA sequences in 25$\mu$l mixture in PCR tubes. The mixture contains 0.6 $\mu$M forward and reverse primer, 0.4 $\mu$M BSA, 2 - 20 ng DNA, 12.5 $\mu$l 2xEconoTaq () and 7.5 $\mu$l MilliQ water.\\
The PCR cycling comprised of an initial 10 min step at 94
The LSU D1-D3  and D8-D10 regions as well as the SSU region were amplified with the D1R-F-, FD8-RB and 18ScomF1-18ScomR1 primer sets respectively (table ~\ref{tbl:PrimerTable}).\\
Sequencing was conducted by Macrogen with Sanger sequencing.
\FloatBarrier
\begin{table}
\caption{List of primers used for phylogenetic elucidation for Heron Island and Cook Island \emph{Gambierdiscus}.}
\label{tbl:PrimerTable}
\begin{tabular}{  | p{2cm} | p{7.5cm} | p{2.5cm} | p{2cm} | }
\hline
\textbf{Name} & \textbf{Sequence} & \textbf{Purpose} & \textbf{Reference} \\
\hline
  \multicolumn{4}{| c |}{\textbf{LSU D1-D3 region}}\\
    \hline
  D1R-F   &  & Amplification \& sequencing &  \\
  D3B &  & Amplification \& sequencing &  \\
\hline
  \multicolumn{4}{| c |}{\textbf{LSU D8-D10 region}}\\
    \hline
   FD8   &  & Amplification \& sequencing & \cite{chinain1999morphology} \\
   \hline
 GLD8\_421F   &  & Sequencing & \cite{nishimura2013genetic} \\
 \hline
 GLD8\_677R   &  & Sequencing & \cite{nishimura2013genetic} \\
 \hline
   RB   &  & Amplification \& sequencing &\cite{chinain1999morphology}  \\
    \hline
  \multicolumn{4}{| c |}{\textbf{SSU region}}\\
    \hline
 18ScomF1 & GCTTGTCTCAAAGATTAAGCCATGC & Amplification \& sequencing & \\
 \hline
 18ScomR1  & CACCTACGGAAACCTTGTTACGAC & Amplification \& sequencing & \\
 \hline
 Dino18SF2  & ATTAATAGGGATAGTTGGGGGC & Amplification \& sequencing & \\
 \hline
 Dino18SR1    & GAGCCAGATRCDCACCCA & Amplification \& sequencing & \\ 
 \hline
G10'F    & TGGAGGGCAAGTCTGGTG & Sequencing & \cite{nishimura2013genetic} \\
\hline
G18'R    & GCATCACAGACCTGTTATTG & Sequencing &  \\
 \hline
\end{tabular}
\end{table}
\FloatBarrier

\subsubsection{Sequence alignment and phylogenetic analysis}
The program used to analyse the sequences and their phylogenetic relationship is Geneious 8.2.1.
The rRNA sequences were aligned using MUSCLE against publicly available sequences from the NCBI database with a maximum of 8 iterations. The alignment was manually screened and adjusted. The 5' and 3' ends were truncated to render all sequences of the same length leaving D1-D3, D10-D8 and SSU at xx bp, 791 bp and 1699 bp respectively. Outliers for LSU and SSU alignments were \emph{Alexandrium tamarense} AY831407.1 and \emph{Prorocentrum lima} AB189780.1 respectively.
PHYML was used for Maximum Likelyhood (ML) tree generation with 1000 bootstraps in the general-time-reversal model.

\subsubsection{Morphology}
Scanning electron microscopy (SEM) was conducted by Mona Hoppenrath at the xxx institute, Germany. The plate structure of isolates HG xx was elucidated from the images as per xx.\\
--- include light microscopy?

\subsubsection{Toxin profile analysis}
Toxin profile analysis was conducted by Tim Harwood at the Cawthorne Institute, New Zealand, via LC-MS.

MBA 

\subsection{Phylogeny of \emph{Gonyaulacales}}

\subsubsection{RNA extraction and analysis}
The RNA was extracted from HG4, HG6 and HG7 using the xx method.

\subsubsection{Transcriptomic databases used}
Microbial marine eukaryote transcriptome project (MMETP)

\subsubsection{Selection of gene candidates for concatenated phylogeny}
include which species used to represent all branches of Gonyaulacales. several representatives per genus?

\subsubsection{Concatenated phylogeny}

\subsection{Search for PKS genes in \emph{Gambierdiscus} transcriptomes}

\subsection{PKS expression in \emph{Gambierdiscus} strains}
\begin{enumerate}
\item culturing in presence of copeopods - elicit defence mechanism in form of toxins?
\item culture with different media - some media said to be better for toxin production..?
\end{enumerate}

\section{Results}

\subsection{Diversity of \emph{Gambierdiscus} at Heron Island and Cook Islands}

\subsubsection{Phylogenetics}

\subsubsection{Morphological analysis}

\subsubsection{Toxin analysis}

\section{Future work and potential obstacles}

\subsection{Phylogeny of \emph{Gonyaulacales}}

\subsection{Search for PKS genes in \emph{Gambierdiscus} transcriptomes}

\subsection{PKS expression in \emph{Gambierdiscus} strains}

\subsection{Proposed project timeline}

\newpage
\bibliographystyle{plain}
\bibliography{review_ref.bib}


\end{document}