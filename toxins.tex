\documentclass[12pt]{article}
\usepackage[hcentering,bindingoffset=20mm]{geometry}
\usepackage{placeins}
\usepackage[numbib]{tocbibind}
\usepackage{rotating}
\usepackage[square,sort,comma,numbers]{natbib}
\usepackage{graphicx}
\usepackage{tabularx}
\linespread{1.3}
%\fontsize{8cm}{1.3em}\selectfont

\usepackage{longtable}
\usepackage{lscape}

\addtolength{\textwidth}{2cm}
\addtolength{\hoffset}{-1cm}


\addtolength{\textheight}{2cm}
\addtolength{\voffset}{-1cm}
\setlength{\parindent}{0pt}

\title{\textbf{The known aspects of \emph{Gambierdiscus} spp. leading to ciguatera fish poisoning}}
%\author{Anna Liza Kretzschmar, Gurjeet Kohli, Hazel Farrel, Shauna Murray}
\date{}
%\usepackage{cite}

\begin{document}

\section{Toxins produced by \emph{Gambierdiscus}}
%table 1.3 - amend with new info
% toxin mod in the food chain - or seperate section?
%\emph{G. toxicus} has been shown to produce a range of toxins other than ciguatoxins \cite{holmes1994purification,murata1993structure} Furthermore, other species within this genus produce ciguatoxin like substances, such as \emph{G. polynesiensis}, \emph{G. australes} and \emph{G. pacificus} \cite{roeder2010characteristic}. 

%trediculous symptomology detailed in sims1987theoretical

Ciguateric seafood is characteristically impossible to discern during culinary preparation as it does not exhibit any obviously recognisable differences to non toxic specimens and is heat resistant \cite{withers1982ciguatera}. Hence toxin presence is usually only diagnosed retrospectively from CFP symptoms in conjunction with recent seafood consumption \cite{sims1987theoretical} as samples of the food consumed are rarely obtained. Due to diagnosis depending on the deductive capability of the medical professional, CFP is predicted to be widely under reported with as little of 20 \% of cases being reported  in Queensland, Australia, alone \cite{lewis2006ciguatera}. The disease can manifest as over 180 symptoms, with varying levels of severity, and differ even between individuals at the same time and source of exposure \cite{sims1987theoretical}. The effects of the disease appear to be cumulative as severity increases with previous exposure \cite{emerson1983preliminary}. Commonly a couple hours after consumption, the initial symptoms are gastrointestinal which can be followed by cardiovascular and neurological manifestations \cite{sims1987theoretical}. They can persist between weeks to years, and recurrence is a common phenomenon induced by a variety of factors such as alcohol, fish or meat consumption \cite{lewis2006ciguatera} as well as sexual activity \cite{lange1992travel}. \\
Symptoms of CFP can vary between geographical locations \cite{molgo2000ciguatera,dickey2010ciguatera}, which likely contributes to the inefficiency of diagnosis. This could be de to the structural difference between CTXs, indeed and or MTXs. Hence accurate determination of CTX and MTX congeners is vital to understanding toxicology, for linked risks and symptoms symptoms of CFP - which is an avenue of investigation that urgently needs attention.\\

In 1979 the great breakthrough in understanding CFP was that an algal species, \emph{G. toxicus}, produces the toxins that cause the disease and that the toxins bioaccumulate up the food chain \citep{adachi1979thecal}. With the advent of further species discovery from 1995 onwards \cite{faust1995observation}, the inconsistency in toxin measurements between isolates tested started to make sense - they are different species. The current challenge is to characterise the species specific toxin profiles, employing a consistent and conclusive methodology. Bioassays provide an excellent indicator to a species' toxicity, however it does not suffice to elucidate a detailed toxin profile - LC-MS analysis is required. \emph{Gambierdiscus} spp. produce more than just CTXs - notably MTXs are also commonly registered \cite{holmes1994purification,murata1993structure}. It has, until recently, been assumed in the literature that CTXs are the sole culprit in CFP as MTXs are water soluble. CTXs have been isolated form the Pacific, Caribbean and the Indian ocean. The  structure varies between the Pacific and the Caribbean congeners, with the structures of the Indian ocean analogues still to be determined. Symptomatology of CFP varies between the three regions, however if the variance is dependent on  the structural difference, or for example due to different precursor modification during biomagnification, is yet to be determined \cite{lewis2006ciguatera}. Understanding the molecular evolution and ecology of the toxins, coupled with which species produce them and under what circumstances, are points that are essential to dealing with harmful algal blooms and their impact on humans and the environment. Whether and which changing environmental cues play a role is imperative for predictive purposes with the advent of global warming \cite{llewellyn2010revisiting}.  \\

\subsection{CTX}
Toxins in the CTX family consist of lipid soluble polyether ladders which act as sodium channel activators \cite{dechraoui1999ciguatoxins}. They are orally effective in humans at the pM and mM concentration range \cite{molgo2000ciguatera}, causing an influx of Na$^{+}$ ions and hence spontaneous action potentials in the cell, especially active on voltage sensitive channels along the nodes of Ranvier \cite{sims1987theoretical,mattei1999neurotoxins,lewis1992action,molgo2000ciguatera}. \\	   
Currently CTXs are prefixed by their origin, where members from the Pacific, Caribbean and Indian ocean are designated P-CTX, C-CTX and I-CTX respectively. % don't know C-PTX subdivision ref... for P-CTX it's \cite{} I-CYX is\cite}
The congeners are further subdivided on a structural basis into type I and type II within P-CTXs, followed by all C-CTXs as type III - the molecular structure of I-CTXs have not been elucidated to such an extent as to assign a type \cite{legrand1997two,hamilton2002multiple,hamilton2002isolation}, An overview of the designations can be found in table x.  \\
Type I P-CTXs consist of 13 rings with 60 C atoms \cite{murata1990structures,lewis1991purification,lewis1993origin}, including the first completely defined CTX which was isolated from moray eels and designated P-CTX-1B \cite{murata1990structures} or also described as CTX-1 \cite{lewis1991purification}. P-CTX-2 and P-CTX-3 were isolated from the same extracts but exhibited an alternate structure and toxicity in mice \cite{lewis1991purification}. It has been suggested by two research groups that P-CTX-1, P-CTX-2 and P-CTX-3 may be derivatives from the dinoflagellate precursors CTX-4A and CTX-4B (also GTX-4B by Murata et al \cite{murata1990structures}) \cite{lewis1993origin,yasumoto2000structural}. So far, P-CTX-4A and P-CTX-4B have been isolated from \emph{Gambierdiscus polynesiensis} culture extracts \cite{chinain2010growth}, while P-CTX-1, P-CTX-2 and P-CTX-3 have not. \\
The structure for type II P-CTX-3C consists of 13 ring and 57 C atoms and has been isolated first from \emph{Gambierdiscus} culture \cite{satake1993structure}, then from \emph{G. polynesiensis} \cite{chinain2010growth}. There are two known congeners of P-CTX-3C which have been isolated - 49-epi-CTX-3C (or CTX-3B by Chinain et al \cite{chinain2010growth}) and M-seco-CTX-3C from \emph{Gambierdiscus} \cite{satake1993structure} and \emph{G. polynesiensis} \cite{chinain2010growth}. Two type II CTXs have been isolated from Moray eel, 2,3-dihydroxy-CTX-3C (or CTX-2A1) and 51-hydroxy-CTX-3C \cite{satake1998isolation}, and are suspected to constitute oxygenated metabolites of P-CTX-3. \\ %\cite{yasumoto2000structural} check ref as well as if P-CTX3 or P-CTX-3C 
Structurally C-CTXs were resolved to consist of 14 rings and 62 C atoms, with multiple congeners isolated from carnivorous fish - C-CTX-1, C-CTX-2, C-CTX-1127, C-CTX-1141, C-CTX-1143, C-CTX-1157, C-CTX-1159 \cite{vernoux1997isolation,lewis1998structure,pottier2003identification,pottier2002characterisation}. Up to date, no C-CTXs have been isolated from any of the endemic \emph{Gambierdiscus} spp., however Fraga et al have shown that \emph{G. excentricus} from the Canary Islands produce CTX and MTX like compounds - the characterisation of these toxins is in process \cite{fraga2011gambierdiscus}. \\
The most recently discovered group CTXs is from the Indian ocean, where I-CTX-1, I-CTX-2, I-CTX-3 and I-CTX-4 were isolated from carnivorous fish. Their structure has not yet been elucidated, preventing classification to a sub type \cite{hamilton2002multiple,hamilton2002isolation}. It has been elucidated that they have a higher molecular weight in comparison to P-CTXs and C-CTXs \cite{caillaud2010update,hamilton2002multiple,hamilton2002isolation}. \\ %I-CTX-1 toxic to mice via intraperitoneal injection \cite{hamilton2002isolation}. 

It has been suggested that the evident variety of CTX congeners, found even in the same ecosystem, is due to modification to the toxin structure as it passes up the food chain which can increase potency up to 10-fold \cite{hokama1996human,lewis2006ciguatera}. This theory is backed by a study conducted by Mak et al, examining the CTX composition of different members of a ciguateric food web \cite{mak2013pacific}.

%Indigenous population's understanding of ciguateric fish may help prevent CFP. 
%Experiemntation of P-CTX-1 injeted into rats intravenously showed rapid distribution to tissue and high bioavailibility of toxin via Neuro2a assay 
%cite{ledreux2013bioavailability}.

\begin{sidewaystable}[!htbp]
\caption{Different known congeners of CTXs and their toxicity.}
\begin{tabular}{ |  p{4cm} | p{1.3cm} | p{2.7cm} | p{6cm} | p{6cm} | }
\hline
\textbf{Toxin} & \textbf{Type} & \textbf{Molecular Ion [M +H]$^{+}$} & \textbf{Source} & \textbf{Toxicity (LD50, i.p. mice)} \\
\hline
P-CTX-1, P=CTX-1B & I & 1111.6 & Moray eel (\emph{Gymnothorax javanicus}) \cite{murata1990structures,lewis1991purification} & P-CTX-1 0.35 $\mu$g/kg \cite{murata1990structures}; P-CTX-1B 0.25$\mu$g/kg \cite{lewis1991purification} \\
\hline
P-CTX-2 & I & 1095.5 \cite{lewis1991purification} & Moray eel (\emph{Gymnothorax javanicus}) \cite{lewis1991purification} & 2.3 $\mu$g/kg \cite{lewis1991purification} \\
\hline
 P-CTX-3 & I & 1095.5 \cite{lewis1991purification} & Moray eel (\emph{Gymnothorax javanicus}) \cite{lewis1991purification} & 0.9 $\mu$g/kg \cite{lewis1991purification} \\
\hline
 P-CTX-4A & I & 1061.6 \cite{yasumoto2000structural} & \emph{Gambierdiscus} sp. \cite{yasumoto2000structural}; \emph{G. polynesiensis} \cite{chinain2010growth} & 12$\mu$g/kg \cite{chinain2010growth} \\
\hline
 P-CTX-4B & I & 1061.6 \cite{yasumoto2000structural} & \emph{Gambierdiscus} sp. \cite{yasumoto2000structural}; \emph{G. polynesiensis} \cite{chinain2010growth} & 20$\mu$g/kg \cite{chinain2010growth}\\
\hline
 P-CTX-3C & II & 1023.6 \cite{satake1993structure} &  \emph{Gambierdiscus} sp. \cite{satake1993structure}; \emph{G. polynesiensis} \cite{chinain2010growth} & 2.5$\mu$g/kg \cite{chinain2010growth}\\
\hline
 P-49-epi-CTX-3C & II & 1023.6 \cite{chinain2010growth} & \emph{Gambierdiscus} sp. \cite{satake1993structure}; \emph{G. polynesiensis} \cite{chinain2010growth} & 8$\mu$g/kg\cite{chinain2010growth}\\
\hline
 P-M-seco-CTX-3C & II & 1041.6 \cite{chinain2010growth} &\emph{Gambierdiscus} sp. \cite{satake1993structure}; \emph{G. polynesiensis} \cite{chinain2010growth} & 10$\mu$g/kg \cite{chinain2010growth}\\
\hline
 C-CTX-1 & III & 1141.6 \cite{vernoux1997isolation,pottier2002characterisation} & Horse-eye jack (\emph{Caranx latus}) \cite{vernoux1997isolation,pottier2002characterisation} & 3.6$\mu$g/kg \cite{vernoux1997isolation}\\
\hline
 C-CTX-2 & III & 1141.6 \cite{vernoux1997isolation,pottier2002characterisation}& Horse-eye jack (\emph{Caranx latus}) \cite{vernoux1997isolation,pottier2002characterisation} & Toxic \cite{vernoux1997isolation}\\
\hline
 I-CTX-1 & N/A & 1141.6 \cite{hamilton2002isolation}& Red bass (\emph{Lutjanus bohar}); Red emperor (\emph{Lutjanus sebae}) \cite{hamilton2002isolation} & Toxic \cite{hamilton2002isolation} \\
\hline
\end{tabular}
\end{sidewaystable}
\FloatBarrier

\subsection{MTX}

MTXs are the largest known proteinacious natural product \cite{yokoyama1988some,murata1993structure} and are highly toxic Ca$^{2+}$ channel activators with a LD$_{50}$ 0.05 $\mu$g/kg - there are only a couple of bacterial, proteinacious toxins with higher potency known \cite{yokoyama1988some,murata1993structure}. However the Ca$^{2+}$ activation is a secondary process of MTX activity, the primary mode of action in mammalian cells  is still unclear \cite{van2000diversity}. They are water soluble, polyether ladder compound first isolated form herbivorous Acanthurids fish gut in 1976 \cite{yasumoto1976toxicity}. Their complex structure and stereochemistry was elucidated in the 90s using stereoscopic studies and partial synthesis \cite{murata1993structure,murata1994structure,satake1995structural,nonomura1996complete,zheng1996complete}. \\
There are three isolated MTXs - designated MTX-1, MTX-2 and MTX-3, isolated from a \emph{Gambierdiscus} sp. in Queensland, Australia \cite{holmes1994purification}. The first two are structurally larger than the third and it is suspected, though unproven, that MTX-1 is synonymous with the MTX first isolated. \\ %who isolated it first? yokoyama? 
Initially in the 70s, MTXs were detected only in the guts of fish \cite{yasumoto1976toxicity}, so the prevalent theory that followed has been that they are irrelevant in CFP research due to the assumption that their water soluble nature would exclude them from being absorbed into flesh in concentrations relevant for bioaccumulation. This popular theory needs to be reconsidered, as earlier in the year a fish feeding study conducted by Kohli et al that demonstrated MTX were present in snapper liver, digestive organs and muscle in concentrations high enough to measure via LC-MS. In this study snapper were fed juvenile mullet, injected with a known quantity of \emph{G. australes} - a strain characterised to only produce MTX \cite{kohli2014feeding}. These findings make it imperative that the involvement of MTX in CFP, as well as the mode of action of MTX in humans, is established. 
%check detection method of yasumoto1976

%Biophysical characteristics of pacific MTXs different to caribbean MTX \cite{lu2013caribbean}. Carribean MTX not been charactrised, so whether the difference in action is structural is speculative. %check lit

%MTX easier to detect and quantify than CTX due to sulphate esters using liquid chromatography-electronspray ionisation-mass spectrometry (T. Harwood pers. com.). Solvolysis (desulphonation) reduces toxicity 100-fold or more \cite{murata1991effect}. 

\begin{table}
\caption{Different known congeners of MTXs and their toxicity.}
\begin{tabular}{ |  p{2cm} | p{2cm} | p{3cm} | p{3cm} | p{4cm} | }
\hline
\textbf{Origin} & \textbf{Toxin} & \textbf{Molecular Ion [M +H]$^{+}$} & \textbf{Source} & \textbf{Toxicity (LD50, i.p. mice)} \\
\hline
 Pacific & MTX-1 & 3422 \cite{holmes1994purification,murata1993structure} & \emph{Gambierdiscus} sp. \cite{holmes1994purification} & 0.05$\mu$g/kg \cite{murata1993structure}\\
\hline
 Pacific & MTX-2 & 3298 \cite{holmes1994purification} & \emph{Gambierdiscus} sp. \cite{holmes1994purification} & 0.08$\mu$g/kg \cite{holmes1994purification}\\
\hline
 Pacific & MTX-3 & 1060   \cite{holmes1994purification} & \emph{Gambierdiscus} sp. \cite{holmes1994purification} & Toxic \cite{holmes1994purification} \\
\hline
\end{tabular}
\end{table}
\FloatBarrier

\subsection{Other}
%insert gambieroxide
Recently, a new toxin has been isolated from \emph{G. toxicus} by Watanabe et al \cite{watanabe2013gambieroxide}. Gambieroxide's stereostructure has been elucidated by NMR and LC/MS/MS, concluding that it is structurally similar to yessotoxin (YTX) which is a lipophillic toxin found in filter feeding shellfish  \cite{tubaro2010yessotoxins}. Known producers of YTX are \emph{Protoceratium reticulatum}, \emph{Lingulodinium polyedrum} and \emph{Gonyaulax spinifera} \cite{tubaro2010yessotoxins} which are not  phylogenetically closely related to \emph{Gambierdiscus}. YTX h. In their structural paper, Watanabe et al state that the biological activity of gambieroxide will be reported at a later time point \cite{watanabe2013gambieroxide}.

\newpage
\bibliographystyle{plain}
\bibliography{review_ref.bib}

%what was Ballantine cited for?

\end{document}