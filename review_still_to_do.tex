\documentclass[12pt]{article}
\usepackage[hcentering,bindingoffset=20mm]{geometry}
\usepackage{placeins}
\usepackage[numbib]{tocbibind}
\usepackage{rotating}
\usepackage[square,sort,comma,numbers]{natbib}
\usepackage{graphicx}
\usepackage{tabularx}
\linespread{1.3}
%\fontsize{8cm}{1.3em}\selectfont

\usepackage{longtable}
\usepackage{lscape}

\addtolength{\textwidth}{2cm}
\addtolength{\hoffset}{-1cm}


\addtolength{\textheight}{2cm}
\addtolength{\voffset}{-1cm}
\setlength{\parindent}{0pt}

\title{\textbf{The known aspects of \emph{Gambierdiscus} spp. leading to ciguatera fish poisoning}}
%\author{Anna Liza Kretzschmar, Gurjeet Kohli, Shauna Murray}
\date{}
%\usepackage{cite}

\begin{document}
\maketitle
plus time line fig

\section{Morphology and phylogenetics of \emph{Gambierdiscus}}
% insert discussion about morph not being enough, eg carpenteri and yasumotoi papers from both gurjeet and shauna plus picture from shauna's book

\subsection{Gambieric acid, gambierol, gambieroxide and ciguaterin}
%insert gambieroxide
Recently, a new toxin has been isolated from \emph{G. toxicus} by Watanabe et al \cite{watanabe2013gambieroxide}. Gambieroxide's stereostructure has been elucidated by NMR and LC-MS/MS, concluding that it is structurally similar to yessotoxin (YTX) which is a lipophillic toxin found in filter feeding shellfish  \cite{tubaro2010yessotoxins}. Known producers of YTX are \emph{Protoceratium reticulatum}, \emph{Lingulodinium polyedrum} and \emph{Gonyaulax spinifera} \cite{tubaro2010yessotoxins} which are not  phylogenetically closely related to \emph{Gambierdiscus}.
- In their structural paper, Watanabe et al state that the biological activity of gambieroxide will be reported at a later time point \cite{watanabe2013gambieroxide}.
%g. yas nerw toxin? check murrauy 14 g yas great barrier reef
% as irt could be caused by CTXs, MTX or one of the reported unidentified metabolites produced by \emph{Gambierdiscus} \cite{endean1993variation,vernoux1997isolation}. 

%ledreux 14, otero12?, mak 13, pawliez 13, robertson 14 rou???13 yogi 14, hung 05, Ledreux 13, Xu 14, Rhodes 14, is habermehl94 incl with shark? Boada? is yasumoto's turban snail included? Wong 05 study outbrwak hong kong?
\begin{table}
\caption{new entries for potential vectors in food web table}
\begin{tabular}{ | p{2cm} | p{3cm} | p{4.5cm} | p{2cm} | p{3cm} | }
\hline
\textbf{Group} & \textbf{Latin name} (Common name) & \textbf{Source} & \textbf{CTX (if detected)} & \textbf{Methods of detection} \\
\hline
  &  \emph{} &  &  & \\
  & \emph{}  &  &  & \\
  & \emph{} &  &  & \\
  &  \emph{} &  &  & \\
  & \emph{}  &  &  & \\
  & \emph{} &  &  & \\
  &  \emph{} &  &  & \\
  & \emph{}  &  &  & \\
  & \emph{} &  &  & \\
  &  \emph{} &  &  & \\
  & \emph{}  &  &  & \\
  & \emph{} &  &  & \\
  &  \emph{} &  &  & \\
  & \emph{}  &  &  & \\
  & \emph{} &  &  & \\
  &  \emph{} &  &  & \\
  & \emph{}  &  &  & \\
  & \emph{} &  &  & \\
\end{tabular}
\end{table}


\begin{table}
\caption{CTXs and congeners detected by various assays in herbivorous fish and other animals.}
\label{tbl:HerbTable}
\begin{tabular}{| p{2cm} | p{3cm} | p{4.5cm} | p{2cm} | p{3cm} | }
\hline
\textbf{Group} & \textbf{Latin name} (Common name) & \textbf{Source} & \textbf{CTX (if detected)} & \textbf{Methods of detection} \\
\hline
 Giant clam &  \emph{Tridacna} sp. (Giant clam) & New Caledonia, French Polynesia \cite{laurent2012ciguatera} & CTX +\cite{laurent2012ciguatera} & MBA \cite{laurent2012ciguatera}; N2A \cite{laurent2012ciguatera}; RBA \cite{laurent2012ciguatera} \\
  & \emph{Hippopus hippopus} (Giant clam)  & Republic of Vanuatu \cite{laurent2012ciguatera} & CTX +\cite{laurent2012ciguatera} & N2A \cite{laurent2012ciguatera}; RBA \cite{laurent2012ciguatera} \\
  \hline
  & \emph{} &  &  & \\
  &  \emph{} &  &  & \\
  & \emph{}  &  &  & \\
  & \emph{} &  &  & \\
  &  \emph{} &  &  & \\
  & \emph{}  &  &  & \\
  & \emph{} &  &  & \\
  &  \emph{} &  &  & \\
  & \emph{}  &  &  & \\
  & \emph{} &  &  & \\
  &  \emph{} &  &  & \\
  & \emph{}  &  &  & \\
  & \emph{} &  &  & \\
  &  \emph{} &  &  & \\
  & \emph{}  &  &  & \\
  & \emph{} &  &  & \\
\end{tabular}
\end{table}
\FloatBarrier
\begin{table}
\caption{CTXs and congeners detected by various assays in omnivorous fish and other animals.}
\label{tbl:OmniTable}
\begin{tabular}{ | p{2cm} | p{3cm} | p{4.5cm} | p{2cm} | p{3cm} | }
\hline
\textbf{Group} & \textbf{Latin name} (Common name) & \textbf{Source} & \textbf{CTX (if detected)} & \textbf{Methods of detection} \\
\hline 
  &  \emph{} &  &  & \\
  & \emph{}  &  &  & \\
  & \emph{} &  &  & \\
  &  \emph{} &  &  & \\
  & \emph{}  &  &  & \\
  & \emph{} &  &  & \\
  &  \emph{} &  &  & \\
  & \emph{}  &  &  & \\
  & \emph{} &  &  & \\
  &  \emph{} &  &  & \\
  & \emph{}  &  &  & \\
  & \emph{} &  &  & \\
  &  \emph{} &  &  & \\
  & \emph{}  &  &  & \\
  & \emph{} &  &  & \\
  &  \emph{} &  &  & \\
  & \emph{}  &  &  & \\
  & \emph{} &  &  & \\
\end{tabular}
\end{table}

\FloatBarrier
\begin{longtable}{  | p{2cm} | p{3cm} | p{4.5cm}  | p{2cm} | p{3cm}  | }
\caption{CTXs and congeners detected by various assays in carnivorous fish and other animals.}\\
\label{tbl:CarnTable}
%\hline
\textbf{Group} & \textbf{Latin name} (Common name) & \textbf{Source} & \textbf{CTX detected} & \textbf{Methods of detection} \\
\hline
Amberjack &  \emph{Seriola dumerili} (Greater amberjack) & Canary Islands \cite{caillaud2012towards}; Selvagens Islands, Portugal \cite{otero2010first}; Hawaii, USA \cite{campora2008detection,hokama1977radioimmunoassay,hokama1983rapid,hokama1985rapid}; Haiti \cite{poli1997identification}; St. Barthelemy Island, Caribbean \cite{vernoux1986heterogeneity}; St. Thomas, Caribbean \cite{granade1976ciguatera} & C-CTX-1 \cite{poli1997identification};C-CTX-1B \cite{otero2010first}; P-CTX-3C and CTX analogues from Carribean and Indic waters \cite{otero2010first} & UPLC/MS \cite{otero2010first}; HPLC/MS \cite{poli1997identification}; TLC \cite{vernoux1986heterogeneity}; BSBA \cite{granade1976ciguatera}; MGBA \cite{campora2008detection,granade1976ciguatera}; MBA \cite{hokama1983rapid,hokama1985rapid,vernoux1986heterogeneity}; S-EIA \cite{hokama1985rapid}; SPIA \cite{otero2010first}; RIA \cite{campora2008detection,hokama1983rapid}; ELISA \cite{campora2008detection}; N2A \cite{caillaud2012towards,campora2008detection}; RBA \cite{} \\
  & \emph{Seriola fasciata} (Lesser amberjack) & Selvagens Islands, Portugal \cite{otero2010first}; West Africa \cite{boada2010ciguatera} & C-CTX-1 \cite{boada2010ciguatera}; C-CTX-1B \cite{otero2010first}; P-CTX-3C and CTX analogues from Carribean and Indic waters \cite{otero2010first} & LCMS/MS \cite{boada2010ciguatera}; UPLC/MS \cite{otero2010first}\\
 & \emph{Seriola rivoliana} (Almaco jack) & Canary Islands \cite{campora2010evaluating}; Hawaii, USA \cite{campora2008detection}; St. Thomas, USA \cite{granade1976ciguatera} & C-CTX-1 \cite{} & LCMS/MS \cite{}; BSBA \cite{granade1976ciguatera}; MGBA \cite{granade1976ciguatera}; ELISA \cite{campora2008detection,campora2010evaluating}; N2A \cite{campora2008detection,campora2010evaluating} \\
 \hline
Baracuda & \emph{Sphyraena barracuda} (Great barracuda) & The Bahamas \cite{o2012linking}; Cameroon, West Africa \cite{bienfang2008ciguatera}; Florida Keys, USA \cite{dechraoui2005use}; Guadeloupe, French West Indies \cite{pottier2003identification,pottier2001ciguatera}; St Barthelemy\cite{pottier2001ciguatera,vernoux1986heterogeneity}; French Polynesia \cite{bagnis1987use} & C-CTX-1 \cite{dechraoui2005use,pottier2003identification} & Cat BA \cite{bagnis1987use}; Chick BA \cite{pottier2001ciguatera}; MQBA \cite{bagnis1987use}, MBA \cite{bagnis1987use}; N2A \cite{o2012linking} \\
 & \emph{Sphyraena jello}  (Pickhandle barracuda) & Hervey Bay, Australia \cite{lewis1984ciguatoxin} & CTX +ve \cite{lewis1984ciguatoxin} & MBA \cite{lewis1984ciguatoxin}; TLC \cite{lewis1984ciguatoxin} \\
 & \emph{Sphyraena} sp. (Barracuda) & California, USA \cite{hokama1990simplified} & CTX +ve \cite{hokama1990simplified} & N2A \cite{hokama1990simplified}; S-EIA \cite{hokama1990simplified}; SPIA \cite{hokama1990simplified} \\
 & \emph{Sphyraena} sp. (Barracuda fish eggs) & Taiwan \cite{hung2005persistent} & CTX +ve \cite{hung2005persistent} & MBA \cite{hung2005persistent}; N2A \cite{hung2005persistent} \\
 \hline
 Eel &  \emph{Gymnothorax funebris} (Green Moray) & St. Barthelemy \cite{vernoux1986heterogeneity}  & CTX +ve \cite{vernoux1986heterogeneity}  & MBA \cite{vernoux1986heterogeneity}; TLC \cite{vernoux1986heterogeneity} \\
  &  \emph{Gymnothorax javanicus} & Tuamotu Archipelago \& Tahiti, French Polynesia \cite{labrousse1996toxicological,murata1990structures,legrand1997two}; Tarawa, Republic of Kiribati \cite{lewis1997characterization}; Hawaii, USA \cite{scheuer1967ciguatoxin} & P-CTX-1 \cite{murata1990structures,lewis1991purification,lewis1997characterization}; P-CTX-2 \cite{lewis1991purification}; P-CTX-3 \cite{lewis1991purification,lewis1997characterization}; 2,3-dihydroxy-CTX-3C \cite{satake1998isolation}; 51-hydroxy-CTX-3C \cite{satake1998isolation}; P-CTX-4B \cite{murata1990structures,lewis1991purification} & DLBA \cite{labrousse1996toxicological}; HPLC/HNMR \cite{murata1990structures,lewis1991purification}; HPLC/MS \cite{lewis1997characterization,satake1998isolation}; MBA \cite{lewis1997characterization,scheuer1967ciguatoxin,satake1998isolation}; TLC \cite{scheuer1967ciguatoxin} \\
  \hline
 Emperor bream & \emph{Lethrinus olivaceus} (Longface emperor) &  Nuku Hiva, French Polynesia \cite{darius2007ciguatera} & CTX +ve \cite{darius2007ciguatera} & RBA \cite{darius2007ciguatera} \\
  & \emph{Lethrinus miniatus} (Trumpet emperor) & French Polynesia \cite{bagnis1987use} & CTX +ve \cite{bagnis1987use} & Cat MBA \cite{bagnis1987use}; MBA \cite{bagnis1987use}; MQBA \cite{bagnis1987use} \\
  &  \emph{Monotaxis grandoculis} (Big eye bream) & French Polynesia \cite{bagnis1987use} & CTX +ve \cite{bagnis1987use}  & Cat MBA \cite{bagnis1987use}; MBA \cite{bagnis1987use}; MQBA \cite{bagnis1987use} \\
  \hline
Goatfish  & \emph{Mulloidichthys auriflamma} (Goldstriped goatfish) & Hawaii, USA \cite{hokama1990simplified} & CTX +ve \cite{hokama1990simplified}  & S-EIA \cite{hokama1990simplified}; SPIA \cite{hokama1990simplified}  \\
  & \emph{Mulloidichthys martinicus} (Yellow goatfish) & St. Barthelemy \cite{vernoux1986heterogeneity}  &  CTX +ve \cite{vernoux1986heterogeneity}  &  MBA \cite{vernoux1986heterogeneity}; TLC \cite{vernoux1986heterogeneity} \\
  &  \emph{Parupeneus insularis} (Twosaddle goatfish) & Nuku Hiva, French Polynesia \cite{darius2007ciguatera} & CTX +ve \cite{darius2007ciguatera}  &  RBA \cite{darius2007ciguatera} \\
  \hline
Gastropod  & \emph{Conus} spp. (Cone snails)  & Hawaii, USA \cite{park2000microbial} & CTX +ve \cite{park2000microbial} & Ciguatect \textregistered \cite{park2000microbial} \\
\hline
 Grouper & \emph{Cephalopholis argus} (Blue spotted grouper) & Nuku Hiva, French Polynesia \cite{darius2007ciguatera}; French Polynesia \cite{}; Hawaii, USA \cite{} & CTX +ve \cite{} & Cat BA \cite{}; ELISA \cite{}; MBA \cite{};  MQBA \cite{}; N2A \cite{}; RBA \cite{} \\
  &  \emph{} &  &  & \\
  & \emph{}  &  &  & \\
  & \emph{} &  &  & \\
  &  \emph{} &  &  & \\
  & \emph{}  &  &  & \\
  & \emph{} &  &  & \\
  &  \emph{} &  &  & \\
  & \emph{}  &  &  & \\
  & \emph{} &  &  & \\
  &  \emph{} &  &  & \\
  & \emph{}  &  &  & \\
  & \emph{} &  &  & \\
  &  \emph{} &  &  & \\
  & \emph{}  &  &  & \\
  & \emph{} &  &  & \\
  &  \emph{} &  &  & \\
  & \emph{}  &  &  & \\
  & \emph{} &  &  & \\
  &  \emph{} &  &  & \\
  & \emph{}  &  &  & \\
  & \emph{} &  &  & \\
  &  \emph{} &  &  & \\
  & \emph{}  &  &  & \\
  & \emph{} &  &  & \\
  &  \emph{} &  &  & \\
  & \emph{}  &  &  & \\
  & \emph{} &  &  & \\
  &  \emph{} &  &  & \\
  & \emph{}  &  &  & \\
  & \emph{} &  &  & \\
  &  \emph{} &  &  & \\
  & \emph{}  &  &  & \\
  & \emph{} &  &  & \\
  &  \emph{} &  &  & \\
  & \emph{}  &  &  & \\
  & \emph{} &  &  & \\
  &  \emph{} &  &  & \\
  & \emph{}  &  &  & \\
  & \emph{} &  &  & \\
  &  \emph{} &  &  & \\
  & \emph{}  &  &  & \\
  & \emph{} &  &  & \\
  &  \emph{} &  &  & \\
  & \emph{}  &  &  & \\
  & \emph{} &  &  & \\
  &  \emph{} &  &  & \\
  & \emph{}  &  &  & \\
  & \emph{} &  &  & \\
  &  \emph{} &  &  & \\
  & \emph{}  &  &  & \\
  & \emph{} &  &  & \\
  &  \emph{} &  &  & \\
  & \emph{}  &  &  & \\
  & \emph{} &  &  & \\
  &  \emph{} &  &  & \\
  & \emph{}  &  &  & \\
  & \emph{} &  &  & \\
  &  \emph{} &  &  & \\
  & \emph{}  &  &  & \\
  & \emph{} &  &  & \\
  &  \emph{} &  &  & \\
  & \emph{}  &  &  & \\
  & \emph{} &  &  & \\
  &  \emph{} &  &  & \\
  & \emph{}  &  &  & \\
  & \emph{} &  &  & \\
  &  \emph{} &  &  & \\
  & \emph{}  &  &  & \\
  & \emph{} &  &  & \\
  &  \emph{} &  &  & \\
  & \emph{}  &  &  & \\
  & \emph{} &  &  & \\
  &  \emph{} &  &  & \\
  & \emph{}  &  &  & \\
  & \emph{} &  &  & \\
  &  \emph{} &  &  & \\
  & \emph{}  &  &  & \\
  & \emph{} &  &  & \\
  &  \emph{} &  &  & \\
  & \emph{}  &  &  & \\
  & \emph{} &  &  & \\
\end{longtable}

\section{Conclusion}
\emph{Gambierduiscus} is a genus of BHABs that are responsible for CFP. With the increase in frequency of CFP cases globally, as well as recorded blooms, it is essential to complement our currently fragmentary understanding of the phylogeny, distribution, abundance and species specific toxin profiles in order to monitor for potential health hazards. Aspects include elucidating the species specific toxin profiles with LC-MS, sampling extensively including at different depth and determination of the phylogenetic relationship between species as well as finding toxin markers.
Due to the  advancement of \emph{Gambierdiscus} spp. to new areas and CFP incidence frequency predicted to rise drastically, the mechanism of biomagnification of CTXs through the food chain needs to be scrutinised as our understanding of the trophic components of ciguateric food webs are still expanding.
It is essential to track the presence of CTX in lower trophic level of the food web to predict in which genera bioaccumulation is likely \cite{mak2013pacific} and investigate species other than fish as coastal toxin vectors. Furthermore it needs to be established if MTX can bioaccumulate and play a role in CFP. For this purpose MTX should be tested for along with CTX in suspected contaminated samples.
Finally for proper risk assessment relevant to monitoring by the seafood industry, the field needs to move away from i.p. injections and include long term CTX toxicity studies.


\newpage
\bibliographystyle{plain}
\bibliography{review_ref.bib}

%what was Ballantine cited for?

\end{document}