\documentclass[12pt]{article}
\usepackage[hcentering,bindingoffset=20mm]{geometry}
\usepackage{placeins}
\usepackage[numbib]{tocbibind}
\usepackage{rotating}
\usepackage[square,sort,comma,numbers]{natbib}
\usepackage{graphicx}
\usepackage{tabularx}
\linespread{1.3}
%\fontsize{8cm}{1.3em}\selectfont
\usepackage{gensymb}
\usepackage{longtable}
\usepackage{lscape}

\addtolength{\textwidth}{2cm}
\addtolength{\hoffset}{-1cm}


\addtolength{\textheight}{2cm}
\addtolength{\voffset}{-1cm}
\setlength{\parindent}{0pt}

\title{Figures}
\begin{document}
\begin{figure} 
\includegraphics[scale=.85]{oz-gamb-grey-map.png} 
\caption{(A) Sites where Gambierdiscus isolates have been reported in Australia: site 5 indicates Wapengo Lagoon, NSW; site 6 indicates Merimbula, NSW; site 7 indicates Eden, NSW; and site 8 indicates Exmouth, WA; (B) Great Barrier Reef:​ Site 1 indicates Raine Island, QLD; site 2 indicates Townsville, QLD; Site 3 indicates Heron Island, QLD; Site 4 indicated Platypus Bay, QLD; (C) Heron Island: Sampling site for this study.} 
\label{fig:OzSites}
\end{figure} 

\begin{figure} 
\includegraphics[scale=.4]{petramus-LM.png} 
\caption{Light micrographs of \emph{Gambierdiscus petramus}, strains HG4 (A); HG6 (B); HG7 (C); and HG26 (D). Scale bar equal 10 $\mu$m.​} 
\label{fig:PetLM}
\end{figure} 

\FloatBarrier 
\begin{figure} 
\includegraphics[scale=.22]{MH_epitheca.jpg} 
\caption{SEM micrographs of the epitheca of \emph{Gambierdiscus petramus}, (A, B, C, D) show all plates in full; (E, F, G) show apical pore; (H) shows apical pore of recently divided cell. Scale bar equal 20 $\mu$m unless otherwise specified.} 
\label{fig:epiSEM}
\end{figure} 
\FloatBarrier 

\FloatBarrier 
\begin{figure} 
\includegraphics[scale=.22]{MH_hypotheca.jpg} 
\caption{SEM micrographs of the hypotheca of \emph{Gambierdiscus petramus}, (A, B) show the hypotheca; (C, D, E) show the smaller hypothecal plates close to the sulcus; (F) side view of cell; (G) sulcal plates from hypothecal view. Scale bar equal 20 $\mu$m.} 
\label{fig:hypoSEM}
\end{figure} 
\FloatBarrier

\FloatBarrier 
\begin{figure} 
\includegraphics[scale=.22]{MH_ornamentation.jpg} 
\caption{SEM micrographs of \emph{Gambierdiscus petramus} showing plate ornamentation. Scale bar equal 20 $\mu$m unless otherwise specified.} 
\label{fig:ornSEM}
\end{figure} 
\FloatBarrier

\FloatBarrier 
\begin{figure} 
\includegraphics[scale=.22]{MH_sulcus-HG26-HG4.jpg} 
\caption{Supplementary:SEM micrographs of \emph{Gambierdiscus petramus},(A, B, C, D, E, F) depict the sulcal plates; (C) also shows the cingulum. Scale bar equal 20 $\mu$m unless otherwise specified.} 
\label{fig:sulcSEM}
\end{figure} 
\FloatBarrier
\FloatBarrier 
\begin{figure} 
\includegraphics[scale=.22]{MH_variability.jpg} 
\caption{Supplementary:SEM micrographs of \emph{Gambierdiscus petramus}, (A, B) show the variable teardrop like shape seen in some cels; (C, D, E) show the variability observed in the 2' plate, rectangular, two sides boardering on the 2'' plate and 2 sides boardering on the 3'' respectively. Scale bar equal 20 $\mu$m unless otherwise specified.} 
\label{fig:varSEM}
\end{figure} 
\FloatBarrier

\FloatBarrier 
\begin{figure} 
\includegraphics[scale=.65]{size_chart-grey.png} 
\caption{Visual representation of the size difference between \emph{Gambierdiscus} species (table ~\ref{fig:SizeGraph}). Error bars denote standard deviation of meassurements for \emph{G. petramus}. Data taken from publications as follows: a) Chinain et al \citep{chinain1999morphology}; b) Fraga et al \citep{fraga2014genus}; c) Faust et al \citep{faust1995observation}; d) Gomez et al \citep{gomez2015fukuyoa}; e) Litaker et al \citep{litaker2009taxonomy}; f) Nishimura et al \citep{nishimura2014morphology}; and g) Xu et al \citep{xu2014distribution}} 
\label{fig:SizeGraph}
\end{figure} 
\FloatBarrier 

\begin{figure} 
\includegraphics[scale=.5]{SSU_complex_geo_merge-f.png} 
\caption{Maximum likelihood phylogeny of \textit{Gambierdiscus} species/phylotypes of the SSU rDNA region. Nodal support are Bayesian posterior probability (pp) and bootstrap (bt) values obtained from Bayesian inference analysis and maximum likelihood analysis, respectively. Nodes with strong support (pp/bt = 1.00 / 100) are shown as thick lines. The geographic origin of the isolates is coded as follows: Black = Pacific Ocean; dark grey = Caribbean Ocean; light grey = Atlantic Ocean; white = South China Sea; white with black horizontal stripes = Indian Ocean.}
\label{fig:HGSSU} 
\end{figure} 
\FloatBarrier 
%need to include what the geo legend is

\newpage
\begin{figure} 
\includegraphics[scale=.5]{D8D10_complex_geo_merge-F.png} 
\caption{Maximum likelihood phylogeny of \textit{Gambierdiscus} species/phylotypes of the D8-D10 LSU rDNA region. Nodal support are Bayesian posterior probability (pp) and bootstrap (bt) values obtained from Bayesian inference analysis and maximum likelihood analysis, respectively. Nodes with strong support (pp/bt = 1.00 / 100) are shown as thick lines. The geographic origin of the isolates is coded as follows: Black = Pacific Ocean; dark grey = Caribbean Ocean; light grey = Atlantic Ocean; white = South China Sea; white with black horizontal stripes = Indian Ocean; white with black vertical stripes = Singapore strait; black with white stripes = Gulf of Mexico.} 
\label{fig:HGD8D10}
\end{figure} 
\FloatBarrier 
\newpage
\bibliographystyle{acm}
\bibliography{references.bib}
\end{document}
