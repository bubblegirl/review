\documentclass[12pt]{article}
 \usepackage[hcentering,bindingoffset=20mm]{geometry}
 \usepackage{placeins}
 \usepackage[numbib]{tocbibind}
 \usepackage{rotating}
\usepackage[square,sort,comma,numbers]{natbib}
 \usepackage{graphicx}
 \usepackage{tabularx}
 \linespread{1.3}
 \usepackage{gensymb}
 \usepackage{longtable}
 \usepackage{lscape}
 \usepackage{url}
 \addtolength{\textwidth}{2cm}
 \addtolength{\hoffset}{-1cm}
 
 
 \addtolength{\textheight}{2cm}
 \addtolength{\voffset}{-1cm}
 \setlength{\parindent}{0pt}
 
\title{Development of semi-quantitative PCR assays for the detection and enumeration of \emph{Gambierdiscus lapillus} and \emph{Gambierdiscus polynesiensis} (Gonyaulacales, Dinophyces) at the Great Barrier Reef, Australia.}
\author{me}
\date{}

\begin{document}
\maketitle
\paragraph{}Anna Liza Kretzschmar\\
Plant Functional Ecology and Climate Change Cluster (C3), University of Technology Sydney, Ultimo, 2007 NSW, Australia, anna.kretzschmar@uts.edu.au
\paragraph{}Arjun Verma \\
Plant Functional Ecology and Climate Change Cluster (C3), University of Technology Sydney, Ultimo, 2007 NSW, Australia
\paragraph{}Gurjeet Kohli\\ 
Plant Functional Ecology and Climate Change Cluster (C3), University of Technology Sydney, Ultimo, 2007 NSW, Australia
\paragraph{}Shauna Murray\\ 
Plant Functional Ecology and Climate Change Cluster (C3), University of Technology Sydney, Ultimo, 2007 NSW, Australia
\newpage
\section{Abstract}

\newpage
\section{Introduction}
Benthic dinoflagellates of the genus \emph{Gambierdiscus} are the causative organism of ciguatera fish poisoning (CFP), as they produce the neurotoxic ciguatoxins (CTX) which bioaccumulate. CFP 
- CFP and UNESCO, Gamb, Aust and not knowing causative agent, CTX
- LM monitoring issues - mol;ecular, global cig strategy qPCR
- Vandersea and Nishimura
qPCR is a rapid and accurate molecular tool, an alternative for species identification and enumeration with LM. Development of \emph{Gambierdiscus} spp. specific qPCR primers is part of Element 1 of the global ciguatera strategy plan \cite{globalcig}. Two major principles can be used to devise species specific qPCR methodologies: SYBR Green based assays or assays using TaqMan probes. The former relies on target specific primer design and flourometric meassurement of the chelating agent SYBR Green, which is directly proportional to the amount of PCR product present after each cycle. The latter relies on the release of a flourophore upon incoropration of a target specific probe, indexed to the PCR product present per cycle. Currently the qPCR assays available for  \emph{Gambierdiscus} spp. are from both methodologies. Vandersea et al. (2012) developed assays for \emph{G. belizeanus}, \emph{G. caribaeus}, \emph{G. carpenteri}, \emph{G. carolinianus}, \emph{Gambierdiscus} sp. ribotype 2 and \emph{G. (Fukoyoa) rutzleri} utilising SYBR Green \cite{vandersea2012development}. Nishimura et al. (2016) developed TaqMan probes for  \emph{G. australes}, \emph{G. scabrosus}, \emph{Gambierdiscus} sp. type 2, \emph{Gambierdiscus} sp. type 3 and \emph{G. (Fukoyoa)} cf. \emph{yasumotoi} \cite{nishimura2016quantitative}. This constitutes 6 out of 12 (or 13 with Kirsty's) verified species specific qPCR primer sets for described \emph{Gambierdiscus} spp. and  3 out of 6 undescribed \emph{Gambierdiscus} sp. types/ribotypes , as well as 2 out of 3 qPCR primer sets for species of the genus \emph{Fukoyoa}.
\newpage
\section{Materials and methods}
\subsection{Clonal strains and culturing conditions}
Three strains of \emph{G. lapillus} and two strains of \emph{G. polynesiensis} were isolated from Heron Island, Australia, and Rarotonga, Cook Islands, respectively (Table ~\ref{tbl:StrainTable}). The cultures were maintained F$\bullet$-10 medium at 27 $^{\circ}$C, 60mol$\bullet$-m$^{2}$ $\bullet$-s light in 12hr:12hr light to dark cycles.
\FloatBarrier
\begin{table}
\caption{List of \emph{Gambierdiscus} clonal strains used for the qPCR assay.}
\label{tbl:StrainTable}
\begin{tabular}{  | p{2cm} | p{2cm} | p{2cm}| p{3cm} | p{3cm} | p{2cm} | }
\hline
\textbf{Species}  & \textbf{Collection site} &  \textbf{Collection date} &\textbf{Latitude} & \textbf{Longitude} & \textbf{Strain name} \\
  \hline
   \emph{G. lapillus}   &Heron Island, Australia &July 2014 &23$^{\circ}$ 4420' S&151$^{\circ}$ 9140' E  & HG4 \\
   \hline
&&&&& HG6\\
 \hline
 &&&& &HG7\\
 \hline
\emph{G. polynesiensis}&Rarotonga, Cook Islands&November 2014 &21$^{\circ}$ 2486' S&159$^{\circ}$ 7286' W  & CG14 \\
 \hline
&&&&&CG15\\
    \hline
 \end{tabular}
\end{table}
\subsection{DNA extraction and species specific primer design}
Genomic DNA was extracted using the CTAB method \citep{zhou1999analysis}. Purity and concentration of the extract was measured by Nanodrop (Nanodrop2000, Thermo Scientific), the integrity was visualised on 1\% agarose gel.
Unique primer sets were designed for the small-subunit (SSU) rDNA region of  \emph{G. lapillus}, as reported by Kretzschmar et al., and \emph{G. polynesiensis}, as available in GenBank reference database. The target sequences were aligned against sequences of all other \emph{Gambierdiscus}spp. from GenBank reference database, with the MUSCLE algorithm (maximum of 8 iterations) \citep{edgar2004muscle} used through the Geneious software, version 8.1.7 \citep{kearse2012geneious}. Unique sites were determined manually (Table ~\ref{tbl:PrimerTable}). Primers were synthesised by Integrated DNA Technologies (IA, USA).
Primer sets were tested systematically for secondary product formation for all 3 strains of \emph{G. lapillus} and 2 strains of \emph{G. polynesiensis} (Table ~\ref{tbl:StrainTable}) via standard PCR in 25$\mu$l mixture in PCR tubes. The mixture contained 0.6 $\mu$M forward and reverse primer, 0.4 $\mu$M BSA, 2 - 20 ng DNA, 12.5 $\mu$l 2xEconoTaq (Lucigen) and 7.5 $\mu$l PCR grade water.
The PCR cycling comprised of an initial 10 min step at 94 $^{\circ}$C, followed by 30 cycles of denaturing at 94 $^{\circ}$C for 30 sec, annealing at 60 $^{\circ}$C for 30 sec and extension at 72 $^{\circ}$C for 1 min, finalised with 3 minutes of extension at 72 $^{\circ}$C. Products were visualised on 1\% agarose gel.
\FloatBarrier
\begin{table}
\caption{List of species specific  qPCR primer sets for SSU rDNA.}
\label{tbl:PrimerTable}
\begin{tabular}{  | p{2cm} | p{2cm} | p{2cm} | p{2cm} | p{7cm} | }
\hline
\textbf{DNA target} & \textbf{Amplicon size} & \textbf{Primer name} & \textbf{Synthesis direction of primer} & \textbf{Sequence (5'-3')}  \\
  \hline
   \emph{G. lapillus}   & &qGlapSSU2F & Forward & \\
   \hline
 & &qGlapSSU2R & Reverse & \\
 \hline
\emph{G. polynesiensis}& &qGpolySSU2F& Forward & \\
 \hline
  & &qGpolySSU2R & Reverse &   \\
    \hline
 \end{tabular}
\end{table}

\subsection{Evaluation of primer specificity}
To verify primer set specificity as listed in Table ~\ref{tbl:PrimerTable}, DNA was extracted via CTAB from \emph{G. belizeanus} (CCMP401), \emph{G. australes} (CCMP1650 and CG61) and \emph{G. pacificus} (CAWD149). sPIKEY 8A - \emph{G. xx} (CAWD232) DNA was extracted using a PowerSoil™ DNA isolation kit (Mo Bio Inc., CA, USA) by Dr. Kirsty Smith, Cawthron Institute, New Zealand. Cross reactivity for \emph{G. scabrosus} () DNA was extracted using DNeasy Plant Mini Kit (Quiagen, Tokyo, Japan) according to the manufacturer's protocol, by Dr. Nishimura, Kochi University, Japan. For all extracted samples, the presence and integrity of genomic DNA was  assessed on 1\% agarose gel. Primer sets designed for \emph{G. lapillus} and \emph{G. polynesiensis} were tested for cross-reactivity against all other \emph{Gambierdiscus} spp. available via PCR. PCR amplicons were visually assessed on 1\% agarose gel.

\subsection{Evaluation of primer sensitivity}
 The qPCR reaction mixture contained 10 $\mu$l SYBR Select Master Mix (Thermo Fisher Scientific), 7 $\mu$l MilliQ water, 0.5 $\mu$M forward and reverse primers and 2 - 20 ng DNA template, for a final volume of 20 $\mu$l.

\subsection{Calibration curve construction}
Standard curves were constructed via two methods: (1) synthetic gene fragment based (henceforth referred to as gene based calibration) and (2) cell based calibration curves. The former was constructed from a 10-fold serial dilution of a synthesised fragment containing the SSU target sequence, forward and reverse primer sites and 50bp flanking both primer sites matching sequencing results. Cell-based standard curves were constructed using 10-fold dilutions of gDNA extract of known cell concentrations.
\subsubsection{Gene based calibration curve}
A synthetic gBlocks gene fragment spanning the target SSU sequence, primer sites and 50bp on either end was synthesised for both \emph{G. lapillus} and \emph{G. polynesiensis}.
\subsubsection{Cell based calibration curve}
Three cultures of \emph{G. lapillus} and two cultures of \emph{G. polynesiensis} (Table ~\ref{tbl:StrainTable}) were used to construct cell based standard curves. Cells were counted microscopically () with a Sedqick Rafter Counting Chamber. DNA was extracted using xx kit, as per the manufacturer's instructions. The gDNA extracts were 10-fold serial diluted.

%\subsection{Quantification of SSU rDNA copy number per cell of \emph{G. lapullis} and \emph{G. polynesiensis}}

%\subsection{Toxicity of \emph{G. lapillus}}

\subsection{Screening environmental samples for \emph{G. lapullis} and \emph{G. polynesiensis} abundance}
A total of 33 macroalgal sites around Heron Island were targeted, in tripplicate as biological replicates.
-fig showing sampling sites
-table showing samples, substrate, ID, result? 

\newpage
\section{Results}
\subsection{species specific primer design}
-

\FloatBarrier
\begin{table}
\caption{Cross-reactivity of qPCR primer sets.}
\label{tbl:CrossreactTable}
\begin{tabular}{  | p{4cm} | p{3cm} | p{2cm} | p{2.5cm} | p{2.5cm} | }
\hline
\textbf{Template} & \textbf{Strain name} & \textbf{gDNA gel band} & \textbf{GlapSSU2F-GlapSSU2R} & \textbf{GpolySSUF-GpolySSUR}  \\
  \hline
\emph{G. australes} & CCMP1650 &+&-&- \\
  \hline
 & CG61 &+&-&- \\
   \hline
 \emph{G. belizeanus}&CCMP401&+&-&-\\
   \hline
 \emph{G. lapillus}&HG1&+&+&-\\
   \hline
 &HG4&+&+&-\\
   \hline
 &HG6&+&+&-\\
   \hline
   &HG7&+&+&-\\
   \hline
   &HG26&+&+&-\\
  \hline
   \emph{G. pacificus}&CAWD149&+&-&-\\
  \hline
   \emph{G. polynesiensis}&CG14&+&-&+\\
   \hline
   &CG15&+&-&+\\
  \hline
  \emph{G. scabrosus}&&+&-&-\\
    \hline
 \end{tabular}
\end{table}

\subsection{Evaluation of primer specificity}

\subsection{Evaluation of primer sensitivity}
-figs qPCR std curves
\FloatBarrier 
\begin{figure} 
%\includegraphics[scale=.22]{MH_variability.jpg} 
\caption{qPCR standard curves of \emph{G. lapillus} strains HG4: (A) HG6: (B) HG7:(C). Error bars represent the deviation of technical replicates during reactions.} 
\label{fig:lapiStd}
\end{figure} 
\FloatBarrier

\FloatBarrier 
\begin{figure} 
%\includegraphics[scale=.22]{MH_variability.jpg} 
\caption{qPCR standard curves of \emph{G. polynesiensis} strains CG14: (A) CG15: (B). Error bars represent the deviation of technical replicates during reactions.} 
\label{fig:polyStd}
\end{figure} 
\FloatBarrier
-Ct of known gBlocks gives copy number - extrapolate those to SSU rDNA copies for cell. can compare to
\subsection{Quantification of SSU rDNA copy number per cell of \emph{G. lapullis} and \emph{G. polynesiensis}}

\subsection{Screening environmental samples for \emph{G. lapullis} and \emph{G. polynesiensis} abundance}
\FloatBarrier
\begin{longtable}{ | p{1cm} | p{1cm} | p{3cm} | p{4cm} | p{4cm} | }
\caption{Screening of macroalgal samples of \emph{G. lapillus} and \emph{G. polynesiensis}and cell density estimates via qPCR.}\\
\hline
\label{tbl:MacroalgaeTable}
1&A&&&\\
\hline
&B&&&\\
\hline
&C&&&\\
\hline
2&A&&&\\
\hline
&B&&&\\
\hline
&C&&&\\
\hline
3&A&&&\\
\hline
&B&&&\\
\hline
&C&&&\\
\hline
4&A&&&\\
\hline
&B&&&\\
\hline
&C&&&\\
\hline
5&A&&&\\
\hline
&B&&&\\
\hline
&C&&&\\
\hline
6&A&&&\\
\hline
&B&&&\\
\hline
&C&&&\\
\hline
7&A&&&\\
\hline
&B&&&\\
\hline
&C&&&\\
\hline
8&A&&&\\
\hline
&B&&&\\
\hline
&C&&&\\
\hline
9&A&&&\\
\hline
&B&&&\\
\hline
&C&&&\\
\hline
10&A&&&\\
\hline
&B&&&\\
\hline
&C&&&\\
\hline
11&A&&&\\
\hline
&B&&&\\
\hline
&C&&&\\
\hline
12&A&&&\\
\hline
&B&&&\\
\hline
&C&&&\\
\hline
13&A&&&\\
\hline
&B&&&\\
\hline
&C&&&\\
\hline
14&A&&&\\
\hline
&B&&&\\
\hline
&C&&&\\
\hline
15&A&&&\\
\hline
&B&&&\\
\hline
&C&&&\\
\hline
16&A&&&\\
\hline
&B&&&\\
\hline
&C&&&\\
\hline
17&A&&&\\
\hline
&B&&&\\
\hline
&C&&&\\
\hline
18&A&&&\\
\hline
&B&&&\\
\hline
&C&&&\\
\hline
19&A&&&\\
\hline
&B&&&\\
\hline
&C&&&\\
\hline
20&A&&&\\
\hline
&B&&&\\
\hline
&C&&&\\
\hline
21&A&&&\\
\hline
&B&&&\\
\hline
&C&&&\\
\hline
22&A&&&\\
\hline
&B&&&\\
\hline
&C&&&\\
\hline
23&A&&&\\
\hline
&B&&&\\
\hline
&C&&&\\
\hline
24&A&&&\\
\hline
&B&&&\\
\hline
&C&&&\\
\hline
25&A&&&\\
\hline
&B&&&\\
\hline
&C&&&\\
\hline
27&A&&&\\
\hline
&B&&&\\
\hline
&C&&&\\
\hline
28&A&&&\\
\hline
&B&&&\\
\hline
&C&&&\\
\hline
29&A&&&\\
\hline
&B&&&\\
\hline
&C&&&\\
\hline
30&A&&&\\
\hline
&B&&&\\
\hline
&C&&&\\
\hline
31&A&&&\\
\hline
&B&&&\\
\hline
&C&&&\\
\hline
32&A&&&\\
\hline
&B&&&\\
\hline
&C&&&\\
\hline
33&A&&&\\
\hline
&B&&&\\
\hline
&C&&&\\
 \hline
\end{longtable}
\FloatBarrier

\FloatBarrier
\begin{table}
\caption{Screening of tow net samples of \emph{G. lapillus} and \emph{G. polynesiensis}and cell density estimates via qPCR.}
\label{tbl:NetTable}
\begin{tabular}{  | p{4cm} |  p{4cm} | p{4cm} | }
\hline
\textbf{Sample ID}  & \textbf{\emph{G. lapillus}} & \textbf{\emph{G. polynesiensis}}  \\
\hline
34 &&\\
\hline
35&&\\
\hline
36&&\\
\hline
 \end{tabular}
\end{table}
\FloatBarrier
\newpage
\section{Discussion}
- RE normalization issue
- need to record surface or macroalgae for quantitative qPCR
- difference in abundance even in bio replicates
\newpage
\section{Conclusion}

\section{Acknowledgements}
Gratitude to Dr. Adachi and Dr. Nishimura for supplying \emph{G. scabrosus} as well as Dr. Kirsty Smith and Dr. Lesley Rhodes for supplying \emph{G. xx}, both used for DNA for cross reactivity assessment.
\FloatBarrier
\newpage
\bibliographystyle{acm}
\bibliography{references.bib}


\end{document}
