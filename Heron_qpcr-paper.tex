\documentclass[12pt]{article}
 \usepackage[hcentering,bindingoffset=20mm]{geometry}
 \usepackage{placeins}
 \usepackage[numbib]{tocbibind}
 \usepackage{rotating}
\usepackage[square,sort,comma,numbers]{natbib}
 \usepackage{graphicx}
 \usepackage{tabularx}
 \linespread{1.3}
 \usepackage{gensymb}
 \usepackage{longtable}
 \usepackage{lscape}
 \usepackage{url}
 \addtolength{\textwidth}{2cm}
 \addtolength{\hoffset}{-1cm}
 
 
 \addtolength{\textheight}{2cm}
 \addtolength{\voffset}{-1cm}
 \setlength{\parindent}{0pt}
 
\title{Development of semi-quantitative PCR assays for the detection and enumeration of \emph{Gambierdiscus} spp. (Gonyaulacales, Dinophyces).}
\author{me}
\date{}

\begin{document}
\maketitle
\paragraph{}Anna Liza Kretzschmar\\
Plant Functional Ecology and Climate Change Cluster (C3), University of Technology Sydney, Ultimo, 2007 NSW, Australia, anna.kretzschmar@uts.edu.au
\paragraph{}Arjun Verma \\
Plant Functional Ecology and Climate Change Cluster (C3), University of Technology Sydney, Ultimo, 2007 NSW, Australia
\paragraph{}Gurjeet Kohli\\ 
Plant Functional Ecology and Climate Change Cluster (C3), University of Technology Sydney, Ultimo, 2007 NSW, Australia
\paragraph{}Shauna Murray\\ 
Plant Functional Ecology and Climate Change Cluster (C3), University of Technology Sydney, Ultimo, 2007 NSW, Australia
\newpage
\section{Abstract}

\newpage
\section{Introduction}
- CFP and UNESCO, Gamb, Aust and not knowing causative agent, CTX
- LM monitoring issues - mol;ecular, global cig strategy qPCR
- Vandersea and Nishimura

\newpage
\section{Materials and methods}
\subsection{Clonal strains and culturing conditions}
Three strains of \emph{G. lapillus} and two strains of \emph{G. polynesiensis} were isolated from Heron Island, Australia, and Rarotonga, Cook Islands, respectively (Table ~\ref{tbl:StrainTable}). The cultures were maintained F$\bullet$-10 medium at 27 $^{\circ}$C, 60mol$\bullet$-m$^{2}$ $\bullet$-s light in 12hr:12hr light to dark cycles.
\FloatBarrier
\begin{table}
\caption{List of \emph{Gambierdiscus} clonal strains used for the qPCR assay.}
\label{tbl:StrainTable}
\begin{tabular}{  | p{2cm} | p{2cm} | p{2cm}| p{3cm} | p{3cm} | p{2cm} | }
\hline
\textbf{Species}  & \textbf{Collection site} &  \textbf{Collection date} &\textbf{Latitude} & \textbf{Longitude} & \textbf{Strain name} \\
  \hline
   \emph{G. lapillus}   &Heron Island, Australia &July 2014 &23$^{\circ}$ 4420' S&151$^{\circ}$ 9140' E  & HG4 \\
   \hline
&&&&& HG6\\
 \hline
 &&&& &HG7\\
 \hline
\emph{G. polynesiensis}&Rarotonga, Cook Islands&November 2014 &21$^{\circ}$ 2486' S&159$^{\circ}$ 7286' W  & CG14 \\
 \hline
&&&&&CG15\\
    \hline
 \end{tabular}
\end{table}
\subsection{DNA extraction and species specific primer design}
Genomic DNA was extracted using the CTAB method \citep{zhou1999analysis}. Purity and concentration of the extract was measured by Nanodrop (Nanodrop2000, Thermo Scientific), the integrity was visualised on 1\% agarose gel.
Unique primer sets were designed for the small-subunit (SSU) rDNA region of  \emph{G. lapillus}, as reported by Kretzschmar et al., and \emph{G. polynesiensis}, as available in GenBank reference database. The target sequences were aligned against sequences of all other \emph{Gambierdiscus}spp. from GenBank reference database, with the MUSCLE algorithm (maximum of 8 iterations) \citep{edgar2004muscle} used through the Geneious software, version 8.1.7 \citep{kearse2012geneious}. Unique sites were determined manually (Table ~\ref{tbl:PrimerTable}). Primers were synthesised by Integrated DNA Technologies.
Primer sets were tested systematically for secondary product formation for all 3 strains of \emph{G. lapillus} and 2 strains of \emph{G. polynesiensis} (Table ~\ref{tbl:StrainTable}) via standard PCR in 25$\mu$l mixture in PCR tubes. The mixture contained 0.6 $\mu$M forward and reverse primer, 0.4 $\mu$M BSA, 2 - 20 ng DNA, 12.5 $\mu$l 2xEconoTaq (Lucigen) and 7.5 $\mu$l PCR grade water.
The PCR cycling comprised of an initial 10 min step at 94 $^{\circ}$C, followed by 30 cycles of denaturing at 94 $^{\circ}$C for 30 sec, annealing at 60 $^{\circ}$C for 30 sec and extension at 72 $^{\circ}$C for 1 min, finalised with 3 minutes of extension at 72 $^{\circ}$C. Products were visualised on 1\% agarose gel.
\FloatBarrier
\begin{table}
\caption{List of species specific  qPCR primer sets for SSU rDNA.}
\label{tbl:PrimerTable}
\begin{tabular}{  | p{2cm} | p{2cm} | p{2cm} | p{2cm} | p{7cm} | }
\hline
\textbf{DNA target} & \textbf{Amplicon size} & \textbf{Primer name} & \textbf{Synthesis direction of primer} & \textbf{Sequence (5'-3')}  \\
  \hline
   \emph{G. lapillus}   & &GlapSSU2F & Forward & \\
   \hline
 & &GlapSSU2R & Reverse & \\
 \hline
\emph{G. polynesiensis}& &GpolySSUF& Forward & \\
 \hline
  & &GpolySSUR & Reverse &   \\
    \hline
 \end{tabular}
\end{table}

\subsection{Evaluation of primer specificity}
To verify primer set specificity as listed in Table ~\ref{tbl:PrimerTable}, DNA was extracted via CTAB from \emph{G. belizeanus} CCMP401, \emph{G. australes} CCMP1650 and \emph{G. pacificus} CAWD149. Cross reactivity for \emph{G. scabrosus} was checked against 8 strains, DNA extracted using DNeasy Plant Mini Kit (Quiagen, Tokyo, Japan),according to the manufacturer's protocol. For all extracted samples, the presence and integrity of genomic DNA was  assessed on 1\% agarose gel. Primer sets designed for \emph{G. lapillus} and \emph{G. polynesiensis} were tested for cross-reactivity against all other \emph{Gambierdiscus} spp. available via PCR. PCR amplicons were visually assessed on 1\% agarose gel.

\subsection{Evaluation of primer sensitivity}
 The qPCR reaction mixture contained 10 $\mu$l SYBR Select Master Mix (Thermo Fisher Scientific), 7 $\mu$l MilliQ water, 0.5 $\mu$M forward and reverse primers and 2 - 20 ng DNA template, for a final volume of 20 $\mu$l.

\subsection{Calibration curve construction}

\subsubsection{Gene based calibration curve}

\subsubsection{Cell based calibration curve}
%\subsection{Quantification of SSU rDNA copy number per cell of \emph{G. lapullis} and \emph{G. polynesiensis}}

%\subsection{Toxicity of \emph{G. lapillus}}

\subsection{Screening environmental samples for \emph{G. lapullis} and \emph{G. polynesiensis} abundance}

\newpage
\section{Results}
\subsection{species specific primer design}
- 
\subsection{Sample collection}

\subsection{Evaluation of primer specificity}

\subsection{Evaluation of primer sensitivity}

\subsection{Quantification of SSU rDNA copy number per cell of \emph{G. lapullis} and \emph{G. polynesiensis}}

\subsection{Screening environmental samples for \emph{G. lapullis} and \emph{G. polynesiensis} abundance}

\newpage
\section{Discussion}
- RE normalization issue
- need to record surface or macroalgae for quantitative qPCR
\newpage
\section{Conclusion}

\section{Acknowledgements}
Gratitude to Dr. Adachi and Dr. Nishimura for supplying \emph{G. scabrosus} DNA for cross reactivity assessment.
\FloatBarrier
\newpage
\bibliographystyle{acm}
\bibliography{references.bib}


\end{document}