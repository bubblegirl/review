\documentclass[12pt]{article}
 \usepackage[hcentering,bindingoffset=20mm]{geometry}
 \usepackage{placeins}
 \usepackage[numbib]{tocbibind}
 \usepackage{rotating}
\usepackage[square,sort,comma,numbers]{natbib}
 \usepackage{graphicx}
 \usepackage{tabularx}
 \linespread{1.3}
 \usepackage{gensymb}
 \usepackage{longtable}
 \usepackage{lscape}
 \usepackage{url}
 \addtolength{\textwidth}{2cm}
 \addtolength{\hoffset}{-1cm}
 
 
 \addtolength{\textheight}{2cm}
 \addtolength{\voffset}{-1cm}
 \setlength{\parindent}{0pt}
 
\title{Development of semi-quantitative PCR assays for the detection and enumeration of \emph{Gambierdiscus} spp. (Gonyaulacales, Dinophyces).}
\author{me}
\date{}

\begin{document}
\maketitle
\paragraph{}Anna Liza Kretzschmar$^{2}$\\
Plant Functional Ecology and Climate Change Cluster (C3), University of Technology Sydney, Ultimo, 2007 NSW, Australia, anna.kretzschmar@uts.edu.au
\paragraph{}Arjun Verma \\
Plant Functional Ecology and Climate Change Cluster (C3), University of Technology Sydney, Ultimo, 2007 NSW, Australia
\paragraph{}Gurjeet Kohli\\ 
Plant Functional Ecology and Climate Change Cluster (C3), University of Technology Sydney, Ultimo, 2007 NSW, Australia
\paragraph{}Shauna Murray\\ 
Plant Functional Ecology and Climate Change Cluster (C3), University of Technology Sydney, Ultimo, 2007 NSW, Australia
\newpage
\section{Abstract}

\section{Introduction}
- CFP and UNESCO, Gamb, Aust and not knowing causative agent, CTX
- LM monitoring issues - mol;ecular, global cig strategy qPCR
- Vandersea and Nishimura

\section{Materials and methods}
\subsubsection{Clonal strains and culturing conditions}

\subsubsection{DNA extraction and species specific primer design}

\subsubsection{Sample collection}

\subsubsection{Evaluation of primer specificity}
Unique primer sets were designed for \emph{G. lapillus} and \emph{G. polynesiensis} (Table ~\ref{tbl:PrimerTable}).
\FloatBarrier
\begin{table}
\caption{List of species specific  qPCR primer sets for SSU rDNA.}
\label{tbl:PrimerTable}
\begin{tabular}{  | p{2cm} | p{2cm} | p{2cm} | p{2cm} | p{7cm} | }
\hline
\textbf{DNA target} & \textbf{Amplicon size} & \textbf{Primer name} & \textbf{Synthesis direction of primer} & \textbf{Sequence (5'-3')}  \\
  \hline
   \emph{G. lapillus}   & &GlapSSU2F & Forward & \\
   \hline
 & &GlapSSU2R & Reverse & \\
 \hline
\emph{G. polynesiensis}& &GpolySSUF& Forward & \\
 \hline
  & &GpolySSUR & Reverse &   \\
    \hline
 \end{tabular}
\end{table}
\subsubsection{Evaluation of primer sensitivity}

\subsubsection{Quantification of SSU rDNA copy number per cell of \emph{G. lapullis} and \emph{G. polynesiensis}}

\subsubsection{Screening environmental samples for \emph{G. lapullis} and \emph{G. polynesiensis} abundance}

\section{Results}
\subsubsection{species specific primer design}
- 
\subsubsection{Sample collection}

\subsubsection{Evaluation of primer specificity}

\subsubsection{Evaluation of primer sensitivity}

\subsubsection{Quantification of SSU rDNA copy number per cell of \emph{G. lapullis} and \emph{G. polynesiensis}}

\subsubsection{Screening environmental samples for \emph{G. lapullis} and \emph{G. polynesiensis} abundance}

\section{Discussion}
- RE normalization issue
- need to record surface or macroalgae for quantitative qPCR
\section{Conclusion}

\FloatBarrier
\newpage
\bibliographystyle{acm}
\bibliography{references.bib}


\end{document}