\documentclass[12pt]{article}
 \usepackage[hcentering,bindingoffset=20mm]{geometry}
 \usepackage{placeins}
 \usepackage[numbib]{tocbibind}
 \usepackage{rotating}
\usepackage[square,sort,comma,numbers]{natbib}
 \usepackage{graphicx}
 \usepackage{tabularx}
 \linespread{1.3}
 \usepackage{gensymb}
 \usepackage{longtable}
 \usepackage{lscape}
 \usepackage{url}
 \addtolength{\textwidth}{2cm}
 \addtolength{\hoffset}{-1cm}
 
 
 \addtolength{\textheight}{2cm}
 \addtolength{\voffset}{-1cm}
 \setlength{\parindent}{0pt}
 
\title{CHARACTERISATION OF \emph{GAMBIERDISCUS lapillus} SP. NOV. (GONYAULACALES): A NEW DINOFLAGELLATE FROM THE GREAT BARRIER REEF (AUSTRALIA) THAT PRODUCES A CIGUATOXIN ANALOGUE. $^{1}$}
\author{me}
\date{}

\begin{document}
\maketitle
\paragraph{}Anna Liza Kretzschmar$^{2}$\\
Plant Functional Ecology and Climate Change Cluster (C3), University of Technology Sydney, Ultimo, 2007 NSW, Australia, anna.kretzschmar@uts.edu.au
\paragraph{}Arjun Verma \\
Plant Functional Ecology and Climate Change Cluster (C3), University of Technology Sydney, Ultimo, 2007 NSW, Australia
\paragraph{}Tim Harwood\\ 
Cawthron Institute, The Wood, Nelson 7010, New Zealand
\paragraph{}Mona Hoppenrath\\
Senckenberg Research Institute, German Centre for Marine Biodiversity Research, 26382 Wilhelmshaven, Germany
\paragraph{}Shauna Murray\\ 
Plant Functional Ecology and Climate Change Cluster (C3), University of Technology Sydney, Ultimo, 2007 NSW, Australia
\newpage
\section{Abstract}
\textit{Gambierdiscus} is a genus of benthic dinoflagellates which is found worldwide. 
Some species produce neurotoxins (maitotoxins and ciguatoxins) which  bioaccumulate and cause ciguatera fish poisoning (CFP), a potentially fatal food-borne illness that is common worldwide in tropical regions. 
The investigation of toxigenic species of \textit{Gambierdiscus} in CFP endemic regions in Australia is necessary as a first step to determine which species of Gambierdiscus are related to CFP cases occurring in this region.
In this study we characterised five strains of \textit{Gambierdiscus} collected from Heron Island,  Australia, a region in which ciguatera is endemic. Clonal cultures were processed using (i) light microscopy; (ii) scanning electron microscopy; (iii) DNA sequencing based on the nuclear encoded ribosomal  18S and D8-D10 28S regions; (iv) toxicity via mouse bioassay and; (v) toxin profile as determined by Liquid Chromatography-Mass Spectrometry. 
Both the morphological and phylogenetic data indicated that these strains represent a new, toxic species of \emph{Gambierdiscus}, \emph{Gambierdiscus lapillus} sp. nov (plate formula Po, 3', 0a, 7'', 6c, 7-8s, 5''', 0p, 2'''' and distinctive by size and hatchet shaped 2' plate). Culture extracts were found to be toxic using the mouse bioassay. Using chemical analysis, it was determined that they did not contain maitotoxin (MTX1) or known algal-derived ciguatoxin analogs (CTX3B, 3C, CTX4A, 4B), but that they contained putative MTX3, and likely other unknown compounds.

%\paragraph{Keywords:} \emph{Gambierdiscus}, ciguatoxin, ciguatera
%\paragraph{Abbreviations:} CFP, ciguatera fish poisoning; CTX, ciguatoxin; 	LC-MS, liquid chromatography - mass spectrometry; MBA, mouse bioassay; MTX, maitotoxin
\newpage

%-alt to lapillus: cavus, florens, vegrandis
\section{Introduction}

%\subsection{overview, history, diff sp}
\emph{Gambierdiscus} Adachi et Fukuyo is a genus of benthic dinoflagellate genus, epiphytic on many substrates in shallow tropical and sub-tropical waters \citep{marine2014}. 
Species of \emph{Gambierdiscus} have been found to produce the ciguatoxins (CTX), which are potent neurotoxins thought to be responsible for ciguatera fish poisoning (CFP) \citep{berdalet2012global}. 
Since the discovery of the type species \emph{G. toxicus} R.Adachi \& Y.Fukuyo in 1977 \citep{yasumoto1977finding}, extensive research has established that certain species of the genus are the primary sources of CTXs, the causative agent of CFP \citep{chinain1997intraspecific,holmes1998gambierdiscus}. 
Maitotoxins (MTXs) are also commonly produced, however their role in CFP is yet to be established \citep{kohli2014feeding}. 

For over 15 years, \emph{Gambierdiscus} was considered to be a monotypic taxon, but since 1995 new species have been discovered with more extensive sampling in the Atlantic Ocean, Indian Ocean and Pacific \citep{faust1995observation,holmes1998gambierdiscus,litaker2009taxonomy,chinain1999morphology,fraga2011gambierdiscus,nishimura2014morphology}.
Currently 11 species, and six unnamed clades of \emph{Gambierdiscus} have been described based on their distinct morphological and genetic characteristics \citep{adachi1979thecal,faust1995observation,chinain1999morphology,litaker2009taxonomy,nishimura2014morphology,fraga2011gambierdiscus}.  
The genus \emph{Fukuyoa} F. G\'omez, D. Qiu, R.M. Lopes \& S. Lin was separated from \emph{Gambierdiscus} in 2015 \citep{gomez2015fukuyoa}. \emph{Fukuyoa} spp. produce MTX \citep{holmes1998gambierdiscus,holland2013differences} and hence may contribute to CFP \citep{kohli2014feeding}, they will be referred to alongside \emph{Gambierdiscus} where appropriate.
Ascertaining the distribution of species and their toxin production is essential for assessing risk of CFP to local populations as well as the seafood industry.

%\subsection{toxins and CFP}
CFP is the most common non bacterial illness associated with seafood consumption \citep{friedman2008ciguatera}, however the correct diagnosis of the illness is difficult due to extensive list of neurological, gastrointestinal and cardiovascular symptoms \citep{sims1987theoretical}. 
The United Nations Educational Scientific and Cultural Organization's international panel for harmful algal blooms has proposed a coordinated global ciguatera strategy to address the impact of the disease through \emph{Gambierdiscus} spp. detection and monitoring, improved toxin detection in seafood and epidemiological data collation \citep{globalcig}.
The suite of toxins, if any, produced by \emph{Gambierdiscus} varies between species. 
Bioassays are commonly used to assess toxicity, however liquid chromatography-mass spectrometry (LC-MS/MS) analysis is required to unambiguously identify the toxins present \citep{diogened2014chemistry}. 
Some species exclusively produce MTX (\emph{G. australes} \citep{rhodes2014production}) while others solely produce CTX (\emph{G. polynesiensis} \citep{rhodes2014production}) or neither (\emph{G. carpenteri} \citep{kohli2014high}), as confirmed by LC-MS/MS. 

%\subsection{abundance and distribution}
Some \emph{Gambierdiscus} spp. have been classified as endemic to either the Pacific or Atlantic Oceans, while others have been found globally distributed \citep{berdalet2012global,litaker2010global}. %It has been suggested that with more extensive sampling, the distribution is likely to be global for all \emph{Gambierdiscus} species \citep{testerICHA}. 
However, the current understanding about \emph{Gambierdiscus} distribution and abundance is fragmentary due to the paucity of studies worldwide, the continued discovery of new species, difficulty with species identification with light microscopy and the comparatively lower sampling frequency in the Atlantic Ocean \citep{berdalet2012global,nishimura2014morphology}. 

Different species of \emph{Gambierdiscus} usually co-occur at sample sites \citep{litaker2010global}. 
As species can show intra-specific morphological variability \citep{bravo2014cellular}, but can also be highly morphologically similar to one another \citep{kohli2014high}, molecular genetics techniques are necessary to complement the analysis of morphology for species determination. 
The taxonomy of the genus was poorly defined until relatively recently (ie \citep{litaker2009taxonomy,richlen2008phylogeography}) which means that species from studies pre-dating 2009 are difficult to evaluate based on current species delimitations \citep{berdalet2012global}. %Morphology alongside genetic tools can identify species with a high certainty.

%Annual reported CFP cases are estimated up to 500,000 globally and are responsible for 80 to 96 \% of human poisoning from consuming seafood \citep{fleming1998seafood,grandjean2008centers}. ---> check ref
 %As CFP is increasing in frequency in the Pacific \citep{skinner2011ciguatera} and in other regions, an understanding of the distribution and species specific toxicology of \emph{Gambierdiscus} spp. is essential \citep{globalcig}.
 
%\subsection{australia}
Cases of CFP have been reported in Australia, mostly around the Great Barrier Reef (GBR), but a consistent CTX-producing species of \emph{Gambierdiscus} has not yet been confirmed \citep{lewis2006ciguatera}. 
This is likely due to a combination of factors:
Australia has a coastline of ~66,000 km (Fig. ~\ref{fig:OzSites}), of which at least 50\% could be considered tropical or sub-tropical, most of which has never been sampled for \textit{Gambierdiscus} species presence; species identifications have often been uncertain; culturing \textit{Gambierdiscus} requires considerable time and effort; and measuring CTX analogs is also very difficult. 
To date,  the four species \emph{G. belizeanus} M.A.Faust, \emph{G. carpenteri} Kibler, Litaker, Faust, Holland, Vandersea \& Tester, \emph{G. toxicus} and \emph{F. yasumotoi} (M.J.Holmes) F.G\'omez, D.X.Qiu, R.M.Lopes \& Senjie Lin have been identified in Australia (Table ~\ref{tbl:OzTable}), in Queensland and New South Wales. 
\emph{Gambierdiscus} was also detected as part of a benthic dinoflagellate community via pyrosequencing in Western Australia, however could only be identified to genus level \citep{kohli2014cob}.
Three analogues of MTX, designated MTX-1, MTX-2 and MTX-3, isolated from a \emph{Gambierdiscus} sp. in Queensland, Australia \citep{holmes1994purification}. 
As this discovery pre-dates the understanding that \emph{Gambierdiscus} includes more species than \emph{G. toxicus}, it remains unclear from which species these analogues were isolated.  
It is highly likely that many more \emph{Gambierdiscus} species were present even at those particular sites examined, as in each case the studies represent only single or short term sampling events at one location within a site. 
Therefore the toxicity and distribution of \emph{Gambierdiscus} species in Australian coastal waters remains largely unknown. %Consequently, the causative agent for CFP in Australia has not yet been determined. \\

The aim of the current study was to further the continued search for the causative agent/s of CFP in Australia by characterizing a species of \emph{Gambierdiscus} from a ciguateric region using molecular and morphological methods, and investigate the toxicology of the isolates.

 \newpage
\section{Materials and methods}

\subsection{Specimen collection and culture conditions}
Specimens were collected from Heron Island (23$^{\circ}$ 4420' S, 151$^{\circ}$ 9140' E), which is part of the southern GBR region. 
Heron Island constitutes site 3 of previous specimen collection in Australia (Fig. ~\ref{fig:OzSites}).
In October 2013, a sample of \emph{Halimeda} sp. was taken when the average sea surface temperature was 23.8 $^{\circ}$C. 
A second sample of \emph{Padina} sp. was collected in July 2014 at a time when the average sea surface temperature was 19.8 $^{\circ}$C.
Macroalgal samples were taken around the Heron Island research station (Fig. ~\ref{fig:OzSites}). 
The macroalgal samples were stored in 500ml bottles with seawater and processed within 48 hours. 
The samples were shaken to dislodge protists, then concentrated through a 125 $\mu$m sieve \citep{litaker2010global}. 
Single cells were isolated from the samples using a micropipette under an inverted light microscope (Nikon Eclipse Ni), washed three times in sterile seawater, and transferred into F/10 medium \citep{holmes1991strain} to establish clonal cultures.
The culture HG1 was isolated from the sample from October 2013, while the cultures HG4, HG5, HG6, HG7, HG26 were isolated from the sample from July 2014.
The cultures were maintained F/10 medium at 27 $^{\circ}$C, 60$\mu$mol.m$^{2}$.s$^{-1}$ light in 12hr:12hr light to dark cycles.


%\newpage



\subsection{Morphological analyses}
The cultures HG4, HG6, HG7 and HG26 (during exponential growth phase) were fixed with 10 \% Lugol's iodide and examined with an inverted light microscope (Nikon Eclipse TS100 equipped with an Infinite Luminera 1 camera) for cell size. 
Depth (D: measured along dorso-ventral axis) and width (W: measured between the lateral axis) was measured for 30 to 36 cells per strain.
The culture HG26 was fixed with 10 \% Lugol's iodide for plastid and nucleus diagnosis with an inverted light microscope (Nikon Eclipse Ni equipped with an Nikon DS-Qi2 camera).
For plate pattern identification LM with calcofluor staining of the type strain HG7 was carried out under UV (Olympus BX51 microscope equipped with Olympus DP73 camera, Australia). 
The culture was centrifuged at 300g, concentrated and fixed in 4\% Lugol’s iodide, then stained with 1 $\mu$L of calcofluor stock solution (1 mg.mL$^{-1}$) of Fluorescent Brightener 28 (Sigma, St Louis, MO, USA) for illumination.
For HG4, HG5, HG6, HG7 and HG26 30 ml of culture was centrifuged at 15,000 x g and concentrated to 2 ml, preserved with 10 \% Lugol's iodide, and analyzed by scanning electron microscopy (SEM) at the Senckenberg am Meer (Wilhelmshaven, Germany). 
Cells were placed on a 5 $\mu$m Millipore filter, rinsed in distilled water, and dehydrated in a series of increasing ethanol concentrations (30, 50, 70, 85, 90, 100 \%), followed by chemical drying with hexamethyldisilazane at room temperature. 
When completely dry, the sample was mounted on a stub and sputter coated with gold palladium (SCD 050 Bal-Tec, BAL-TEC Pr\"aparations-ger\" atevertrieb, Wallof, Germany). 
Cells were observed using a Tescan VEGA3 microscope (Elekronen-Optik-Service GmbH, Dortmund, Germany) at 15 kV.

\subsection{Genomic DNA extraction}
Genomic DNA was extracted using the CTAB method \citep{zhou1999analysis}. 
Purity and concentration of the extract was measured by Nanodrop (Nanodrop2000, Thermo Scientific), the integrity was visualised on 1\% agarose gel.

\subsection{PCR amplification and sequencing}
Extracted DNA was used as a template to amplify rDNA sequences in 25 $\mu$l mixture in PCR tubes. 
Reactions consisted of a final concentration of 0.6 $\mu$M forward and reverse primer, 0.4 $\mu$M BSA, 2 - 20 ng DNA, 12.5 $\mu$l 2xEconoTaq (Lucigen Corporation, Middleton, WI, USA) and 7.5 $\mu$l PCR grade water.\\
The PCR cycling comprised of an initial 10 min step at 94 $^{\circ}$C, followed by 30 cycles of denaturing at 94 $^{\circ}$C for 30 s, annealing at 55 $^{\circ}$C for 30 s and extension at 72 $^{\circ}$C for 1 min, finalized with 3 minutes of extension at 72 $^{\circ}$C.
The LSU D8-D10 and SSU rDNA regions were amplified with the FD8-RB and 18ScomF1-18ScomR1 primer sets, respectively (Table ~\ref{tbl:PrimerTable}).\\
Sanger sequencing was conducted by Macrogen Inc. (Seoul, Korea).


\subsection{Sequence alignment and phylogenetic analysis}
Sequencing data were aligned with \emph{Gambierdiscus} spp. data from the GenBank reference database (accession numbers as part of Fig. ~\ref{fig:HGD8D10} and Fig. ~\ref{fig:HGSSU}). 
The alignment algorithm MUSCLE, with a maximum of eight iterations \citep{edgar2004muscle}, was used through the Geneious software, version 8.1.7 \citep{kearse2012geneious}. 
Alignments were truncated to the same length (D10-D8 and SSU at 799bp and 1690bp, respectively), and discrepancies removed. 
Alignments are available on request.
Phylogenetic trees were calculated with Bayesian inference (BI) as well as maximum likelihood (ML). 
BI, via Mr. Bayes 3.2.2, was used to estimate the posterior probability (PP) distribution with Metropolis-Coupled Markov Chain Monte Carlo simulations \citep{ronquist2003mrbayes}. 
A random staring tree with three heated and one cold chain(s) with a temperature set at 0.2. Trees were sampled every 100th generation for the 2,000,000 generations generated.
PHYML was used for ML analysis with 1,000 bootstraps (BS) \citep{guindon2003simple}.
The general time reversal model with an estimated gamma distribution was used for both analyses.\\

\subsection{Toxin production}

\subsubsection{Toxicity via mouse bioassay}
The cultures HG4, HG6 and HG7 (1,248,000, 3,982,000, and 1,323,200 cells, respectively) were freeze-dried in late stationary phase (day 36 of culture). 
The weights of the samples were 57.0 mg, 155.0 mg and 23.0 mg, respectively. 
The samples were extracted exhaustively with methanol. 
The extracts were evaporated to a small volume at 35$^{\circ}$C using a rotary evaporator, and the solution aliquotted into glass vials. 
The remaining methanol was removed under a stream of nitrogen, and the extracts freeze-dried overnight. 
The total weight of the extracts, which were all brown green colour, was as follows: HG4, 22.5 mg HG6, 42.9 mg HG7, 15.1 mg. 
The median lethal doses of the test materials by intraperitoneal (i.p.) injection were determined according to the principles of the Organisation for Economic Co-operation and Development Guideline 425. 
Samples were dissolved in 1\% Tween 60 in normal saline immediately before dosing. 
For toxicity by i.p. injection, aliquots of this solution, made up to a total volume of 1 ml with the same solvent, were injected into female Swiss mice, of initial body weight 18-22 g. 
The toxicity of the extract of HG6 by gavage was also investigated, when aliquots of the extract solution were made up to 200 $\mu$l with Tween-saline. 
The mice were monitored intensively during the day of dosing. 
Survivors were examined and weighed each day for the following 13 d, after which they were killed and necropsied. 
Tap water and food (Rat and Mouse Cubes, Speciality Feeds Ltd, Glen Forrest, Western Australia) were available at all times.\\

\subsubsection{Toxin profile screening by liquid chromatography-mass spectrometry}
After 36 days in culture 2 litres of HG4, HG6 and HG7 were centrifuged (15,000g) and the resultant pellet freeze-dried. 
Pellets were extracted with methanol and screened for the presence of maitotoxin (MTX-1) and algal CTXs (CTX-3B, CTX-3C, CTX-4A, CTX-4B) using a LC-MS/MS methods developed by the Cawthron Institute \citep{kohli2014feeding}.
The limit of detection of the analysis was ~1 ng.mL$^{-1}$ for MTX-1 and <1 ng.mL$^{-1}$ for the various CTX analogs monitored.
\newpage
\section{Results}

\subsection{\emph{Gambierdiscus lapillus}}
\subsubsection{Morphology}
%er than all other characterised \emph{Gambierdiscus} spp. (Table ~\ref{tbl:GlobalSizeTable}, Fig. ~\ref{fig:SizeGraph}),
Cells of\textit{ Gambierdiscus lapillus} were anterioposteriorly compressed and round to ellipsoid, sometimes dorsally pointed in apical view (Fig. ~\ref{fig:epiSEM}). 
Cells are small, dorsoventral depth of 40.6 $\mu$m (range of 34.4 - 50.9 $\mu$m; standard deviation 3.3 $\mu$m) and width of 39 $\mu$m (range of 32 - 46.9 $\mu$m; standard deviation 3.2 $\mu$m) with a depth to width ratio of 1.04 (standard deviation 0.06 $\mu$m) (n=128). 
The hypotheca was slightly higher than the epitheca, rarely significantly higher (Fig. ~\ref{fig:hypoSEM}f and Fig. ~\ref{fig:latSEM}a).
The cell surface had a strong  reticulate-foveate ornamentation, apical and antapical plates with more regularly arranged deep depressions nearly all containing a pore, and pre- and postcingular plates more irregularly reticulated with scattered pores (Fig. ~\ref{fig:ornSEM}). 
Wide intercalary bands had a ribbed appearance (Fig. ~\ref{fig:ornSEM}). 
Newly developing plates after cell division had a smooth surface with scattered pores (Fig. ~\ref{fig:epiSEM}h and Fig. ~\ref{fig:hypoSEM}a). 
The plate formula (Kofoid tabulation) was Po, 3', 0a, 7'', 6c, 7-8s, 5''', 0p, 2''''. 
The apical pore plate (Po) is oval with a fish-hook shaped slit-like apical pore, located close to the center of the epitheca, slightly shifted ventrally (Fig.~\ref{fig:epiSEM}e-h). 
Po was about 6.1-5.2 x 3.6-3.7 $\mu$m in size (Fig. ~\ref{fig:epiSEM}e-h). 
28 to 34 large depressions were distributed over the Po plate and arranged in no constant pattern (Fig. ~\ref{fig:epiSEM}e-h); one (Fig. ~\ref{fig:epiSEM}e) or two (Fig. ~\ref{fig:epiSEM}f, g) rows of depressions intruded into the hook. 
As seen by an inside view the depressions contained a pore (Fig.~\ref{fig:epiSEM}h). Also a tiny pore was visible on the Po-plate (suppl. Fig. ~\ref{fig:latSEM}c). 
The largest apical plate was the hatchet-shaped 2' plate, followed by the pentagonal 3' plate and the hexagonal 1' plate (Fig. ~\ref{fig:epiSEM}a-c). The largest precingular plate was plate 3'' (four-sided, covering the left lateral precingular area) and the pentagonal 4’’ plate (dorsal) was nearly of the same size. Plates 5’’ and 6’’ were on the right lateral side. 
Plates 1” and 7’’ were the smallest (Fig. ~\ref{fig:epiSEM}a-c) located ventrally. 
The largest postcingular plate was the 4''' plate (covering most of the left hypotheca), followed by the plates 2''', 3''', 5''' and 1'''. 
The quadrangular first antapical plate (1'''') was small and the larger, relatively narrow pentagonal 2'''' plate that covered the antapex (Fig. ~\ref{fig:hypoSEM}). 
The sulcus was a deep excavation (Fig. ~\ref{fig:hypoSEM}c,g) and composed of seven to eight plates. The narrow and deep cingulum was difficult to observe and consisted of 6 plates (Fig. ~\ref{fig:sulcSEM}c).
A few cells displayed shape-variability of plate 2' (suppl. Fig. ~\ref{fig:varSEM}c-e). 
The shape varied from hatchet shaped (main morphology) to nearly rectangular (suppl. Fig. ~\ref{fig:varSEM}c) and 6-sided (suppl. Fig. ~\ref{fig:varSEM}e). 
Some of the cells were tear-drop shaped rather than ellipsoid (suppl. Fig. ~\ref{fig:varSEM}a-c). 
%\newpage

\subsubsection{Phylogenetics}

Phylogenetic trees for SSU and D8-D10 LSU rDNA regions were calculated and compared BI and ML. 
The SSU rDNA  topologies of BI and ML were the same with all major branches fully supported ( 1.00 / 100, PP and BS respectively). 
Within \emph{Gambierdiscus}, the majority of the branches were fully supported, with all bar one at least relatively well supported. The branch splitting \emph{G. australes} from the major clade including \emph{G. pacificus}, \emph{toxicus} and \emph{G. lapillus} had the lowest support at 0.95 / 60.
For the SSU rDNA region, the major clades of \textit{Gambierdiscus }were fully supported based on ML BS and BI PP. 
The major clade support for the D8D10 LSU rDNA varied from fully supported to unsupported depending on the clade and support between ML BS and BI PP varied (eg. the branching for the major clade containing \emph{G. caribaeus} and \emph{G. carpenteri}). 
The same relationships of the Heron Island isolates HG1, HG4, HG6, HG7 and HG26 to the previously described \emph{Gambierdiscus} spp. as well as uncharacterised ribotypes and species types were found in the phylogenies based on both D8D10 LSU rDNA (Fig. ~\ref{fig:HGD8D10}) and SSU rDNA (Fig. ~\ref{fig:HGSSU}). 
The phylogenies showed that these strains are clearly separated from other \emph{Gambierdiscus} spp. and fall into a new, fully supported clade. 
The new clade falls within the most diverse major clade (8 species, Fig. ~\ref{fig:HGD8D10}). \emph{G. lapillus} is closely related to \emph{G.} sp. type 6, \emph{G. toxicus} and \emph{G. pacificus}. 
The closest following relatives from within their major clade are \emph{G.} sp. type 5, \emph{G. scabrosus}, \emph{G.} ribotype 2 and then \emph{G. belizeanus} in descending order. \\
No differentiation within species based on their geographic origin was determined for the SSU rDNA or LSU D8-D10 rDNA regions (Fig. ~\ref{fig:HGD8D10} and Fig. ~\ref{fig:HGSSU}).
%emph{G. silvae} was also isolated alongside HG4, HG6 , HG7 and HG26.

%\newpage

\subsubsection{Toxicology}

\paragraph{Toxicity via i.p. injection.}
The intraperitoneal LD50 of the extract of HG6 was 0.78 mg$\bullet$-kg, with 95\% confidence limits between 0.40 and 1.60 mg$\bullet$-kg. 
At high dose-levels, stretching movements were observed immediately after injection of the test material, most likely due to irritation. 
This was rapidly followed by cessation of movement and abdominal breathing. 
Respiration rates declined and death occurred within 1.5-4 hours after dosing. 
At necropsy, erythema was observed throughout the gastrointestinal tract of these animals At lower doses, death occurred at up to 24 hours after dosing. The stomachs of these animals were grossly distended with gas and with abnormally fluid contents. An abnormally large amount of material was present in the duodenum and upper jejunum, which was reddish-brown in colour and gelatinous in consistency. Caeca were also enlarged. Erythema in the glandular stomach was noted. No gross changes were observed in any other organ. 
The acute intraperitoneal toxicity of the extract of HG7 was lower than that of HG6, with an LD50 of 12.5 mg$\bullet$-kg, with 95\% confidence interval between 10.1 and 15.3 mg$\bullet$-kg. The symptoms of intoxication and the gross pathology in mice injected with HG7 were the same as those recorded with HG6. 
The acute intraperitoneal toxicity of the extract of HG4 was even lower. 
No deaths were recorded at 100 mg$\bullet$-kg. 
A dose of 150 mg/kg was lethal, however, with the same effects as seen with the other 2 extracts. 
Insufficient material was available to determine a precise LD50, though from the available data it may be concluded that it lies between 100 and 150 mg$\bullet$-kg.

\paragraph{Toxicity via gavage.}
 HG6 was much less toxic by oral administration than by intraperitoneal injection. 
 No effects were observed at a dose of 300 mg$\bullet$-kg, which is 385 times the median lethal dose by intraperitoneal injection. 
 Insufficient material was available for dosing at higher levels.​

\paragraph{Toxin profile.}
The toxin profile showed isolates of \emph{G. lapillus} positive for MTX-3 only. 
An unusual peak in the CTX fraction was observed, indicative of an yet uncharacterised CTX analogue. 
This peak is subject to further investigation. 

\newpage
\section{Discussion}
The \emph{G. lapillus} cells are anterio-posteriorly compressed, poses a narrow 2'''' plate and heavily areolated cell surface, as does \emph{G. belizeanus} (1p plate equivalent)                                                                                                                                                                                           \citep{litaker2009taxonomy}.
 \emph{G. scabrosus} is also morphologically similar to \emph{G. belizeanus}, but can be distinguished by the asymmetry of the 4'' plate (designated 3'' by Nishimura et al) compared to the symmetry of the 4'' plate in \emph{G. belizeanus} \citep{nishimura2014morphology}. 
 Hence the closest morphological relatives to \emph{G. lapillus} are \emph{G. belizeanus} and \emph{G. scabrosus}. 
 The most distinguishing difference between \emph{G. lapillus} and all other \emph{Gambierdiscus} spp. is the species' diminutive size (Table ~\ref{tbl:GlobalSizeTable} and Fig. ~\ref{fig:SizeGraph}). 
 Size is a feature that has been previously used to distinguish species of \emph{Gambierdiscus} from each other \citep{litaker2009taxonomy}. 
\emph{G. lapillus} also differs to \emph{G. scabrosus} due to it's symmetric 4'' plate (Fig. ~\ref{fig:epiSEM}a), hence it is closer morphologically to \emph{G. belizeanus}. 
The morphological point of difference between the two species lies in the 2' plate. The 2' plate of \emph{G. lapillus} is hatchet shaped (Fig. ~\ref{fig:epiSEM}a-d) while the \emph{G. belizeanus}'s is rectangular \citep{faust1995observation}. 
In terms of the genetic relatedness of the new species and \textit{G. belizeanus}, they are relatively distantly related (Fig. ~\ref{fig:HGD8D10} and Fig. ~\ref{fig:HGSSU}). \\
A small subset of isolates displayed some variability morphological features, and were different to the majority of cells, in the 2' plate as well as cell shape (Fig. ~\ref{fig:varSEM}). 
Bravo et al recorded a change in cell shape and size during the life cycle of a clonal strain of \emph{Gambierdiscus} sp., where a small proportion of the cells observed displayed this variability. 
They postulate that this was previously unrecorded due to their large number of observations compared to the rest of the literature \citep{bravo2014cellular}. 
We suggest that the variability we observed is similarly due to screening numerous cells of 4 strains of \emph{G. lapillus}. 
The variability in 2' plate is negligible. The hatchet shape of this plate should still be used as a diagnostic tool, though confirmation by cell size is essential.\\
Kohli et al observed a similar difference from the original species description in the morphology of strains of  \emph{G. carpenteri} \citep{kohli2014high}. 
Strains of \emph{G. carpenteri} isolated from southern NSW, Australia, were identified by a combination of morphological features and genetic sequencing. These strains were 99\% identical to the type strains of \textit{G. carpenteri}. 
However, their morphological features differed from those described for \textit{G. carpenteri}, and were more similar to those of \emph{G. toxicus} (broad 2'''' plates, pointed dorsal ends, and the lack of a dorsal rostrum) \citep{kohli2014high,litaker2009taxonomy}.
The potential intraspecific morphological variability of \emph{Gambierdiscus} spp. indicates that the identification of \emph{Gambierdiscus} spp. needs to be approached with caution, and preferably verified with molecular analysis.\\
The sister groups to \emph{G. lapillus} in the phylogenetic analyses were found to be \emph{G. pacificus} and \emph{G. toxicus}, 
which was fully supported (Fig. ~\ref{fig:HGD8D10} and Fig. ~\ref{fig:HGSSU}). For the D8-D10 LSU region, the confidence of the inter species relationship within the major clade containing \emph{G. lapillus} was lower.  
While the resolution of \emph{G. lapillus} as a separate species is fully supported, the other branches within that major clade were more ambiguous and support differed between BI and ML with the latter consistently lower than the former. 
Branches were well supported or better in BI phylogeny, while in the ML phylogeny some were unsupported. 
The topology of the clade resembles that of the phylogeny by Xu et al, when the \emph{G.} sp. type 5 and \emph{G.} sp. type 6 were presented \citep{xu2014distribution}, hence it can be assumed that the BI phylogeny is accurate.
 %The BI of the species' relationships was still high, whereas the ML confidence was reduced. %ref other papers.. xu for example
The discrepancy in confidence between SSU and D8-D10 LSU could be due to the length of the fragment, as the SSU fragment is almost twice the length as the LSU fragment. Alternatively it could be due to lower selection pressure, in effect higher mutation rate, in the D8-D10 region compared to the SSU region. Murray et al found that the D1-D6 LSU region in dinoflagellates had a substitution rate 4-8\% times that of the SSU region \citep{murray2005improving}. 
Whether this faster substitution rate applies to the D8-D10 region is unknown, however the possibility should be taken into consideration. \\
Morphologically \emph{G. lapillus} is distinct from \emph{G. pacificus} due to it's smooth cell surface \citep{chinain1999morphology} while \emph{G. lapillus} is discrete from \emph{G. toxicus} based on the broad 2'''' plate of \emph{G. toxicus} \citep{litaker2009taxonomy}.\\

Berdalet et al postulated that species of Gambierdiscus may be cosmopolitan, and the fact that some species were reported from only one or few locations was likely due to under sampling \citep{litaker2010global}. 
It has been postulated that \textit{Gambierdiscus} strains of the same species  in warmer areas display higher toxicity than those in cooler regions \citep{bomber1989epiphytism}. 
To test whether genetic divergence exists between strains from the same species from different localities, where possible a range of isolates from a variety of geographic locations were included in the phylogenetic study. 
The phylogenetic analysis of the D8-D10 LSU rDNA (Fig. ~\ref{fig:HGD8D10})  \emph{G. belizeanus}, \emph{G. caribaeus}, \emph{G. carpenteri},  \emph{G. carolinianus} and \emph{G. toxicus} showed strains of one locality dispersed alongside strains from another geographic origin, rather than strains from different locations clustering together. 
Examining the strain distribution in light of their geographic origin for the SSU rDNA fragment (Fig. ~\ref{fig:HGSSU}) shows a similar trend. 
Strains of \emph{G. carpenteri}, \emph{G. caribaeus} and \emph{G. toxicus} showed closer relationship to geographically removed strains than ones from their own location.
This indicates that the rDNA fragments do not show regional level divergence intraspecifically. 
Regions within rDNA are likely too conserved to show fine scale biogeographical patterns, and more investigation using tools such as microsatellites or RADseq may reveal regional differences.
%The indication that geography does not facilitate divergence within \emph{Gambierdiscus} needs to be tested with concatenated multi gene phylogenetic analysis and ideally with genes involved in toxin production.\\
%genetic distsance calculation like in fraga 14?

CFP is endemic to the GBR in Australia, yet the particular species of \textit{Gambierdiscus} that are directly correlated with CFP accumulation in local fish are still not known.
\emph{G. belizeanus} was found at Heron Island, GBR, \citep{murray2014molecular}, in which CTX and MTX production in \emph{G. belizeanus} has been indicated via bioassay \citep{chinain2010growth,holland2013differences}. 
Hence the ecosystem around Heron Island is likely part of the ciguateric web (in effect a food web where CTXs are found at different trophic levels through bioaccumulation) in the GBR, a decisive factor in selecting the sampling site.
To our knowledge, this study constitutes the first report of a clearly identified strain of a \textit{Gambierdiscus} species that consistently produces a probable congener of CTX in Australia. The LC-MS toxin profile of \emph{G. lapillus} shows an undescribed congener of CTX.
The toxicity of \emph{G. lapillus} is equally unusual. 
Whole cell extract MBAs were conducted to query the toxicity as it would be encountered in an environmental setting. The pathology of the mice was unusual and not indicative of previously reported MTX or CTX intoxication (Pers. comm. Rex Munday). 
Whether the abnormal pathology is due to the unknown CTX congener or another toxin produced by this species is unknown. 
Further fractionation of whole cell extracts into aqueous methanol and dichloromethane partitions, which are expected to contain MTX from CTX respectively \citep{satake1993structure} could be conducted to elucidate which toxin is likely causing the pathology.
The toxicity of \emph{Gambierdiscus} spp. in culture have been observed to fluctuate wildly (pers. comm. Tim Harwood). 
This was observed in the highly variable toxicity of \emph{G. lapillus} isolates in the MBA, where the toxicity of HG6 was 128- to 190-fold as high as HG4.  
The recorded intra species toxicity variation of \emph{G. polynesiensis} via MBA was only 2-fold \citep{chinain2010growth}. Proposed reasons for changing toxin production are phosphorous limitation in the media, different stages of the growth phase and growth rate \citep{sperr1996variation,chinain2010growth}. 
However the HG4, HG6 and HG7 were grown under the same conditions and harvested simultaneously which indicates an intra strain variation in toxin production rather than external influence on toxin production regulation.\\
% whole cell extracts to mimc enviro - incl MTX because Kohli et al. Disprove  Litaker et al 2010


%Sampling around Australia has been limited (fig. ~\ref{fig:OzSites}) and constitute single points in time. As \emph{Gambierdiscus} spp. tend to co-habitate, it is likely that other \emph{Gambierdiscus} spp. are present at Heron Island which we were unable to establish in culture.

%\emph{Gambierdiscus} is a genus within the order \emph{Gonyaulacales}, whose evolutionary relationship has not yet been resolved \citep{gentekaki2014large}. To understand when and where the threat of CFP originates, it is essential to identify the species that produce the toxin at the base of the food chain. Currently \emph{Gambierdiscus} spp. have been implicated, whether the production of CTX and MTX is restricted to a \emph{Gambierdiscus} spp. or if it  persists in closely related genera is unknown. The identification of the genes involved in the toxin production is essential for monitoring CFP.

 \paragraph{\textbf{\emph{Gambierdiscus lapillus} sp. nov. Kretzschmar, Hoppenrath et Murray}}
 \paragraph{Description:} Cells anterioposteriorly compressed and round to ellipsoid in apical view. 
 Size of cells smaller than all other characterised \emph{Gambierdiscus} spp., dorsoventral depth 40.6 $\pm$3.3 $\mu$m, width 39 $\pm$3.2 $\mu$m and depth to width ratio 1.04 $\pm$0.06. 
 The cell surface is highly areolated with pores. 
 The plate formula is Po, 4', 0a, 6'', 6c, 8s, 6''', 1p, 2''''. 
Apical pore complex Po is oval with a fish-hook shaped slit central of epitheca. Distinguishing feature to \emph{G. belizeanus}, closest morphological relative, is hatchet-shaped 2' plate. Largest apical plate is hatchet-shaped 2' plate, then pentagonal 3' plate and smallest hexagonal 1' plate. Largest precingular plate is 3'', then 4'', 5'', 6'', 2'' plates, 1'' plate is smallest. 
Largest postcingular plate is the 4''' plate, then 2''', 3''', 5''' and smallest 1''' plates. Antapical 1'''' plate is smaller than the 2'''' plate, both quadrangular. Pentagonal and narrow 1p plate. 
Deep and excavated sulcus bordering on cingulum. Plastids present, dense, distributed throughout cell. 
 \paragraph{Etymology:} %The epithet refers to disposition of one of the strains, HG6. The growth rate exceeded all other strains in culture at 27 $^{\circ}$C and subsequently terminated. The colloquialism that it lived fast and died young was applied, the strain was designated the rock and roll strain or \emph{G. lapillus} (Latin petram = rock).
\paragraph{Holotype:} Fig. xx; SEM-stub (designation xxx) deposited at Senckenberg am Meer, German Centre for Marine Biodiversity Research, Centre of Excellence for Dinophyte Taxonomy, Germany.  \\

\paragraph{Isotype:}Lugol-fixed subsample of strain HG 7 (designation xxx) deposited at the Senckenberg am Meer, German Centre for Marine Biodiversity Research, Centre of Excellence for Dinophyte Taxonomy,, Germany.  \\

\paragraph{Type locality:} Heron Island (23.4420$^{\circ}$ S, 151.9140$^{\circ}$ E), Great Barrier Reef, Australia, South Pacific Ocean.
%\paragraph{Distribution:} Marine, associated as epiphyte to seaweeds. Observed at Heron Island, Australia, attached to  \emph{Halimeda} sp. and \emph{Padina} sp. with an average sea surface temperature was 23.8 $^{\circ}$C. October 2013 fromThe cultures HG4, HG5, HG6, HG7 and HG 26 were isolated in July 2014 from 
\paragraph{Further information:} Type strain deposited in the culture collection … number … Molecular characterization of strain HG 7, SSU (KU558932,) and LSU (KU558928).
\paragraph{Remarks:} \emph{Gambierdiscus lapillus} can be genetically identified rDNA sequences deposited in GenBank SSU: HG1 KU558929, HG4 KU558930, HG6 KU558931, HG26 KU558933; and D8-D10 LSU: HG1 KU558925, HG4 KU558926, HG6 KU558927. 

\newpage
\section{Acknowledgements}
Thank you to Dr Rex Munday of AgResearch for conducting the toxicology investigation. 
Janis Ortgies, Senckenberg am Meer, is thanked for help with the SEM preparations.

\section{Conflict of interest disclosure}
The authors report no conflict of interest in conducting this study.
\newpage

\begin{figure} 
% \includegraphics[scale=.85]{oz-gamb-grey-map.png} 
\caption{(A) Sites where Gambierdiscus isolates have been reported in Australia: site 5 indicates Wapengo Lagoon, NSW; site 6 indicates Merimbula, NSW; site 7 indicates Eden, NSW; and site 8 indicates Exmouth, WA; (B) Great Barrier Reef:​ Site 1 indicates Raine Island, QLD; site 2 indicates Townsville, QLD; Site 3 indicates Heron Island, QLD; Site 4 indicated Platypus Bay, QLD; (C) Heron Island: Sampling site for this study.} 
\label{fig:OzSites}
\end{figure}
\begin{table}
\caption{\emph{Gambierdiscus} spp. identified around Australia, including toxicological data available. N/D denotes not detected. NA denotes not attempted.}
\label{tbl:OzTable}
\begin{tabular}{ | p{3.3cm} | p{4cm} | p{4.5cm} | p{2.3cm} | }
\hline
 \textbf{\emph{Gambierdiscus} sp.} & \textbf{Site identified from fig. ~\ref{fig:OzSites}} & \textbf{Toxicity}  & \textbf{LC-MS profile}  \\
 \hline
 \emph{G. belizeanus}  & Site 3: Heron Island, GBR \citep{murray2014molecular} & CTX +ve via RBA \citep{chinain2010growth}; MTX +ve via HELA \citep{holland2013differences} & NA  \\
 \hline
 \emph{G. carpenteri} &Site 5: Wapengo Lagoon, NSW \citep{kohli2014high}; site 6:  Merimbula, NSW  \citep{kohli2014high}& MTX +ve via MBA \citep{kohli2014high} & N/D \citep{kohli2014high}\\
 \hline
 \emph{G. toxicus} & Site 2: Townsville, GBR \citep{hallegraeff2010algae}; Site 7: Eden, NSW \citep{hallegraeff2010algae} & CTX +ve via RBA \citep{chinain2010growth}; MTX +ve via MBA \citep{chinain1999morphology} & NA \\
  \hline
  \emph{Gambierdiscus} sp. & Site 8: Exmouth, WA \citep{kohli2014cob}& NA & NA \\
  \hline
 \emph{F. yasumotoi}  & Site 1: Raine Island, GBR \citep{murray2014molecular}; Site 2: Townsville, GBR \citep{murray2014molecular}; site 3: Heron Island, GBR \citep{murray2014molecular}& MTX +ve via MBA \citep{holmes1998gambierdiscus} & N/D \citep{rhodes2014gambierdiscus}\\
  \hline
\end{tabular}
\end{table}

\FloatBarrier
\begin{table}
\caption{List of primers used for phylogenetic analysis of \emph{Gambierdiscus} strains, synthesised by Integrated DNA technologies.}
\label{tbl:PrimerTable}
\begin{tabular}{  | p{2cm} | p{7.5cm} | p{2.5cm} | p{2cm} | }
\hline
\textbf{Name} & \textbf{Sequence (5'-3')} & \textbf{Purpose} & \textbf{Reference} \\
\hline
    \multicolumn{4}{| c |}{\textbf{LSU D8-D10 region}}\\
    \hline
   FD8   & GGATTGGCTCTGAGGGTTGGG & Amplification \& sequencing & \citep{chinain1999morphology} \\
   \hline
 GLD8\_421F   & ACAGCCAAGGGAACGGGCTT & Sequencing & \citep{nishimura2013genetic} \\
 \hline
 GLD8\_677R   & TGTGCCGCCCCAGCCAAACT & Sequencing & \citep{nishimura2013genetic} \\
 \hline
   RB   & GATAGGAAGAGCCGACATCGA & Amplification \& sequencing &\citep{chinain1999morphology}  \\
    \hline
  \multicolumn{4}{| c |}{\textbf{SSU region}}\\
    \hline
 18ScomF1 & GCTTGTCTCAAAGATTAAGCCATGC & Amplification \& sequencing & \citep{zhang2005phylogeny} \\
 \hline
 18ScomR1  & CACCTACGGAAACCTTGTTACGAC & Amplification \& sequencing &  \citep{zhang2005phylogeny}  \\
 \hline
 Dino18SF2  & ATTAATAGGGATAGTTGGGGGC & Sequencing &  \citep{zhang2008mitochondrial}\\
 \hline
 Dino18SR1    & GAGCCAGATRCDCACCCA & Sequencing &  \citep{zhang2008mitochondrial}\\ 
 \hline
G10'F    & TGGAGGGCAAGTCTGGTG & Sequencing & \citep{nishimura2013genetic} \\
\hline
G18'R    & GCATCACAGACCTGTTATTG & Sequencing &  \citep{litaker2005reclassification} \\
 \hline
\end{tabular}
\end{table}

\FloatBarrier
\begin{longtable}{ | p{5cm} | p{2cm} | p{6cm} | p{2cm} |}
\caption{Morphological meassurements for all characterised \emph{Gambierdiscus} spp. and \emph{Fukoyoa} spp. Values in parentheses are $\pm$ standard deviation. NA stands for not available.}\\
\hline
\label{tbl:GlobalSizeTable}
\textbf{Taxa} &  Cell size ($\mu$m) (depth-width-ratio) & Morphologically distinctive features & References  \\
 \hline
\textit{G. australes }M.Chinian \& M.A.Faust	& (86$\pm$5.1)-(77$\pm$3.7)-(1.12) &  Narrow 2'''' plate; smooth cell surface; 2’ rectangular shaped; smaller than \textit{G. excentricus} & \citep{litaker2009taxonomy,chinain1999morphology}  \\
 \hline
 \textit{G. belizeanus }M.A.Faust	& (63$\pm$2.2)-(58$\pm$2.5)-(1.07$\pm$0.08) & Narrow 2'''' plate; heavily reticulate-foveate cell surface & \citep{chinain1999morphology,litaker2009taxonomy,faust1995observation} \\
 \hline
 \textit{G. caribaeus} Vandersea, Litaker, Faust, Kibler, Holland \& Tester	& (82.2$\pm$7.6)-(81.9$\pm$7.9)-(1) & Broad 2'''' plate; 2 rectangular shaped; symmetric 4'' & \citep{litaker2009taxonomy}  \\
 \hline
 \textit{G. carolinianus} Litaker, Vandersea, Faust, Kibler, Holland \& Tester & (78.2$\pm$4.8)-(87.1$\pm$7.1)-(0.9) & Broad 2'''' plate; 2' hatchet shaped; dorsal end 2'''' oblique; larger cell size than \textit{G. polynesiensis} & \citep{litaker2009taxonomy} \\
 \hline
\textit{G. carpenteri} Kibler, Litaker, Faust, Holland, Vandersea \& Tester &	(81.7$\pm$6.4)-(74.8$\pm$5.9)-(1.09) & Broad 2'''' plate; 2' rectangular shaped; asymmetric 4” & \citep{litaker2009taxonomy}  \\
 \hline
\textit{G. excentricus	}S.Fraga & (97.8$\pm$8.0)-(83.0$\pm$10.0)-(1.18) & Narrow 2'''' plate; smooth cell surface; 2' rectangular shaped; cell size bigger than \textit{G. australes} & \citep{litaker2009taxonomy} \\
 \hline
\textit{G. pacificus} M.Chinain \& M.Faust& (70$\pm$4.7)-(63$\pm$3.6)-(1.11) & Narrow 2'''' plate; smooth cell surface; 2’ hatchet shaped & \citep{chinain1999morphology,litaker2009taxonomy} \\
 \hline
\textit{G. polynesiensis} M.Chinain \& M.Faust & (69$\pm$4.5)-(69$\pm$3.6)-(1.1) & Broad 2'''' plate; 2' hatchet shaped; dorsal end 2'''' oblique; smaller cell size than \textit{G. carolinianus} &	\citep{chinain1999morphology,litaker2009taxonomy} \\ 
 \hline
\textit{G. scabrosus} T.Nishimura, Shinya Sato \& M.Adachi	& (63.2$\pm$5.7)-(58.2$\pm$5.7)-(1.09$\pm$0.07) & Narrow 2'''' plate; reticulate-foveate; asymmetric 3” plate & \citep{nishimura2014morphology,nishimura2013genetic,kuno2010genetic}  \\
 \hline
 \textit{G. silvae}	S.Fraga \& F.Rodr{\'i}guez & (69$\pm$8.0)-(64$\pm$9.0)-(1.1) & Narrow 2'''' plate; heavily reticulate-foveate & \citep{fraga2014genus,litaker2010global} \\
 \hline
\textit{G. toxicus} R.Adachi \& Y.Fukuyo & (93$\pm$5.7)-(83$\pm$2.3)-(1.12) & Broad 2'''' plate; 2; hatchet shaped; dorsal end 2'''' pointed & \citep{litaker2009taxonomy,adachi1979thecal,chinain1997intraspecific,richlen2008phylogeography}\\
 \hline
 \emph{Gambierdiscus} sp. type 4	& (67$\pm$5)-(68.8$\pm$5.6)-(0.97) & Evenly distributed pores; larger than \textit{Gambierdiscus sp. type 5}; 2’ hatchet shaped; broad 2'''' plate 
 & \citep{xu2014distribution}\\
 \hline
 \emph{Gambierdiscus} sp. type 5 & (54.8$\pm$4.6)-(53.7$\pm$6.3)-(1.02) & Evenly distributed pores; smaller than \textit{Gambierdiscus sp. type 4}; 2’ rectangular shaped; narrow 2'''' plate &  \citep{xu2014distribution} \\
 \hline
 \textit{F. paulensis} F.Gómez, D.Qiu, R.M.Lopes \& S.Lin & (50$\pm$3)-(45$\pm$2)-(1.2) & Smaller than \textit{F. yasumotoi} and larger than \textit{F. rutzleri} & \citep{gomez2015fukuyoa} \\
 \hline
\textit{F. rutzleri }(Faust, Litaker, Vandersea, Kibler, Holland \& Tester) F.Gómez, D.Qiu, R.M.Lopes \& S.Lin& (43$\pm$5.1)-(36$\pm$6.0)-(1.19) & Smaller than \textit{F. yasumotoi}; cell size less than 42μm &  \citep{litaker2009taxonomy}\\
 \hline
\textit{F. yasumotoi }(M.J.Holmes) F.Gómez, D.Qiu, R.M.Lopes \& S.Lin& (56.8$\pm$5.6)-(51.7$\pm$5.6)-(1.1) & Larger than \textit{F. ruetzleri}; cell size exceeds 42 μm & \citep{litaker2009taxonomy,holmes1998gambierdiscus}  \\
 \hline
\textit{G. lapillus}  & (40.6$\pm$3.3)(39$\pm$3.2)-(1.04$\pm$0.06) & Narrow 2'''' plate; heavily reticulate-foveate; hatchet-shaped 2' plate & This study\\
   \hline
\end{longtable}
\FloatBarrier

\FloatBarrier 
\begin{figure} 
%\includegraphics[scale=.02]{lapillus-LM.png} 
\caption{Light micrographs of \emph{Gambierdiscus lapillus}, strains HG4 (A); HG6 (B); HG7 (C); and HG26 (D). Scale bar equal 10 $\mu$m.​} 
\label{fig:PetLM}
\end{figure} 

\FloatBarrier 
\begin{figure} 
%\includegraphics[scale=.22]{MH_epitheca.jpg} 
\caption{SEM micrographs of the epitheca of \emph{Gambierdiscus petramus}, (A, B, C, D) show all plates in full; (E, F, G) show apical pore; (H) shows apical pore of recently divided cell. Scale bar equal 20 $\mu$m unless otherwise specified.} 
\label{fig:epiSEM}
\end{figure} 
\FloatBarrier 

\FloatBarrier 
\begin{figure} 
%\includegraphics[scale=.22]{MH_hypotheca.jpg} 
\caption{SEM micrographs of the hypotheca of \emph{Gambierdiscus petramus}, (A, B) show the hypotheca; (C, D, E) show the smaller hypothecal plates close to the sulcus; (F) side view of cell; (G) sulcal plates from hypothecal view. Scale bar equal 20 $\mu$m.} 
\label{fig:hypoSEM}
\end{figure} 
\FloatBarrier

\FloatBarrier 
\begin{figure} 
%\includegraphics[scale=.22]{MH_ornamentation.jpg} 
\caption{SEM micrographs of \emph{Gambierdiscus petramus} showing plate ornamentation. Scale bar equal 20 $\mu$m unless otherwise specified.} 
\label{fig:ornSEM}
\end{figure} 
\FloatBarrier

\FloatBarrier 
\begin{figure} 
%\includegraphics[scale=.1]{size_chart-grey.png} 
\caption{Visual representation of the size difference between \emph{Gambierdiscus} species (table ~\ref{fig:SizeGraph}). Error bars denote standard deviation of meassurements for \emph{G. lapillus}. Data taken from publications as follows: a) Chinain et al \citep{chinain1999morphology}; b) Fraga et al \citep{fraga2014genus}; c) Faust et al \citep{faust1995observation}; d) Gomez et al \citep{gomez2015fukuyoa}; e) Litaker et al \citep{litaker2009taxonomy}; f) Nishimura et al \citep{nishimura2014morphology}; and g) Xu et al \citep{xu2014distribution}} 
\label{fig:SizeGraph}
\end{figure} 


\FloatBarrier 
\begin{figure} 
%\includegraphics[scale=.1]{SSU_complex_geo_merge-f.png} 
\caption{Maximum likelihood phylogeny of \textit{Gambierdiscus} species/phylotypes of the SSU rDNA region. Nodal support are Bayesian posterior probability (PP) and bootstrap (BS) values obtained from Bayesian inference analysis and maximum likelihood analysis, respectively. Nodes with strong support (PP/BS = 1.00 / 100) are shown as thick lines.}
\label{fig:HGSSU} 
\end{figure} 
\FloatBarrier 

%\newpage
\begin{figure} 
%\includegraphics[scale=.1]{D8D10_complex_geo_merge-F.png} 
\caption{Maximum likelihood phylogeny of \textit{Gambierdiscus} species/phylotypes of the LSU D8-D10 rDNA region. Nodal support are Bayesian posterior probability (PP) and bootstrap (BS) values obtained from Bayesian inference analysis and maximum likelihood analysis, respectively. Nodes with strong support (PP/BS = 1.00 / 100) are shown as thick lines.} 
\label{fig:HGD8D10}
\end{figure} 


\FloatBarrier
\begin{table}
\caption{Morphological meassurements of \emph{G. lapillus }sp. nov. strains collected from Heron Island, Australia. Between 30 to 36 cells were counted per strain. Values in parentheses are $\pm$ standard deviation. \textbf{--- for supplementary data}}
\label{tbl:SizeTable}
\begin{tabular}{ | p{2cm} | p{2.5cm} | p{2.5cm} | p{2.5cm} | }
\hline
 \textbf{Strain} & \textbf{D ($\mu$m)} & \textbf{W ($\mu$m)}  & \textbf{D:W ratio}  \\
 \hline
 HG4  & 39 ($\pm$2.6) & 38.1 ($\pm$3.1) & 1.03 ($\pm$0.06) \\

 HG6  & 43.2 ($\pm$3.0) & 40.9 ($\pm$3.0) & 1.06 ($\pm$0.04)  \\

 HG7  & 39.2 ($\pm$2.8) & 38.4 ($\pm$2.8) & 1.02 ($\pm$0.07)  \\

 HG26  & 40.7 ($\pm$2.7) & 38.5 ($\pm$3.2) & 1.06 ($\pm$0.05) \\
 \hline
\end{tabular}
\end{table}
\FloatBarrier

\FloatBarrier 
\begin{figure} 
%\includegraphics[scale=.22]{MH_sulcus-HG26-HG4.jpg} 
\caption{Supplementary:SEM micrographs of \emph{Gambierdiscus petramus},(A, B, C, D, E, F) depict the sulcal plates; (C) also shows the cingulum. Scale bar equal 20 $\mu$m unless otherwise specified.} 
\label{fig:sulcSEM}
\end{figure} 
\FloatBarrier

\FloatBarrier 
\begin{figure} 
%\includegraphics[scale=.22]{MH_sulcus-HG26.jpg} 
\caption{Supplementary:SEM micrographs of \emph{Gambierdiscus petramus},(A) General cell view in lateral view; (B) cingular plates (with the suture between C3
and C4 visible); (C) Po plate with a small pore next to the large depressions. Scale bar equal 20 $\mu$m unless otherwise specified.} 
\label{fig:latSEM}
\end{figure} 
\FloatBarrier

\FloatBarrier 
\begin{figure} 
%\includegraphics[scale=.22]{MH_variability.jpg} 
\caption{Supplementary:SEM micrographs of \emph{Gambierdiscus petramus}, (A, B) show the variable teardrop like shape seen in some cels; (C, D, E) show the variability observed in the 2' plate, rectangular, two sides boardering on the 2'' plate and 2 sides boardering on the 3'' respectively. Scale bar equal 20 $\mu$m unless otherwise specified.} 
\label{fig:varSEM}
\end{figure} 
\FloatBarrier
\newpage
\bibliographystyle{acm}
\bibliography{references.bib}


\end{document}
