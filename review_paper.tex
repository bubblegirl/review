\documentclass[12pt]{article}
\usepackage[hcentering,bindingoffset=20mm]{geometry}
\usepackage{placeins}
\usepackage[numbib]{tocbibind}
\usepackage{rotating}
\usepackage[square,sort,comma,numbers]{natbib}
\usepackage{graphicx}
\usepackage{tabularx}
\linespread{1.3}
%\fontsize{8cm}{1.3em}\selectfont

\usepackage{longtable}
\usepackage{lscape}

\addtolength{\textwidth}{2cm}
\addtolength{\hoffset}{-1cm}


\addtolength{\textheight}{2cm}
\addtolength{\voffset}{-1cm}
\setlength{\parindent}{0pt}

\title{\textbf{The known aspects of \emph{Gambierdiscus} spp. leading to ciguatera fish poisoning}}
%\author{Anna Liza Kretzschmar, Gurjeet Kohli, Hazel Farrel, Shauna Murray}
\date{}
%\usepackage{cite}

\begin{document}
\maketitle
\tableofcontents
\section{Abstract}

\section{Introduction}
% HABs, testing shit
Harmful algal bloom-forming dinoflagellates have become an increasingly observed phenomena on the global scale. They cause oxygen depletion which limits its availability to other organisms, block sunlight for the benthos and potentially produce harmful compounds \cite{grandjean2008centers}, an example for the latter is \emph{Gambierdiscus spp.} \cite{lehane2000ciguatera}. This genus is especially prevalent in the literature as they produce toxins responsible for ciguatera fish	poisoning (CFP)	in humans through bioaccumulation in tropical and sub-tropical fish. It is classically reported in regions bordering on the Pacific and Indian ocean, however in the last decade \emph{G. toxicus} has been found as far into the northern hemisphere as the Mediterranean Sea. This expansion of habitat may be due t global warming \cite{aligizaki2008morphological}. \
A study correlating CFP related calls from the US national poison centre data with storms and sea surface temperature, this showed that sea surface temperature was a relevant factor for CFP incidence and extrapolate that climate change related sea temperature rise could escalate CFP by 200 - 400 \% in the US alone \cite{garces2012habitat}

% history, species confusion, differenc ein toxicity within same species
Vast diversity of phytoplankton recently discovered at varying taxonomic levels \cite{simon2009diversity}. 
New molecular methods promise many more species to be documented \cite{murray2012genetic,murray2012transcriptomics}.
Currently only 160 described benthic dinoflagellates \cite{taylor2008dinoflagellate} %check plus explain benthic
Benthic dinoflagellate implicated in CFP in 1977 sparked interest in this phylum \cite{yasumoto1977finding}.
Genus \emph{Gambierdiscus} named after location first sampled, Gmabier Islands in French Polynesia, long believed to be monophyletic with \emph{Gambierdiscus toxicus} \cite{adachi1979thecal} %when were more species discovered?
\emph{Gambierdiscus} spp. produce ciguatoxins (CTX) and maitotoxins (MTX) \cite{chinain1997intraspecific,chinain1999morphology,chinain1999seasonal,chinain2010growth,holmes1998gambierdiscus,rhodes2010toxic,fraga2011gambierdiscus,holland2013differences}
CFP most common nonbacterial illness associated with fish consumption \cite{friedman2008ciguatera} %rephrase
CFP cases 50,000 - 500,000 /year globally \cite{fleming1998seafood}.
CTX orally accumulative from herbivorous and carnivorous fish consumption causing CFP \cite{bagnis1979clinical,gillespie1987possible,sims1987theoretical} %MTX implicatd in any of these or was that just Gurjeet setting up his work?
Increase in HAB events reported globally \cite{hallegraeff2010ocean} % HAB impact on enviro and humans: \cite{garces2012habitat}
CFP increase of 60\% in last decade in Pacific Islands \cite{skinner2011ciguatera}
New \emph{Gambierdiscus} species discovered, evidence that they may all have individual toxin profiles \cite{chinain2010growth,fraga2011gambierdiscus,holland2013differences}
In other dinoflagellates, toxin producton varies at species level and within species.
HAB species are monitored for early detection of CFP outbreak.
Climate change is impacting the marine environment - a model ocean for changing conditions is the Mediterranean Sea \cite{lejeusne2010climate}, where \emph{Gambierdiscus} has extended it's habitat to since the early 2000s \cite{aligizaki2008morphological}.

\section{Morphology and phylogenetics of \emph{Gambierdiscus}}
% table 1.1



\begin{sidewaystable}[!htbp]
\caption{Presently characterised \emph{Gambierdiscus} spp. and their taxonomic and genetic identifications.}
\begin{tabular}{ |  p{4cm} | p{4cm} | p{4cm} | p{4cm} | p{4cm} | }
\hline
\textbf{Species} & \textbf{Cell size (depth-width-length)($mu$m)} & \textbf{Morphological characteristics} & \textbf{Genetics (Genbank \#)} & \textbf{References} \\
\hline
 \emph{G. } & $\pm$ &  &  & \\
\hline
 \emph{G. } &  &  &  & \\
\hline
 \emph{G. } &  &  &  & \\
\hline
 \emph{G. } &  &  &  & \\
\hline
 \emph{G. } &  &  &  & \\
\hline
 \emph{G. } &  &  &  & \\
\hline
 \emph{G. } &  &  &  & \\
\hline
 \emph{G. } &  &  &  & \\
\hline
 \emph{G. } &  &  &  & \\
\hline
 \emph{G. } &  &  &  & \\
\hline
 \emph{G. } &  &  &  & \\
\hline
 \emph{G. } &  &  &  & \\
\hline
 \emph{G. } &  &  &  & \\
\hline
 \emph{G. } &  &  &  & \\
\hline
 \emph{G. } &  &  &  & \\
\hline
 \emph{G. } &  &  &  & \\
\hline
 \emph{G. } &  &  &  & \\
\hline
 \emph{G. } &  &  &  & \\
\hline
 \emph{G. } &  &  &  & \\
\hline
 \emph{G. } &  &  &  & \\
\hline
\end{tabular}
\end{sidewaystable}
\FloatBarrier
\section{Toxins produced by \emph{Gambierdiscus}}
%table 1.3 - amend with new info
% toxin mod in the food chain - or seperate section?
The agents triggering ciguatoxin production in algal blooms and whether this is dependent on environmental cues or strain dependent is unclear \cite{lewis2006ciguatera}.
\emph{G. toxicus} has been shown to produce a range of toxins other than ciguatoxins \cite{holmes1994purification,murata1993structure} Furthermore, other species within this genus produce ciguatoxin like substances, such as \emph{G. polynesiensis}, \emph{G. australes} and \emph{G. pacificus} \cite{roeder2010characteristic}. Understanding the molecular evolution and ecology of the toxins produced by \emph{Gambierdiscus} spp. are essential to dealing with harmful algal blooms and their impact on humans and the environment. \\
Studies pertaining to the toxcicity of \emph{Gambierdicus} towards copepods and their nauplii were conducted \cite{lee2014toxicity}, as copepds feed on microalgae attached to macroalgae. They could constitute the primary consumers in the food chain, which can lead to toxin accumulation \cite{raisuddin2007copepod} resulting in CFP. The impact of \emph{G. yasumotoi} on the model copepod \emph{Tigriopus japonicus} was assessed in a study conducted by Lee et al, finding that the presence of \emph{G. yasumotoi} cells, or to a lesser extent growth media stock, cause immobility and mortality on adult copepods, where increased \emph{G. yasumotoi} concentration increased the effect. Furthermore nauplii were affected at lower concentrations \cite{lee2014toxicity}.
It has been shown that ciguatera toxins passing through the food chain in the Pacific undergo biotransformation. Some modifications can increase the potency of the toxins up to 10-fold, compared to the toxin produced by \textit{G.toxicus} \cite{lewis2006ciguatera}.
%table 1.3

In xxxx the great breakthrough in understanding CFP was that an algal specied, \emph{G.  toxicus}, produces the toxins that cause the disease and that the concenration increases up the food chain. With the advent of further species discovery in xxx, the nconsistency in txin production between isolates started to make sense - they are different species. The current challenge is to characterise the toxin profiles of the different species, which is often conducted by various assays. However, \emph{Gambierdiscus} spp. produce more than just CTXs - notably MTX is also commonly produced, as well as some others (?). It has, until recently, been assumed in the literature that CTX is the sole culprit in CFP as MTX is water soluble. Bioassays provide an excellent indicator to a species' toxicity, however it does not suffice to elucidate a detailed toxin profile - LC-MS analysis is required. CTX structure structure varies between the Pacific and teh Caribbean, with the Indian ocean structures still to be determined. Symptomlogy of CFP also varies between regions. However if the variance is dependent on  the structural difference, or for example due to different precursor modification during biomagnificatiion, is yet to be determined. Furthermore <insert MTX>. Understanding the molecular evolution and ecology of the toxins, coupled with which species produce them and under what circumstances, are points that are essential to dealing with harmful algal blooms and CFP.  

\subsection{CTX}
The family of CTX toxins consist of lipid soluble polyether ladders which act as sodium channel activators \cite{}. The exact mode of action is yet to be determined - it has been proposed that sodium channel activation is merely a secondary action of the toxin \cite{}. They are orally effective in humans at the pM and mM concentration range \cite{molgo2000ciguatera}, causing an influx of Na$^{+}$ ions and hence spontaneous action potentials in the cell, especially active on voltage sensitive channels along the nodes of Ranvier \cite{sims1987theoretical,mattei1999neurotoxins,lewis1992action,molgo2000ciguatera}. They are characteristically impossible to discern while preparing sea food as they do not cause any obviously recognisable differences to healthy specimens and are thermostable \cite{withers1982ciguatera}. Hence CTX presence is usually only diagnosed retrospectively from CFP symptoms in conjunction with seafood consumption \cite{}. Samples of the food consumed are rarely obtained retrospectively. Due to diagnosis depending on the deductive capability of the medical professional, CFP is predicted to be widely under reported with as little of 20 \% of cases being reported \cite{}. The disease can manifest as over 180 symptoms, with varying levels of severity, and differ even between individuals at the same time of exposure \cite{}. The effects of the disease appear to be cumulative as severity increases with previous exposure \cite{}. Commonly a couple hours after consumption, the initial symptoms are gastrointestinal which can be followed by cardiovascular and neurological manifestations \cite{sims1987theoretical}. They can persist between weeks to years, with the longest recorded case of xx decades \cite{}. Recurrence is a common phenomenon induced by a variety of factors such as alcohol, fish or meat consumption as well as intercourse \cite{}. 

The group of CTX toxins is diverse, as their structure varies geographically and  each with a number of congeners. Currently they are prefixed by their origin, where CTX from the Pacific, Caribbean and Indian ocean are designated P-CTX, C-CTX and I-CTX respectively \cite{}. %Rather than just 
P-CTX further divided into type I and type II on structural basis \cite{legrand1997two}, folowed by all C-CTXs as type III \cite{}. The atomic structure of I-CTXs have not been elucidated to such an extent as to assign a type \cite{hamilton2002multiple,hamilton2002isolation}.  \\
Type I P-CTXs consist of 13 rings with 60 C atoms \cite{murata1990structures,lewis1991purification,lewis1993origin} and the first completely defined CTX was isolated from moray eels and designated P-CTX-1B \cite{murata1990structures} or also described as CTX-1 \cite{lewis1991purification}. P-CTX-2 and P-CTX-3 were isolated from the same extracts but exhibited an alternate structure and toxicity in mice \cite{lewis1991purification}. It has been suggested by twlo research groups that P-CTX-1, P-CTX-2 and P-CTX-3 may be derivatives from the dinoflagellate precursors CTX-4A and CTX-4B (also GTX-4B by Murata et al \cite{murata1990structures}) \cite{lewis1993origin,yasumoto2000structural}. So far, P-CTX-4A and P-CTX-4B have been isolated from \emph{Gambierdiscus polynesiensis} culture extracts \cite{chinain2010growth}, while P-CTX-1, P-CTX-2 and P-CTX-3 have not. \\
The structure for type II P-CTX-3C has been resolved to consist of 13 ring and 57 C atoms and has been isolated first from \emph{Gambierdiscus} culture \cite{satake1993structure}, then from \emph{G. polynesiensis} \cite{chinain2010growth}. There are two known congeners of P-CTX-3C which have been isolated - 49-epi-CTX-3C (or CTX-3B by Chinain et al \cite{chinain2010growth}) and M-seco-CTX-3C from \emph{Gambierdiscus} \cite{satake1993structure} and \emph{G. polynesiensis} \cite{chinain2010growth}.

Two congeners have been isolated 4 Also isolated from Moray eel was 2,3-dihydroxy-CTX-3C (or CTX-2A1) and 51-hydroxy-CTX-3C \cite{satake1998isolation} which jave been suggested to be oxygenated metabolites of CTX-3 \cite{yasumoto2000structural}
C-CTX: 14 rings and 62 C atoms \cite{vernoux1997isolation,lewis1998structure,pottier2003identification,pottier2002characterisation}. Multiple congeners isolated from carnivorous fish C-CTX-1, C-CTX-2, C-CTX-1141, C-CTX-1127, C-CTX-1143, C-CTX-1157, C-CTX-1159 \cite{vernoux1997isolation,lewis1998structure,pottier2003identification,pottier2002characterisation}. No C-CTXs isolated from \emph{Gambierdiscus} sp. unlike P-CTX but \emph{G. excentricus} shown to produce CTXs from this region. CTX profile in process of characterisation \cite{fraga2011gambierdiscus}.
I-CTX: most recently discovered. C-CTX-1, C-CTX-2, C-CTX-3, C-CTX-4 isolated from carnivorous fish. have higher MW than P-CTX and C-CTX \cite{caillaud2010update,hamilton2002multiple,hamilton2002isolation}. Structures of I-CTXs not determined yet \cite{hamilton2002multiple,hamilton2002isolation}, I-CTX-1 toxic to mice via intraperitoneal injection \cite{hamilton2002isolation}

It has been suggested that the evident variety of CTX congeners even in the same eco system is due to modification to the toxin structure as it passes up the food chain \cite{hokama1996human}. This theory is backed by a study conducted by Mak et al, examining the CTX composition of different members of a ciguateric food web \cite{mak2013pacific}.
Symptoms of CFP can vary between geographical locations \cite{molgo2000ciguatera,dickey2010ciguatera}, which likely contributes to the inefficiency of diagnosis. This could be de to the structural difference between CTXs. Hence accurate determination of exact CTX congeners is vital to understanding toxicology, for linked risks and symptoms symptoms of CFP - which is an avenue of investigation that urgently needs attention.


% sodium channel activation causes influx of  ions which causes cell depolarisation \cite{mattei1999neurotoxins,lewis1992action,molgo2000ciguatera}.
% Causes spontaneous action potentials in cells affected, with the potential to cause a variety of symptoms - gastrointestinal, neurological and cardiovascular (only in severe cases) \cite{sims1987theoretical}.
%symptoms can vary between geographical locations \cite{molgo2000ciguatera,dickey2010ciguatera}. could be due to structural differences between toxins - need to characterise CTXs. 
Indigenous population's understanding of ciguateric fish may help prevent CFP. 
%Broadly, CTXs are thermostable, liposoluble cyclic plyether ladders. They are further divided into subgroups based on structure and geography P-CTX (Pacific region), C-CTX (Caribbean region) and I-CTX (Indian Ocean). P-CTX further divided into type I and type II on structural basis \cite{legrand1997two}
type I P-CTX: 13 rings 60 C atoms \cite{murata1990structures,lewis1991purification,lewis1993origin} first structurally completely defined CTX P-CTX1B isolated from marray eels \cite{murata1990structures} or also described as CTX-1 \cite{lewis1991purification}. From same extracts CTX-2 and CTX-3 were also isolated - slightly different structure and toxicity in mice \cite{lewis1991purification}. CTX-1, CTX-2 and CTX-3 may be derivatives from dinoflagellate precursors CTX-4A and CTX-4B (also GTX-4B by Murata et al \cite{murata1990structures}) \cite{lewis1993origin,yasumoto2000structural}. CTX-4A and CTX-4B have been isolated from \emph{Gambierdiscus polynesiensis} culture extracts \cite{chinain2010growth}.
type II P-CTX: CTX-3C has 13 rings 57 C atoms, isolated from \emph{Gambierdiscus} culture \cite{satake1993structure}, then from \emph{G. polynesiensis} \cite{chinain2010growth}. Two congeners have been isolated 49-epi-CTX-3C (or CTX-3B by Chinain et al \cite{chinain2010growth}) and M-seco-CTX-3C from \emph{Gambierdiscus} \cite{satake1993structure} and \emph{G. polynesiensis} \cite{chinain2010growth}. Also isolated from Moray eel was 2,3-dihydroxy-CTX-3C (or CTX-2A1) and 51-hydroxy-CTX-3C \cite{satake1998isolation} which jave been suggested to be oxygenated metabolites of CTX-3 \cite{yasumoto2000structural}
C-CTX: 14 rings and 62 C atoms \cite{vernoux1997isolation,lewis1998structure,pottier2003identification,pottier2002characterisation}. Multiple congeners isolated from carnivorous fish C-CTX-1, C-CTX-2, C-CTX-1141, C-CTX-1127, C-CTX-1143, C-CTX-1157, C-CTX-1159 \cite{vernoux1997isolation,lewis1998structure,pottier2003identification,pottier2002characterisation}. No C-CTXs isolated from \emph{Gambierdiscus} sp. unlike P-CTX but \emph{G. excentricus} shown to produce CTXs from this region. CTX profile in process of characterisation \cite{fraga2011gambierdiscus}.
I-CTX: most recently discovered. C-CTX-1, C-CTX-2, C-CTX-3, C-CTX-4 isolated from carnivorous fish. have higher MW than P-CTX and C-CTX \cite{caillaud2010update,hamilton2002multiple,hamilton2002isolation}. Structures of I-CTXs not determined yet \cite{hamilton2002multiple,hamilton2002isolation}, I-CTX-1 toxic to mice via intraperitoneal injection \cite{hamilton2002isolation}
Experiemntation of P-CTX-1 injeted into rats intravenously showed rapid distribution to tissue and high bioavailibility of toxin via Neuro2a assay 
cite{ledreux2013bioavailability}.
Suggested that the change to CTX precursor structure as it passes up the food chain is responsible for the apparent diversity of CTXs isolated from seafood \cite{hokama1996human}.

\begin{sidewaystable}[!htbp]
\caption{Different known congeners of CTXs and their toxicity.}
\begin{tabular}{ |  p{4cm} | p{2cm} | p{2cm} | p{6cm} | p{6cm} | }
\hline
\textbf{Toxin} & \textbf{Type} & \textbf{Molecular Ion [M +H]$^{+}$} & \textbf{Source} & \textbf{Toxicity (LD50, i.p. mice)} \\
\hline
P-CTX-1, P=CTX-1B & I & 1111.6 & Moray eel (\emph{Gymnothorax javanicus}) \cite{murata1990structures,lewis1991purification} & P-CTX-1 0.35 $\mu$g/kg \cite{murata1990structures}; P-CTX-1B 0.25$\mu$g/kg \cite{lewis1991purification} \\
\hline
P-CTX-2 & I & 1095.5 \cite{lewis1991purification} & Moray eel (\emph{Gymnothorax javanicus}) \cite{lewis1991purification} & 2.3 $\mu$g/kg \cite{lewis1991purification} \\
\hline
 P-CTX-3 & I & 1095.5 \cite{lewis1991purification} & Moray eel (\emph{Gymnothorax javanicus}) \cite{lewis1991purification} & 0.9 $\mu$g/kg \cite{lewis1991purification} \\
\hline
 P-CTX-4A & I & 1061.6 \cite{yasumoto2000structural} & \emph{Gambierdiscus} sp. \cite{yasumoto2000structural}; \emph{G. polynesiensis} \cite{chinain2010growth} & 12$\mu$g/kg \cite{chinain2010growth} \\
\hline
 P-CTX-4B & I & 1061.6 \cite{yasumoto2000structural} & \emph{Gambierdiscus} sp. \cite{yasumoto2000structural}; \emph{G. polynesiensis} \cite{chinain2010growth} & 20$\mu$g/kg \cite{chinain2010growth}\\
\hline
 P-CTX-3C & II & 1023.6 \cite{satake1993structure} &  \emph{Gambierdiscus} sp. \cite{satake1993structure}; \emph{G. polynesiensis} \cite{chinain2010growth} & 2.5$\mu$g/kg \cite{chinain2010growth}\\
\hline
 P-49-epi-CTX-3C & II & 1023.6 \cite{chinain2010growth} & \emph{Gambierdiscus} sp. \cite{satake1993structure}; \emph{G. polynesiensis} \cite{chinain2010growth} & 8$\mu$g/kg\cite{chinain2010growth}\\
\hline
 P-M-seco-CTX-3C & II & 1041.6 \cite{chinain2010growth} &\emph{Gambierdiscus} sp. \cite{satake1993structure}; \emph{G. polynesiensis} \cite{chinain2010growth} & 10$\mu$g/kg \cite{chinain2010growth}\\
\hline
 C-CTX-1 & III & 1141.6 \cite{vernoux1997isolation,pottier2002characterisation} & Horse-eye jack (\emph{Caranx latus}) \cite{vernoux1997isolation,pottier2002characterisation} & 3.6$\mu$g/kg \cite{vernoux1997isolation}\\
\hline
 C-CTX-2 & III & 1141.6 \cite{vernoux1997isolation,pottier2002characterisation}& Horse-eye jack (\emph{Caranx latus}) \cite{vernoux1997isolation,pottier2002characterisation} & Toxic \cite{vernoux1997isolation}\\
\hline
 I-CTX-1 & N/A & 1141.6 \cite{hamilton2002isolation}& Red bass (\emph{Lutjanus bohar}); Red emperor (\emph{Lutjanus sebae}) \cite{hamilton2002isolation} & Toxic \cite{hamilton2002isolation} \\
\hline
\end{tabular}
\end{sidewaystable}
\FloatBarrier

\subsection{MTX}
largest known and highly toxic non-proteinous natural product \cite{yokoyama1988some,murata1993structure}. water-soluble polyether ladder compound first isolated form herbivorous Arcanthurids fish gut in 1976 \cite{yasumoto1976toxicity}. structure and stereochemistry elucidated in the 1990s using stereoscopic studies and partial synthesis .. complex.. \cite{murata1993structure,murata1994structure,satake1995structural,nonomura1996complete,zheng1996complete}.
2 large MTX-1, MTX-2 and 1 small MTX-3 described from different strains of \emph{Gambierdiscus} sp. isolated from Qeensland, Australia \cite{holmes1994purification}. MTX-1 may be synonymous with initially desxribed MTX but unproven. Highly potent calcium channel activator LD$_{50}$ 0.05 $\mu$g/kg, only a couple more potent proteinaceous toxins produced by bacteria \cite{yokoyama1988some,murata1993structure}.
Calcium channel activation of MTX is secondary action due to membrane depolarisation - primary mode of action on mammalian cell unclear \cite{van2000diversity}.
Biophysical characteristics of pacific MTXs different to caribbean MTX \cite{lu2013caribbean}. Carribean MTX not been charactrised, so whether the difference in action is structural is speculative. %check lit
MTX low tendency to accumulate in fish flesh rather than intestines \cite{yasumoto1976toxicity}, however pacific island nationals commonly eat whole fish hence could have part in CFP. Despite long the running assumption that MTX should not be a causative agent, partial or otherwise, of CFP due to the toxin's water solubility. The widely accepted theory was that due to that property, MTX would not be absorbed into the fish flesh in concentrations relevant for bioaccumulation. This was disputed earlier in the year by a fish feeding study conducted by Kohli et al. where snapper were fed juvenile mullet, injected with a known quantity of \emph{G. australes} - a strain characterised to only produce MTX. This study definitively demonstrated MTX presence snapper liver, digstive organs and muscle via LC-MS \cite{kohli2014feeding}. These findings make it imperative that the involvement of MTX in CFP, as well as the mode of action of MTX in humans, is established.   % check for ref about island nation eating practices
MTX easier to detect and quantify than CTX due to sulphate esters using liquid chromatography-electronspray ionisation-mass spectrometry (T. Harwood pers. com.). Solvolysis (desulphonation) reduces toxicity 100-fold or more \cite{murata1991effect}.

MTXs are the largest known non-proteinacious natural product and are highly toxic as potent calcium channel activators active in the xx range. Structurally they are defined as a water soluble, polyether ladder compound and has largely been assumed to be irrelevant in CFP research as it was assumed that it's watersoluble nature would exclude it from beng absorbed into fish flesh. This assumption needs to be reconsidered after a fish feeding study by Kohli et al in 2013 detected MTX in  snapper flesh. It is essential that the role of MTX in CFP is elucidated. 

\begin{sidewaystable}[!htbp]
\caption{Different known congeners of MTXs and their toxicity.}
\begin{tabular}{ |  p{4cm} | p{4cm} | p{4cm} | p{4cm} | p{4cm} | }
\hline
\textbf{Origin} & \textbf{Toxin} & \textbf{Molecular Ion [M +H]$^{+}$} & \textbf{Source} & \textbf{Toxicity (LD50, i.p. mice)} \\
\hline
 Pacific & MTX-1 & 3422 \cite{holmes1994purification,murata1993structure} & \emph{Gambierdiscus} sp. \cite{holmes1994purification} & 0.05$\mu$g/kg \cite{murata1993structure}\\
\hline
 Pacific & MTX-2 & 3298 \cite{holmes1994purification} & \emph{Gambierdiscus} sp. \cite{holmes1994purification} & 0.08$\mu$g/kg \cite{holmes1994purification}\\
\hline
 Pacific & MTX-3 & 1060   \cite{holmes1994purification} & \emph{Gambierdiscus} sp. \cite{holmes1994purification} & Toxic \cite{holmes1994purification} \\
\hline
\end{tabular}
\end{sidewaystable}
\FloatBarrier

\subsection{Other}
%insert gambieroxide

\section{Toxcicity of different \emph{Gambierdiscus} species}
clear evidence of \emph{Gambierdiscus} spp. producing CTX and MTX \cite{murata1990structures,holmes1991strain,satake1993structure,holmes1994purification,satake1996isolation}.
wild and culturable strains \emph{Gambierdiscus} shown not to produce CTX in quantities that can be detected \cite{gillespie1985significance,holmes1990toxicity}.
Most studies pre-date recognition of other \emph{Gambierdiscus} species and hence automatically fall under the caveat of \emph{Gambierdiscus toxicus}. Differentiating/characterising toxin profile for individual species is imperative. \emph{G. belizeanus} discovered 1995 \cite{faust1995observation}, leaving 16 years of research mis-attributed to \emph{G. toxicus}
Table 1.2 shows known toxcicity of each \emph{Gambierdiscus} species as detected via various essays and Liquid chromatography-Mass spectrometry (LC-MS).
\emph{G. polynesiensis}: P-CTX produced type I (CTX-4A, CTX-4B) and type II (CTX-3Cm M-seco-CTX-3C, 49-epi-CTX-3C); P-CTX-3C major toxin produced as detected by LC-MS. Two differnt strains were tested - produce same toxins but in different proportions \cite{chinain2010growth}.
Using human erythrocyte lysis assay (HELA) found haemolytic activity in 56 strains of 6 species \emph{G. belizeanus, G. caribaeus, G. carolinianus, G. carpenteri, Gambierdiscus} ribotype 2. intraspecific toxicity varied marginally between some strains but toxicity of each strain was consitent over 2 year study period \cite{holland2013differences}.HELA assay meassures haemolytic activity, proposed to be a direct correlation to the concentration of MTX present. However no specific link of MTX and haemolysis established in literature. %check that shit out
mouse bioassay (MBA) showed water soluble fraction of \emph{G. polynesiensis} extracts demonstrated toxicity, indicating presence of MTXs \cite{chinain1999morphology}. However toxins with this effect have not been characterised form this strain. %check for update
Caribbean species \emph{G. excentricus} may prduce CTX and MTX due to demonstrated neuroblastoma cytotoxicity assay(N2A) \cite{fraga2011gambierdiscus}, however toxin profile needs verification with LC-MS.
\emph{G. australes} extracts demonstrated toxicity from both lipo-soluble and water-soluble fractions, indicatng presence of both CTX and MTX via MBA. However no CTX detected via LC-MS \cite{rhodes2010toxic}.
Different \emph{G. australes} strain from French Polynesia was shown to produce CTXs in a receptor binding assay (RBA), concentrations low compared to \emph{G. polynesiensis} \cite{chinain2010growth}. (these results are interesting and need further investigation).
Bioassays are important for toxicity determination, however only LC-MS based analysis techniques can determine exact toxin profile of different species - currently this technique has only been applied to 2 \emph{Gambierdiscus} species for toxin profiling. area of research needs urgent attention.
New ribotypes 4 \- 6 at Marakei in the republic of Kiribati and different toxcicity! \emph{Gambierdiscus} from ciguateric site were up to 4 \- fold more toxic than \emph{Gambierdiscus} from reference site \cite{xu2014distribution}

\section{Geographic abundance of \emph{Gambierdiscus}}
% global map with references
%difference in CTXs
use table 1.2?
Distributed in coastal zones at tropical and subtropical latitudes.
Distribution not well understood due to previous assumption that all \emph{Gambierdiscus} isolates detected are \emph{Gambierdiscus toxicus}.


	\begin{landscape}
	\begin{longtable}{ | p{2.5cm} | p{7cm} | p{3cm} | p{3cm} | p{3.5cm} | }
	\caption{Geographical distribution of different \emph{Gambierdiscus} spp. and their associated toxin profile.} \\
	\hline
	\textbf{Species} & \textbf{Geographical Distribution} & \textbf{CTX toxicity via assays} & \textbf{MTX toxicity via assays} & \textbf{Toxicity through LC-MS} \\
	\hline
	\emph{G. toxicus} & Tahiti, French Polynesia \cite{adachi1979thecal,chinain1999morphology}; Mexican Caribbean \cite{hernandez2004species}; New Caledonia \cite{chinain1999morphology}; Reunion Island \cite{chinain1999morphology}; Nha Trang, Vietnam \cite{roeder2010characteristic} & MBA -ve \cite{chinain1999morphology}; RBA +ve \cite{chinain2010growth} & MBA +ve \cite{chinain1999morphology} & N/A \\
	\hline
	\emph{G. belizeanus} & Belize \cite{faust1995observation}, Florida, USA \cite{litaker2009taxonomy}; Mexican Caribbean \cite{hernandez2004species}; Malaysia \cite{}; Pakistan \cite{munir2011occurrence}; Queensland, Australia \cite{}; St. Barthelemy Island, Caribbean \cite{litaker2010global} & RBA +ve \cite{chinain2010growth} & HELA +ve \cite{holland2013differences} & N/A \\
	\hline
	\emph{G. yasumotoi} & Singapore \cite{holmes1998gambierdiscus}; Japan \cite{nishimura2013genetic}; Mexican Caribbean \cite{hernandez2004species}; Queensland, Australia \cite{}; Nha Trang, Vietnam \cite{} & MBA +ve \cite{} & N/A & N/A \\
	\hline
	\emph{G. australes} & French Polynesia \cite{chinain1999morphology}; Japan \cite{nishimura2013genetic}; Cook Islands \cite{rhodes2010toxic}; Hawaii, USA \cite{litaker2009taxonomy}; Pakistan \cite{munir2011occurrence} & MBA +ve \cite{rhodes2010toxic}; RBA +ve \cite{chinain2010growth} & HELA +ve \cite{holland2013differences}; MBA +ve \cite{rhodes2010toxic} & CTX N/D \cite{}; MTX +ve \cite{}\\
	\hline
	\emph{G. pacificus} & French Polynesia \cite{chinain1999morphology}; Marshall \& Society Islands, Micronesia \cite{litaker2010global}; Kota Kinabalu \& Sipandan Island, Malaysia \cite{mohammad2005marine}; Nha Trang, Vietnam \cite{} & MBA +ve \cite{chinain1999morphology} & MVA +ve \cite{chinain1999morphology} & N/A \\
	\hline
	\emph{G. polynesiensis} & French Polynesia \cite{chinain1999morphology}; Canary Islands \cite{fraga2011gambierdiscus}; Pakistan \cite{munir2011occurrence}; Nha Trang, Vietnam \cite{} & MBA +ve \cite{chinain1999morphology}; RBA +ve \cite{chinain2010growth} & MBA +ve \cite{chinain1999morphology} & CTX +ve \cite{chinain2010growth}; MTX N/A \\
	\hline
	\emph{G. caribaeus} & Florida, USA \cite{litaker2009taxonomy}; Belize \cite{litaker2009taxonomy}; Tahiti, French Polynesia \cite{litaker2009taxonomy}; Palau, Hawai, USA \cite{litaker2009taxonomy}; Flower Garden Banks National Marine Sanctuary, USA \cite{holland2013differences}; Ocho Rios, Jamaica \cite{}; Bahamas \cite{litaker2010global}; Grand Cayman Islands \cite{}; Tol-truk, Micronesia \cite{litaker2010global}; Jeju Island, Korea \cite{jeong2012first} & N/A & HELA +ve \cite{holland2013differences} & N/A \\
	\hline
	\emph{G. carolinianus} & North Carolina, USA \cite{litaker2009taxonomy}; Bermuda, Mexico \cite{litaker2010global}; Puerto Rico \cite{holland2013differences}; Flower Garden Banks National Marine Sanctuary, USA \cite{holland2013differences}; Ocho Rios, Jamaica \cite{holland2013differences}; Crete, Greece \cite{holland2013differences} & N/A & HELA +ve \cite{holland2013differences} & N/A\\
	\hline
	\emph{G. carpenteri} & Belize \cite{litaker2009taxonomy}; Guam, USA \cite{litaker2009taxonomy}; Fiji \cite{litaker2009taxonomy}; Hawaii, USA \cite{}; Dry Tortugas, USA \cite{holland2013differences}; Flower Garden Banks National Marine Sanctuary, USA \cite{holland2013differences}; Ocho Rios, Jamaica \cite{}; Merimbula \& Exmouth, Australia \cite{kohli2014cob} & N/A & HELA +ve \cite{holland2013differences} & N/A \\
	\hline
	\emph{G. ruetzleri} & North Carolina, USA \cite{litaker2009taxonomy}; Belize \cite{litaker2009taxonomy} & N/A & HELA +ve \cite{holland2013differences} & N/A \\
	\hline
	\emph{G. excentricus} & Canary Islands \cite{fraga2011gambierdiscus} & NCBA +ve \cite{fraga2011gambierdiscus} & NCBA +ve \cite{fraga2011gambierdiscus} & N/A \\
	\hline
	\emph{Gambierdiscus} ribotype 1 & Belize \cite{litaker2010global} & N/A & N/A & N/A \\
	\hline
	\emph{Gambierdiscus} ribotype 2 & Belize \cite{litaker2010global}; Martinique, France \cite{litaker2010global}; Puerto Rico \cite{holland2013differences} & N/A & HELA +ve \cite{holland2013differences} & N/A\\
	\hline
	\emph{Gambierdiscus} ribotype 3 & & & & \\
	\hline
	\emph{Gambierdiscus} ribotype 4 & & & & \\
	\hline
	\emph{Gambierdiscus} ribotype 5 & & & & \\
	\hline
	\emph{Gambierdiscus} ribotype 6 & & & & \\
	\hline
	\emph{Gambierdiscus} sp. type 1 & Japan \cite{nishimura2013genetic} & MBA +ve \cite{nishimura2013genetic} & MBA +ve \cite{nishimura2013genetic} & N/A \\
	\hline
	\emph{Gambierdiscus} sp. type 2 & Japan \cite{nishimura2013genetic} & MBA -ve \cite{nishimura2013genetic} & MBA -ve \cite{nishimura2013genetic} & N/A \\
	\hline
	\emph{Gambierdiscus} sp. type 3 & Japan \cite{nishimura2013genetic} & MBA -ve \cite{nishimura2013genetic} & MBA +ve \cite{nishimura2013genetic} & N/A \\
	\hline
	\end{longtable}
	\end{landscape}
	\FloatBarrier
\subsection{Pacific and Indian Ocean regions}
named after Gambier Islands, where firt identified \cite{adachi1979thecal}
\emph{G. toxicus, G. belizeanus, G. yasumotoi, G. australes, G. pacificus, G. polynesiensis, G. caribaeus, G. carpenteri} reported in Hawaii, Pacific Islands, Australia, South East Asia, Northern Indian Ocean %\cite{} table1.2
Three ribotypes suggesting new species found off coast f Japan \cite{nishimura2013genetic}
\emph{Gambierdiscus} of undetermined species found in Philippines \cite{gillespie1987possible} Hong kong \cite{lu2004harmful} Indonesia \cite{praseno1996hab} Mauritius \cite{hurbungs2002seasonal}. Also Mexican Pacific coast \cite{ceballos2006analisis} regions around Madagascar \cite{grzebyk1994ecology}. 
Reports of CFP in Madagascar with 20\% mortality rate \cite{habermehl1994severe}. CFP very common in Mexican-Carribean but not reflected in literature \cite{hernandez2004species}.
First \emph{G. toxicus} description in caribbean \cite{besada1982observations}.

\subsection{Atlantic Ocean region}
\emph{Gambierdiscus} was reported for the first time in 2003 near Crete in the Mediterranean Sea, as an alien species emigrating to a new habitat, which could herald an emerging region for CFP \cite{aligizaki2008morphological}.
early \emph{Gambierdiscus} look alike was spotted 1948 at Cape Verde Islands \cite{silva1956contribution} and 1979 Key Largos Florida \cite{taylor1979description}
\emph{Gambierdiscus} spp. been described from east coast of USA, Carribean and Mediterranean - see table 1.2 for ref.
\emph{Gambierdiscus} with species not defined reported at Cyprus, Rhodes and Saronikos Gulf \cite{aligizaki2009toxic,aligizaki2010diversity}, French West Indies \cite{lobel1988assessment}, Cuba \cite{delgado2006epiphytic}, Veracruz \cite{okolodkov2007seasonal}.
Central and South America not species specific: Costa Rica and Brazil (M. Montero pers. comm. in \cite{parsons2012gambierdiscus})
Africa not species specific: Angola \cite{berdalet2012global} but CFP reported in Canary Islands and Cameroon \cite{bienfang2008ciguatera} indicating \emph{Gambierdiscus} presence.

\subsubsection{Global distribution of \emph{Gambierdiscus} spp.}
%fix up with new info
\emph{Gambierdiscus} species have been classified as endemic to either Pacific or Atlantic \cite{berdalet2012global,litaker2010global}
Only in Pacific: \emph{G. australes} \emph{G. pacificus} %ref table 1.2
Only in Atlantic: \emph{G. ruetzleri} \emph{G. excentricus} \emph{Gambierdiscus} riobotype 1 and 2 %ref table 1.2
Both: \emph{G. belizeanus} \emph{G. caribeaus} \emph{G. carpenteri} \emph{G. carolinianus} \cite{litaker2010global,litaker2009taxonomy,berdalet2012global}
\emph{G. yasumotoi} Pacific but one report in Mexican-Carribean \cite{hernandez2004species} BUT before discovery of other globular species \emph{G. ruetzleri} which is distributed in Atlantic region \cite{litaker2009taxonomy}.
\emph{G. toxicus} distribution is dubious due to being default species in literature - often wrongly identified 
\emph{G. polynesiensis} found thrughout Pacific table 1.2 but once found near Canary Islands in Atlantic \cite{fraga2011gambierdiscus}
In hundreds of samples from Atlantic (Carribean/Gulf of Mexico/West Indies/Southeast US coast Florida to North Carolina), no Pacific-specific species have been detected \cite{berdalet2012global,litaker2010global}
Most of Pacific has not been sampled for \emph{Gambierdiscus} spp. so assertion cannot be made in return about Atlantic specific species.
compounded by under reporting and under sampling globally, and Pacific specifically, means that assertions about endemism or restricted distrubution can not be made conclusively.
Multiple \emph{Gambierdiscus} spp. can co-occur in a region eg Heron Island Australia, and found in isolation eg Merimbula where only \emph{G. carpenteri} is found %references? Merimbula data published yet?
Localised benthic blooms of \emph{Gambierdiscus} noted in literature in both Pacific and Atlantic regions \cite{nakajima1981toxicity,withers1984ciguatera,chinain1999seasonal,darius2007ciguatera}.
Cell density in blooms between 10,000 and 100,000 cell g$^{-1}$ wet weight algae \cite{litaker2010global}. No estimates for cell density of \emph{Gambierdiscus} which leads to CFP. Onset of CFP may be dependent on other factors too - species level variation in toxicity. Unidentified species of \emph{Gambierdiscus} found around Crete but no CFP outbreaks \cite{caillaud2010update}.
\emph{Gambierdiscus} spp. usually below 1,000 cells g$^{-1}$ wet weight algae \cite{litaker2010global} but in some locations those cell densities are consistent throughout the year \cite{chinain1999seasonal}.
Consistent exposure of low toxin concentration could cause CFP-related build up in fish. Need to understand relationship beween abundance and CFP outbreak. Benthic dinoflaggelate sampling is difficult due to sediments etc. Also \emph{Gambierdiscus} cell distribution can be uneven over small distances which causes difficulty in estimating large areas \cite{lobel1988assessment,ballantine1988population,litaker2010global}.

\section{Ciguatera fish poisoning /  food web?}

Ciguatera manifests as gastrointestinal, cardiovascular and neurological symptoms which can last weeks to months, years in severe cases, and be re-triggered by consumption of alcohol or certain foods \cite{lewis2006ciguatera}. 
\\

Ciguaterra poisoning is responsible for 80 – 96 \% of human poisoning from consuming fish and estimated at around 500 000 cases a year \cite{grandjean2008centers}, however correct diagnosis is problematic. It relies on the medical professional connecting the symptoms with recent fish consumption while eliminating other possible marine poison. In Queensland alone it is estimated that less that 20 \% of cases are reported \cite{lewis2006ciguatera}.\\

The symptoms of CFP can vary widely with a myriad of symptoms. Adams produces a detailed report of eight people contracting CFP from the same meal, hence assumed the same toxin profile, but eliciting varied symptoms and length of the condition \cite{adams1993outbreak}.

Symptoms of chronic CFP cases worsen/relapse in relation to sexual activity \cite{lange1992travel}.

The distribution of P-CTX-1, P-CTX-2 and P-CTX-3 in a ciguateric reef system has been elucidated by Mak et al, with emphasis on biotransformation of P-CTX as it moves through the food web as well as importance of species/genus specific dietary habits impacting the P-CTX concentration of fishes \cite{mak2013pacific}.
Essential to elucidate the presence of CTX in lower trophic level of the food web to predict in which genera bioaccumulation is likely \cite{mak2013pacific}
A study conducted in ciguateric reef systems in the Republic of Kiribati tracked the components of the food web part of CTX bioaccumulation. Of samples tested, 54\% of herbivorous, 72\% of omnivorous and 76\% of carnivorous fish tested positive for P-CTX presence. Grazers predominantly exhibited P-CTX-2 while the principal CTX found in piscivorous fish was P-CTX-1 \cite{mak2013pacific}.

\section{Detection of CTX and MTX in seafood}
% epic table 1.4
Name of ciguatera derived from "cigua", native Cubans description of a turban shelled snail and implicated in outbreak of sickness in Spanish explorers in 1500s \cite{gudger1930poisonous}. CTX in turban snail \emph{Turbo argyrstoma} confirmed \cite{yasumoto1976toxicity}.
Majority cases reported from large reef fish, eg. \cite{hokama2001ciguatera,lewis2001changing,dechraoui2005use,laurent2005ciguatera}, major factor in diagnostics due to rare retention of fish sample for testing.
Estimated less than 20\% of CFP cases reported globally \cite{dickey2010ciguatera}. Common misdiagnosis for CFP likely due to excess of 175 symptoms\cite{sims1987theoretical}, potential variance of symptoms with portion size \cite{wong2008features}, individual susceptibility or toxin accumulation with age \cite{bagnis1979clinical,glaziou1993study} and could be associated with other illnesses such as decompression sickness \cite{adams1993outbreak}, chronic fatigue syndrome, multiple sclerosis \cite{lindsay1997chronic,ting2001ciguatera} and brain tumours \cite{lindsay1997chronic}.
several hundred fish species implicated in CFP \cite{}. %need ref from gurjeet
With presumed under reporting and other limitations mentioned, compounded by absence of an adequate commercial test kit, cannot express exact figure for number of cases. Carnivorous fish are main cause of CFP, however herbivorous fish (eg. surgeonfish, parrotfish pose important accumulative intermediate in food chain \cite{cruz2006macroalgal,randall1958review}.
Table 4: summary of over 90 fish species and other marine fauna which have tested positive for CTX from ciguatera prone regions and following reported outbreaks. Mostly mid-latitude tropical and sub-tropical zones, fitting with \emph{Gambierdiscus} distribution in Table 2.
CFP also reported in no-endemic areas due to increased seafood import \cite{glaziou1994epidemiology,ting2001ciguatera}. Studies predominantly focus on large reef fish, but toxin accumulation detected in eels, sea cucumbers, sarfish, seals and jelly fish. %ref in table 2
Sharks suspected after CFP outbreak but no remnant samples left availible for testing \cite{boisier1995fatal,lehane2000ciguatera}. Urgent requirement to fill deficit of knowledge regarding species other than fish as coastal toxin vectors.
CTX is focus for CFP studies rather than in conjunction with MTX.
MBA used to confirm MTX in \emph{Ctenochaetus striatus} (striped bristletooth) \cite{bagnis1987use} %this article is about CTX only - MTX not mentioned in text.
knowledge gap: do MTX bioaccumulate up the food chain? and role in CFP?
on small island nations, native fisherman know ciguatera prone zones and fish species to avoid. However in French Polynesia, CTX detected in fish considered to be safe \cite{darius2007ciguatera}.
Toxin profiles and structures elucidated via chromatographic techniques such as HPCL, UPLC and LC-MS with nuclear magnetic resonance (NMR) \cite{legrand1989isolation,murata1990structures,murata1990structures,satake1996isolation} and radio ligand binding (RLB) \cite{hamilton2002multiple,hamilton2002isolation}. Effective methods, however expensive and require special training - ie not routine in the field and unlikely to become so. 
toxin confirmation via UPLC/HPLC then LC-MS: isolation and fractionation of CTX to compare against their known molecular weights (table 3). Rapid analysis method proposed \cite{lewis2009rapid}, however acquisition of standards difficult due to limited natural CTX compounds \cite{berdalet2012global} and synthetic production is possible \cite{hirama2001total} although highly complex. Lacking consistent source for reference, reliable quantification of CTX and congeners is not possible. Compounding problem is co-elution peaks of similar compounds; inhibiting/promoting matrixes remain unresolved.
Biological assays for ciguateric fish detection using chickens \cite{}, cats \cite{larson1967ciguatera}, mongooses \cite{hokama1977radioimmunoassay},diptera larvae \cite{labrousse1996toxicological}, brine shrimp \cite{granade1976ciguatera} and mosquitoes \cite{bagnis1987use}. Each essay problematic due to toxin specificity and quantification, inefficiencies, ethical considerations summary: \cite{dickey2010ciguatera}
MBA via intraperitoneal injection most commonly used bioassay (see table 4) even though it does not provide linear dose-response relationship with CTX toxicity \cite{hoffman1983mouse}. 
Many alternative biochemical assays proposed to replace biological assays for sea food testing. Radioimmunoassay \cite{hokama1977radioimmunoassay} preceeded the financially more viable as well as higher throughput enzyme-linked immunosorbent assay (ELISA) \cite{hokama1983rapid}, has been refined to reliably detect picogram concentration of CTX in fish tissue \cite{campora2008detection,campora2010evaluating}.
Stick enzyme immunoassay (SEIA) \cite{hokama1985rapid} and solid phase immunoassay (SPIA) \cite{hokama1990simplified} progressed to the development of commercial kits (i.e. Cigua-check\textregistered \ and Ciguatect\textregistered), although these have yielded a considerable amount of false positives and false negatives \cite{wong2005study}. Cigua-check\textregistered \ is no longer manufactured.
Alternate screening options are sodium channel binding (N2A) \cite{dickey2010ciguatera} and RBA \cite{poli1997identification,darius2007ciguatera} which are both recommended by European Food Standard Association \cite{} %ummm... check this shit out
Alternate assays do not discern between specific CTX and MTX congeners; need development and validation via LC-MS analysis. Problems impacting progress: lack of available purified standards \cite{}, different toxin congeners and unidentified metabolites produced by \emph{Gambierdiscus} in fish samples \cite{endean1993variation,vernoux1997isolation}.
Kohli et al. detected the presence of \emph{G. australes} in the viscera of snapper via nested PCR. This could provide an effective, inexpensive and widely established method to detect whether suspected ciguateric fish have consumed \emph{Gambierdiscus}, and furthermore this approach can be species specific giving an indication to the possible toxin profile present in the fish \cite{kohli2014feeding}. 

In 2005 Hung et al established that the CFP of three family members was caused by CTX present in baracuda fish eggs \cite{hung2005persistent}. This indicates that bioaccumulated CTX can be passed on to offspring which start their life cycle with a symptom inducing concentration of CTX.

\begin{sidewaystable}[!htbp]
\caption{CTXs and congeners detected by various assays in herbivorous fish and other animals.}
\begin{tabular}{ |  p{4.5cm} | p{5cm} | p{4.5cm} | p{5cm} | }
\hline
\textbf{Latin name (Common name)} & \textbf{Source} & \textbf{CTX (if detected)} & \textbf{Methods of detection} \\
\hline
  &  &  & \\
\hline
  &  &  & \\
\hline
  &  &  & \\
\hline
  &  &  & \\
\hline
  &  &  & \\
\hline
  &  &  & \\
\hline
  &  &  & \\
\hline
  &  &  & \\
\hline
  &  &  & \\
\hline
  &  &  & \\
\hline
  &  &  & \\
\hline
  &  &  & \\
\hline
  &  &  & \\
\hline
  &  &  & \\
\hline
  &  &  & \\
\hline
  &  &  & \\
\hline
\end{tabular}
\end{sidewaystable}

\begin{sidewaystable}[!htbp]
\caption{CTXs and congeners detected by various assays in omnivorous fish and other animals.}
\begin{tabular}{ |  p{4.5cm} | p{5cm} | p{4.5cm} | p{5cm} | }
\hline
\textbf{Latin name (Common name)} & \textbf{Source} & \textbf{CTX (if detected)} & \textbf{Methods of detection} \\
\hline
  &  &  & \\
\hline
  &  &  & \\
\hline
  &  &  & \\
\hline
  &  &  & \\
\hline
  &  &  & \\
\hline
  &  &  & \\
\hline
  &  &  & \\
\hline
  &  &  & \\
\hline
  &  &  & \\
\hline
\end{tabular}
\end{sidewaystable}

\begin{landscape}
\begin{longtable}{ |  p{4.5cm} | p{5cm} | p{4.5cm} | p{5cm} | }
\caption{CTXs and congeners detected by various assays in carnivorous fish and other animals.}\\
\hline
\textbf{Latin name (Common name)} & \textbf{Source} & \textbf{CTX (if detected)} & \textbf{Methods of detection} \\
\hline
 \textbf{Seriola dumerili} (Greater amberjack) & Canary Islands \cite{caillaud2012towards}; Selvagens Islands, Portugal \cite{otero2010first}; Hawaii, USA \cite{campora2008detection,hokama1977radioimmunoassay,hokama1983rapid,hokama1985rapid}; Haiti \cite{poli1997identification}; St. Barthelemy Island, Caribbean \cite{vernoux1986heterogeneity}; St. Thomas, Caribbean \cite{granade1976ciguatera} & C-CTX-1 \cite{poli1997identification};C-CTX-1B \cite{otero2010first}; P-CTX-3C and CTX analogues from Carribean and Indic waters \cite{otero2010first} & UPLC/MS \cite{otero2010first}; HPLC/MS \cite{poli1997identification}; TLC \cite{vernoux1986heterogeneity}; BSBA \cite{granade1976ciguatera}; MGBA \cite{campora2008detection,granade1976ciguatera}; MBA \cite{hokama1983rapid,hokama1985rapid,vernoux1986heterogeneity}; S-EIA \cite{hokama1985rapid}; SPIA \cite{otero2010first}; RIA \cite{campora2008detection,hokama1983rapid}; ELISA \cite{campora2008detection}; N2A \cite{caillaud2012towards,campora2008detection}; RBA \cite{} \\
\hline
 \textbf{Seriola fasciata} (Lesser amberjack) & Selvagens Islands, Portugal \cite{otero2010first}; West Africa \cite{boada2010ciguatera} & C-CTX-1 \cite{boada2010ciguatera}; C-CTX-1B \cite{otero2010first}; P-CTX-3C and CTX analogues from Carribean and Indic waters \cite{otero2010first} & LCMS/MS \cite{boada2010ciguatera}; UPLC/MS \cite{otero2010first}\\
\hline
 \emph{Seriola rivoliana} (Almaco jack) & Canary Islands \cite{campora2010evaluating}; Hawaii, USA \cite{campora2008detection}; St. Thomas, USA \cite{granade1976ciguatera} & C-CTX-1 \cite{} & LCMS/MS \cite{}; BSBA \cite{granade1976ciguatera}; MGBA \cite{granade1976ciguatera}; ELISA \cite{campora2008detection,campora2010evaluating}; N2A \cite{campora2008detection,campora2010evaluating} \\
\hline
  &  &  & \\
\hline
  &  &  & \\
\hline
  &  &  & \\
\hline
  &  &  & \\
\hline
  &  &  & \\
\hline
  &  &  & \\
\hline
  &  &  & \\
\hline
  &  &  & \\
\hline
  &  &  & \\
\hline
  &  &  & \\
\hline
  &  &  & \\
\hline
  &  &  & \\
\hline
  &  &  & \\
\hline
  &  &  & \\
\hline
  &  &  & \\
\hline
  &  &  & \\
\hline
  &  &  & \\
\hline
  &  &  & \\
\hline
  &  &  & \\
\hline
  &  &  & \\
\hline
  &  &  & \\
\hline
  &  &  & \\
\hline
  &  &  & \\
\hline
  &  &  & \\
\hline
  &  &  & \\
\hline
  &  &  & \\
\hline
  &  &  & \\
\hline
  &  &  & \\
\hline
  &  &  & \\
\hline
  &  &  & \\
\hline
  &  &  & \\
\hline
  &  &  & \\
\hline
  &  &  & \\
\hline
  &  &  & \\
\hline
  &  &  & \\
\hline
  &  &  & \\
\hline
  &  &  & \\
\hline
  &  &  & \\
\hline
  &  &  & \\
\hline
  &  &  & \\
\hline
  &  &  & \\
\hline
  &  &  & \\
\hline
  &  &  & \\
\hline
  &  &  & \\
\hline
  &  &  & \\
\hline
  &  &  & \\
\hline
  &  &  & \\
\hline
  &  &  & \\
\hline
  &  &  & \\
\hline
  &  &  & \\
\hline
  &  &  & \\
\hline
  &  &  & \\
\hline
  &  &  & \\
\hline
  &  &  & \\
\hline
  &  &  & \\
\hline
  &  &  & \\
\hline
  &  &  & \\
\hline
  &  &  & \\
\hline
  &  &  & \\
\hline
  &  &  & \\
\hline
  &  &  & \\
\hline
  &  &  & \\
\hline
  &  &  & \\
\hline
  &  &  & \\
\hline
  &  &  & \\
\hline
  &  &  & \\
\hline
  &  &  & \\
\hline
  &  &  & \\
\hline
  &  &  & \\
\hline
  &  &  & \\
\hline
\end{longtable}
\end{landscape}

\FloatBarrier
\newpage
\bibliographystyle{plain}
\bibliography{review_ref.bib}

%what was Ballantine cited for?

\end{document}