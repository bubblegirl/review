\documentclass[12pt]{article}
\usepackage[hcentering,bindingoffset=20mm]{geometry}
\usepackage{placeins}
\usepackage[numbib]{tocbibind}
\usepackage{rotating}
\usepackage[square,sort,comma,numbers]{natbib}
\usepackage{graphicx}
\usepackage{tabularx}
\linespread{1.3}
%\fontsize{8cm}{1.3em}\selectfont
\usepackage{gensymb}
\usepackage{longtable}
\usepackage{lscape}

\addtolength{\textwidth}{2cm}
\addtolength{\hoffset}{-1cm}


\addtolength{\textheight}{2cm}
\addtolength{\voffset}{-1cm}
\setlength{\parindent}{0pt}

\title{\textbf{ \emph{Gambierdiscus} spp. - the harmful algal bloom genus responsible for ciguatera fish poisoning.}}
\author{Anna Liza Kretzschmar, Gurjeet Kohli, Shauna Murray}
\date{}
%\usepackage{cite}

\begin{document}
\maketitle
%.wwwproxy0.library.unsw.edu.au
\section{Introduction}
A vast biodiversity of phytoplankton was recently been uncovered at varying taxonomic levels with the advent of readily accessible population and species genetics \cite{simon2009diversity}. Ongoing development of molecular techniques herald rapid future advancement in species discovery and phylogeny elucidation \cite{murray2012genetic,murray2012transcriptomics}. 
Harmful algal bloom (HAB) forming dinoflagellates are an increasingly observed phenomenon on the global scale and hence have become a focus for research \cite{berdalet2012global,hallegraeff2010ocean,garces2012habitat}. They cause oxygen depletion which limits availability to other organisms, block sunlight to the benthos and potentially produce harmful compounds \cite{grandjean2008centers}, an example for the latter is \emph{Gambierdiscus} spp. \cite{lehane2000ciguatera}.
Interest in benthic dinoflagellates was sparked in 1977, when the phylum was implicated in ciguatera fish poisoning (CFP) \cite{yasumoto1977finding}.  
Currently there are only approximately 200 described benthic dinoflagellates, 85\% of which are marine and 15\% from continental waters \cite{gomez2012quantitative}.
In 1979 it was discovered that \emph{Gambierdiscus toxicus} was responsible for toxin production resulting in CFP \cite{adachi1979thecal}. The genus has been recognised as the primary producer of ciguatoxins (CTXs) and maitotoxins (MTXs) \cite{chinain1997intraspecific,holmes1998gambierdiscus}. CTXs bioaccumulate through the food web until consumption of seafood with orally accumulative effective levels cause CFP \cite{bagnis1979clinical,gillespie1987possible,sims1987theoretical}. Whether MTXs play a role in CFP needs to be further investigated \cite{kohli2014feeding}. For over 15 years \emph{Gambierdiscus} was considered to be a monotypic taxon, but since 1995 new species are being discovered with the aid of genetic analysis tools to illuminate the genus phylogeny \cite{faust1995observation,holmes1998gambierdiscus,litaker2009taxonomy,chinain1999morphology,fraga2011gambierdiscus,nishimura2014morphology}. Evidence suggests that the production of CTXs and MTXs production varies between, and potentially within, species \cite{chinain2010growth,holland2013differences}. \\%instert phylo need?
CFP is the most common non bacterial illness associated with seafood consumption, posing a global problem exacerbated by seafood trade \cite{friedman2008ciguatera}. %CFP problem ref! + seafood trade
Annual reported CFP cases are estimated up to 500,000 a year globally and are responsible for 80 to 96 \% of human poisoning from consuming seafood \cite{fleming1998seafood,grandjean2008centers}, however correct diagnosis is problematic. It relies on the medical professional connecting the symptoms with recent fish consumption while eliminating other possible marine poisons. In Queensland, Australia, alone it is estimated that less than 20 \% of cases are reported \cite{lewis2006ciguatera}.
The symptoms of CFP can vary widely with a myriad of symptoms which have been exhaustively described by Sims et al \cite{sims1987theoretical}. CFP manifests as gastrointestinal, cardiovascular and neurological symptoms which can last weeks to months, years in severe cases, and be re-triggered by consumption of alcohol or certain foods \cite{lewis2006ciguatera}. Adams produces a detailed report of eight people contracting CFP from the same meal, hence assumed the same toxin profile, but eliciting varied symptoms and length of the condition \cite{adams1993outbreak}. 
A study correlating CFP related calls from the US national poison centre data with storms and sea surface temperature showed that sea surface temperature was a relevant factor for CFP incidence and extrapolated that climate change related sea temperature rise could escalate CFP by 200 - 400 \% in the US alone \cite{garces2012habitat}. A study of preferential substrate colonisation showed that \emph{Gambierdiscus} favoured dead corals, which will increase in availability with coral bleaching \cite{grzebyk1994ecology}. Incidentally, over the last decade, CFP incidence has increased by 60\% in Pacific Islands \cite{skinner2011ciguatera}.
Classically CFP is reported in regions bordering on the Caribbean, Pacific and Indian oceans, however there is evidence that \emph{Gambierdiscus} is expanding it's habitat. Climate change is impacting the marine environment - a model ocean for changing conditions is the Mediterranean Sea \cite{lejeusne2010climate}, where \emph{Gambierdiscus} has been detected since the early 2000s \cite{aligizaki2008morphological}.
Understanding the phylogeny and species specific toxicology of \emph{Gambierdiscus} spp. is essential as part as global HAB species monitoring for early warning system for CFP outbreaks \cite{berdalet2012global}.


\section{Morphology and phylogenetics of \emph{Gambierdiscus}}

When \emph{Gambierdiscus} was first described by Adachi and Fukuyo in 1979 it was believed to be a monotypic taxon with \emph{G. toxicus} as it's sole representative \cite{adachi1979thecal}. Inconsistency in morphology and toxicity were observed which led to another species discovery 15 years later, species discovery is now partially based on differences in ribosomal RNA (rRNA) genes \cite{faust1995observation,holmes1990toxicity,holmes1991strain,chinain1997intraspecific,richlen2008phylogeography,bomber1988epiphytic,bomber1989epiphytism,bomber1989genitic,morton1993response}. Currently 13 species and 6 ribotypes have been described, as per table ~\ref{tbl:MorphTable}, based on their distinct morphological and genetic characteristics. New discovery of species with increased global sampling is likely. \\

The ensuing descriptions contain an overview of the main morphological characteristics for currently described \emph{Gambierdiscus} sp., the terminology used has been conformed to the proposed terms from Hoppenrath et al \cite{hoppenrath2013taxonomy}. The initial species description detailed \emph{Gambierdiscus} to be large (60-100 $\mu$m), armoured with a distinct plate pattern and fishhook shaped accessory pore. Species fall into two categories - antero-posteriorly compressed, also described as lenticular, or slightly laterally compressed, or globular, as per figure xx - from Shauna's book. %insert from Shauna's book
To date there are only two globular species (\emph{G. yasumotoi} and \emph{G. ruetzleri}) which can be distinguished from each other by cell size, size and 2' apical and 2'''' antapical plate and the depth to width ratio as per Litaker et al. \cite{litaker2009taxonomy}. \\
The other species are antero-posteriorly compressed and further divided by the 1p posterior intercalary plate being narrow (\emph{G. australes}, \emph{G. belizeanus}, \emph{G. excentricus} \emph{G. pacificus}, \emph{G. scabrosus} and \emph{G. silvae}) or broad (\emph{G. polynesiensis}, \emph{G. carolinianus}, \emph{G. toxicus}, \emph{G. caribaeus} and \emph{G. carpenteri}). \\
The narrow 1p posterior intercalary plated species are subdivided by heavily reticulate-foveate (\emph{G. belizeanus}, \emph{G. scabrosus} and \emph{G. silvae}) and smooth cell surfaces (\emph{G. australes}, \emph{G. excentricus} and \emph{G. pacificus}). Further distinction for the reticulate-foveate species can be seen where \emph{G. belizeanus} and \emph{G. scabrosus} pose symmetric and asymmetric 4'' precingular plates respectively \cite{nishimura2014morphology}. %cannot find distinction whether G. silvae has symmetric or asymmetric 4’’ precingular plate - will try and work out if this is due to differential plate pattern assignment
Smooth plated species can be distinguished by the shape of the 2' apical plate, which can be hatchet shaped (\emph{G. pacificus}) or rectangular ( \emph{G.australes} and \emph{G. excentricus}). The conventional rectangularly shaped 2' apical plated species can be differentiated by size - where \emph{G. excentricus} is 1.5 times wider and deeper than \emph{G. australes}, as well as other features described in their initial descriptions \cite{chinain1999morphology,fraga2011gambierdiscus}. \\
Species within the broad 1p posterior intercalary plate category can be set apart by possessing a rectangular shaped 2' apical plate (\emph{G. caribaeus} and \emph{G. carpenteri}) or a hatchet shaped 2' apical plate (\emph{G. carolinianus}, \emph{G. polynesiensis} and \emph{G. toxicus}).
\emph{G. toxicus} can be discerned by a pointed dorsal end to the 1p posterior intercalary plate. \emph{G. carolinianus} and \emph{polynesiensis} can be differentiated as outlined by Litaker et al \cite{litaker2009taxonomy}. The rectangularly shaped 2' apical plate possessing species \emph{G. caribaeus} and \emph{G. carpenteri} can be differentiated by the shape of the 4'' precingular plate, which is symmetric and asymmetric respectively. \\
Other morphological distinctions, such as the size and shape of the sulcal plates can be found in the original descriptions as well as Litaker et al's account \cite{litaker2009taxonomy,fraga2011gambierdiscus,faust1995observation,holmes1998gambierdiscus,chinain1999morphology}. %insert new species shit
These features can be readily identified using scanning electron microscopy. However the strain specific variability of morphology within species, such as in size and shape of individual plates, makes this a support tool for classification which should be verified with genetic analysis. \\
%Nishimura 14 - light micro unreliable due to subtle differences
%insert more spin, especially what Shauna said 

An essential tool for genetic analysis to discern between the different species of \emph{Gambierdiscus} is rRNA analysis, specifically comparing regions of the large subunit (LSU) rRNA, small subunit (SSU) rRNA and internal transcribed spacer (ITS) rRNA. From such analysis, the current consensus is that \emph{Gambierdiscus} is monophyletic, with the lenticular and globular species forming two distinct clades \cite{chinain1999morphology,litaker2009taxonomy,fraga2011gambierdiscus,richlen2008phylogeography,kuno2010genetic,litaker2010global,nishimura2013genetic}. \\
Phylogenetic analysis concluded that the globular clade containing \emph{G. ruetzleri} and \emph{G. yasumotoi} diverged early in the genus' evolution \cite{litaker2009taxonomy,nishimura2013genetic}. The two species are also the most closely related within the genus. 
Based on D8-D10 LSU rRNA sequences, \emph{Gambierdiscus} ribotype 2 was designated putative new phylotypes by Litaker et al as well as \emph{Gambierdiscus} species type 4, 5 and 6 by Xu et al \cite{litaker2010global,xu2014distribution}. \emph{Gambierdiscus} species types 2 and 3 are also supported by SSU rRNA sequences \cite{nishimura2013genetic,kuno2010genetic}. Genetic data implies that these are separate species, as was the case for \emph{G. scabrosus}, previously \emph{Gambierdiscus} species type 1 \cite{nishimura2013genetic,nishimura2014morphology},  and \emph{G. silvae}, previously Gambierdiscus ribotype 1 \cite{fraga2014genus}, but morphological circumscription was required to confirm their status as new species. \\

Phylogeny is a powerful tool that will be essential in elucidating the evolutionary relationships of species and genera and, in a HAB setting, potential identification and monitoring of toxin genes. However, the efficiency of this technique is directly dependent on a database to which sequences can be compared. At the ICHA conference in 2014, Siano, Edvardson and Karlsen mentioned in their respective presentations the need for a unified, accessible dinoflagellate database to advance the field \cite{sianoICHA,edvardsonICHA,karlsenICHA}.\\
%ok, so... p-values from Gurjeet's stuff?
% NZ strain of G yas Rhodes_14 shows super similarity between G yas and G rutz but was decided on G yas on SEM data --- insert what was requested during discussion - 
Shauna, picture from your book will go in here and a paragraph on why morphology isn't enough for species identification, referring to both your and Gurjeet's work on \emph{G carpenteri} and \emph{G. yasumotoi} still to come.

\begin{longtable}{ |  p{2cm} | p{2.7cm} | p{4.5cm} | p{3.1cm} | p{1cm} | }
\caption{Presently characterised \emph{Gambierdiscus} spp. and their taxonomic and genetic identifications; table adapted from Kholi \cite{kohli2013Gambierdiscus}}\\
\label{tbl:MorphTable}
\textbf{Species} & \textbf{Cell size ($\mu$m) (depth-width-length)} & \textbf{Morphological characteristics} & \textbf{Genetics (Genbank \#)} & \textbf{References} \\
\hline
 \emph{G. australes} & (72.5$\pm$3.8)-(63.4$\pm$5.0)-(38.7$\pm$3.8) & Antero-posteriously compressed species; narrow 1p plate; smooth cell surface; 2' rectangular shaped; smaller than \emph{G. excentricus} & SSU: EF202891-96; D1-D3 LSU: EF202969-72; D8-D10 LSU: EF498072-74  & \cite{chinain1999morphology,litaker2009taxonomy} \\
\hline
 \emph{G. belizeanus} & (61.7$\pm$3.1)-(60.0$\pm$4.5)-(48.1$\pm$4.2) & Antero-posteriously compressed species; narrow 1p plate; different 2' plate symmetry and size; heavily reticulate-foveate cell surface & SSU: EF202876-77; D1-D3 LSU: EF202940-43; D8-D10 LSU: EF498028-34  & \cite{litaker2009taxonomy,faust1995observation} \\
\hline
 \emph{G. caribaeus} & (82.2$\pm$7.6)-(81.9$\pm$7.9)-(60.0$\pm$6.2) & Antero-posteriously compressed species; broad 1p plate; 2’ rectangular shaped; symmetric 4’’ & SSU: EF202914-28; D1-D3 LSU: EF202929-37, EF202983, EF202985; D8-D10 LSU: EF498045-71 & \cite{litaker2009taxonomy} \\
\hline
 \emph{G. carolinianus} & (78.2$\pm$4.8)-(87.1$\pm$7.1)-(51.4$\pm$5.2) & Antero-posteriously compressed species; broad 1p plate; 2’ hatchet shaped; dorsal end 1p oblique; larger cell size than \emph{G. polynesiensis} & SSU: EF202897-EF202901; D1-D3 LSU: EF202973-75; D8-D10 LSU: EF498035-37  & \cite{litaker2009taxonomy} \\
\hline
 \emph{G. carpenteri} & (81.7$\pm$6.4)-(74.8$\pm$5.9)-(50.2$\pm$6.1) & Antero-posteriously compressed species; broad 1p plate; 2’ rectangular shaped; asymmetric 4'' & SSU: EF202908-13; D1-D3 LSU: EF202938-39, EF202984; D8-D10 LSU: EF498038-44  & \cite{litaker2009taxonomy} \\
\hline
  \emph{G. excentricus} & (97.8$\pm$8.0)-(83.0$\pm$10.0)-(37.0$\pm$3.0) & Antero-posteriously compressed species; narrow 1p plate; smooth cell surface; 2’ rectangular shaped; cell size bigger than \emph{G. australes} (1.5 times wider and deeper) & D1-D3 LSU: HQ877874, JF303063, JF303065-71 ; D8-D10 LSU: JF303073-76  & \cite{litaker2009taxonomy} \\
\hline
  \emph{G. pacificus} & (58.5$\pm$3.9)-(53.6$\pm$4.1)-(40.4$\pm$3.6) & Antero-posteriously compressed species; narrow 1p plate; smooth cell surface; 2' hatchet shaped & SSU: EF202861-65; D1-D3 LSU: EF202944-47; D8-D10 LSU: EF498012-13, EF498015-16  & \cite{litaker2009taxonomy,chinain1999morphology} \\
\hline
 \emph{G. polynesiensis} & (66.3$\pm$3.0)-(60.5$\pm$5.9)-(44.3$\pm$5.1) & Antero-posteriously compressed species; broad 1p plate; 2’ hatchet shaped; dorsal end 1p oblique; smaller cell size than \emph{G. carolinianus} & SSU: EF202902-07; D1-D3 LSU: EF202976-82; D8-D10 LSU: EF498076-80  & \cite{litaker2009taxonomy,chinain1999morphology} \\
\hline 
 \emph{G. ruetzleri} & (45.5$\pm$3.3)-(37.5$\pm$3.0)-(51.6$\pm$4.9) & Globular species smaller than \emph{G. yasumotoi}; cell size less than 42$\mu$m & SSU: EF202853-60; D1-D3 LSU: EF202962-64; D8-D10 LSU: EF498081-85 & \cite{litaker2009taxonomy} \\
 \hline
 \emph{G. scabrosus} & (63.2$\pm$5.7)-(58.2$\pm$5.7)-(37.3$\pm$3.5) & Antero-posteriously compressed species; narrow 1p plate; reticulate-foveate; asymmetric 4'' plate & SSU: AB764229-76; D8-D10 LSU: AB765908-12  & \cite{nishimura2013genetic,nishimura2014morphology,kuno2010genetic} \\ %is 3'' plate right?
\hline
\emph{G. silvae} & (69$\pm$8)-(64$\pm$9)-(46$\pm$5)  & Antero-posteriorly  compressed; narrow 1p plate; heavily reticulate-foveate; & D8-D10 LSU: GU968512-20, GU968523 & \cite{litaker2010global,fraga2014genus} \\
\hline
 \emph{G. toxicus} & (93.0$\pm$5.5)-(83.0$\pm$2.3)-(54.0$\pm$1.5) & Antero-posteriously compressed species; broad 1p plate; 2’ hatchet shaped; dorsal end 1p pointed & SSU: EF202878-90; D1-D3 LSU: EF202951-61; D8-D10 LSU: EF498017-27 & \cite{litaker2009taxonomy,adachi1979thecal,chinain1997intraspecific,richlen2008phylogeography} \\
 \hline
  \emph{G. yasumotoi} & (56.8$\pm$5.6)-(51.7$\pm$5.6)-(62.4$\pm$4.3) & Globular species larger than \emph{G. ruetzleri}; cell size exceeds 42 $\mu$m & SSU: EF202846-52; D1-D3 LSU: EF202965-68; D8-D10 LSU: EF498087-89 & \cite{holmes1998gambierdiscus,litaker2009taxonomy} \\
  \hline
  \multicolumn{5}{| c |}{\textbf{Genetically described phylotypes}}\\
    \hline
\emph{Gambier- discus} ribotype 2 & N/A & N/A & D8-D10 LSU: GU968499-500, GU968503, GU968505, GU968507-11  & \cite{litaker2010global} \\
\hline
\emph{Gambier- discus} sp. type 2 & N/A & N/A & SSU: AB764277-96; D8-D10 LSU: AB765913-18 & \cite{kuno2010genetic,nishimura2013genetic} \\
\hline
\emph{Gambier- discus} sp. type 3 & N/A & N/A & SSU: AB764296-300; D8-D10 LSU: AB765923-24 & \cite{nishimura2013genetic} \\
\hline
\emph{Gambier- discus} sp. type 4  & (65.9-72.4$\pm$4.1-4.2)-(64.5-68.9$\pm$5.0)-(N/A) & Antero-posteriorly compressed species;evenly distributed pores; larger than \emph{Gambierdiscus} sp. type 5; 2' hatchet shaped; broad 1p plate & D8-D10 LSU: KJ125080, KJ125114-15,KJ125119-20, KJ125122 etc  & \cite{xu2014distribution} \\
\hline
\emph{Gambier- discus} sp. type 5  & (54.8$\pm$4.6)-(53.7$\pm$6.3)- (N/A) & Antero-posteriorly compressed species;evenly distributed pores; smaller than \emph{Gambierdiscus} sp. type 4; 2' rectangular shaped; narrow 1p plate & D8-D10 LSU : KJ125132-35 & \cite{xu2014distribution} \\
\hline
 \emph{Gambier- discus} sp. type 6 & N/A & N/A & D8-D10 LSU: KJ125108-09, KJ125111-13 & \cite{xu2014distribution} \\
 \hline
\end{longtable}
\FloatBarrier

\section{Toxins produced by \emph{Gambierdiscus}}
%table 1.3 - amend with new info
% toxin mod in the food chain - or seperate section?
%\emph{G. toxicus} has been shown to produce a range of toxins other than ciguatoxins \cite{holmes1994purification,murata1993structure} Furthermore, other species within this genus produce ciguatoxin like substances, such as \emph{G. polynesiensis}, \emph{G. australes} and \emph{G. pacificus} \cite{roeder2010characteristic}. 

%trediculous symptomology detailed in sims1987theoretical

Ciguateric seafood is characteristically impossible to discern during culinary preparation as it does not exhibit any obviously recognisable differences to non toxic specimens and is heat resistant \cite{withers1982ciguatera}. Hence toxin presence is usually only diagnosed retrospectively from CFP symptoms in conjunction with recent seafood consumption \cite{sims1987theoretical} as samples of the food consumed are rarely obtained. Due to diagnosis depending on the deductive capability of the medical professional, CFP is predicted to be widely under reported with as little of 20 \% of cases being reported  in Queensland, Australia, alone \cite{lewis2006ciguatera}. The disease can manifest as over 180 symptoms, with varying levels of severity, and differ even between individuals at the same time and source of exposure \cite{sims1987theoretical}. The effects of CFP appear to be cumulative as severity increases with previous exposure \cite{emerson1983preliminary}. Commonly a couple hours after consumption, the initial symptoms are gastrointestinal which can be followed by cardiovascular and neurological manifestations \cite{sims1987theoretical}. They can persist between weeks to years, and recurrence is a common phenomenon induced by a variety of factors such as alcohol, fish or meat consumption \cite{lewis2006ciguatera} as well as sexual activity \cite{lange1992travel}. \\
Symptoms of CFP can vary between geographical locations \cite{molgo2000ciguatera,dickey2010ciguatera}, which likely contributes to the inefficiency of diagnoses. This could be due to the structural difference between CTXs, and potentially MTXs. Hence accurate determination of CTX and MTX congeners is vital to understanding toxicology, for linked risks and symptoms of CFP - which is an avenue of investigation that urgently needs attention.\\

In 1979 the great breakthrough in understanding CFP was that an algal species, \emph{G. toxicus}, produces the toxins that cause the disease and that the toxins bioaccumulate up the food chain \cite{adachi1979thecal}. With the advent of further species discovery from 1995 onwards \cite{faust1995observation}, the inconsistency in toxin measurements between isolates tested started to make sense - they are different species. The current challenge is to characterise the species specific toxin profiles, employing a consistent and conclusive methodology. Bioassays provide an excellent indicator to a species' toxicity, however it does not suffice to elucidate a detailed toxin profile - liquid chromatography-mass spectrometry (LC-MS) or tandem LC-MS/MS analysis is required \cite{diogened2014chemistry}. \emph{Gambierdiscus} spp. produce more than just CTXs - notably MTXs are also commonly registered \cite{holmes1994purification,murata1993structure}. It has, until recently, been assumed in the literature that CTXs are the sole culprit in CFP as MTXs are water soluble. CTXs have been isolated form the Pacific, Caribbean and the Indian ocean. The  structure varies between the Pacific and the Caribbean congeners, with the structures of the Indian ocean analogues still to be determined. Symptomatology of CFP varies between the three regions, however if the variance is dependent on  the structural difference, or for example due to different precursor modification during biomagnification, is yet to be determined \cite{lewis2006ciguatera}. Understanding the molecular evolution and ecology of the toxins, coupled with which species produce them and under what circumstances, are points that are essential to dealing with harmful algal blooms and their impact on humans and the environment. Whether and which changing environmental cues play a role is imperative for predictive purposes with the advent of global warming \cite{llewellyn2010revisiting}.  \\

\subsection{CTX}
Toxins in the CTX family consist of lipid soluble polyether ladders which act as sodium channel activators \cite{dechraoui1999ciguatoxins}. They are orally effective in humans at the picomolar (pM) and millimolar (mM) concentration range \cite{molgo2000ciguatera}, causing an influx of Na$^{+}$ ions and hence spontaneous action potentials in the cell, especially active on voltage sensitive channels along the nodes of Ranvier \cite{sims1987theoretical,mattei1999neurotoxins,lewis1992action,molgo2000ciguatera}. \\
Currently CTXs are prefixed by their origin, where members from the Pacific, Caribbean and Indian ocean are designated P-CTX, C-CTX and I-CTX respectively. % don't know C-PTX subdivision ref... for P-CTX it's \cite{} I-CYX is\cite}
The congeners are further subdivided on a structural basis into type I and type II within P-CTXs, followed by all C-CTXs as type III - the molecular structure of I-CTXs have not been elucidated to such an extent as to assign a type \cite{legrand1997two,hamilton2002multiple,hamilton2002isolation}, An overview of the designations can be found in table ~\ref{tbl:CTXTable}.  \\
Type I P-CTXs consist of 13 rings with 60 C atoms \cite{murata1990structures,lewis1991purification,lewis1993origin}, including the first completely defined CTX which was isolated from moray eels and designated P-CTX-1B \cite{murata1990structures} or also described as CTX-1 \cite{lewis1991purification}. P-CTX-2 and P-CTX-3 were isolated from the same extracts but exhibited an alternate structure and toxicity in mice \cite{lewis1991purification}. It has been suggested by two research groups that P-CTX-1, P-CTX-2 and P-CTX-3 may be derivatives from the dinoflagellate precursors CTX-4A and CTX-4B (also GTX-4B by Murata et al \cite{murata1990structures}) \cite{lewis1993origin,yasumoto2000structural}. So far, P-CTX-4A and P-CTX-4B have been isolated from \emph{G. polynesiensis} culture extracts \cite{chinain2010growth}, while P-CTX-1, P-CTX-2 and P-CTX-3 have not. \\
The structure for type II P-CTX-3C consists of 13 ring and 57 C atoms and has been isolated first from \emph{Gambierdiscus} culture \cite{satake1993structure}, then from \emph{G. polynesiensis} \cite{chinain2010growth}. There are two known congeners of P-CTX-3C which have been isolated - 49-epi-CTX-3C (or CTX-3B by Chinain et al \cite{chinain2010growth}) and M-seco-CTX-3C from \emph{Gambierdiscus} \cite{satake1993structure} and \emph{G. polynesiensis} \cite{chinain2010growth}. Two type II CTXs have been isolated from Moray eel, 2,3-dihydroxy-CTX-3C (or CTX-2A1) and 51-hydroxy-CTX-3C \cite{satake1998isolation}, and are suspected to constitute oxygenated metabolites of P-CTX-3. \\ %\cite{yasumoto2000structural} check ref as well as if P-CTX3 or P-CTX-3C 
Structurally C-CTXs were resolved to consist of 14 rings and 62 C atoms, with multiple congeners isolated from carnivorous fish - C-CTX-1, C-CTX-2, C-CTX-1127, C-CTX-1141, C-CTX-1143, C-CTX-1157, C-CTX-1159 \cite{vernoux1997isolation,lewis1998structure,pottier2003identification,pottier2002characterisation}. Up to date, no C-CTXs have been isolated from any of the endemic \emph{Gambierdiscus} spp., however Fraga et al have shown that \emph{G. excentricus} from the Canary Islands produce CTX and MTX like compounds - the characterisation of these toxins is in process \cite{fraga2011gambierdiscus}. \\
The most recently discovered group CTXs is from the Indian ocean, where I-CTX-1, I-CTX-2, I-CTX-3 and I-CTX-4 were isolated from carnivorous fish. Their structure has not yet been established, preventing classification to a sub type \cite{hamilton2002multiple,hamilton2002isolation}. It has been elucidated that they have a higher molecular weight in comparison to P-CTXs and C-CTXs \cite{caillaud2010update,hamilton2002multiple,hamilton2002isolation}. \\ %I-CTX-1 toxic to mice via intraperitoneal injection \cite{hamilton2002isolation}. 

It has been suggested that the evident variety of CTX congeners, found even in the same ecosystem, is due to modification to the toxin structure as it passes up the food chain which can increase potency up to 10-fold \cite{hokama1996human,lewis2006ciguatera}. This theory is supported by a study conducted by Mak et al, examining the CTX composition of different members of a ciguateric food web \cite{mak2013pacific}.

%Indigenous population's understanding of ciguateric fish may help prevent CFP. 
%Experiemntation of P-CTX-1 injeted into rats intravenously showed rapid distribution to tissue and high bioavailibility of toxin via Neuro2a assay 
%cite{ledreux2013bioavailability}.

\begin{table}
\caption{Different known congeners of CTXs and their toxicity.}
\label{tbl:CTXTable}
\begin{tabular}{  | p{2cm} | p{1.5cm} | p{2.5cm} | p{4cm} | p{4cm} |}
\hline
\textbf{Toxin} & \textbf{Type} & \textbf{Molecular Ion [M +H]$^{+}$} & \textbf{Source} & \textbf{Toxicity (LD50, i.p. mice)} \\
\hline
P-CTX-1, P-CTX-1B & I & 1111.6 & Moray eel (\emph{Gymnothorax javanicus}) \cite{murata1990structures,lewis1991purification} & P-CTX-1 0.35 $\mu$g/kg \cite{murata1990structures}; P-CTX-1B 0.25$\mu$g/kg \cite{lewis1991purification} \\
\hline
P-CTX-2 & I & 1095.5 \cite{lewis1991purification} & Moray eel (\emph{Gymnothorax javanicus}) \cite{lewis1991purification} & 2.3 $\mu$g/kg \cite{lewis1991purification} \\
\hline
 P-CTX-3 & I & 1095.5 \cite{lewis1991purification} & Moray eel (\emph{Gymnothorax javanicus}) \cite{lewis1991purification} & 0.9 $\mu$g/kg \cite{lewis1991purification} \\
 \hline
 P-CTX-4A & I & 1061.6 \cite{yasumoto2000structural} & \emph{Gambierdiscus} sp. \cite{yasumoto2000structural}; \emph{G. polynesiensis} \cite{chinain2010growth} & 12$\mu$g/kg \cite{chinain2010growth} \\
 \hline
 P-CTX-4B & I & 1061.6 \cite{yasumoto2000structural} & \emph{Gambierdiscus} sp. \cite{yasumoto2000structural}; \emph{G. polynesiensis} \cite{chinain2010growth} & 20$\mu$g/kg \cite{chinain2010growth}\\
 \hline
 P-CTX-3C & II & 1023.6 \cite{satake1993structure} &  \emph{Gambierdiscus} sp. \cite{satake1993structure}; \emph{G. polynesiensis} \cite{chinain2010growth} & 2.5$\mu$g/kg \cite{chinain2010growth}\\
 \hline
 P-49-epi-CTX-3C & II & 1023.6 \cite{chinain2010growth} & \emph{Gambierdiscus} sp. \cite{satake1993structure}; \emph{G. polynesiensis} \cite{chinain2010growth} & 8$\mu$g/kg\cite{chinain2010growth}\\
 \hline
 P-M-seco-CTX-3C & II & 1041.6 \cite{chinain2010growth} &\emph{Gambierdiscus} sp. \cite{satake1993structure}; \emph{G. polynesiensis} \cite{chinain2010growth} & 10$\mu$g/kg \cite{chinain2010growth}\\
 \hline
 C-CTX-1 & III & 1141.6 \cite{vernoux1997isolation,pottier2002characterisation} & Horse-eye jack (\emph{Caranx latus}) \cite{vernoux1997isolation,pottier2002characterisation} & 3.6$\mu$g/kg \cite{vernoux1997isolation}\\
 \hline
 C-CTX-2 & III & 1141.6 \cite{vernoux1997isolation,pottier2002characterisation}& Horse-eye jack (\emph{Caranx latus}) \cite{vernoux1997isolation,pottier2002characterisation} & Toxic \cite{vernoux1997isolation}\\
 \hline
 I-CTX-1 & N/A & 1141.6 \cite{hamilton2002isolation}& Red bass (\emph{Lutjanus bohar}); Red emperor (\emph{Lutjanus sebae}) \cite{hamilton2002isolation} & Toxic \cite{hamilton2002isolation} \\
 \hline
\end{tabular}
\end{table}
\FloatBarrier

\subsection{MTX}

MTXs are the largest known proteinacious natural product \cite{yokoyama1988some,murata1993structure} and are highly toxic Ca$^{2+}$ channel activators with a LD$_{50}$ 0.05 $\mu$g/kg - there are only a couple of known bacterial, proteinacious toxins with higher potency \cite{yokoyama1988some,murata1993structure}. However the Ca$^{2+}$ activation is a secondary process of MTX activity, the primary mode of action in mammalian cells  is still unclear \cite{van2000diversity}. They are water soluble, polyether ladder compound first isolated form herbivorous Acanthurids fish gut in 1976 \cite{yasumoto1976toxicity}. Their complex structure and stereochemistry was elucidated in the 1990s using stereoscopic studies and partial synthesis \cite{murata1993structure,murata1994structure,satake1995structural,nonomura1996complete,zheng1996complete}. \\
There are three isolated MTXs (table ~\ref{tbl:MTXTable}) - designated MTX-1, MTX-2 and MTX-3, isolated from a \emph{Gambierdiscus} sp. in Queensland, Australia \cite{holmes1994purification}. The first two are structurally larger than the third and it is suspected, though unproven, that MTX-1 is synonymous with the MTX first isolated. \\ %who isolated it first? yokoyama? 

%Biophysical characteristics of pacific MTXs different to caribbean MTX \cite{lu2013caribbean}. Carribean MTX not been charactrised, so whether the difference in action is structural is speculative. %check lit

%MTX easier to detect and quantify than CTX due to sulphate esters using liquid chromatography-electronspray ionisation-mass spectrometry (T. Harwood pers. com.). Solvolysis (desulphonation) reduces toxicity 100-fold or more \cite{murata1991effect}. 

\begin{table}
\caption{Different known congeners of MTXs and their toxicity.}
\label{tbl:MTXTable}
\begin{tabular}{ |  p{2cm} | p{2cm} | p{3cm} | p{3cm} | p{4cm} | }
\hline
\textbf{Origin} & \textbf{Toxin} & \textbf{Molecular Ion [M +H]$^{+}$} & \textbf{Source} & \textbf{Toxicity (LD50, i.p. mice)} \\
\hline
 Pacific & MTX-1 & 3422 \cite{holmes1994purification,murata1993structure} & \emph{Gambierdiscus} sp. \cite{holmes1994purification} & 0.05$\mu$g/kg \cite{murata1993structure}\\
 \hline
 Pacific & MTX-2 & 3298 \cite{holmes1994purification} & \emph{Gambierdiscus} sp. \cite{holmes1994purification} & 0.08$\mu$g/kg \cite{holmes1994purification}\\
 \hline
 Pacific & MTX-3 & 1060   \cite{holmes1994purification} & \emph{Gambierdiscus} sp. \cite{holmes1994purification}; \emph{G. australes}, \emph{G. pacificus} \cite{rhodes2014production} &  0.08$\mu$g/kg for \emph{G. australes}  \cite{rhodes2014production} \\
 \hline
\end{tabular}
\end{table}
\FloatBarrier
%can i use and cite gurjeets chemdraws?
\subsection{Other}
Further toxins reportedly synthesised by \emph{Gambierdiscus} spp. are are gambieric acid, gambierol and gambieroxide \cite{watanabe2013gambieroxide,satake1993gambierol,nagai1992gambieric}.
Gambierol is less toxic than CTX with 80 $\mu$g/kg i.p. and 150 $\mu$g/kg orally in MBA, but fast acting with lethality setting in within 3 hrs\cite{ito2003pathological}. Gambierol could play a potential role in CFP, which to date has not been investigated. Though less toxic than CTX, it is still a highly toxic substance which could bioaccumulate in the food chain \cite{rhodes2014production}.\\
Gambieric acids A, B, C and D showed potent antifungal properties but no toxicity via MBA at 1000 $\mu$g/kg i.p. and hence are not of concern as a health hazard, including CFP \cite{rhodes2014production,nagai1992gambieric}.
Recently gambieroxide was isolated from \emph{G. toxicus} by Watanabe et al \cite{watanabe2013gambieroxide}. Gambieroxide's stereostructure has been elucidated by NMR and LC-MS/MS, concluding that it is structurally similar to yessotoxin (YTX) which is a lipophillic toxin found in filter feeding shellfish  \cite{tubaro2010yessotoxins}. Known producers of YTX are \emph{Protoceratium reticulatum}, \emph{Lingulodinium polyedrum} and \emph{Gonyaulax spinifera} \cite{tubaro2010yessotoxins} which are not  phylogenetically closely related to \emph{Gambierdiscus}.  Gambieroxide needs to be further investigated for structure, toxicity and bioaccumulative potential for assessment as a health threat, including CFP. \\

\section{Toxicity of different \emph{Gambierdiscus} spp.}
\emph{Gambierdiscus} spp. produce CTXs and MTXs \cite{murata1990structures,holmes1991strain,satake1993structure,holmes1994purification,satake1996isolation}, however, some wild and culturable strains have been recorded to not produce either or both toxins in measurable quantities \cite{gillespie1985significance,holmes1990toxicity}. Hence toxin production varies between species. Most studies were conducted in the 16 year time period pre-dating the discovery of \emph{G. belizeanus} in 1995 \cite{faust1995observation} and automatically fell under the caveat of the previously assumed type species \emph{Gambierdiscus toxicus}. Table ~\ref{tbl:GeoTable} shows known toxicity of each \emph{Gambierdiscus} species as detected via various essays and whether they have been characterised by LC-MS. \\
The Caribbean species \emph{G. excentricus} was tested with a neuroblastoma cytotoxicity assay (N2A) which indicated both MTX and CTX toxicity \cite{fraga2011gambierdiscus}. However the toxin profile needs to be extrapolated with LC-MS.
Chinain et al identified the toxins produced by two strains of \emph{G. polynesiensis} with LC-MS. The major toxin produced was P-CTX-3C, other type II (M-seco-CTX-3C, 49-epi-CTX-3C) and type I (CTX-4A, CTX-4B) toxins were detected. The different strains produced the same toxins, but at different proportions \cite{chinain2010growth}. In an earlier study, Chinain et al detected toxicity of the water soluble fraction of \emph{G. polynesiensis} in a mouse bioassay (MBA), implicating MTX activity \cite{chinain1999morphology}. Rhodes et al produced contradictory evidence as they found no LC-MS measurable MTX in a Cook Island strain of \emph{G. polynesiensis} \cite{rhodes2014production}.
\emph{G. australes} extracts isolated by Rhodes et al indicated CTX and MTX presence by displaying toxic activity in both lipid and water soluble fractions when tested with MBA. However, LC-MS/MS analysis did not detect any CTXs and MTXs were not tested for. A follow up study on the same strain with LC-MS also found no CTXs, but potent MTX-3 like compounds were detected \cite{rhodes2014production,rhodes2010toxic}. In contrast, three different \emph{G. australes} strains isolated from French Polynesia, tested with a radiobinding assay (RBA), produced P-CTX-3C like compounds \cite{chinain2010growth}. The Cook Islands isolate from Rhodes et al was tested against against a P-CTX-3C standard for LC-MS with a negative result \cite{rhodes2014production}. If the French Polynesian strains's RBA results are verified with LC-MS analysis, this could be a case in point for species specific, differential toxin production. \\
Holland et al conducted a large scale two year toxicity screening of 56 strains from 6 species of \emph{G. belizeanus, G. caribaeus, G. carolinianus, G. carpenteri, G. rutzleri} and  \emph{Gambierdiscus} ribotype 2, using human erythrocyte lysis assay (HELA) \cite{holland2013differences}. Haemolytic activity is a well established property of MTX \cite{igarashi1999mechanisms}, which is the implication from Holland's study. It showed that the haemolytic activity varied marginally between strains but stayed consistent for each strain over the two year sampling period, indicating that MTX production, and congener suite, varies between the species but remains consistent over time \cite{holland2013differences}.
In contrast, a recent study of CTX-like toxicity using mouse neuroblastoma assay (MNA) of the new \emph{Gambiersiscus} ribotypes 4, 5 and 6 found that the isolates from a ciguateric site were up to 4\- fold more toxic compared to isolates from a geographically closely related non-ciguateric reference site \cite{xu2014distribution}. This observed difference in MTX and CTX production staying constant and varying respectively, could be due to CTX production increasing in a ciguateric setting, or something other than temperature variation changes MTX production. The findings of both of these studies need to be verified with LC-MS. \\
These case studies show that while bioassays are important to indicate toxicity, LC-MS studies of every species need to be conducted to define their toxin profiles. Completing toxin characterisation for each individual species is imperative to ascertain the possible implication of species migration and HABs on CFP predictions and ecological impact. 

\section{Local abundance of \emph{Gambierdiscus} spp.}
A correlation of available average abundance data of \emph{Gambierdiscus} spp. in the Pacific and the Atlantic by Litaker et al found cell concentrations up to 1,000,000 cells g $^{-1}$ wet weight algae, however the most commonly found abundance was between 100 to 1,000 cells g$^{-1}$ wet weight algae \cite{litaker2010global}. In some locations those cell densities are consistent throughout the year \cite{chinain1999seasonal}.
Localised benthic HABs (BHABs) of \emph{Gambierdiscus} noted in literature in both Pacific and Atlantic regions \cite{nakajima1981toxicity,withers1984ciguatera,chinain1999seasonal,darius2007ciguatera}.
Cell density of \emph{Gambierdiscus} spp. in blooms varies between 10,000 and 100,000 cell g$^{-1}$ wet weight algae \cite{chinain1999seasonal}. In the absence of an estimate for cell density of \emph{Gambierdiscus} which leads to CFP, it has been proposed  that high cell concentrations of \emph{Gambierdiscus} equate high CFP risk, especially in a bloom scenario \cite{litaker2010global}. 
There are a number of observations which stand in contradiction to this statement. For example unidentified \emph{Gambierdiscus} spp. have been found around Crete but no CFP outbreaks have been recorded and the only two properly identified strains of \emph{G. toxicus} were found to be a non toxic \cite{caillaud2010update,chinain2010growth}.
Furthermore, the sampling within a ciguateric system in the Republic of Kiribati showed comparatively low colonisation of the macro algae \emph{Halimeda} with 0 - 174 cells g$^{-1}$ wet weight algae but, 91 \% of fish sampled in the area exceeded the quarantine threshold of 0.01$\mu$g/kg P-CTX-1 equivalent \cite{xu2014distribution,chan2011spatial}. The authors postulate that \emph{Gambierdiscus} spp. could preferentially colonise another substrate, though Parson et al did not find such a correlation upon extensive sampling over three years - \emph{Halimedia} and \emph{Laurencia} spp. were favoured, though the hydrodynamics and species of Gambierdiscus seem to relevant to colonisation \cite{parsonICHA}. On this basis, Xu et als other hypothesis that a yet unidentified Gambierdiscus sp. supplements the 6 low to medium toxic strains reported seems more likely \cite{xu2014distribution,bomber1988r}. Another option could be that low level, consistent exposure of CFP causes build up in fish.
Another option could be that low level, consistent exposure of CFP causes build up in fish. An added complication in estimating abundances is that benthic dinoflagellate sampling is difficult due to sediments and Gambierdiscus cell distribution can be uneven over small distances which causes difficulty in estimating large areas  \cite{lobel1988assessment,ballantine1988population,litaker2010global}.
 Gambierdiscus spp. are usually found co-occurring at sample sites, dating as far back as the first description of G. toxicus [Litaker]. This makes identification of species composition difficult and reliant on molecular techniques for conclusive results. Furthermore, recent research by Adachi et al suggests that depth of the sampling site needs to be taken into consideration. They found a comparative change in species composition in the same area by depth. On the Japanese coast they found the highest Gambierdiscus cell concentration at 30 m - and a different species composition, some introduced/absent, in comparison to 0-3 m. As most predatory fish that are implicated in CFP dwell at 10-30 m, this is a relevant depth to investigate species composition. The impact of the irradiation factor should be considered and compared when sampling \cite{adachiICHA}. An exception to the common observation of multi-species occurrence is when G. carpenteri was found in isolation at three separate sampling sites in temperate Australian waters. Recorded cell densities were 8256 g $^{-1}$ wet weight algae \cite{kohli2014high}. No CTX or MTX was detected in this strain via LC-MS/MS, however the MTX-like fraction did display toxicity via MBA. Whether this strain could have contributed to the CFP incidences in the region on the basis of MTX production is yet to be established  \cite{kohli2014high}. 
To glean meaningful and comparable data on Gambierdiscus abundance and distribution, unified methodology need to be established and implemented. The measurement of cells per g $^{-1}$ wet weight algae is misleading as weight does not equate surface area available for colonisation. Surface area measured in cm$^{2}$is more adequate, however more time intensive especially in field studies \cite{parsonICHA} [Lobel]. Tester et al have suggested a solution to this problem - using an artificial substrate made out of window screen mesh. They have extensively studied the comparable time colonisation frequency of Gambierdiscus and shown that 24 hr incubation at the field site is comparable to algae present \cite{tester2ICHA,tester2014sampling}. It appears that the risk of CFP is species and hence toxin profile dependent\cite{berdalet2012global}. Consequently the potential impact of the Gambierdiscus bloom is highly species dependent with outcomes ranging from an annoyance to a health hazard. It is vital that toxin production of each species is elucidated for monitoring purposes. \\%Considering that a low toxicity representatives of \emph{Gambierdiscus} could participate in a bloom, or toxin production is not be triggered, 


\section{Geographic abundance of \emph{Gambierdiscus}}
% global map with references - GEOHAB report too close?
%Xu 14 and Nishimura 13 have info on gambierdiscus phylogeny and clades!
%difference in CTXs
The current state of biogeographic knowledge for most benthic dinoflagellates is poor - \emph{Gambierdiscus} is comparatively widely sampled, but the issues surrounding species identification confounds elucidating a clear picture \cite{marine2014}.

Some \emph{Gambierdiscus} species have been classified as endemic to either Pacific or Atlantic, while others have been found globally distributed \cite{berdalet2012global,litaker2010global}. It has been suggested that with more extensive sampling, the distribution is likely to be global for all \emph{Gambierdiscus} species \cite{testerICHA}. However currently, due to the relatively recent progress regarding the phylogeny within the genus as well as the continued discovery of new species, understanding about \emph{Gambierdiscus} distribution and abundance is fragmentary, as per table ~\ref{tbl:GeoTable}. \\
Species identified only in the Pacific include \emph{G. australes} and \emph{G. pacificus}, while Atlantic endemic species are \emph{G. ruetzleri}, \emph{G. excentricus}, \emph{G. silvae} and \emph{Gambierdiscus} ribotype 2. Species found in both regions include \emph{G. belizeanus} \emph{G. caribaeus} \emph{G. carpenteri} \emph{G. carolinianus} \cite{litaker2010global,litaker2009taxonomy,berdalet2012global,fraga2014genus}. %incl scabrosus! \\
The global distribution and abundance of \emph{Gambierdiscus} is not well understood. The primary contributing factor to this uncertainty is the default species designation of \emph{G. toxicus} over decades. Detangling the phylogeny of \emph{Gambierdiscus} was initiated by Litaker et al in 2009 based on morphology and genetic markers \cite{litaker2009taxonomy}. Since then, several new species types and ribotypes have been recorded. To consolidate the phylogeny, genetic analysis needs to be consistently employed. Older data based purely on morphology needs to be dismissed for species identification as some species appear similar. For example, G. yasumotoi has been primarily reported in the Pacific, except for one report in the Mexican-Caribbean \cite{hernandez2004species}. However this singular record precedes the discovery of the other globular species \emph{G. ruetzleri}, which is endemic to the Atlantic region \cite{litaker2009taxonomy}. A recent report of this species at the southern Kuwait coast and the Gulf of Aqaba in Jordan \cite{saburova2013new}, could indicate that this species is globally distributed also \cite{xu2014distribution}.
Similarly, \emph{G. polynesiensis} found throughout Pacific table ~\ref{tbl:GeoTable} but once found near Canary Islands in the Atlantic \cite{fraga2011gambierdiscus}. \\
 Table ~\ref{tbl:GeoTable} shows the global distribution of each \emph{Gambierdiscus} sp. as well as their respective recorded toxicities - the interrelationship between the two is imperative for HAB monitoring \cite{testerICHA}.
Another considerable factor hindering the elucidation of abundance an distribution is sampling infrequency and inconsistent between the regions. In the hundreds of samples from Atlantic (Caribbean/Gulf of Mexico/West Indies/South-east US coast Florida to North Carolina), no Pacific-specific species have been detected \cite{berdalet2012global,litaker2010global}. However most of Pacific has not been sampled for \emph{Gambierdiscus} spp. so a similar conclusion about Atlantic species in the Pacific cannot be made and assertions about endemism or restricted distribution can not be made conclusively.

%amend with Ledreux 13, Xu 14, Rhodes 14, 
% aligizaki 09 suez canal?
	
	
	%\begin{landscape}
	\begin{longtable}{ | p{2cm} | p{5.5cm} | p{2.3cm} | p{2.3cm} | p{2.3cm} | }
	\caption{Geographical distribution of different \emph{Gambierdiscus} spp. and their associated toxin profile; table adapted from Kholi \cite{kohli2013Gambierdiscus}} \\
	\label{tbl:GeoTable}
	\textbf{Species} & \textbf{Geographical Distribution} & \textbf{CTX toxicity via assays} & \textbf{MTX toxicity via assays} & \textbf{Toxicity through LC-MS} \\
	\hline
	\emph{G. australes} & French Polynesia \cite{chinain1999morphology}; Japan \cite{nishimura2013genetic}; Cook Islands \cite{rhodes2010toxic}; Hawaii, USA \cite{litaker2009taxonomy}; Pakistan \cite{munir2011occurrence} & MBA +ve \cite{rhodes2010toxic}; RBA +ve \cite{chinain2010growth} & HELA +ve \cite{holland2013differences}; MBA +ve \cite{rhodes2010toxic} & CTX N/D \cite{}; MTX +ve \cite{}\\
	\hline
	\emph{G. belizeanus} & Belize \cite{faust1995observation}, Florida, USA \cite{litaker2009taxonomy}; Gulf of Aqaba, Jordan \cite{saburova2013new}; Mexican Caribbean \cite{hernandez2004species}; Malaysia \cite{leaw2011first}; Pakistan \cite{munir2011occurrence}; Marakei, Republic of Kiribati \cite{xu2014distribution}; Queensland, Australia \cite{}; St. Barthelemy \cite{litaker2010global} & RBA +ve \cite{chinain2010growth} & HELA +ve \cite{holland2013differences} & N/A \\ %saburova 13 found in jordan
	\hline
	\emph{G. caribaeus} & Florida, USA \cite{litaker2009taxonomy}; Belize \cite{litaker2009taxonomy}; Tahiti, French Polynesia \cite{litaker2009taxonomy}; Palau, Hawai, USA \cite{litaker2009taxonomy}; Flower Garden Banks National Marine Sanctuary, USA \cite{holland2013differences}; Ocho Rios, Jamaica \cite{}; Bahamas \cite{litaker2010global}; Grand Cayman Islands \cite{}; Tol-truk, Micronesia \cite{litaker2010global}; Jeju Island, Korea \cite{jeong2012first} & N/A & HELA +ve \cite{holland2013differences} & N/A \\
	\hline
	\emph{G. carolinianus} & North Carolina, USA \cite{litaker2009taxonomy}; Bermuda, Mexico \cite{litaker2010global}; Puerto Rico \cite{holland2013differences}; Flower Garden Banks National Marine Sanctuary, USA \cite{holland2013differences}; Ocho Rios, Jamaica \cite{holland2013differences}; Crete, Greece \cite{holland2013differences} & N/A & HELA +ve \cite{holland2013differences} & N/A\\
	\hline
	\emph{G. carpenteri} & Belize \cite{litaker2009taxonomy}; Guam, USA \cite{litaker2009taxonomy}; Republic of Fiji \cite{litaker2009taxonomy}; Hawaii, USA \cite{}; Dry Tortugas, USA \cite{holland2013differences}; Flower Garden Banks National Marine Sanctuary, USA \cite{holland2013differences}; Ocho Rios, Jamaica \cite{}; Merimbula \& Exmouth, Australia \cite{kohli2014cob}; Marakei, Republic of Kiribati \cite{xu2014distribution} & N/A & HELA +ve \cite{holland2013differences} & N/A \\
	\hline
	\emph{G. excentricus} & Canary Islands \cite{fraga2011gambierdiscus} & NCBA +ve \cite{fraga2011gambierdiscus} & NCBA +ve \cite{fraga2011gambierdiscus} & N/A \\
	\hline
		\emph{G. pacificus} & French Polynesia \cite{chinain1999morphology}; Marshall \& Society Islands, Micronesia \cite{litaker2010global}; Kota Kinabalu \& Sipandan Island, Malaysia \cite{mohammad2005marine}; Marakei, Republic of Kiribati \cite{xu2014distribution}; Nha Trang, Vietnam \cite{} & MBA +ve \cite{chinain1999morphology} & MBA +ve \cite{chinain1999morphology} & N/A \\
		\hline
	\emph{G. polynesiensis} & French Polynesia \cite{chinain1999morphology}; Canary Islands \cite{fraga2011gambierdiscus}; Pakistan \cite{munir2011occurrence}; Nha Trang, Vietnam \cite{} & MBA +ve \cite{chinain1999morphology}; RBA +ve \cite{chinain2010growth} & MBA +ve \cite{chinain1999morphology} & CTX +ve \cite{chinain2010growth}; MTX N/A \\
	\hline
	\emph{G. ruetzleri} & North Carolina, USA \cite{litaker2009taxonomy}; Belize \cite{litaker2009taxonomy} & N/A & HELA +ve \cite{holland2013differences} & N/A \\
	\hline
	\emph{G. scabrosus} & Japan \cite{nishimura2013genetic} & MBA +ve \cite{nishimura2013genetic} & MBA +ve \cite{nishimura2013genetic} & N/A \\
	\hline
	\emph{G. silvae} & Belize \cite{litaker2010global,fraga2014genus} &  N/A & N/A & N/A \\
	\hline
	\emph{G. toxicus} & Tahiti, French Polynesia \cite{adachi1979thecal,chinain1999morphology}; Mexican Caribbean \cite{hernandez2004species}; New Caledonia \cite{chinain1999morphology}; Reunion Island \cite{chinain1999morphology}; Nha Trang, Vietnam \cite{roeder2010characteristic} & MBA -ve \cite{chinain1999morphology}; RBA +ve \cite{chinain2010growth} & MBA +ve \cite{chinain1999morphology} & N/A \\
	\hline
	\emph{G. yasumotoi} & Singapore \cite{holmes1998gambierdiscus}; Japan \cite{nishimura2013genetic}; Gulf of Aqaba, Jordan \cite{saburova2013new}; Kuwait \cite{saburova2013new}; Mexican Caribbean \cite{hernandez2004species}; Queensland, Australia \cite{}; Great Barrier Reef, Australia \cite{murray2014molecular}; Nha Trang, Vietnam \cite{} & MBA +ve \cite{} & N/A & N/A \\ 
	\hline
	\emph{Gambier- discus} ribotype 2 & Belize \cite{litaker2010global}; Martinique, France \cite{litaker2010global}; Puerto Rico \cite{holland2013differences} & N/A & HELA +ve \cite{holland2013differences} & N/A\\
	\hline
	\emph{Gambier- discus} ribotype 3 & & & & \\ %where from?
	\emph{Gambier- discus} sp. type 2 & Japan \cite{nishimura2013genetic} & MBA -ve \cite{nishimura2013genetic} & MBA -ve \cite{nishimura2013genetic} & N/A \\
	\hline
	\emph{Gambier- discus} sp. type 3 & Japan \cite{nishimura2013genetic} & MBA -ve \cite{nishimura2013genetic} & MBA +ve \cite{nishimura2013genetic} & N/A \\
	\hline
	\emph{Gambier- discus} sp. type 4 &  Marakei, Republic of Kiribati \cite{xu2014distribution}& MNA +ve \cite{xu2014distribution} & N/A & N/A \\
	\hline
	\emph{Gambier- discus} sp. type 5 & Marakei, Republic of Kiribati \cite{xu2014distribution}& MNA +ve \cite{xu2014distribution} & N/A & N/A \\
	\hline
	\emph{Gambier- discus} sp. type 6 & Marakei, Republic of Kiribati \cite{xu2014distribution}& MNA +ve \cite{xu2014distribution} & N/A & N/A \\
\hline
	\end{longtable}
	%\end{landscape}
	\FloatBarrier
\subsection{Pacific and Indian Ocean regions}
The genus is named after Gambier Islands in French Polynesia, where it was first identified \cite{adachi1979thecal} and shortly after in the Caribbean \cite{besada1982observations}.
\emph{G. toxicus}, \emph{G. belizeanus}, \emph{G. yasumotoi}, \emph{G. australes}, \emph{G. pacificus}, \emph{G. polynesiensis}, \emph{G. caribaeus}, \emph{G. carpenteri} reported in Hawaii, Pacific Islands, Australia, South East Asia, Northern Indian Ocean.
\emph{G. scabrosus} and two new \emph{Gambierdiscus} ribotypes have been reported off the coast of Japan \cite{nishimura2013genetic,nishimura2014morphology}. \emph{G. scabrosus} was described by Nishimura et al in 2014, however the toxicity data in table ~\ref{tbl:GeoTable} is from their publication in the previous year where the isolate was designated \emph{Gambierdiscus} sp. type 1 \cite{nishimura2013genetic,nishimura2014morphology}. \\
A study published this year reported \emph{G. carpenteri} blooms in Australia in a warm water system of 16.6 - 17\celsius shows that this genus can thrive in regions outside the tropical zones \cite{kohli2014high}.
Furthermore, \emph{Gambierdiscus} of undetermined species found in Philippines \cite{gillespie1987possible}, Hong kong \cite{lu2004harmful}, Indonesia \cite{praseno1996hab}, Mauritius \cite{hurbungs2002seasonal}, the Mexican Pacific coast \cite{ceballos2006analisis} and regions around Madagascar \cite{grzebyk1994ecology}. Hernandez et al report that CFP is very common in Mexican-Caribbean, but that this not reflected in literature \cite{hernandez2004species}.\\
A report of CFP outbreak in 1994 in Madagascar is unusual as it reports a 20\% mortality rate from a sample size of 500 people. However the facilities to confirm CTX presence were not available \cite{habermehl1994severe}. 
%needs to include kiribati - Mak13

\subsection{Atlantic Ocean region}
An early \emph{Gambierdiscus} look alike was recorded in 1948 at Cape Verde Islands \cite{silva1956contribution} and 1979 Key Largos Florida \cite{taylor1979description}.
\emph{Gambierdiscus} spp. have been described from east coast of USA, Caribbean and Mediterranean as per table ~\ref{tbl:GeoTable}. The latter was reported for the first time in 2003 near Crete in the Mediterranean Sea as an alien species emigrating to a new habitat, which could herald an emerging region for CFP \cite{aligizaki2008morphological,zenetos2009aquatic}.
Undefined \emph{Gambierdiscus} spp. have been more frequently reported at Cyprus, Rhodes and Saronikos Gulf \cite{aligizaki2009toxic,aligizaki2010diversity}, French West Indies \cite{lobel1988assessment}, Cuba \cite{delgado2006epiphytic} and Veracruz \cite{okolodkov2007seasonal}.
In Central and South America non specified species have been recorded in Costa Rica and Brazil (M. Montero pers. comm. in \cite{parsons2012gambierdiscus}).
Recent reports of CFP have emerged from Africa, specifically in Angola \cite{berdalet2012global}, the Canary Islands \cite{boada2010ciguatera} and Cameroon \cite{bienfang2008ciguatera} indicating expansion of \emph{Gambierdiscus} habitat or migratory vector species. %Aligizaki 08 harmful algae news alledgedly found gam in cannary islands? need to find the damn paper to check!
%boda et al report CTX confirmation via LC-MS in seafood

\section{Detection of CTX in seafood}
% and MTX been checked for??
The name of CFP is derived from the word "cigua", a native Cuban description of a turban shelled snail which was implicated in an outbreak of sickness in Spanish explorers in 1500s \cite{gudger1930poisonous}. Later the presence of CTX was confirmed the in turban snail \emph{Turbo argyrstoma} \cite{yasumoto1976toxicity}, however the majority of cases reported occur from large reef fish \cite{hokama2001ciguatera,lewis2001changing,dechraoui2005use,laurent2005ciguatera}. The recent consumption of large reef fish is one of the major factors in diagnostics of CFP due to rare retention of fish sample for testing. \\
The inconsistency of diagnostics contributes to the estimated report rate of CFP at less than 20\% globally \cite{dickey2010ciguatera}. Contributing factors to common misdiagnosis for CFP likely include the excess of 175 gastrointestinal, neurological and cardiovascular symptoms \cite{sims1987theoretical}; potential variance of symptoms with portion size \cite{wong2008features,mak2013pacific}; different symptomatology between individuals with the same exposure source and exacerbation of manifestation with repeat exposure \cite{bagnis1979clinical,glaziou1993study}; and that it could be associated with other illnesses such as decompression sickness \cite{adams1993outbreak}, chronic fatigue syndrome, multiple sclerosis \cite{lindsay1997chronic,ting2001ciguatera} and brain tumours \cite{lindsay1997chronic}. \\
The under reporting and diagnostic issues mentioned, compounded by absence of an adequate commercial test kit \cite{wong2005study}, the exact figure for number of global CFP cases cannot be determined. Carnivorous fish are main cause of CFP, however herbivorous fish (eg. surgeonfish, parrotfish) pose important intermediate vector in the food chain \cite{cruz2006macroalgal,randall1958review,mak2013pacific}.%insert biomagnification here? % reference bollocksing the 2nd commercial test kits
Tables ~\ref{tbl:HerbTable}, ~\ref{tbl:OmniTable} and ~\ref{tbl:CarnTable} show a summary of fish species and other marine fauna which have tested positive for CTX from ciguatera prone regions and following reported outbreaks, though several hundred unverified fish species implicated in CFP \cite{}. %need ref from gurjeet FAO 2004 
Most of the entries stem from mid-latitude tropical and sub-tropical zones, meshing with the distribution of \emph{Gambierdiscus} in table ~\ref{tbl:GeoTable}. Tables \ref{tbl:HerbTable}, ~\ref{tbl:OmniTable} and ~\ref{tbl:CarnTable}, adapted from Kholi, \cite{kohli2013Gambierdiscus} list CTX detection only, as it is currently not common practice to test for MTXs in suspected ciguateric specimens. 
Studies predominantly focus on large reef fish, but CTXs in concentrations above the toxicity  threshold have been detected in such unlikely vectors as octopus, lobster, and jellyfish \cite{mak2013pacific,zlotnick1995ciguatera}.
Sharks have been suspected after CFP outbreaks but no remnant samples were left available for testing \cite{boisier1995fatal,lehane2000ciguatera,habermehl1994severe}. In the case of a Madagascan outbreak in 1994, the mortality rate from implicated shark consumption was 20\% \cite{habermehl1994severe}. 
Studies pertaining to the toxicity of \emph{Gambierdiscus} towards copepods and their nauplii were conducted \cite{lee2014toxicity}, as copepods feed on microalgae attached to macroalgae. They could constitute the primary consumers in the food chain, which can lead to toxin accumulation \cite{raisuddin2007copepod} and constitute as the first component of the ciguateric food web. The impact of \emph{G. yasumotoi} on the model copepod \emph{Tigriopus japonicus} was assessed in a study conducted by Lee et al, finding that the presence of \emph{G. yasumotoi} cells, or to a lesser extent growth media stock, cause immobility and mortality on adult copepods, where an the concentration of \emph{G. yasumotoi} present correlated with effect. Furthermore nauplii had a lower toxicity threshold \cite{lee2014toxicity}. \\
It has been shown that ciguatera toxins passing through the food chain in the Pacific undergo biotransformation. %different types between different trophic levels
Some modifications can increase the potency of the toxins up to 10-fold, compared to the toxin produced by the \emph{Gambierdiscus} sp. under observation \cite{lewis2006ciguatera}.
In 2005 Hung et al established that the CFP of three family members was caused by CTX present in Baracuda fish eggs \cite{hung2005persistent}. This indicates that bioaccumulated CTXs can be passed on to offspring which start their life cycle with CTX concentrations above the toxicity threshold to elicit CFP.  \\ %contrats with study describing loss of CTXs over time
On small island nations, native fisherman know ciguatera prone zones and fish species to avoid so the local community relies on garnered knowledge. However in French Polynesia, CTXs were detected in fish considered to be safe by local lore  \cite{darius2007ciguatera}.
The development of reliable, inexpensive and laymen operable detection method for ciguateric food sources has been a decades long quest. Initially a stick enzyme immunoassay (SEIA) \cite{hokama1985rapid} was developed, followed by a solid phase immunoassay (SPIA) \cite{hokama1990simplified}.These methods progressed to the development of commercial kits (i.e. Cigua-check\textregistered \ and Ciguatect\textregistered), although these have yielded a considerable amount of false positives and false negatives \cite{wong2005study}. Cigua-check\textregistered is no longer manufactured.%find dickey 1994 Evaluation of a solid-phase immunobead assay for detection of ciguatera-related biotoxins in Caribbean finfish for ciguatect
A host of bioassays have also been developed, though these are problematic due to toxin specificity and quantification, inefficiencies and ethical considerations \cite{dickey2010ciguatera}.
Bioassays for ciguateric fish detection have used chickens \cite{}, cats \cite{larson1967ciguatera}, mongooses \cite{hokama1977radioimmunoassay}, diptera larvae \cite{labrousse1996toxicological}, brine shrimp \cite{granade1976ciguatera} and mosquitoes \cite{bagnis1987use}.
Many alternative biochemical assays have been proposed to replace bioassays for sea food testing. Radioimmunoassay \cite{hokama1977radioimmunoassay} preceded the financially more viable, higher throughput enzyme-linked immunosorbent assay (ELISA) \cite{hokama1983rapid}, which has been refined to reliably detect pg concentration of CTX in fish tissue \cite{campora2008detection,campora2010evaluating}. Currently, the most commonly used detection method is MBA via intraperitoneal (i.p.) injection, as per table ~\ref{tbl:GeoTable}, even though it does not provide linear dose-response relationship with CTX toxicity \cite{hoffman1983mouse}. 
Munday identified a major flaw with this system – i.p. is not the standard uptake of toxins in humans and results from this method do not give a dose dependent ratio to the oral route. Similarly, administration by gavage gives a skewered toxicity threshold result due to murine stomach contents being solid, so liquids forced in through gavage are absorbed preferentially without dilution \cite{mundyICHA}. While the difference in LD50 between administration routes isn’t available for CTX and MTX, for other the algal BHAB toxins palytoxin and pinnatoxin F LD50 (µg/kg) respectively for i.p. is 0.6 and 14.4, via gavage 640 and 25, while LD50 for voluntary feeding is >2500 and 50 \cite{botana2014seafood}. CODEX Alimentarius Comission, the international food standards organisation set up by WHO and the food and agriculture organisation of the UN have endorsed that voluntary feeding should take precedence in seafood toxin studies []. Elucidating the impact of low level oral accumulation of prevalent toxins in seafood, such as CTX, should be of high priority \cite{mundyICHA}.
To reliably elucidate toxin profiles and structures in samples, it is imperative to employ chromatographic techniques such as high-performance liquid chromatography (HPLC), ultra-performance liquid chromatography (UPLC) and LC-MS with nuclear magnetic resonance (NMR) \cite{legrand1989isolation,murata1990structures,murata1990structures,satake1996isolation,diogened2014chemistry} and radio ligand binding (RLB) \cite{hamilton2002multiple,hamilton2002isolation}. These methods are effective and informative, however their limitation lies in expense and requirement of special training hence they are not routine in the field and unlikely to become so. %reference?
Confirmation of CTX or MTX congener(s) present is conducted by UPLC/HPLC then LC-MS, where the toxin is first isolated and the fractionated to compare against their known molecular weights (tables ~\ref{tbl:CTXTable} and ~\ref{tbl:MTXTable}). For expedient field application, a rapid analysis method was proposed \cite{lewis2009rapid}, however acquisition of standards difficult due to limited natural CTX compounds \cite{berdalet2012global} and though synthetic production is possible, it is highly complex \cite{hirama2001total}. A compounding problem has been the co-elution peaks of similar compounds. However the Cawthron Institute in New Zealand has developed a method of fractionation and UPLC-MS/MS technique which can detect MTX-1, MTX-3, P-CTX-1, P-CTX-3B, P-CTX-3C and P-CTX-4B in a single run \cite{selwoodICHA}.
Alternate screening options are sodium channel binding (N2A) \cite{dickey2010ciguatera} and RBA \cite{poli1997identification,darius2007ciguatera} which are both recommended by European Food Standard Association \cite{}. %ummm... check this shit out
However, these assays have the same limitation of not discerning between CTX and MTX congeners, so a positive result would need validation via LC-MS analysis \cite{diogened2014chemistry,mak2013pacific}. 

%MBA used to confirm MTX in \emph{Ctenochaetus striatus} (striped bristletooth) \cite{bagnis1987use}  ---> this article is about CTX only - MTX not mentioned in text.



\FloatBarrier
\begin{longtable}[l]{ | p{2cm} | p{3cm} | p{4.5cm} | p{2cm} | p{3cm} | }
	\caption{CTXs and congeners detected by various assays in herbivorous fish and other vectors.}\\
	\hline
	\label{tbl:HerbTable}
	\textbf{Group} & \textbf{Latin name} (Common name) & \textbf{Source} & \textbf{CTX (if detected)} & \textbf{Methods of detection} \\
	\hline
	Giant clam 	& \emph{Hippopus hippopus} (Horses hoof) & Republic of Vanuatu \cite{laurent2012ciguatera} & CTX +\cite{laurent2012ciguatera} & N2A \cite{laurent2012ciguatera}; RBA \cite{laurent2012ciguatera} \\
	& \emph{Tridacna maxima} (Small giant clam) & Raivavae, French Polynesia \cite{pawlowiez2013evaluation} & CTX +ve\cite{pawlowiez2013evaluation} & N2A \cite{pawlowiez2013evaluation} \\
	& \emph{Tridacna} sp. (Giant clam) & New Caledonia, French Polynesia \cite{laurent2012ciguatera} & CTX +\cite{laurent2012ciguatera} & MBA \cite{laurent2012ciguatera}; N2A \cite{laurent2012ciguatera}; RBA \cite{laurent2012ciguatera} \\
	& \emph{Tripneustes gratilla} (Hawaiian sea urchin) & Raivavae, French Polynesia \cite{pawlowiez2013evaluation} & CTX +ve\cite{pawlowiez2013evaluation} & N2A \cite{pawlowiez2013evaluation} \\
	\hline
	Parrotfish & \emph{Chlorurus frontalis} (Pacific slopehead parrotfish) & Tubuai, French Polynesia \cite{darius2007ciguatera} & CTX +ve \cite{darius2007ciguatera} & RBA \cite{darius2007ciguatera} \\
	& \emph{Chlorurus microrhinos} (Steepheaded parrotfish) & Tubuai, French Polynesia \cite{darius2007ciguatera}; French Polynesia \cite{chinain2014mail} & CTX +ve \cite{darius2007ciguatera,chinain2014mail} & N2A//RBA \cite{chinain2014mail}; RBA \cite{darius2007ciguatera} \\
	& \emph{Hipposcarus longiceps} (Pacific longnose parrotfish) & French Polynesia \cite{chinain2014mail}; Republic of Kiribati \cite{mak2013pacific} & P-CTX-2 \cite{mak2013pacific}; P-CTX-3 \cite{mak2013pacific} & LC-MS/MS \cite{mak2013pacific}; N2A//RBA \cite{chinain2014mail}; N2A \cite{mak2013pacific} \\
	& \emph{Leptoscarus vaigiensis} (Marbled parrotfish) & French Polynesia \cite{chinain2014mail} & CTX +ve \cite{chinain2014mail} & N2A//RBA \cite{chinain2014mail} \\
	& \emph{Scarus altipinnis} (Filament-finned parrot fish) & Tubuai, French Polynesia \cite{darius2007ciguatera}, French Polynesia \cite{chinain2014mail} & CTX +ve \cite{darius2007ciguatera,chinain2014mail} & N2A//RBA \cite{chinain2014mail}; RBA \cite{darius2007ciguatera} \\
	& \emph{Scarus ghobban} (Blue-barred parrotfish) & Tubuai, French Polynesia \cite{darius2007ciguatera}; Republic of Kiribati \cite{mak2013pacific} & P-CTX-1 \cite{mak2013pacific}; P-CTX-2 \cite{mak2013pacific}; P-CTX-3 \cite{mak2013pacific} & LC-MS/MS \cite{mak2013pacific}; N2A \cite{mak2013pacific}; RBA \cite{darius2007ciguatera} \\
	& \emph{Scarus gibbus} (Heavy beak parrotfish) & French Polynesia \cite{bagnis1987use,satake1996isolation,chinain2014mail}; Tahiti, French Polynesia \cite{pompon1983ciguatera} & P-CTX-4A \cite{satake1996isolation} & HPLC/HNMR \cite{satake1996isolation}; MBA \cite{bagnis1987use,satake1996isolation,pompon1983ciguatera}; MQBA \cite{bagnis1987use}; N2A//RBA \cite{chinain2014mail}; \\
	& \emph{Scarus jonesi} & French Polynesia \cite{bagnis1987use} & CTX +ve \cite{bagnis1987use} & Cat BA \cite{bagnis1987use}; MBA \cite{bagnis1987use}; MQBA \cite{bagnis1987use} \\
	& \emph{Scarus rubroviolaceus} & Nuku Hiva, French Polynesia \cite{darius2007ciguatera} & CTX +ve \cite{darius2007ciguatera} & RBA \cite{darius2007ciguatera}\\
	& \emph{Scarus russelii} (Eclipse parrotfish) & Republic of Kiribati \cite{mak2013pacific} & P-CTX-1 \cite{mak2013pacific}; P-CTX-2 \cite{mak2013pacific}; P-CTX-3 \cite{mak2013pacific} & LC-MS/MS \cite{mak2013pacific}; N2A \cite{mak2013pacific} \\
    \hline
	Rabbitfish & \emph{Siganus argenteus} (treamlined spinefoot) & Republic of Kiribati \cite{mak2013pacific} & P-CTX-1 \cite{mak2013pacific} & LC-MS/MS \cite{mak2013pacific}; N2A \cite{mak2013pacific} \\
	 & \emph{Siganus rivulatus} (Marbled spinefoot) & Eastern Mediterranean \cite{bentur2007ciguatoxin} & CTX +ve \cite{bentur2007ciguatoxin} & Cigua-check \textregistered \cite{bentur2007ciguatoxin}\\
	\hline
	Sea cucumber & \emph{Holothuria} spp. & Hawaii, USA \cite{park2000microbial} & CTX +ve \cite{park2000microbial} & Ciguatect \textregistered \cite{park2000microbial} \\
	\hline
	Surgeonfish &\emph{Acanthurus dussumieri} (Dussumier's surgeonfish) & Hawaii, USA \cite{hokama1993evaluation} & CTX +ve \cite{hokama1993evaluation} & MBA \cite{hokama1993evaluation}; S-EIA \cite{hokama1993evaluation}; SPIA \cite{hokama1993evaluation} \\
	& \emph{Acanthurus leucopareius} (Whitebar surgeonfish) & French Polynesia \cite{chinain2014mail} & CTX +ve \cite{chinain2014mail} & N2A//RBA \cite{chinain2014mail} \\ & \emph{Acanthurus nigricans} (Whitecheek surgeonfish) & French Polynesia \cite{chinain2014mail} & CTX +ve \cite{chinain2014mail} & N2A//RBA \cite{chinain2014mail} \\
	& \emph{Acanthurus lineatus} (Lined surgeonfish) & Republic of Kiribati \cite{mak2013pacific} & P-CTX-1 \cite{mak2013pacific}; P-CTX-2 \cite{mak2013pacific}; P-CTX-3 \cite{mak2013pacific} & LC-MS/MS \cite{mak2013pacific}; N2A \cite{mak2013pacific} \\
	& \emph{Acanthurus maculiceps} (White-freckled surgeonfish) & Republic of Kiribati \cite{mak2013pacific} & P-CTX-1 \cite{mak2013pacific} & LC-MS/MS \cite{mak2013pacific}; N2A \cite{mak2013pacific} \\
		&  \emph{Acanthurus mata} (Elongated surgeonfish) & Republic of Kiribati \cite{mak2013pacific} & P-CTX-1 \cite{mak2013pacific}; P-CTX-2 \cite{mak2013pacific}; P-CTX-3 \cite{mak2013pacific} & LC-MS/MS \cite{mak2013pacific}; N2A \cite{mak2013pacific} \\
	& \emph{Acanthurus nigricans} (Whitecheek surgeonfish) & Republic of Kiribati \cite{mak2013pacific} & P-CTX-1 \cite{mak2013pacific}& LC-MS/MS \cite{mak2013pacific}; N2A \cite{mak2013pacific} \\
	& \emph{Acanthurus nigroris} (Bluelined surgeonfish) & Hawaii, USA \cite{hokama1993evaluation} & CTX +ve \cite{hokama1993evaluation} & MBA \cite{hokama1993evaluation}; S-EIA \cite{hokama1993evaluation}; SPIA \cite{hokama1993evaluation} \\
	& \emph{Acanthurus olivaceus} (Orangeband surgeonfish) & Hawaii, USA \cite{hokama1993evaluation} & CTX +ve \cite{hokama1993evaluation} & MBA \cite{hokama1993evaluation}; S-EIA \cite{hokama1993evaluation}; SPIA \cite{hokama1993evaluation} \\
	& \emph{Acanthurus }sp. & Hawaii, USA \cite{hokama1990simplified} & CTX +ve \cite{hokama1990simplified} & S-EIA \cite{hokama1990simplified}; SPIA \cite{hokama1990simplified} \\
	& \emph{Acanthurus xanthopterus} (Yellowfin surgeonfish) & Nuku Hiva, French Polynesia \cite{darius2007ciguatera}; Republic of Kiribati \cite{mak2013pacific} &  P-CTX-1 \cite{mak2013pacific}; P-CTX-2 \cite{mak2013pacific}; P-CTX-3 \cite{mak2013pacific} & LC-MS/MS \cite{mak2013pacific}; N2A \cite{mak2013pacific}; RBA \cite{darius2007ciguatera}\\
	& \emph{Ctenochaetus striatus} (Striped bristletooth) & Nuku Hiva, French Polynesia \cite{darius2007ciguatera}; French Polynesia \cite{chinain2014mail}; Republic of Kiribati \cite{mak2013pacific} & P-CTX-1 \cite{mak2013pacific}; P-CTX-2 \cite{mak2013pacific}; P-CTX-3 \cite{mak2013pacific} & LC-MS/MS \cite{mak2013pacific}; N2A//RBA \cite{chinain2014mail}; N2A \cite{mak2013pacific}; RBA \cite{darius2007ciguatera} \\
	\hline
	\end{longtable}
	\FloatBarrier
	
	\begin{longtable}[l]{ | p{2cm} | p{3cm} | p{4.5cm} | p{2cm} | p{3cm} | }
	\caption{CTXs and congeners detected by various assays in omnivorous fish and other vectors.}\\
	\hline
	\label{tbl:OmniTable}
	\textbf{Group} & \textbf{Latin name} (Common name) & \textbf{Source} & \textbf{CTX (if detected)} & \textbf{Methods of detection} \\
	\hline
	Butterflyfish & \emph{Chaetodon auriga} (Threadfin butterflyfish) & Republic of Kiribati \cite{mak2013pacific} & P-CTX-1 \cite{mak2013pacific}; P-CTX-2 \cite{mak2013pacific}; P-CTX-3 \cite{mak2013pacific} & LC-MS/MS \cite{mak2013pacific}; N2A \cite{mak2013pacific} \\
	\hline
	Crustaceans & \emph{Carpilius convexus} (Convex reef crab) & Republic of Kiribati \cite{mak2013pacific} & P-CTX-1 \cite{mak2013pacific} & LC-MS/MS \cite{mak2013pacific}; N2A \cite{mak2013pacific} \\
	& \emph{Charybdis paucidentata} (Red swimming crab) & Republic of Kiribati \cite{mak2013pacific} & P-CTX-1 \cite{mak2013pacific} & LC-MS/MS \cite{mak2013pacific}; N2A \cite{mak2013pacific} \\
	& \emph{Panulirus penicillatus} (Pronghorn spiny lobster) & Republic of Kiribati \cite{mak2013pacific} & P-CTX-1 \cite{mak2013pacific} & LC-MS/MS \cite{mak2013pacific}; N2A \cite{mak2013pacific} \\
	\hline
	Jellyfish & \emph{Cnidaria} sp. & American Samoa, US \cite{zlotnick1995ciguatera} & CTX +ve \cite{zlotnick1995ciguatera} & SPIA \cite{zlotnick1995ciguatera} \\
	\hline
	Mackerel & \emph{Scomberomorus cavalla} (Kink mackerel) & Florida, USA \cite{dickey2008ciguatera}; St. Barthelemey \cite{pottier2001ciguatera,vernoux1986heterogeneity}; Guadeloupe \cite{pottier2001ciguatera} & C-CTX-1 \cite{dickey2008ciguatera}; C-CTX-2 \cite{dickey2008ciguatera} & Chick BA \cite{pottier2001ciguatera}; LC-MS/MS \cite{dickey2008ciguatera}; MBA \cite{vernoux1986heterogeneity}; N2A \cite{dickey2008ciguatera} TLC \cite{vernoux1986heterogeneity} \\
	& \emph{Scomberomorus commerson} (Spanish mackerel) & Hervey Bay, Australia \cite{} & & \\ % this shit is entirely out of whack
	\hline
	Mullet & \emph{Crenimugil crenilabis} (Fringelip mullet) & French Polynesia \cite{bagnis1987use}; Nuku Hiva, French Polynesia \cite{darius2007ciguatera} & CTX +ve \cite{darius2007ciguatera} & MBA \cite{bagnis1987use}; MQBA \cite{bagnis1987use}; RBA \cite{darius2007ciguatera}\\
	& \emph{Liza vaigiensis} (Thinlip grey mullet) & Nuku Hiva, French Polynesia \cite{darius2007ciguatera} & CTX +ve \cite{darius2007ciguatera} & RBA \cite{darius2007ciguatera} \\
	\hline
	Knifejaw & \emph{Oplegnathus punctatus} (Spotted knifejaw) & Miyazaki, Japan \cite{yogi2011detailed} & CTX-3C \cite{yogi2011detailed} & HPLC/MS \cite{yogi2011detailed}\\
	\hline
Pufferfish & \emph{Arothron nigropunctatus} (Blackspotted puffer) & Republic of Kiribati \cite{mak2013pacific} & P-CTX-1 \cite{mak2013pacific}; P-CTX-2 \cite{mak2013pacific}; P-CTX-3 \cite{mak2013pacific} & LC-MS/MS \cite{mak2013pacific}; N2A \cite{mak2013pacific} \\
\hline
	Salmon & Farmed salmon & Chile \cite{ebesu1994first} & CTX +ve \cite{ebesu1994first} & SPIA \cite{ebesu1994first}\\
	\hline
	Sea chub & \emph{Kyphosus cinerascens} (Blue sea chub) & Tubuai , French Polynesia \cite{darius2007ciguatera}; Nuku Hiva, French Polynesia \cite{darius2007ciguatera}; Rapa, French Polynesia \cite{pawlowiez2013evaluation} & CTX +ve \cite{pawlowiez2013evaluation,darius2007ciguatera} & N2A \cite{pawlowiez2013evaluation}; RBA \cite{darius2007ciguatera} \\
	\hline
	Starfish & \emph{Ophiocoma} spp. (Brittle stars) & Hawaii, USA \cite{park2000microbial} & CTX +ve \cite{park2000microbial} & Ciguatect \textregistered \cite{park2000microbial} \\
	\hline
Triggerfish	& \emph{Balistapus undulatus} (Orange-lined triggerfish) & Republic of Kiribati \cite{mak2013pacific} & P-CTX-1 \cite{mak2013pacific}; P-CTX-2 \cite{mak2013pacific}; P-CTX-3 \cite{mak2013pacific} & LC-MS/MS \cite{mak2013pacific}; N2A \cite{mak2013pacific} \\
\hline
	\end{longtable}
	%Lionfish & \emph{Pterois volitans} (Red lionfish) & St. Croix, US Virgin Islands \cite{robertson2013invasive}; St. John, US Virgin Islands \cite{robertson2013invasive}; St. Thomas, US Virgin Islands \cite{robertson2013invasive} & CTX +ve\cite{robertson2013invasive} & N2A \cite{robertson2013invasive} \\
		\FloatBarrier
	\begin{longtable}[l]{ | p{2cm} | p{3cm} | p{4.5cm} | p{2cm} | p{3cm} | }
	\caption{CTXs and congeners detected by various assays in carnivorous fish and other vectors.}\\
	\hline
	\label{tbl:CarnTable}
	\textbf{Group} & \textbf{Latin name} (Common name) & \textbf{Source} & \textbf{CTX detected} & \textbf{Methods of detection} \\
	\hline
	Amberjack & \emph{Seriola dumerili} (Greater amberjack) & Canary Islands \cite{caillaud2012towards}; Selvagens Islands, Portugal \cite{otero2010first}; Hawaii, USA \cite{campora2008detection,hokama1977radioimmunoassay,hokama1983rapid,hokama1993evaluation}; Haiti \cite{poli1997identification}; St. Barthelemy \cite{vernoux1986heterogeneity}; St. Thomas, US Virgin Islands \cite{granade1976ciguatera} & C-CTX-1 \cite{poli1997identification};C-CTX-1B \cite{otero2010first}; P-CTX-3C and CTX analogues from Carribean and Indic waters \cite{otero2010first} & UPLC/MS \cite{otero2010first}; HPLC/MS \cite{poli1997identification}; TLC \cite{vernoux1986heterogeneity}; BSBA \cite{granade1976ciguatera}; MGBA \cite{campora2008detection,granade1976ciguatera}; MBA \cite{hokama1983rapid,hokama1993evaluation,vernoux1986heterogeneity}; S-EIA \cite{hokama1993evaluation}; SPIA \cite{otero2010first}; RIA \cite{campora2008detection,hokama1983rapid}; ELISA \cite{campora2008detection}; N2A \cite{caillaud2012towards,campora2008detection}; RBA \cite{} \\
	& \emph{Seriola fasciata} (Lesser amberjack) & Selvagens Islands, Portugal \cite{otero2010first}; West Africa \cite{boada2010ciguatera} & C-CTX-1 \cite{boada2010ciguatera}; C-CTX-1B \cite{otero2010first}; P-CTX-3C and CTX analogues from Carribean and Indic waters \cite{otero2010first} & LCMS/MS \cite{boada2010ciguatera}; UPLC/MS \cite{otero2010first}\\
	& \emph{Seriola rivoliana} (Almaco jack) & Canary Islands \cite{campora2010evaluating}; Hawaii, USA \cite{campora2008detection}; St. Thomas, US Virgin Islands \cite{granade1976ciguatera} & C-CTX-1 \cite{} & LCMS/MS \cite{}; BSBA \cite{granade1976ciguatera}; MGBA \cite{granade1976ciguatera}; ELISA \cite{campora2008detection,campora2010evaluating}; N2A \cite{campora2008detection,campora2010evaluating} \\
	\hline
	Angelfish  & \emph{Pomacanthus imperator} (Emperor angelfish) & Republic of Kiribati \cite{mak2013pacific} & P-CTX-1 \cite{mak2013pacific}; P-CTX-2 \cite{mak2013pacific}; P-CTX-3 \cite{mak2013pacific} & LC-MS/MS \cite{mak2013pacific}; N2A \cite{mak2013pacific} \\
	\hline
	Baracuda & \emph{Sphyraena barracuda} (Great barracuda) & The Bahamas \cite{o2012linking}; Cameroon, West Africa \cite{bienfang2008ciguatera}; Florida Keys, USA \cite{dechraoui2005use}; Guadeloupe, French West Indies \cite{pottier2003identification,pottier2001ciguatera}; St Barthelemy\cite{pottier2001ciguatera,vernoux1986heterogeneity}; French Polynesia \cite{bagnis1987use,chinain2014mail} & C-CTX-1 \cite{dechraoui2005use,pottier2003identification}; CTX-1 \cite{pottier2003identification}; CTX-2 and isomers \cite{pottier2003identification}; CTX congeners and other \cite{pottier2003identification}& Cat BA \cite{bagnis1987use}; Chick BA \cite{pottier2001ciguatera}; MQBA \cite{bagnis1987use}, MBA \cite{bagnis1987use,pottier2002characterisation,pottier2003identification}; N2A//RBA \cite{chinain2014mail}; N2A \cite{o2012linking} \\
	& \emph{Sphyraena jello} (Pickhandle barracuda) & Hervey Bay, Australia \cite{lewis1984ciguatoxin} & CTX +ve \cite{lewis1984ciguatoxin} & MBA \cite{lewis1984ciguatoxin}; TLC \cite{lewis1984ciguatoxin} \\
	& \emph{Sphyraena} sp. (Barracuda) & California, USA \cite{hokama1990simplified} & CTX +ve \cite{hokama1990simplified} & N2A \cite{hokama1990simplified}; S-EIA \cite{hokama1990simplified}; SPIA \cite{hokama1990simplified} \\
	& \emph{Sphyraena} sp. (Barracuda fish eggs) & Taiwan \cite{hung2005persistent} & CTX +ve \cite{hung2005persistent} & MBA \cite{hung2005persistent}; N2A \cite{hung2005persistent} \\
	\hline
	Butterflyfish & \emph{Chaetodon meyeri} (Scrawled butterflyfish) & Republic of Kiribati \cite{mak2013pacific} & P-CTX-1 \cite{mak2013pacific}; P-CTX-2 \cite{mak2013pacific}; P-CTX-3 \cite{mak2013pacific} & LC-MS/MS \cite{mak2013pacific}; N2A \cite{mak2013pacific} \\
	 & \emph{Forcipiger longirostris} (Longnose butterflyfish) & Republic of Kiribati \cite{mak2013pacific} & P-CTX-1 \cite{mak2013pacific}; P-CTX-3 \cite{mak2013pacific} & LC-MS/MS \cite{mak2013pacific}; N2A \cite{mak2013pacific} \\
	 \hline
	Eel & \emph{Gymnothorax flavimarginatus} (Yellow-edged moray) & Republic of Kiribati \cite{mak2013pacific} & P-CTX-1 \cite{mak2013pacific}; P-CTX-2 \cite{mak2013pacific}; P-CTX-3 \cite{mak2013pacific} & LC-MS/MS \cite{mak2013pacific}; N2A \cite{mak2013pacific} \\
	& \emph{Gymnothorax funebris} (Green Moray) & St. Barthelemy \cite{vernoux1986heterogeneity} & CTX +ve \cite{vernoux1986heterogeneity} & MBA \cite{vernoux1986heterogeneity}; TLC \cite{vernoux1986heterogeneity} \\
	& \emph{Gymnothorax javanicus} & Tuamotu Archipelago \& Tahiti, French Polynesia \cite{labrousse1996toxicological,murata1990structures,legrand1997two}; French Polynesia \cite{chinain2014mail}; Tarawa, Republic of Kiribati \cite{lewis1997characterization}; Republic of Kiribati \cite{mak2013pacific}; Hawaii, USA \cite{scheuer1967ciguatoxin} & P-CTX-1 \cite{murata1990structures,lewis1991purification,lewis1997characterization,mak2013pacific}; P-CTX-2 \cite{lewis1991purification,mak2013pacific}; P-CTX-3 \cite{mak2013pacific,lewis1991purification,lewis1997characterization}; 2,3-dihydroxy-CTX-3C \cite{satake1998isolation}; 51-hydroxy-CTX-3C \cite{satake1998isolation}; P-CTX-4B \cite{murata1990structures,lewis1991purification} & DLBA \cite{labrousse1996toxicological}; HPLC/HNMR \cite{murata1990structures,lewis1991purification}; HPLC/MS \cite{lewis1997characterization,satake1998isolation}; LC-MS/MS \cite{mak2013pacific}; MBA \cite{lewis1997characterization,scheuer1967ciguatoxin,satake1998isolation}; N2A//RBA \cite{chinain2014mail}; N2A \cite{mak2013pacific}; TLC \cite{scheuer1967ciguatoxin} \\
	\hline
	Emperor bream & \emph{Lethrinus olivaceus} (Longface emperor) & Nuku Hiva, French Polynesia \cite{darius2007ciguatera} & CTX +ve \cite{darius2007ciguatera} & RBA \cite{darius2007ciguatera} \\
	& \emph{Lethrinus miniatus} (Trumpet emperor) & French Polynesia \cite{bagnis1987use} & CTX +ve \cite{bagnis1987use} & Cat MBA \cite{bagnis1987use}; MBA \cite{bagnis1987use}; MQBA \cite{bagnis1987use} \\
	& \emph{Lethrinus xanthochilus} (Yellowlip emperor) & French Polynesia \cite{chinain2014mail} & CTX +ve \cite{chinain2014mail} & N2A//RBA \cite{chinain2014mail} \\
	& \emph{Monotaxis grandoculis} (Big eye bream) & French Polynesia \cite{bagnis1987use,chinain2014mail} Republic of Kiribati \cite{mak2013pacific} & P-CTX-1 \cite{mak2013pacific} & Cat MBA \cite{bagnis1987use}; LC-MS/MS \cite{mak2013pacific}; MBA \cite{bagnis1987use}; MQBA \cite{bagnis1987use}; N2A//RBA \cite{chinain2014mail}; N2A \cite{mak2013pacific}\\
	\hline
	Goatfish & \emph{Mulloidichthys auriflamma} (Goldstriped goatfish) & Hawaii, USA \cite{hokama1990simplified} & CTX +ve \cite{hokama1990simplified} & S-EIA \cite{hokama1990simplified}; SPIA \cite{hokama1990simplified} \\
	& \emph{Mulloidichthys martinicus} (Yellow goatfish) & St. Barthelemy \cite{vernoux1986heterogeneity} & CTX +ve \cite{vernoux1986heterogeneity} & MBA \cite{vernoux1986heterogeneity}; TLC \cite{vernoux1986heterogeneity} \\
	& \emph{Parupeneus insularis} (Twosaddle goatfish) & Nuku Hiva, French Polynesia \cite{darius2007ciguatera} & CTX +ve \cite{darius2007ciguatera} & RBA \cite{darius2007ciguatera} \\
		& \emph{Parupeneus trifasciatus} (Indian doublebar goatfish) & Republic of Kiribati \cite{mak2013pacific} & P-CTX-1 \cite{mak2013pacific} & LC-MS/MS \cite{mak2013pacific}; N2A \cite{mak2013pacific} \\
	\hline
	Gastropod & \emph{Conus} spp. (Cone snails) & Hawaii, USA \cite{park2000microbial} & CTX +ve \cite{park2000microbial} & Ciguatect \textregistered \cite{park2000microbial} \\
	\hline
	Grouper & \emph{Cephalopholis argus} (Blue spotted grouper) & Nuku Hiva, French Polynesia \cite{darius2007ciguatera}; French Polynesia \cite{bagnis1987use}; Hawaii, USA \cite{campora2008detection}; Republic of Kiribati \cite{mak2013pacific} & P-CTX-1 \cite{mak2013pacific}; P-CTX-2 \cite{mak2013pacific}; P-CTX-3 \cite{mak2013pacific} & Cat BA \cite{bagnis1987use}; ELISA \cite{campora2008detection}; LC-MS/MS \cite{mak2013pacific}; MBA \cite{bagnis1987use}; MQBA \cite{bagnis1987use}; N2A \cite{campora2008detection,mak2013pacific}; RBA \cite{darius2007ciguatera} \\
	& \emph{Cephalopholis miniata} (Coral grouper) & Arafura Sea, Australia \cite{lucas1997pacific}; Republic of Fiji \cite{arnett2007ciguatera,dickey2008ciguatera} & P-CTX-1 \cite{arnett2007ciguatera,lucas1997pacific,dickey2008ciguatera} & HPLC/MS \cite{lucas1997pacific}; MBA \cite{lucas1997pacific}; N2A \cite{arnett2007ciguatera,dickey2008ciguatera} \\
	& \emph{Cephalopholis sexmaculata} (Sixspot grouper) & Republic of Kiribati \cite{mak2013pacific} & P-CTX-1 \cite{mak2013pacific} & LC-MS/MS \cite{mak2013pacific}; N2A \cite{mak2013pacific} \\
	& \emph{Epinephelus coeruleopunctatus} (Whitespotted grouper) & Rapa, French Polynesia \cite{pawlowiez2013evaluation}; French Polynesia \cite{chinain2014mail} Republic of Kiribati \cite{mak2013pacific} & P-CTX-1 \cite{mak2013pacific}; P-CTX-2 \cite{mak2013pacific}; P-CTX-3 \cite{mak2013pacific} &  LC-MS/MS \cite{mak2013pacific};N2A//RBA \cite{chinain2014mail}; N2A \cite{mak2013pacific,pawlowiez2013evaluation} \\
	& \emph{Epinephelus coioides} (Orange-spotted grouper) & Hong Kong, China \cite{wong2005study} & CTX +ve \cite{wong2005study} & MBA \cite{wong2005study} \\
	& \emph{Epinephelus fasciatus} (blacktip grouper) & French Polynesia \cite{chinain2014mail} & CTX +ve \cite{chinain2014mail} & N2A//RBA \cite{chinain2014mail} \\
	& \emph{Epinephelus fuscoguttatus} (Brown-marbled grouper) &  Republic of Kiribati \cite{mak2013pacific} & P-CTX-1 \cite{mak2013pacific}; P-CTX-2 \cite{mak2013pacific}; P-CTX-3 \cite{mak2013pacific} & LC-MS/MS \cite{mak2013pacific}; N2A \cite{mak2013pacific} \\
	& \emph{Epinephelus lanceolatus} (Giant grouper) & Hong Kong, China \cite{wong2009solid} & CTX +ve \cite{wong2009solid} & MBA \cite{wong2009solid} \\
	& \emph{Epinephelus merra} (Honeycomb grouper) & French Polynesia \cite{chinain2014mail}; Republic of Kiribati \cite{mak2013pacific} & P-CTX-1 \cite{mak2013pacific} &  LC-MS/MS \cite{mak2013pacific};N2A//RBA \cite{chinain2014mail}; N2A \cite{mak2013pacific} \\
	& \emph{Epinephelus microdon} (Marble grouper) & French Polynesia \cite{bagnis1987use} & CTX +\cite{bagnis1987use} & Cat BA \cite{bagnis1987use}; MBA \cite{bagnis1987use}; MQBA \cite{bagnis1987use} \\
	& \emph{Epinephelus multinotatus} (White-blotched grouper) & Republic of Kiribati \cite{mak2013pacific} & P-CTX-1 \cite{mak2013pacific}; P-CTX-2 \cite{mak2013pacific}; P-CTX-3 \cite{mak2013pacific} & LC-MS/MS \cite{mak2013pacific}; N2A \cite{mak2013pacific} \\
	& \emph{Epinephelus mystacinus} (Misty grouper) & St. Thomas, US Virgin Islands \cite{granade1976ciguatera} & CTX +ve \cite{granade1976ciguatera} & BSBA \cite{granade1976ciguatera}; MGBA \cite{granade1976ciguatera} \\
	& \emph{Epinephelus morio} (Red grouper) & St. Barthelemy \cite{vernoux1986heterogeneity} & CTX +ve\cite{vernoux1986heterogeneity} & MBA \cite{vernoux1986heterogeneity}; TLC \cite{vernoux1986heterogeneity} \\
	& \emph{Epinephelus polyphekadion} (Camouflage grouper) & French Polynesia \cite{chinain2014mail}; Republic of Kiribati \cite{mak2013pacific} & P-CTX-1 \cite{mak2013pacific}; P-CTX-2 \cite{mak2013pacific}; P-CTX-3 \cite{mak2013pacific} &  LC-MS/MS \cite{mak2013pacific}; N2A//RBA \cite{chinain2014mail}; N2A \cite{mak2013pacific}  \\
	& \emph{Epinephelus} sp. & Baja California, Mexico \cite{lechuga1995documented} & CTX +ve \cite{lechuga1995documented} & MBA \cite{lechuga1995documented} \\
	& \emph{Epinephelus spilotoceps} (Foursaddle grouper) & Republic of Kiribati \cite{mak2013pacific} & P-CTX-1 \cite{mak2013pacific}; P-CTX-2 \cite{mak2013pacific}; P-CTX-3 \cite{mak2013pacific} & LC-MS/MS \cite{mak2013pacific}; N2A \cite{mak2013pacific} \\
	& \emph{Epinephelus tauvina} (Greasy grouper) & Republic of Kiribati \cite{mak2013pacific} & P-CTX-1 \cite{mak2013pacific}; P-CTX-2 \cite{mak2013pacific}; P-CTX-3 \cite{mak2013pacific} & LC-MS/MS \cite{mak2013pacific}; N2A \cite{mak2013pacific} \\
	& \emph{Mycteroperca bonaci} (Black grouper) & Key Largo, USA \cite{dickey2008ciguatera} & C-CTX-1 \cite{dickey2008ciguatera}; C-CTX-2 \cite{dickey2008ciguatera} & LC-MS/MS \cite{dickey2008ciguatera}; N2A \cite{dickey2008ciguatera} \\
	& \emph{Mycteroperca prionura} (Sawtail grouper) & Baja California, Mexico \cite{sierra1998overview} & CTX-1 \cite{sierra1998overview} & HPLC/MS \cite{sierra1998overview}; MBA \cite{sierra1998overview} \\
	& \emph{Mycteroperca} sp. & Baja California, Mexico \cite{lechuga1995documented} & CTX +ve \cite{lechuga1995documented} & MBA \cite{lechuga1995documented} \\
	& \emph{Mycteroperca venenosa} (Yellowfin grouper) & Guadeloupe, French West Indies \cite{}; St. Barthelemy \cite{} & CTX +ve \cite{} & Chick BA \cite{}\\
	& \emph{Plectropomus aerolatus} (Squaretail coral grouper) & Hong Kong, China \cite{wong2005study} & CTX +ve \cite{wong2005study} & MBA \cite{wong2005study} \\
	& \emph{Plectropomus laevis} (Blacksaddles coral grouper) & Hong Kong, China \cite{wong2008features}; French Polynesia \cite{chinain2014mail} & CTX +ve \cite{wong2008features,chinain2014mail} &  MBA \cite{wong2008features}; N2A//RBA \cite{chinain2014mail}\\
	& \emph{Plectropomus leopardus} (Leopard coral grouper) & Tahiti, French Polynesia \cite{pompon1983ciguatera}; Tubuai, French Polynesia \cite{darius2007ciguatera}; French Polynesia \cite{bagnis1987use}; Hong Kong, China \cite{wong2005study} & CTX +ve \cite{wong2005study,darius2007ciguatera,bagnis1987use,pompon1983ciguatera} & Cat BA \cite{bagnis1987use}; MBA \cite{wong2005study,bagnis1987use,pompon1983ciguatera}; MQBA \cite{bagnis1987use}; RBA \cite{darius2007ciguatera} \\
	& \emph{Plectropomus} spp. (Coral trout) & Great Barrier Reef, Australia \cite{lewis1992multiple} & CTX-1 \cite{lewis1992multiple}; CTX-2 \cite{lewis1992multiple}; CTX-3 \cite{lewis1992multiple} & HPLC/MS \cite{lewis1992multiple}; MBA \cite{lewis1992multiple} \\
	& \emph{Variola albimarginata} & Hong Kong, China \cite{wong2008features} & CTX +ve \cite{wong2008features} & MBA \cite{wong2008features} \\
	& \emph{Variola louti} (Yellow-edged lyretail) & Republic of Kiribati \cite{mak2013pacific} & P-CTX-1 \cite{mak2013pacific}; P-CTX-2 \cite{mak2013pacific} & LC-MS/MS \cite{mak2013pacific}; N2A \cite{mak2013pacific} \\
	\hline
	Grunt & \emph{Pomadasys maculatus} (Blotched javelin) & Platypus Bay, Australia \cite{lewis1992multiple} & CTX-1 \cite{lewis1992multiple}; CTX-2 \cite{lewis1992multiple}; CTX-3 \cite{lewis1992multiple} & HPLC/MS \cite{lewis1992multiple}; MBA \cite{lewis1992multiple} \\
	\hline
	Hawkfish & \emph{Paracirrhites hemistictus} (Whitespot hawkfish) & Republic of Kiribati \cite{mak2013pacific} & P-CTX-1 \cite{mak2013pacific}; P-CTX-2 \cite{mak2013pacific} & LC-MS/MS \cite{mak2013pacific}; N2A \cite{mak2013pacific} \\
	\hline
	Hogfish & \emph{Bodianus bilunulatus} (Tarry hogfish) & Hawaii, USA \cite{hokama1993evaluation} & CTX +ve \cite{hokama1993evaluation} & MBA \cite{hokama1993evaluation}; S-EIA \cite{hokama1993evaluation} \\
	& \emph{Bodianus rufus} (Spanish hogfish) & St. Barthelemey \cite{vernoux1986heterogeneity} & CTX +ve \cite{vernoux1986heterogeneity} & MBA \cite{vernoux1986heterogeneity}; TLC \cite{vernoux1986heterogeneity} \\
	& \emph{Bodianus} sp. & Hawaii, USA \cite{hokama1990simplified} & CTX +ve \cite{hokama1990simplified} & SPIA \cite{hokama1990simplified}\\
	\hline
Octopus & \emph{Octopodidae} (Octopus) & Republic of Kiribati \cite{mak2013pacific} & P-CTX-1 \cite{mak2013pacific} & LC-MS/MS \cite{mak2013pacific}; N2A \cite{mak2013pacific} \\
\hline
	Jacks \& Scads & \emph{Caranx ignobilis} (Giant trevally) & Tubuai, French Polynesia \cite{darius2007ciguatera}; St. Barthelemey \cite{vernoux1986heterogeneity} & CTX +ve \cite{hokama1993evaluation,darius2007ciguatera} & MBA \cite{hokama1993evaluation}; S-EIA \cite{hokama1993evaluation}; RBA \cite{darius2007ciguatera} \\
	& \emph{Caranx latus} (Horse-eye jack) & French West Indies \cite{pottier2002characterisation}; St. Barthelemey \cite{vernoux1997isolation,lewis1998structure}; The Bahamas \cite{larson1967ciguatera}; St. Thomas, US Virgin Islands \cite{granade1976ciguatera} & C-CTX-1 \cite{pottier2002characterisation,vernoux1997isolation,lewis1998structure}; C-CTX-2 \cite{vernoux1997isolation,pottier2002characterisation,lewis1998structure}; 10 new congeners \cite{pottier2002characterisation} & BSBA \cite{granade1976ciguatera}; Cat BA \cite{larson1967ciguatera}; HPLC/MS \cite{pottier2002characterisation,vernoux1997isolation,lewis1998structure}; MBA \cite{pottier2002characterisation,vernoux1997isolation} ; MGBA \cite{granade1976ciguatera}\\
	& \emph{Caranx melampygus} (Bluefin trevally) & French Polynesia \cite{bagnis1987use,chinain2014mail}; Nuku Hiva, French Polynesia \cite{darius2007ciguatera} & CTX +ve \cite{bagnis1987use,darius2007ciguatera,chinain2014mail} & Cat BA \cite{bagnis1987use}; MBA \cite{bagnis1987use}; MQBA \cite{bagnis1987use}; N2A//RBA \cite{chinain2014mail}; RBA \cite{darius2007ciguatera} \\
	& \emph{Caranx papuensis} (Brassy trevally) & Tubuai, French Polynesia \cite{darius2007ciguatera} & CTX +ve \cite{darius2007ciguatera} & RBA \cite{darius2007ciguatera} \\
	& \emph{Caranx} sp. (Trevally) & Hawaii, USA \cite{hokama1993evaluation,hokama1990simplified} & CTX +ve \cite{hokama1990simplified,hokama1993evaluation} & MBA \cite{hokama1993evaluation}; S-EIA \cite{hokama1993evaluation,hokama1990simplified}; SPIA \cite{hokama1990simplified,hokama1993evaluation} \\
	& \emph{Pseudocaranx dentex} (Hard-tail jack) & Rapa, French Polynesia \cite{pawlowiez2013evaluation}; French Polynesia \cite{chinain2014mail} & CTX +ve \cite{chinain2014mail,pawlowiez2013evaluation} & N2A \cite{pawlowiez2013evaluation}; N2A//RBA \cite{chinain2014mail} \\
	\hline
	Lionfish & \emph{Pterois volitans} (Red lionfish) & St. Croix, US Virgin Islands \cite{robertson2013invasive}; St. John, US Virgin Islands \cite{robertson2013invasive}; St. Thomas, US Virgin Islands \cite{robertson2013invasive} & CTX +ve\cite{robertson2013invasive} & N2A \cite{robertson2013invasive} \\
	\hline
	Moorish Idol & \emph{zanclus cornutus} (Moorish idol) & Republic of Kiribati \cite{mak2013pacific} & P-CTX-1 \cite{mak2013pacific} & LC-MS/MS \cite{mak2013pacific}; N2A \cite{mak2013pacific} \\
	\hline
Porcupinefish & \emph{Diodon hystrix} (Spot-fin porcupinefish) & Republic of Kiribati \cite{mak2013pacific} & P-CTX-1 \cite{mak2013pacific}; P-CTX-3 \cite{mak2013pacific} & LC-MS/MS \cite{mak2013pacific}; N2A \cite{mak2013pacific} \\
& \emph{Diodon liturosus} (Black-blotched porcupinefish) & Republic of Kiribati \cite{mak2013pacific} & P-CTX-1 \cite{mak2013pacific}; P-CTX-3 \cite{mak2013pacific} & LC-MS/MS \cite{mak2013pacific}; N2A \cite{mak2013pacific} \\
\hline
	Seal & \emph{Monachus schauinslandi} (Hawaiian monk seal) & Hawaii, USA \cite{bottein2011identification} & P-CTX-3C \cite{bottein2011identification} & LC-MS/MS \cite{bottein2011identification}; N2A \cite{bottein2011identification} \\
	\hline
	Snapper & \emph{Aphareus furca} (Black forktail snapper) & Hawaii, USA \cite{hokama1990simplified} & CTX +ve \cite{hokama1990simplified} & S-EIA \cite{hokama1990simplified}; SPIA \cite{hokama1990simplified} \\
	& \emph{Aprion virescens} (Blue green snapper) & French Polynesia \cite{bagnis1987use} & CTX +ve \cite{bagnis1987use} & Cat BA \cite{bagnis1987use}; MBA \cite{bagnis1987use}; MQBA \cite{bagnis1987use} \\
	& \emph{Lutjanus argentimaculatus} (Mangrove red snapper) & Hong Kong \cite{wong2008features} & CTX +ve \cite{wong2008features} & MBA \cite{wong2008features} \\
	& \emph{Lutjanus bohar} (Two spot red snapper) & Nuku Hiva, French Polynesia \cite{darius2007ciguatera}; Tubuai, French Polynesia \cite{darius2007ciguatera}; French Polynesia \cite{bagnis1987use,chinain2014mail}; Minamitorishima Island, Japan \cite{yogi2011detailed}; Republic of Mauritius \cite{hamilton2002multiple,hamilton2002isolation}; Republic of Kiribati \cite{mak2013pacific}; Hawaii, USA \cite{hokama1990simplified} & I-CTX-1 \cite{hamilton2002multiple,hamilton2002isolation}; P-CTX-1B \cite{yogi2011detailed}; P-CTX-1 \cite{mak2013pacific}; P-CTX-2 \cite{mak2013pacific}; P-CTX-3 \cite{mak2013pacific} & HPLC/MS \cite{hamilton2002multiple,hamilton2002isolation}; Cat BA \cite{bagnis1987use}; LC-MS/MS \cite{yogi2011detailed,mak2013pacific}; MBA \cite{hamilton2002multiple,bagnis1987use,hamilton2002isolation}; MGBA \cite{hamilton2002multiple,hamilton2002isolation}; MQBA \cite{bagnis1987use}; N2A//RBA \cite{chinain2014mail}; N2A \cite{mak2013pacific}; RBA \cite{darius2007ciguatera} \\
	& \emph{Lutjanus buccanella} (Blackfin snapper) & St. Croix, US Virgin Islands \cite{hoffman1983mouse} & CTX +ve \cite{hoffman1983mouse} & MBA \cite{hoffman1983mouse}; TLC \cite{hoffman1983mouse} \\
	& \emph{Lutjanus fulvus} (Blacktail snapper) & French Polynesia \cite{chinain2014mail}; Republic of Kiribati \cite{mak2013pacific} & P-CTX-1 \cite{mak2013pacific}; P-CTX-2 \cite{mak2013pacific} & LC-MS/MS \cite{mak2013pacific}; N2A//RBA \cite{chinain2014mail}; N2A \cite{mak2013pacific} \\
	& \emph{Lutjanus gibbus} (Humpback red snapper) & Nuku Hiva, French Polynesia \cite{darius2007ciguatera}; French Polynesia \cite{bagnis1987use,chinain2014mail} & CTX +ve \cite{darius2007ciguatera,bagnis1987use,chinain2014mail} & MBA \cite{bagnis1987use}; MQBA \cite{bagnis1987use}; N2A//RBA \cite{chinain2014mail}; RBA \cite{darius2007ciguatera} \\
	& \emph{Lutjanus griseus} (Grey snapper) & French West Indies \cite{pottier2002analysis} & C-CTX-1 plus isomers and congeners \cite{pottier2002analysis} & HPLC/MS \cite{pottier2002analysis}; MBA \cite{pottier2002analysis}\\
	& \emph{Lutjanus kasmira} (Bluestripe snapper) & Hawaii, USA \cite{hokama1993evaluation};& CTX +ve \cite{hokama1993evaluation} & MBA \cite{hokama1993evaluation}; S-EIA \cite{hokama1993evaluation}; SPIA \cite{hokama1993evaluation} \\
	& \emph{Lutjanus monostigma} (One-spot snapper)& Nuku Hiva, French Polynesia \cite{darius2007ciguatera}; French Polynesia \cite{chinain2014mail} & CTX +ve \cite{darius2007ciguatera,chinain2014mail} & N2A//RBA \cite{chinain2014mail}; RBA \cite{darius2007ciguatera} \\
	& \emph{Lutjanus rivulatus} (Blubberlip snapper) & French Polynesia \cite{chinain2014mail} & CTX +ve \cite{chinain2014mail} & N2A//RBA \cite{chinain2014mail} \\
	& \emph{Lutjanus sebae} (Red emperor) & Republic of Mauritius \cite{hamilton2002multiple,hamilton2002isolation} & I-CTX-1 \cite{hamilton2002multiple,hamilton2002isolation}; I-CTX-2 \cite{hamilton2002multiple,hamilton2002isolation}; I-CTX-3 \cite{hamilton2002multiple,hamilton2002isolation}; I-CTX-4 \cite{hamilton2002multiple,hamilton2002isolation} & HPLC/MS \cite{hamilton2002multiple,hamilton2002isolation}; HPLC/MS/RLB \cite{hamilton2002multiple,hamilton2002isolation}; MBA \cite{hamilton2002multiple,hamilton2002isolation}; MGBA \cite{hamilton2002multiple,hamilton2002isolation} \\ %THE FUCK THERE IS ONLY MEANT TO BE I-CTX-1
	& \emph{Lutjanus} spp. & Antigua \cite{hokama1990simplified}; Okinawa, Japan \cite{yogi2011detailed}; Baja California, Mexico \cite{parrilla1992outbreaks}; St. Thomas, US Virgin Islands \cite{granade1976ciguatera}; West Africa \cite{bienfang2008ciguatera} & P-CTX-1B \cite{yogi2011detailed} & BSBA \cite{granade1976ciguatera}; LC-MS/MS \cite{yogi2011detailed}; MBA \cite{parrilla1992outbreaks}; MGBA \cite{granade1976ciguatera}; N2A \cite{bienfang2008ciguatera}; S-EIA \cite{hokama1990simplified}; SPIA \cite{hokama1990simplified} \\
	& \emph{Lutjanus stellatus} (Star snapper) & Hong Kong, China \cite{wong2008features} & CTX +ve \cite{wong2008features} & MBA \cite{wong2008features} \\
	& \emph{Macolor niger} (Black and white snapper) & Republic of Kiribati \cite{mak2013pacific} & P-CTX-1 \cite{mak2013pacific} & LC-MS/MS \cite{mak2013pacific}; N2A \cite{mak2013pacific} \\
	\hline
	Squirrelfish and Soldierfish & \emph{Myripristis berndti} (Blotcheye soldierfish) & Republic of Kiribati \cite{mak2013pacific} & P-CTX-1 \cite{mak2013pacific} & LC-MS/MS \cite{mak2013pacific}; N2A \cite{mak2013pacific} \\
	& \emph{Myripristis kuntee} (Epaulette soldierfish) & Hawaii, USA \cite{hokama1993evaluation} & CTX +ve \cite{hokama1993evaluation} & MBA \cite{hokama1993evaluation}; S-EIA \cite{hokama1993evaluation}; SPIA \cite{hokama1993evaluation}\\
	& \emph{Sargocentron spiniferum} (Sabre squirrelfish) & Nuku Hiva, French Polynesia \cite{darius2007ciguatera} & CTX +ve \cite{darius2007ciguatera} & RBA \cite{darius2007ciguatera} \\
	& \emph{Sargocentron tiere} (Blue lined squirrelfish) & Republic of Kiribati \cite{mak2013pacific} & P-CTX-1 \cite{mak2013pacific}; P-CTX-2 \cite{mak2013pacific}; P-CTX-3 \cite{mak2013pacific} & LC-MS/MS \cite{mak2013pacific}; N2A \cite{mak2013pacific} \\
	\hline
	Tilefish & \emph{Malacanthus plumieri} (Sand tilefish) & St. Barthelemey \cite{vernoux1986heterogeneity} & CTX +ve \cite{vernoux1986heterogeneity} & MBA \cite{vernoux1986heterogeneity}; TLC \cite{vernoux1986heterogeneity} \\
	\hline
	Triggerfish & \emph{Balistoides viridescens} (Titan triggerfish) & French Polynesia \cite{chinain2014mail} & CTX +ve \cite{chinain2014mail} & N2A//RBA \cite{chinain2014mail} \\
	\hline
	Tuna & \emph{Gymnosarda unicolor} (Dogtooth tuna) & Nuku Hiva, French Polynesia \cite{darius2007ciguatera}; French Polynesia \cite{bagnis1987use} & CTX +ve \cite{darius2007ciguatera,bagnis1987use} & Cat BA \cite{bagnis1987use}; MBA \cite{bagnis1987use}; MQBA \cite{bagnis1987use}; RBA \cite{darius2007ciguatera}\\
	\hline
	Unicorn fish & \emph{Naso brachycentron} ( Humpback unicornfish) & Nuku Hiva, French Polynesia \cite{darius2007ciguatera} & CTX +ve \cite{darius2007ciguatera} & RBA \cite{darius2007ciguatera} \\
	& \emph{Naso brevirostris} (Spotted unicornfish) & Nuku Hiva, French Polynesia \cite{darius2007ciguatera} & CTX +ve \cite{darius2007ciguatera} & RBA \cite{darius2007ciguatera} \\
	& \emph{Naso hexacanthus} (Sleek unicornfish) & Nuku Hiva, French Polynesia \cite{darius2007ciguatera} & CTX +ve \cite{darius2007ciguatera} & RBA \cite{darius2007ciguatera} \\
	& \emph{Naso lituratus} (Orangespine unicornfish) & Nuku Hiva, French Polynesia \cite{darius2007ciguatera}; French Polynesia \cite{chinain2014mail,bagnis1987use} & CTX +ve \cite{darius2007ciguatera,bagnis1987use,chinain2014mail} & Cat BA \cite{bagnis1987use}; MBA \cite{bagnis1987use}; MQBA \cite{bagnis1987use}; N2A//RBA \cite{chinain2014mail}; RBA \cite{darius2007ciguatera}\\
	& \emph{Naso unicornis} (Bluespine unicornfish) & Nuku Hiva, French Polynesia \cite{darius2007ciguatera}; French Polynesia \cite{chinain2014mail} & CTX +ve \cite{darius2007ciguatera,chinain2014mail} & N2A//RBA \cite{chinain2014mail}; RBA \cite{darius2007ciguatera} \\
	\hline
	Wrasse & \emph{Cheilinus undulatus} (Humphead wrasse) & Hong Kong, China \cite{wong2005study,wong2009solid}; French Polynesia \cite{bagnis1987use} & CTX +ve \cite{bagnis1987use,wong2005study,wong2009solid} & Cat BA \cite{bagnis1987use}; MBA \cite{bagnis1987use,wong2005study,wong2009solid}; MQBA \cite{bagnis1987use} \\
	& \emph{Coris aygula} (Clown coris) & Tubuai, French Polynesia \cite{darius2007ciguatera}; Republic of Kiribati \cite{mak2013pacific} & P-CTX-1 \cite{mak2013pacific}; P-CTX-2 \cite{mak2013pacific} & LC-MS/MS \cite{mak2013pacific}; N2A \cite{mak2013pacific}; RBA \cite{darius2007ciguatera} \\
	& \emph{Epibulus insidiator} (Slingjaw wrasse) & Republic of Kiribati \cite{mak2013pacific} & P-CTX-1 \cite{mak2013pacific} & LC-MS/MS \cite{mak2013pacific}; N2A \cite{mak2013pacific} \\
	& \emph{Semicossyphus} sp. & Baja California, Mexico \cite{lechuga1995documented} & CTX +ve \cite{lechuga1995documented} & MBA \cite{lechuga1995documented}\\
	\hline
\end{longtable}


\FloatBarrier


\section{Trophic transfer of toxins through the food web}

Stemming from the theory proposed by Randall et al that ciguateric toxins would be an environmental factor and the discovery by Yasumoto et al in 1977 that herbivorous or omnivorous fish consume \emph{Gambierdiscus} cells directly, it is widely accepted that the fish retain CTXs and start the bioaccumualtion process resulting in CFP \cite{randall1958review,yasumoto1977finding}.
However, the numbers of studies testing trophic transfer and toxin retention are limited.
A three year study published in 1966 by Banner et al found no significant decrease in toxicity in 129 specimens of red snapper (\emph{Lutjanus bohar}) kept in captivity \cite{banner1966retention}. This study pre-dates meaningful toxin analysis by current standards as toxicity was assessed by feeding the fish to mongoose and rate their subsequent behaviour and survival. The red snapper were caught and transported from the Line Islands, of which the Republic of Kiribati is a part of. That area is a well known ciguateric region where Mak et al found that \emph{Lutjanus bohar} was highly toxic, second only to the Moray eel \emph{Gymnothorax javanicus} \cite{mak2013pacific}. It can be tentatively extrapolated that the toxicity observed is ciguatera related, but which toxin or congener caused the mongoose to respond is not clear. \\

In the study by Mak et al tracked the components of a partial food web in a ciguateric reef systems in the Republic of Kiribati. Distribution of P-CTX-1, P-CTX-2 and P-CTX-3 in the ciguateric environment was elucidated with emphasis on biomagnification of P-CTX as it moves through the food web as well as importance of species/genus specific dietary habits impacting the P-CTX concentration of fishes. Of 205 fishes and 15 invertebrates tested, 54\% of herbivorous, 72\% of omnivorous and 76\% of carnivorous fish as well as a lobster and an octopus tested positive for P-CTX presence. Grazers predominantly exhibited P-CTX-2 while the principal CTX found in piscivorous fish was P-CTX-1. Using nitrogen accumulation as an indication of position in the food web, they found an indication that P-CTX-1, unlike P-CTX-2 and -3, possessed a relatively high biomagnification tendency. This positive correlation was observable both for individual samples and when grouped into their corresponding families. They postulate that P-CTX-2 and -3 could be precursors to P-CTX-1 \cite{mak2013pacific}. This study gives an unprecedented insight into P-CTX-1, -2 and -3 abundance in different trophic strata. The author postulates that this study reflects differential, genus specific suitability to toxin accumulation - where genera with lower tolerance would be eliminated from the food chain \cite{makICHA}. It does, however not yield insight in that first step of toxin transfer in the trophic system, from Gambierdiscus spp. to herbivore or omnivore. The following feeding studies illuminate this as yet cryptic step. 

A study conducted by Ledreux et al examined the CTX retention by mullet (\emph{Mugil cephalus}) by feeding known quantities of \emph{G. polynesiensis} which showed rapid clearance of CTXs from blood and tissue \cite{ledreux2014dynamics}. 
Mullet received 9 toxic feeding, two days apart after day 2, and were tested at several time intervals with N2A. Eight hours after a single exposure, they found 63\% of the detected CTXs in the muscle, which dropped to 15\% after 12 hrs. The peak presence of CTXs in was 8 hrs post exposure, however the concentration diminished with repeat exposure in muscle and blood from 63\% and 22\% to 21\% and 1\% respectively, between the first and the 9th (day 16) feeding. Ledreux et al found that CTXs had a half-life of 4.2 hrs in mullet and a mean retention time of 5.9 hrs. As they discuss in the paper, their results show no indication of biomagnification due to the quick clearance of toxin, increased clearance efficiency with repeat exposure and the extrapolated biomagnification factor indicates that a predator would also clear the toxins  \cite{ledreux2014dynamics}. \\
 N2A is a useful assay to indicate CTX toxicity, but it has a broad range of neurotoxin detection and is  hence not specific to CTXs \cite{ebesu2012comment} and it cannot differentiate between different congeners. Ledreux et al did state that their results were specific to CTX-3 and did not encompass CTX-4, and since the latter is more lipophillic, this may result in misleading toxin retention results \cite{ledreux2014dynamics}. The \emph{G. polynesiensis} strain used was TB-92. whose toxin profile was characterised by Chinain et al \cite{chinain2010growth}. Their LC-MS analysis showed that CTX-4A and -4B account for 17\% of the CTX profile and 3\% were labelled miscellaneous. Furthermore they found that 50\% of toxicity during the life cycle of \emph{G. polynesiensis} TB-92, as assessed via behaviour and lethality upon i.p. injection in mice, was accountable to the methanolic "MTX-like" fraction extraction from culture \cite{chinain2010growth}. Ledreux et al potentially only tested a 40\% of the total known toxins and not including CTX-4 which has been proposed as the precursor for CTX-1B through biomagnification \cite{murata1990structures}. Another consideration is that the \emph{Gambierdiscus} cells were at the late exponential growth phase when frozen for feeding by Ledreux et al. In the toxin profiling conducted by Chinain et al they found that CTX-3C production was lowest at day 15, around the late exponential growth phase, with 3 pg CTX-3C equivalent per cell \cite{chinain2010growth}. This may contribute to the concentration of CTXs in blood extracts being too low for Ledreux et al to analyse with LC-MS. \\
The choice of fish species to conduct the study with is questionable. While two different omnivorous mullet genera have been implicated in harbouring CTXs as per table ~\ref{tbl:OmniTable}, out of 73 cases of CFP in Florida between 1954 to 1992, only one case was attributed to a \emph{Mugil} sp. by hearsay \cite{de1994distribution}. In the study conducted by Mak et al on 205 coral reef fish as part of a ciguateric food web, specimens were screened for CTX-1, -2 and -3 via LC-MS. While no mullet species were included in the study, it did find that of 24 omnivores specimens, across 14 species, they presented with a comparatively lower concentration of CTXs than herbivores as well as carnivores and that CTX-3 was generally the least represented of the three CTXs tested \cite{mak2013pacific}. 
The choices of toxin and fish species monitored may be inadequate for assessing CTX trophic transfer. It would be of value to repeat the extensive study conducted by Banner et al in 1966 using LC-MS analysis to clarity the results obtained by Ledreux et al.\\
%show chinain Fig 3 to Shauna - is D15 late exponential growth phase? because that's the very lowest CTX-3C conc in the entire profile
%carnivorous, toxic fish usuallu contain CTX-1B.. check Mak. and has it been suggested that CTX-4 is the precursor? yes \cite{murata1990structures}


Another recent feeding study conducted by Kohli et al illuminates the current lack of knowledge regarding the possibility of MTX biomagnification and hence a potential role in CFP. In this study snapper (\emph{Pagrus auratus}) were fed juvenile herbivorous mullet (\emph{Aldrichetta forsteri}), injected with a known quantity of\emph{ G. australes} - a strain characterised to only produce MTX  \cite{kohli2014feeding}. T They demonstrated that MTX were present in snapper liver, digestive organs and muscle in concentrations high enough to measure via LC-MS. As no toxicity threshold for MTX has been established in humans, the authors extrapolate, based on the toxicity of other known neurotoxins, that the concentrations of MTX detected could be harmful in an averagely sized meal of fish flesh\cite{kohli2014feeding}.
%The tyoe of mullet used in this study is herbivorous, and while there is no other data availible for MTX accumulation in seafood, if it can be assumed hat they are generally accumulated along with CTXs
By injecting the deceased mullet with \emph{G. australes} rather than feeding the cells to them, effectively the first step of trophic transfer is skipped along with possible indications whether MTX, like CTX, transforms as it passes through the food chain. //

The findings in these feeding studies elucidate that there is a distinct lack of information regarding the mechanism of trophic toxin transfer. Ledreux et al revealed a distinct lack of information regarding the mechanism of CTX transfer between the first and second trophic levels, which is the basis for the generally accepted theory on biomagnification resulting in CFP. Kohli et al show that it is imperative that the involvement of MTX in CFP, as well as the mode of action of MTX in humans, is established. Monitoring of MTXs alongside CTXs in ciguateric sea food should ideally be adopted as standard practice. The studies also highlight the need for careful consideration when mimicking environmental processes, as far as known, by choice of fish species, mode of Gambierdiscus delivery and analysis via LC-MS analysis.

%Initially in the 70s, MTXs were detected only in the guts of fish \cite{yasumoto1976toxicity}, so the prevalent theory that followed has been that they are irrelevant in CFP research due to the assumption that their water soluble nature would exclude them from being absorbed into flesh in concentrations relevant for biomagnification. This popular theory needs to be reconsidered, as earlier in the year a fish feeding study conducted by Kohli et al that 
%check detection method of yasumoto1976

\section{Conclusion}
\emph{Gambierduiscus} is a genus of BHABs that are responsible for CFP. With the increase in frequency of CFP cases globally, as well as recorded blooms, it is essential to complement our currently fragmentary understanding of the phylogeny, distribution, abundance and species specific toxin profiles in order to monitor for potential health hazards. Aspects include elucidating the species specific toxin profiles with LC-MS, sampling extensively including at different depth and determination of the phylogenetic relationship between species as well as finding toxin markers.
Due to the  advancement of \emph{Gambierdiscus} spp. to new areas and CFP incidence frequency predicted to rise drastically, the mechanism of biomagnification of CTXs through the food chain needs to be scrutinised as our understanding of the trophic components of ciguateric food webs are still expanding.
It is essential to track the presence of CTX in lower trophic level of the food web to predict in which genera bioaccumulation is likely \cite{mak2013pacific} and investigate species other than fish as coastal toxin vectors. Furthermore it needs to be established if MTX can bioaccumulate and play a role in CFP. For this purpose MTX should be tested for along with CTX in suspected contaminated samples.
Finally for proper risk assessment relevant to monitoring by the seafood industry, the field needs to move away from i.p. injections and include long term CTX toxicity studies.

\newpage
\bibliographystyle{plain}
\bibliography{review_ref.bib}

%what was Ballantine cited for?

\end{document}
